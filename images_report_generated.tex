% Auto-generated LaTeX Images Report
% Generated for IEEE-style academic paper
% Include this file with: % Auto-generated LaTeX Images Report
% Generated for IEEE-style academic paper
% Include this file with: % Auto-generated LaTeX Images Report
% Generated for IEEE-style academic paper
% Include this file with: % Auto-generated LaTeX Images Report
% Generated for IEEE-style academic paper
% Include this file with: \input{images_report_generated.tex}
% Requires: \usepackage{graphicx}

\subsection{1000 nodes Module Change}

\begin{figure}[!htbp]
\centering
\includegraphics[width=0.95\linewidth]{images/1000_nodes_Module_Change/(a)_Travel_Time_bar_chart.png}
\caption{(a) Travel Time}
\label{fig:1000_nodes_Module_Change_a__Travel_Time_bar_chart}
\end{figure}

This bar chart presents the relationship between modules and total travel time for the 1000 nodes Module Change experimental scenario. The x-axis displays modules values ranging from 4 to 7, while the y-axis quantifies total travel time. The single-series visualization facilitates analysis of how the dependent variable responds to changes in the independent parameter setting.

Analysis of the plotted data reveals that total travel time ranges from 6779.83 (at modules = 4) to 6808.90 (at modules = 5), representing a span of 29.07 units. The overall trend is increasing, with values rising from 6779.83 at the initial setting to 6808.90 at the final setting. 

These results indicate that modules configuration meaningfully impacts total travel time in this experimental context. The relatively modest variation suggests that this parameter has limited influence on the measured metric within the tested range. Confidence in these findings is high given the direct correspondence between CSV data and plotted values. Future analysis should consider incorporating error bars representing variance across multiple experimental runs to strengthen statistical validity.

\begin{figure}[!htbp]
\centering
\includegraphics[width=0.95\linewidth]{images/1000_nodes_Module_Change/(b)_Energy_consumption_bar_chart.png}
\caption{(b) Energy consumption}
\label{fig:1000_nodes_Module_Change_b__Energy_consumption_bar_chart}
\end{figure}

This bar chart presents the relationship between modules and total energy consumed for the 1000 nodes Module Change experimental scenario. The x-axis displays modules values ranging from 4 to 7, while the y-axis quantifies total energy consumed. The single-series visualization facilitates analysis of how the dependent variable responds to changes in the independent parameter setting.

Analysis of the plotted data reveals that total energy consumed ranges from 793.93 (at modules = 4) to 795.16 (at modules = 5), representing a span of 1.22 units. The overall trend is increasing, with values rising from 793.93 at the initial setting to 795.16 at the final setting. 

These results indicate that modules configuration meaningfully impacts total energy consumed in this experimental context. The relatively modest variation suggests that this parameter has limited influence on the measured metric within the tested range. Confidence in these findings is high given the direct correspondence between CSV data and plotted values. Future analysis should consider incorporating error bars representing variance across multiple experimental runs to strengthen statistical validity.

\begin{figure}[!htbp]
\centering
\includegraphics[width=0.95\linewidth]{images/1000_nodes_Module_Change/(c)_Distance_covered_bar_chart.png}
\caption{(c) Distance covered}
\label{fig:1000_nodes_Module_Change_c__Distance_covered_bar_chart}
\end{figure}

This bar chart presents the relationship between modules and total distance covered for the 1000 nodes Module Change experimental scenario. The x-axis displays modules values ranging from 4 to 7, while the y-axis quantifies total distance covered. The single-series visualization facilitates analysis of how the dependent variable responds to changes in the independent parameter setting.

Analysis of the plotted data reveals that total distance covered ranges from 4422.50 (at modules = 4) to 4437.52 (at modules = 5), representing a span of 15.02 units. The overall trend is increasing, with values rising from 4422.50 at the initial setting to 4437.52 at the final setting. 

These results indicate that modules configuration meaningfully impacts total distance covered in this experimental context. The relatively modest variation suggests that this parameter has limited influence on the measured metric within the tested range. Confidence in these findings is high given the direct correspondence between CSV data and plotted values. Future analysis should consider incorporating error bars representing variance across multiple experimental runs to strengthen statistical validity.

\begin{figure}[!htbp]
\centering
\includegraphics[width=0.95\linewidth]{images/1000_nodes_Module_Change/(d)_Runtime_bar_chart.png}
\caption{(d) Runtime}
\label{fig:1000_nodes_Module_Change_d__Runtime_bar_chart}
\end{figure}

This bar chart presents the relationship between modules and run time for the 1000 nodes Module Change experimental scenario. The x-axis displays modules values ranging from 4 to 7, while the y-axis quantifies run time. The single-series visualization facilitates analysis of how the dependent variable responds to changes in the independent parameter setting.

Analysis of the plotted data reveals that run time ranges from 187.29 (at modules = 4) to 396.10 (at modules = 5), representing a span of 208.81 units. The overall trend is increasing, with values rising from 187.29 at the initial setting to 226.36 at the final setting. 

These results indicate that modules configuration meaningfully impacts run time in this experimental context. The substantial variation observed (coefficient of variation exceeding 20\%) suggests that parameter tuning could yield significant performance improvements. Confidence in these findings is high given the direct correspondence between CSV data and plotted values. Future analysis should consider incorporating error bars representing variance across multiple experimental runs to strengthen statistical validity.

\begin{figure}[!htbp]
\centering
\includegraphics[width=0.95\linewidth]{images/1000_nodes_Module_Change/(e)_Number_of_module_swapped_bar_chart.png}
\caption{(e) Number of module swapped}
\label{fig:1000_nodes_Module_Change_e__Number_of_module_swapped_bar_chart}
\end{figure}

This bar chart presents the relationship between modules and modules for the 1000 nodes Module Change experimental scenario. The x-axis displays modules values ranging from 4 to 7, while the y-axis quantifies modules. The single-series visualization facilitates analysis of how the dependent variable responds to changes in the independent parameter setting.

Analysis of the plotted data reveals that modules ranges from 4.00 (at modules = 4) to 7.00 (at modules = 7), representing a span of 3.00 units. The overall trend is increasing, with values rising from 4.00 at the initial setting to 7.00 at the final setting. 

These results indicate that modules configuration meaningfully impacts modules in this experimental context. The substantial variation observed (coefficient of variation exceeding 20\%) suggests that parameter tuning could yield significant performance improvements. Confidence in these findings is high given the direct correspondence between CSV data and plotted values. Future analysis should consider incorporating error bars representing variance across multiple experimental runs to strengthen statistical validity.


\clearpage

\subsection{1000 nodes Swapping Time}

\begin{figure}[!htbp]
\centering
\includegraphics[width=0.95\linewidth]{images/1000_nodes_Swapping_Time/(a)_Travel_Time_bar_chart.png}
\caption{(a) Travel Time}
\label{fig:1000_nodes_Swapping_Time_a__Travel_Time_bar_chart}
\end{figure}

This bar chart presents the relationship between swapping time and total travel time for the 1000 nodes Swapping Time experimental scenario. The x-axis displays swapping time values ranging from 1 to 4, while the y-axis quantifies total travel time. The single-series visualization facilitates analysis of how the dependent variable responds to changes in the independent parameter setting.

Analysis of the plotted data reveals that total travel time ranges from 6771.90 (at swapping time = 1) to 6882.78 (at swapping time = 4), representing a span of 110.88 units. The overall trend is increasing, with values rising from 6771.90 at the initial setting to 6882.78 at the final setting. 

These results indicate that swapping time configuration meaningfully impacts total travel time in this experimental context. The relatively modest variation suggests that this parameter has limited influence on the measured metric within the tested range. Confidence in these findings is high given the direct correspondence between CSV data and plotted values. Future analysis should consider incorporating error bars representing variance across multiple experimental runs to strengthen statistical validity.

\begin{figure}[!htbp]
\centering
\includegraphics[width=0.95\linewidth]{images/1000_nodes_Swapping_Time/(b)_Energy_consumption_bar_chart.png}
\caption{(b) Energy consumption}
\label{fig:1000_nodes_Swapping_Time_b__Energy_consumption_bar_chart}
\end{figure}

This bar chart presents the relationship between swapping time and total energy consumed for the 1000 nodes Swapping Time experimental scenario. The x-axis displays swapping time values ranging from 1 to 4, while the y-axis quantifies total energy consumed. The single-series visualization facilitates analysis of how the dependent variable responds to changes in the independent parameter setting.

Analysis of the plotted data reveals that total energy consumed ranges from 794.98 (at swapping time = 3) to 795.16 (at swapping time = 1), representing a span of 0.18 units. The overall trend is decreasing, with values declining from 795.16 at the initial setting to 794.98 at the final setting. Notably, the minimum value occurs at an intermediate swapping time setting (3), suggesting non-monotonic behavior that warrants further investigation.

These results indicate that swapping time configuration meaningfully impacts total energy consumed in this experimental context. The relatively modest variation suggests that this parameter has limited influence on the measured metric within the tested range. Confidence in these findings is high given the direct correspondence between CSV data and plotted values. Future analysis should consider incorporating error bars representing variance across multiple experimental runs to strengthen statistical validity.

\begin{figure}[!htbp]
\centering
\includegraphics[width=0.95\linewidth]{images/1000_nodes_Swapping_Time/(c)_Distance_covered_bar_chart.png}
\caption{(c) Distance covered}
\label{fig:1000_nodes_Swapping_Time_c__Distance_covered_bar_chart}
\end{figure}

This bar chart presents the relationship between swapping time and total distance covered for the 1000 nodes Swapping Time experimental scenario. The x-axis displays swapping time values ranging from 1 to 4, while the y-axis quantifies total distance covered. The single-series visualization facilitates analysis of how the dependent variable responds to changes in the independent parameter setting.

Analysis of the plotted data reveals that total distance covered ranges from 4436.55 (at swapping time = 3) to 4437.52 (at swapping time = 1), representing a span of 0.97 units. The overall trend is decreasing, with values declining from 4437.52 at the initial setting to 4436.55 at the final setting. Notably, the minimum value occurs at an intermediate swapping time setting (3), suggesting non-monotonic behavior that warrants further investigation.

These results indicate that swapping time configuration meaningfully impacts total distance covered in this experimental context. The relatively modest variation suggests that this parameter has limited influence on the measured metric within the tested range. Confidence in these findings is high given the direct correspondence between CSV data and plotted values. Future analysis should consider incorporating error bars representing variance across multiple experimental runs to strengthen statistical validity.

\begin{figure}[!htbp]
\centering
\includegraphics[width=0.95\linewidth]{images/1000_nodes_Swapping_Time/(d)_Runtime_bar_chart.png}
\caption{(d) Runtime}
\label{fig:1000_nodes_Swapping_Time_d__Runtime_bar_chart}
\end{figure}

This bar chart presents the relationship between swapping time and run time for the 1000 nodes Swapping Time experimental scenario. The x-axis displays swapping time values ranging from 1 to 4, while the y-axis quantifies run time. The single-series visualization facilitates analysis of how the dependent variable responds to changes in the independent parameter setting.

Analysis of the plotted data reveals that run time ranges from 190.96 (at swapping time = 3) to 399.38 (at swapping time = 4), representing a span of 208.42 units. The overall trend is increasing, with values rising from 363.28 at the initial setting to 399.38 at the final setting. Notably, the minimum value occurs at an intermediate swapping time setting (3), suggesting non-monotonic behavior that warrants further investigation.

These results indicate that swapping time configuration meaningfully impacts run time in this experimental context. The substantial variation observed (coefficient of variation exceeding 20\%) suggests that parameter tuning could yield significant performance improvements. Confidence in these findings is high given the direct correspondence between CSV data and plotted values. Future analysis should consider incorporating error bars representing variance across multiple experimental runs to strengthen statistical validity.

\begin{figure}[!htbp]
\centering
\includegraphics[width=0.95\linewidth]{images/1000_nodes_Swapping_Time/(e)_Number_of_module_swapped_bar_chart.png}
\caption{(e) Number of module swapped}
\label{fig:1000_nodes_Swapping_Time_e__Number_of_module_swapped_bar_chart}
\end{figure}

This bar chart presents the relationship between swapping time and total module swapped for the 1000 nodes Swapping Time experimental scenario. The x-axis displays swapping time values ranging from 1 to 4, while the y-axis quantifies total module swapped. The single-series visualization facilitates analysis of how the dependent variable responds to changes in the independent parameter setting.

Analysis of the plotted data reveals that total module swapped ranges from 37.00 (at swapping time = 1) to 37.00 (at swapping time = 1), representing a span of 0.00 units. The values remain relatively stable across the parameter range, with minimal net change between initial (37.00) and final (37.00) settings. 

These results indicate that swapping time configuration meaningfully impacts total module swapped in this experimental context. The relatively modest variation suggests that this parameter has limited influence on the measured metric within the tested range. Confidence in these findings is high given the direct correspondence between CSV data and plotted values. Future analysis should consider incorporating error bars representing variance across multiple experimental runs to strengthen statistical validity.


\clearpage

\subsection{1000 nodes Threshold}

\begin{figure}[!htbp]
\centering
\includegraphics[width=0.95\linewidth]{images/1000_nodes_Threshold/(a)_Travel_Time_bar_chart.png}
\caption{(a) Travel Time}
\label{fig:1000_nodes_Threshold_a__Travel_Time_bar_chart}
\end{figure}

This bar chart presents the relationship between threshold and total travel time for the 1000 nodes Threshold experimental scenario. The x-axis displays threshold values ranging from 5 to 20, while the y-axis quantifies total travel time. The single-series visualization facilitates analysis of how the dependent variable responds to changes in the independent parameter setting.

Analysis of the plotted data reveals that total travel time ranges from 6808.90 (at threshold = 5) to 6808.90 (at threshold = 5), representing a span of 0.00 units. The values remain relatively stable across the parameter range, with minimal net change between initial (6808.90) and final (6808.90) settings. 

These results indicate that threshold configuration meaningfully impacts total travel time in this experimental context. The relatively modest variation suggests that this parameter has limited influence on the measured metric within the tested range. Confidence in these findings is high given the direct correspondence between CSV data and plotted values. Future analysis should consider incorporating error bars representing variance across multiple experimental runs to strengthen statistical validity.

\begin{figure}[!htbp]
\centering
\includegraphics[width=0.95\linewidth]{images/1000_nodes_Threshold/(b)_Energy_consumption_bar_chart.png}
\caption{(b) Energy consumption}
\label{fig:1000_nodes_Threshold_b__Energy_consumption_bar_chart}
\end{figure}

This bar chart presents the relationship between threshold and total energy consumed for the 1000 nodes Threshold experimental scenario. The x-axis displays threshold values ranging from 5 to 20, while the y-axis quantifies total energy consumed. The single-series visualization facilitates analysis of how the dependent variable responds to changes in the independent parameter setting.

Analysis of the plotted data reveals that total energy consumed ranges from 795.16 (at threshold = 5) to 795.16 (at threshold = 5), representing a span of 0.00 units. The values remain relatively stable across the parameter range, with minimal net change between initial (795.16) and final (795.16) settings. 

These results indicate that threshold configuration meaningfully impacts total energy consumed in this experimental context. The relatively modest variation suggests that this parameter has limited influence on the measured metric within the tested range. Confidence in these findings is high given the direct correspondence between CSV data and plotted values. Future analysis should consider incorporating error bars representing variance across multiple experimental runs to strengthen statistical validity.

\begin{figure}[!htbp]
\centering
\includegraphics[width=0.95\linewidth]{images/1000_nodes_Threshold/(c)_Distance_covered_bar_chart.png}
\caption{(c) Distance covered}
\label{fig:1000_nodes_Threshold_c__Distance_covered_bar_chart}
\end{figure}

This bar chart presents the relationship between threshold and total distance covered for the 1000 nodes Threshold experimental scenario. The x-axis displays threshold values ranging from 5 to 20, while the y-axis quantifies total distance covered. The single-series visualization facilitates analysis of how the dependent variable responds to changes in the independent parameter setting.

Analysis of the plotted data reveals that total distance covered ranges from 4437.52 (at threshold = 5) to 4437.52 (at threshold = 5), representing a span of 0.00 units. The values remain relatively stable across the parameter range, with minimal net change between initial (4437.52) and final (4437.52) settings. 

These results indicate that threshold configuration meaningfully impacts total distance covered in this experimental context. The relatively modest variation suggests that this parameter has limited influence on the measured metric within the tested range. Confidence in these findings is high given the direct correspondence between CSV data and plotted values. Future analysis should consider incorporating error bars representing variance across multiple experimental runs to strengthen statistical validity.

\begin{figure}[!htbp]
\centering
\includegraphics[width=0.95\linewidth]{images/1000_nodes_Threshold/(d)_Runtime_bar_chart.png}
\caption{(d) Runtime}
\label{fig:1000_nodes_Threshold_d__Runtime_bar_chart}
\end{figure}

This bar chart presents the relationship between threshold and run time for the 1000 nodes Threshold experimental scenario. The x-axis displays threshold values ranging from 5 to 20, while the y-axis quantifies run time. The single-series visualization facilitates analysis of how the dependent variable responds to changes in the independent parameter setting.

Analysis of the plotted data reveals that run time ranges from 225.32 (at threshold = 15) to 396.10 (at threshold = 20), representing a span of 170.78 units. The overall trend is increasing, with values rising from 287.89 at the initial setting to 396.10 at the final setting. Notably, the minimum value occurs at an intermediate threshold setting (15), suggesting non-monotonic behavior that warrants further investigation.

These results indicate that threshold configuration meaningfully impacts run time in this experimental context. The substantial variation observed (coefficient of variation exceeding 20\%) suggests that parameter tuning could yield significant performance improvements. Confidence in these findings is high given the direct correspondence between CSV data and plotted values. Future analysis should consider incorporating error bars representing variance across multiple experimental runs to strengthen statistical validity.

\begin{figure}[!htbp]
\centering
\includegraphics[width=0.95\linewidth]{images/1000_nodes_Threshold/(e)_Number_of_module_swapped_bar_chart.png}
\caption{(e) Number of module swapped}
\label{fig:1000_nodes_Threshold_e__Number_of_module_swapped_bar_chart}
\end{figure}

This bar chart presents the relationship between threshold and total module swapped for the 1000 nodes Threshold experimental scenario. The x-axis displays threshold values ranging from 5 to 20, while the y-axis quantifies total module swapped. The single-series visualization facilitates analysis of how the dependent variable responds to changes in the independent parameter setting.

Analysis of the plotted data reveals that total module swapped ranges from 37.00 (at threshold = 5) to 37.00 (at threshold = 5), representing a span of 0.00 units. The values remain relatively stable across the parameter range, with minimal net change between initial (37.00) and final (37.00) settings. 

These results indicate that threshold configuration meaningfully impacts total module swapped in this experimental context. The relatively modest variation suggests that this parameter has limited influence on the measured metric within the tested range. Confidence in these findings is high given the direct correspondence between CSV data and plotted values. Future analysis should consider incorporating error bars representing variance across multiple experimental runs to strengthen statistical validity.


\clearpage

\subsection{1000 nodes Traffic}

\begin{figure}[!htbp]
\centering
\includegraphics[width=0.95\linewidth]{images/1000_nodes_Traffic/(a)_Travel_Time_bar_chart.png}
\caption{(a) Travel Time}
\label{fig:1000_nodes_Traffic_a__Travel_Time_bar_chart}
\end{figure}

This bar chart presents the relationship between traffic and total travel time for the 1000 nodes Traffic experimental scenario. The x-axis displays traffic values ranging from High to Low, while the y-axis quantifies total travel time. The single-series visualization facilitates analysis of how the dependent variable responds to changes in the independent parameter setting.

Analysis of the plotted data reveals that total travel time ranges from 6269.66 (at traffic = Low) to 8004.34 (at traffic = High), representing a span of 1734.68 units. The overall trend is decreasing, with values declining from 8004.34 at the initial setting to 6269.66 at the final setting. 

These results indicate that traffic configuration meaningfully impacts total travel time in this experimental context. The substantial variation observed (coefficient of variation exceeding 20\%) suggests that parameter tuning could yield significant performance improvements. Confidence in these findings is high given the direct correspondence between CSV data and plotted values. Future analysis should consider incorporating error bars representing variance across multiple experimental runs to strengthen statistical validity.

\begin{figure}[!htbp]
\centering
\includegraphics[width=0.95\linewidth]{images/1000_nodes_Traffic/(b)_Energy_consumption_bar_chart.png}
\caption{(b) Energy consumption}
\label{fig:1000_nodes_Traffic_b__Energy_consumption_bar_chart}
\end{figure}

This bar chart presents the relationship between traffic and total energy consumed for the 1000 nodes Traffic experimental scenario. The x-axis displays traffic values ranging from High to Low, while the y-axis quantifies total energy consumed. The single-series visualization facilitates analysis of how the dependent variable responds to changes in the independent parameter setting.

Analysis of the plotted data reveals that total energy consumed ranges from 757.04 (at traffic = High) to 883.10 (at traffic = Low), representing a span of 126.06 units. The overall trend is increasing, with values rising from 757.04 at the initial setting to 883.10 at the final setting. 

These results indicate that traffic configuration meaningfully impacts total energy consumed in this experimental context. The relatively modest variation suggests that this parameter has limited influence on the measured metric within the tested range. Confidence in these findings is high given the direct correspondence between CSV data and plotted values. Future analysis should consider incorporating error bars representing variance across multiple experimental runs to strengthen statistical validity.

\begin{figure}[!htbp]
\centering
\includegraphics[width=0.95\linewidth]{images/1000_nodes_Traffic/(c)_Distance_covered_bar_chart.png}
\caption{(c) Distance covered}
\label{fig:1000_nodes_Traffic_c__Distance_covered_bar_chart}
\end{figure}

This bar chart presents the relationship between traffic and total distance covered for the 1000 nodes Traffic experimental scenario. The x-axis displays traffic values ranging from High to Low, while the y-axis quantifies total distance covered. The single-series visualization facilitates analysis of how the dependent variable responds to changes in the independent parameter setting.

Analysis of the plotted data reveals that total distance covered ranges from 4374.17 (at traffic = High) to 4806.46 (at traffic = Low), representing a span of 432.29 units. The overall trend is increasing, with values rising from 4374.17 at the initial setting to 4806.46 at the final setting. 

These results indicate that traffic configuration meaningfully impacts total distance covered in this experimental context. The relatively modest variation suggests that this parameter has limited influence on the measured metric within the tested range. Confidence in these findings is high given the direct correspondence between CSV data and plotted values. Future analysis should consider incorporating error bars representing variance across multiple experimental runs to strengthen statistical validity.

\begin{figure}[!htbp]
\centering
\includegraphics[width=0.95\linewidth]{images/1000_nodes_Traffic/(d)_Runtime_bar_chart.png}
\caption{(d) Runtime}
\label{fig:1000_nodes_Traffic_d__Runtime_bar_chart}
\end{figure}

This bar chart presents the relationship between traffic and run time for the 1000 nodes Traffic experimental scenario. The x-axis displays traffic values ranging from High to Low, while the y-axis quantifies run time. The single-series visualization facilitates analysis of how the dependent variable responds to changes in the independent parameter setting.

Analysis of the plotted data reveals that run time ranges from 368.03 (at traffic = High) to 407.25 (at traffic = Mid), representing a span of 39.22 units. The overall trend is increasing, with values rising from 368.03 at the initial setting to 382.14 at the final setting. 

These results indicate that traffic configuration meaningfully impacts run time in this experimental context. The relatively modest variation suggests that this parameter has limited influence on the measured metric within the tested range. Confidence in these findings is high given the direct correspondence between CSV data and plotted values. Future analysis should consider incorporating error bars representing variance across multiple experimental runs to strengthen statistical validity.

\begin{figure}[!htbp]
\centering
\includegraphics[width=0.95\linewidth]{images/1000_nodes_Traffic/(e)_Number_of_module_swapped_bar_chart.png}
\caption{(e) Number of module swapped}
\label{fig:1000_nodes_Traffic_e__Number_of_module_swapped_bar_chart}
\end{figure}

This bar chart presents the relationship between traffic and total module swapped for the 1000 nodes Traffic experimental scenario. The x-axis displays traffic values ranging from High to Low, while the y-axis quantifies total module swapped. The single-series visualization facilitates analysis of how the dependent variable responds to changes in the independent parameter setting.

Analysis of the plotted data reveals that total module swapped ranges from 35.00 (at traffic = High) to 41.00 (at traffic = Low), representing a span of 6.00 units. The overall trend is increasing, with values rising from 35.00 at the initial setting to 41.00 at the final setting. 

These results indicate that traffic configuration meaningfully impacts total module swapped in this experimental context. The relatively modest variation suggests that this parameter has limited influence on the measured metric within the tested range. Confidence in these findings is high given the direct correspondence between CSV data and plotted values. Future analysis should consider incorporating error bars representing variance across multiple experimental runs to strengthen statistical validity.


\clearpage

\subsection{1500 nodes Module Change}

\begin{figure}[!htbp]
\centering
\includegraphics[width=0.95\linewidth]{images/1500_nodes_Module_Change/(a)_Travel_time_bar_chart.png}
\caption{(a) Travel time}
\label{fig:1500_nodes_Module_Change_a__Travel_time_bar_chart}
\end{figure}

This bar chart presents the relationship between modules and total travel time for the 1500 nodes Module Change experimental scenario. The x-axis displays modules values ranging from 4 to 7, while the y-axis quantifies total travel time. The single-series visualization facilitates analysis of how the dependent variable responds to changes in the independent parameter setting.

Analysis of the plotted data reveals that total travel time ranges from 10133.53 (at modules = 5) to 10275.22 (at modules = 4), representing a span of 141.69 units. The overall trend is decreasing, with values declining from 10275.22 at the initial setting to 10148.53 at the final setting. Notably, the minimum value occurs at an intermediate modules setting (5), suggesting non-monotonic behavior that warrants further investigation.

These results indicate that modules configuration meaningfully impacts total travel time in this experimental context. The relatively modest variation suggests that this parameter has limited influence on the measured metric within the tested range. Confidence in these findings is high given the direct correspondence between CSV data and plotted values. Future analysis should consider incorporating error bars representing variance across multiple experimental runs to strengthen statistical validity.

\begin{figure}[!htbp]
\centering
\includegraphics[width=0.95\linewidth]{images/1500_nodes_Module_Change/(b)_Energy_consumed_bar_chart.png}
\caption{(b) Energy consumed}
\label{fig:1500_nodes_Module_Change_b__Energy_consumed_bar_chart}
\end{figure}

This bar chart presents the relationship between modules and total energy consumed for the 1500 nodes Module Change experimental scenario. The x-axis displays modules values ranging from 4 to 7, while the y-axis quantifies total energy consumed. The single-series visualization facilitates analysis of how the dependent variable responds to changes in the independent parameter setting.

Analysis of the plotted data reveals that total energy consumed ranges from 1346.58 (at modules = 5) to 1351.12 (at modules = 4), representing a span of 4.55 units. The overall trend is decreasing, with values declining from 1351.12 at the initial setting to 1347.93 at the final setting. Notably, the minimum value occurs at an intermediate modules setting (5), suggesting non-monotonic behavior that warrants further investigation.

These results indicate that modules configuration meaningfully impacts total energy consumed in this experimental context. The relatively modest variation suggests that this parameter has limited influence on the measured metric within the tested range. Confidence in these findings is high given the direct correspondence between CSV data and plotted values. Future analysis should consider incorporating error bars representing variance across multiple experimental runs to strengthen statistical validity.

\begin{figure}[!htbp]
\centering
\includegraphics[width=0.95\linewidth]{images/1500_nodes_Module_Change/(c)_Distance_covered_bar_chart.png}
\caption{(c) Distance covered}
\label{fig:1500_nodes_Module_Change_c__Distance_covered_bar_chart}
\end{figure}

This bar chart presents the relationship between modules and total distance covered for the 1500 nodes Module Change experimental scenario. The x-axis displays modules values ranging from 4 to 7, while the y-axis quantifies total distance covered. The single-series visualization facilitates analysis of how the dependent variable responds to changes in the independent parameter setting.

Analysis of the plotted data reveals that total distance covered ranges from 6533.82 (at modules = 5) to 6628.04 (at modules = 4), representing a span of 94.22 units. The overall trend is decreasing, with values declining from 6628.04 at the initial setting to 6545.22 at the final setting. Notably, the minimum value occurs at an intermediate modules setting (5), suggesting non-monotonic behavior that warrants further investigation.

These results indicate that modules configuration meaningfully impacts total distance covered in this experimental context. The relatively modest variation suggests that this parameter has limited influence on the measured metric within the tested range. Confidence in these findings is high given the direct correspondence between CSV data and plotted values. Future analysis should consider incorporating error bars representing variance across multiple experimental runs to strengthen statistical validity.

\begin{figure}[!htbp]
\centering
\includegraphics[width=0.95\linewidth]{images/1500_nodes_Module_Change/(d)_Run_time_bar_chart.png}
\caption{(d) Run time}
\label{fig:1500_nodes_Module_Change_d__Run_time_bar_chart}
\end{figure}

This bar chart presents the relationship between modules and total travel time for the 1500 nodes Module Change experimental scenario. The x-axis displays modules values ranging from 4 to 7, while the y-axis quantifies total travel time. The single-series visualization facilitates analysis of how the dependent variable responds to changes in the independent parameter setting.

Analysis of the plotted data reveals that total travel time ranges from 10133.53 (at modules = 5) to 10275.22 (at modules = 4), representing a span of 141.69 units. The overall trend is decreasing, with values declining from 10275.22 at the initial setting to 10148.53 at the final setting. Notably, the minimum value occurs at an intermediate modules setting (5), suggesting non-monotonic behavior that warrants further investigation.

These results indicate that modules configuration meaningfully impacts total travel time in this experimental context. The relatively modest variation suggests that this parameter has limited influence on the measured metric within the tested range. Confidence in these findings is high given the direct correspondence between CSV data and plotted values. Future analysis should consider incorporating error bars representing variance across multiple experimental runs to strengthen statistical validity.

\begin{figure}[!htbp]
\centering
\includegraphics[width=0.95\linewidth]{images/1500_nodes_Module_Change/(e)_Module_swapped_bar_chart.png}
\caption{(e) Module swapped}
\label{fig:1500_nodes_Module_Change_e__Module_swapped_bar_chart}
\end{figure}

This bar chart presents the relationship between modules and modules for the 1500 nodes Module Change experimental scenario. The x-axis displays modules values ranging from 4 to 7, while the y-axis quantifies modules. The single-series visualization facilitates analysis of how the dependent variable responds to changes in the independent parameter setting.

Analysis of the plotted data reveals that modules ranges from 4.00 (at modules = 4) to 7.00 (at modules = 7), representing a span of 3.00 units. The overall trend is increasing, with values rising from 4.00 at the initial setting to 7.00 at the final setting. 

These results indicate that modules configuration meaningfully impacts modules in this experimental context. The substantial variation observed (coefficient of variation exceeding 20\%) suggests that parameter tuning could yield significant performance improvements. Confidence in these findings is high given the direct correspondence between CSV data and plotted values. Future analysis should consider incorporating error bars representing variance across multiple experimental runs to strengthen statistical validity.


\clearpage

\subsection{1500 nodes Swapping Time}

\begin{figure}[!htbp]
\centering
\includegraphics[width=0.95\linewidth]{images/1500_nodes_Swapping_Time/(a)_Travel_time_bar_chart.png}
\caption{(a) Travel time}
\label{fig:1500_nodes_Swapping_Time_a__Travel_time_bar_chart}
\end{figure}

This bar chart presents the relationship between swapping time and total travel time for the 1500 nodes Swapping Time experimental scenario. The x-axis displays swapping time values ranging from 1 to 4, while the y-axis quantifies total travel time. The single-series visualization facilitates analysis of how the dependent variable responds to changes in the independent parameter setting.

Analysis of the plotted data reveals that total travel time ranges from 10089.02 (at swapping time = 1) to 10262.19 (at swapping time = 4), representing a span of 173.17 units. The overall trend is increasing, with values rising from 10089.02 at the initial setting to 10262.19 at the final setting. 

These results indicate that swapping time configuration meaningfully impacts total travel time in this experimental context. The relatively modest variation suggests that this parameter has limited influence on the measured metric within the tested range. Confidence in these findings is high given the direct correspondence between CSV data and plotted values. Future analysis should consider incorporating error bars representing variance across multiple experimental runs to strengthen statistical validity.

\begin{figure}[!htbp]
\centering
\includegraphics[width=0.95\linewidth]{images/1500_nodes_Swapping_Time/(b)_Energy_consumed_bar_chart.png}
\caption{(b) Energy consumed}
\label{fig:1500_nodes_Swapping_Time_b__Energy_consumed_bar_chart}
\end{figure}

This bar chart presents the relationship between swapping time and total energy consumed for the 1500 nodes Swapping Time experimental scenario. The x-axis displays swapping time values ranging from 1 to 4, while the y-axis quantifies total energy consumed. The single-series visualization facilitates analysis of how the dependent variable responds to changes in the independent parameter setting.

Analysis of the plotted data reveals that total energy consumed ranges from 1346.58 (at swapping time = 2) to 1349.76 (at swapping time = 1), representing a span of 3.18 units. The overall trend is decreasing, with values declining from 1349.76 at the initial setting to 1346.59 at the final setting. Notably, the minimum value occurs at an intermediate swapping time setting (2), suggesting non-monotonic behavior that warrants further investigation.

These results indicate that swapping time configuration meaningfully impacts total energy consumed in this experimental context. The relatively modest variation suggests that this parameter has limited influence on the measured metric within the tested range. Confidence in these findings is high given the direct correspondence between CSV data and plotted values. Future analysis should consider incorporating error bars representing variance across multiple experimental runs to strengthen statistical validity.

\begin{figure}[!htbp]
\centering
\includegraphics[width=0.95\linewidth]{images/1500_nodes_Swapping_Time/(c)_Distance_covered_bar_chart.png}
\caption{(c) Distance covered}
\label{fig:1500_nodes_Swapping_Time_c__Distance_covered_bar_chart}
\end{figure}

This bar chart presents the relationship between swapping time and total distance covered for the 1500 nodes Swapping Time experimental scenario. The x-axis displays swapping time values ranging from 1 to 4, while the y-axis quantifies total distance covered. The single-series visualization facilitates analysis of how the dependent variable responds to changes in the independent parameter setting.

Analysis of the plotted data reveals that total distance covered ranges from 6533.82 (at swapping time = 2) to 6552.92 (at swapping time = 1), representing a span of 19.10 units. The overall trend is decreasing, with values declining from 6552.92 at the initial setting to 6534.02 at the final setting. Notably, the minimum value occurs at an intermediate swapping time setting (2), suggesting non-monotonic behavior that warrants further investigation.

These results indicate that swapping time configuration meaningfully impacts total distance covered in this experimental context. The relatively modest variation suggests that this parameter has limited influence on the measured metric within the tested range. Confidence in these findings is high given the direct correspondence between CSV data and plotted values. Future analysis should consider incorporating error bars representing variance across multiple experimental runs to strengthen statistical validity.

\begin{figure}[!htbp]
\centering
\includegraphics[width=0.95\linewidth]{images/1500_nodes_Swapping_Time/(d)_Run_time_bar_chart.png}
\caption{(d) Run time}
\label{fig:1500_nodes_Swapping_Time_d__Run_time_bar_chart}
\end{figure}

This bar chart presents the relationship between swapping time and total travel time for the 1500 nodes Swapping Time experimental scenario. The x-axis displays swapping time values ranging from 1 to 4, while the y-axis quantifies total travel time. The single-series visualization facilitates analysis of how the dependent variable responds to changes in the independent parameter setting.

Analysis of the plotted data reveals that total travel time ranges from 10089.02 (at swapping time = 1) to 10262.19 (at swapping time = 4), representing a span of 173.17 units. The overall trend is increasing, with values rising from 10089.02 at the initial setting to 10262.19 at the final setting. 

These results indicate that swapping time configuration meaningfully impacts total travel time in this experimental context. The relatively modest variation suggests that this parameter has limited influence on the measured metric within the tested range. Confidence in these findings is high given the direct correspondence between CSV data and plotted values. Future analysis should consider incorporating error bars representing variance across multiple experimental runs to strengthen statistical validity.

\begin{figure}[!htbp]
\centering
\includegraphics[width=0.95\linewidth]{images/1500_nodes_Swapping_Time/(e)_Module_swapped_bar_chart.png}
\caption{(e) Module swapped}
\label{fig:1500_nodes_Swapping_Time_e__Module_swapped_bar_chart}
\end{figure}

This bar chart presents the relationship between swapping time and total module swapped for the 1500 nodes Swapping Time experimental scenario. The x-axis displays swapping time values ranging from 1 to 4, while the y-axis quantifies total module swapped. The single-series visualization facilitates analysis of how the dependent variable responds to changes in the independent parameter setting.

Analysis of the plotted data reveals that total module swapped ranges from 64.00 (at swapping time = 1) to 64.00 (at swapping time = 1), representing a span of 0.00 units. The values remain relatively stable across the parameter range, with minimal net change between initial (64.00) and final (64.00) settings. 

These results indicate that swapping time configuration meaningfully impacts total module swapped in this experimental context. The relatively modest variation suggests that this parameter has limited influence on the measured metric within the tested range. Confidence in these findings is high given the direct correspondence between CSV data and plotted values. Future analysis should consider incorporating error bars representing variance across multiple experimental runs to strengthen statistical validity.


\clearpage

\subsection{1500 nodes Threshold}

\begin{figure}[!htbp]
\centering
\includegraphics[width=0.95\linewidth]{images/1500_nodes_Threshold/(a)_Travel_time_bar_chart.png}
\caption{(a) Travel time}
\label{fig:1500_nodes_Threshold_a__Travel_time_bar_chart}
\end{figure}

This bar chart presents the relationship between threshold and total travel time for the 1500 nodes Threshold experimental scenario. The x-axis displays threshold values ranging from 5 to 20, while the y-axis quantifies total travel time. The single-series visualization facilitates analysis of how the dependent variable responds to changes in the independent parameter setting.

Analysis of the plotted data reveals that total travel time ranges from 10133.53 (at threshold = 10) to 10148.53 (at threshold = 5), representing a span of 15.00 units. The overall trend is decreasing, with values declining from 10148.53 at the initial setting to 10133.53 at the final setting. Notably, the minimum value occurs at an intermediate threshold setting (10), suggesting non-monotonic behavior that warrants further investigation.

These results indicate that threshold configuration meaningfully impacts total travel time in this experimental context. The relatively modest variation suggests that this parameter has limited influence on the measured metric within the tested range. Confidence in these findings is high given the direct correspondence between CSV data and plotted values. Future analysis should consider incorporating error bars representing variance across multiple experimental runs to strengthen statistical validity.

\begin{figure}[!htbp]
\centering
\includegraphics[width=0.95\linewidth]{images/1500_nodes_Threshold/(b)_Energy_consumed_bar_chart.png}
\caption{(b) Energy consumed}
\label{fig:1500_nodes_Threshold_b__Energy_consumed_bar_chart}
\end{figure}

This bar chart presents the relationship between threshold and total energy consumed for the 1500 nodes Threshold experimental scenario. The x-axis displays threshold values ranging from 5 to 20, while the y-axis quantifies total energy consumed. The single-series visualization facilitates analysis of how the dependent variable responds to changes in the independent parameter setting.

Analysis of the plotted data reveals that total energy consumed ranges from 1346.58 (at threshold = 10) to 1347.93 (at threshold = 5), representing a span of 1.36 units. The overall trend is decreasing, with values declining from 1347.93 at the initial setting to 1346.58 at the final setting. Notably, the minimum value occurs at an intermediate threshold setting (10), suggesting non-monotonic behavior that warrants further investigation.

These results indicate that threshold configuration meaningfully impacts total energy consumed in this experimental context. The relatively modest variation suggests that this parameter has limited influence on the measured metric within the tested range. Confidence in these findings is high given the direct correspondence between CSV data and plotted values. Future analysis should consider incorporating error bars representing variance across multiple experimental runs to strengthen statistical validity.

\begin{figure}[!htbp]
\centering
\includegraphics[width=0.95\linewidth]{images/1500_nodes_Threshold/(c)_Distance_covered_bar_chart.png}
\caption{(c) Distance covered}
\label{fig:1500_nodes_Threshold_c__Distance_covered_bar_chart}
\end{figure}

This bar chart presents the relationship between threshold and total distance covered for the 1500 nodes Threshold experimental scenario. The x-axis displays threshold values ranging from 5 to 20, while the y-axis quantifies total distance covered. The single-series visualization facilitates analysis of how the dependent variable responds to changes in the independent parameter setting.

Analysis of the plotted data reveals that total distance covered ranges from 6533.82 (at threshold = 10) to 6545.22 (at threshold = 5), representing a span of 11.40 units. The overall trend is decreasing, with values declining from 6545.22 at the initial setting to 6533.82 at the final setting. Notably, the minimum value occurs at an intermediate threshold setting (10), suggesting non-monotonic behavior that warrants further investigation.

These results indicate that threshold configuration meaningfully impacts total distance covered in this experimental context. The relatively modest variation suggests that this parameter has limited influence on the measured metric within the tested range. Confidence in these findings is high given the direct correspondence between CSV data and plotted values. Future analysis should consider incorporating error bars representing variance across multiple experimental runs to strengthen statistical validity.

\begin{figure}[!htbp]
\centering
\includegraphics[width=0.95\linewidth]{images/1500_nodes_Threshold/(d)_Run_time_bar_chart.png}
\caption{(d) Run time}
\label{fig:1500_nodes_Threshold_d__Run_time_bar_chart}
\end{figure}

This bar chart presents the relationship between threshold and total travel time for the 1500 nodes Threshold experimental scenario. The x-axis displays threshold values ranging from 5 to 20, while the y-axis quantifies total travel time. The single-series visualization facilitates analysis of how the dependent variable responds to changes in the independent parameter setting.

Analysis of the plotted data reveals that total travel time ranges from 10133.53 (at threshold = 10) to 10148.53 (at threshold = 5), representing a span of 15.00 units. The overall trend is decreasing, with values declining from 10148.53 at the initial setting to 10133.53 at the final setting. Notably, the minimum value occurs at an intermediate threshold setting (10), suggesting non-monotonic behavior that warrants further investigation.

These results indicate that threshold configuration meaningfully impacts total travel time in this experimental context. The relatively modest variation suggests that this parameter has limited influence on the measured metric within the tested range. Confidence in these findings is high given the direct correspondence between CSV data and plotted values. Future analysis should consider incorporating error bars representing variance across multiple experimental runs to strengthen statistical validity.

\begin{figure}[!htbp]
\centering
\includegraphics[width=0.95\linewidth]{images/1500_nodes_Threshold/(e)_Module_swapped_bar_chart.png}
\caption{(e) Module swapped}
\label{fig:1500_nodes_Threshold_e__Module_swapped_bar_chart}
\end{figure}

This bar chart presents the relationship between threshold and total module swapped for the 1500 nodes Threshold experimental scenario. The x-axis displays threshold values ranging from 5 to 20, while the y-axis quantifies total module swapped. The single-series visualization facilitates analysis of how the dependent variable responds to changes in the independent parameter setting.

Analysis of the plotted data reveals that total module swapped ranges from 63.00 (at threshold = 5) to 64.00 (at threshold = 10), representing a span of 1.00 units. The overall trend is increasing, with values rising from 63.00 at the initial setting to 64.00 at the final setting. 

These results indicate that threshold configuration meaningfully impacts total module swapped in this experimental context. The relatively modest variation suggests that this parameter has limited influence on the measured metric within the tested range. Confidence in these findings is high given the direct correspondence between CSV data and plotted values. Future analysis should consider incorporating error bars representing variance across multiple experimental runs to strengthen statistical validity.


\clearpage

\subsection{1500 nodes Traffic}

\begin{figure}[!htbp]
\centering
\includegraphics[width=0.95\linewidth]{images/1500_nodes_Traffic/(a)_Travel_time_bar_chart.png}
\caption{(a) Travel time}
\label{fig:1500_nodes_Traffic_a__Travel_time_bar_chart}
\end{figure}

This bar chart presents the relationship between traffic and total travel time for the 1500 nodes Traffic experimental scenario. The x-axis displays traffic values ranging from High to Low, while the y-axis quantifies total travel time. The single-series visualization facilitates analysis of how the dependent variable responds to changes in the independent parameter setting.

Analysis of the plotted data reveals that total travel time ranges from 8682.70 (at traffic = Low) to 11868.31 (at traffic = High), representing a span of 3185.61 units. The overall trend is decreasing, with values declining from 11868.31 at the initial setting to 8682.70 at the final setting. 

These results indicate that traffic configuration meaningfully impacts total travel time in this experimental context. The substantial variation observed (coefficient of variation exceeding 20\%) suggests that parameter tuning could yield significant performance improvements. Confidence in these findings is high given the direct correspondence between CSV data and plotted values. Future analysis should consider incorporating error bars representing variance across multiple experimental runs to strengthen statistical validity.

\begin{figure}[!htbp]
\centering
\includegraphics[width=0.95\linewidth]{images/1500_nodes_Traffic/(b)_Energy_consumed_bar_chart.png}
\caption{(b) Energy consumed}
\label{fig:1500_nodes_Traffic_b__Energy_consumed_bar_chart}
\end{figure}

This bar chart presents the relationship between traffic and total energy consumed for the 1500 nodes Traffic experimental scenario. The x-axis displays traffic values ranging from High to Low, while the y-axis quantifies total energy consumed. The single-series visualization facilitates analysis of how the dependent variable responds to changes in the independent parameter setting.

Analysis of the plotted data reveals that total energy consumed ranges from 1303.59 (at traffic = High) to 1443.98 (at traffic = Low), representing a span of 140.40 units. The overall trend is increasing, with values rising from 1303.59 at the initial setting to 1443.98 at the final setting. 

These results indicate that traffic configuration meaningfully impacts total energy consumed in this experimental context. The relatively modest variation suggests that this parameter has limited influence on the measured metric within the tested range. Confidence in these findings is high given the direct correspondence between CSV data and plotted values. Future analysis should consider incorporating error bars representing variance across multiple experimental runs to strengthen statistical validity.

\begin{figure}[!htbp]
\centering
\includegraphics[width=0.95\linewidth]{images/1500_nodes_Traffic/(c)_Distance_covered_bar_chart.png}
\caption{(c) Distance covered}
\label{fig:1500_nodes_Traffic_c__Distance_covered_bar_chart}
\end{figure}

This bar chart presents the relationship between traffic and total distance covered for the 1500 nodes Traffic experimental scenario. The x-axis displays traffic values ranging from High to Low, while the y-axis quantifies total distance covered. The single-series visualization facilitates analysis of how the dependent variable responds to changes in the independent parameter setting.

Analysis of the plotted data reveals that total distance covered ranges from 6463.22 (at traffic = Mid) to 6661.07 (at traffic = Low), representing a span of 197.85 units. The overall trend is increasing, with values rising from 6493.82 at the initial setting to 6661.07 at the final setting. Notably, the minimum value occurs at an intermediate traffic setting (Mid), suggesting non-monotonic behavior that warrants further investigation.

These results indicate that traffic configuration meaningfully impacts total distance covered in this experimental context. The relatively modest variation suggests that this parameter has limited influence on the measured metric within the tested range. Confidence in these findings is high given the direct correspondence between CSV data and plotted values. Future analysis should consider incorporating error bars representing variance across multiple experimental runs to strengthen statistical validity.

\begin{figure}[!htbp]
\centering
\includegraphics[width=0.95\linewidth]{images/1500_nodes_Traffic/(d)_Run_time_bar_chart.png}
\caption{(d) Run time}
\label{fig:1500_nodes_Traffic_d__Run_time_bar_chart}
\end{figure}

This bar chart presents the relationship between traffic and total travel time for the 1500 nodes Traffic experimental scenario. The x-axis displays traffic values ranging from High to Low, while the y-axis quantifies total travel time. The single-series visualization facilitates analysis of how the dependent variable responds to changes in the independent parameter setting.

Analysis of the plotted data reveals that total travel time ranges from 8682.70 (at traffic = Low) to 11868.31 (at traffic = High), representing a span of 3185.61 units. The overall trend is decreasing, with values declining from 11868.31 at the initial setting to 8682.70 at the final setting. 

These results indicate that traffic configuration meaningfully impacts total travel time in this experimental context. The substantial variation observed (coefficient of variation exceeding 20\%) suggests that parameter tuning could yield significant performance improvements. Confidence in these findings is high given the direct correspondence between CSV data and plotted values. Future analysis should consider incorporating error bars representing variance across multiple experimental runs to strengthen statistical validity.

\begin{figure}[!htbp]
\centering
\includegraphics[width=0.95\linewidth]{images/1500_nodes_Traffic/(e)_Module_swapped_bar_chart.png}
\caption{(e) Module swapped}
\label{fig:1500_nodes_Traffic_e__Module_swapped_bar_chart}
\end{figure}

This bar chart presents the relationship between traffic and total module swapped for the 1500 nodes Traffic experimental scenario. The x-axis displays traffic values ranging from High to Low, while the y-axis quantifies total module swapped. The single-series visualization facilitates analysis of how the dependent variable responds to changes in the independent parameter setting.

Analysis of the plotted data reveals that total module swapped ranges from 63.00 (at traffic = High) to 69.00 (at traffic = Low), representing a span of 6.00 units. The overall trend is increasing, with values rising from 63.00 at the initial setting to 69.00 at the final setting. 

These results indicate that traffic configuration meaningfully impacts total module swapped in this experimental context. The relatively modest variation suggests that this parameter has limited influence on the measured metric within the tested range. Confidence in these findings is high given the direct correspondence between CSV data and plotted values. Future analysis should consider incorporating error bars representing variance across multiple experimental runs to strengthen statistical validity.


\clearpage

\subsection{2000 nodes Module Change}

\begin{figure}[!htbp]
\centering
\includegraphics[width=0.95\linewidth]{images/2000_nodes_Module_Change/(a)_Travel_time_bar_chart.png}
\caption{(a) Travel time}
\label{fig:2000_nodes_Module_Change_a__Travel_time_bar_chart}
\end{figure}

This bar chart presents the relationship between modules and total travel time for the 2000 nodes Module Change experimental scenario. The x-axis displays modules values ranging from 4 to 7, while the y-axis quantifies total travel time. The single-series visualization facilitates analysis of how the dependent variable responds to changes in the independent parameter setting.

Analysis of the plotted data reveals that total travel time ranges from 13169.12 (at modules = 7) to 14159.32 (at modules = 4), representing a span of 990.20 units. The overall trend is decreasing, with values declining from 14159.32 at the initial setting to 13169.12 at the final setting. 

These results indicate that modules configuration meaningfully impacts total travel time in this experimental context. The relatively modest variation suggests that this parameter has limited influence on the measured metric within the tested range. Confidence in these findings is high given the direct correspondence between CSV data and plotted values. Future analysis should consider incorporating error bars representing variance across multiple experimental runs to strengthen statistical validity.

\begin{figure}[!htbp]
\centering
\includegraphics[width=0.95\linewidth]{images/2000_nodes_Module_Change/(b)_Energy_consumed_bar_chart.png}
\caption{(b) Energy consumed}
\label{fig:2000_nodes_Module_Change_b__Energy_consumed_bar_chart}
\end{figure}

This bar chart presents the relationship between modules and total energy consumed for the 2000 nodes Module Change experimental scenario. The x-axis displays modules values ranging from 4 to 7, while the y-axis quantifies total energy consumed. The single-series visualization facilitates analysis of how the dependent variable responds to changes in the independent parameter setting.

Analysis of the plotted data reveals that total energy consumed ranges from 2030.72 (at modules = 7) to 2154.01 (at modules = 4), representing a span of 123.28 units. The overall trend is decreasing, with values declining from 2154.01 at the initial setting to 2030.72 at the final setting. 

These results indicate that modules configuration meaningfully impacts total energy consumed in this experimental context. The relatively modest variation suggests that this parameter has limited influence on the measured metric within the tested range. Confidence in these findings is high given the direct correspondence between CSV data and plotted values. Future analysis should consider incorporating error bars representing variance across multiple experimental runs to strengthen statistical validity.

\begin{figure}[!htbp]
\centering
\includegraphics[width=0.95\linewidth]{images/2000_nodes_Module_Change/(c)_Distance_covered_bar_chart.png}
\caption{(c) Distance covered}
\label{fig:2000_nodes_Module_Change_c__Distance_covered_bar_chart}
\end{figure}

This bar chart presents the relationship between modules and total distance covered for the 2000 nodes Module Change experimental scenario. The x-axis displays modules values ranging from 4 to 7, while the y-axis quantifies total distance covered. The single-series visualization facilitates analysis of how the dependent variable responds to changes in the independent parameter setting.

Analysis of the plotted data reveals that total distance covered ranges from 8588.97 (at modules = 7) to 9234.69 (at modules = 4), representing a span of 645.72 units. The overall trend is decreasing, with values declining from 9234.69 at the initial setting to 8588.97 at the final setting. 

These results indicate that modules configuration meaningfully impacts total distance covered in this experimental context. The relatively modest variation suggests that this parameter has limited influence on the measured metric within the tested range. Confidence in these findings is high given the direct correspondence between CSV data and plotted values. Future analysis should consider incorporating error bars representing variance across multiple experimental runs to strengthen statistical validity.

\begin{figure}[!htbp]
\centering
\includegraphics[width=0.95\linewidth]{images/2000_nodes_Module_Change/(d)_Run_time_bar_chart.png}
\caption{(d) Run time}
\label{fig:2000_nodes_Module_Change_d__Run_time_bar_chart}
\end{figure}

This bar chart presents the relationship between modules and total travel time for the 2000 nodes Module Change experimental scenario. The x-axis displays modules values ranging from 4 to 7, while the y-axis quantifies total travel time. The single-series visualization facilitates analysis of how the dependent variable responds to changes in the independent parameter setting.

Analysis of the plotted data reveals that total travel time ranges from 13169.12 (at modules = 7) to 14159.32 (at modules = 4), representing a span of 990.20 units. The overall trend is decreasing, with values declining from 14159.32 at the initial setting to 13169.12 at the final setting. 

These results indicate that modules configuration meaningfully impacts total travel time in this experimental context. The relatively modest variation suggests that this parameter has limited influence on the measured metric within the tested range. Confidence in these findings is high given the direct correspondence between CSV data and plotted values. Future analysis should consider incorporating error bars representing variance across multiple experimental runs to strengthen statistical validity.

\begin{figure}[!htbp]
\centering
\includegraphics[width=0.95\linewidth]{images/2000_nodes_Module_Change/(e)_Module_swapped_bar_chart.png}
\caption{(e) Module swapped}
\label{fig:2000_nodes_Module_Change_e__Module_swapped_bar_chart}
\end{figure}

This bar chart presents the relationship between modules and modules for the 2000 nodes Module Change experimental scenario. The x-axis displays modules values ranging from 4 to 7, while the y-axis quantifies modules. The single-series visualization facilitates analysis of how the dependent variable responds to changes in the independent parameter setting.

Analysis of the plotted data reveals that modules ranges from 4.00 (at modules = 4) to 7.00 (at modules = 7), representing a span of 3.00 units. The overall trend is increasing, with values rising from 4.00 at the initial setting to 7.00 at the final setting. 

These results indicate that modules configuration meaningfully impacts modules in this experimental context. The substantial variation observed (coefficient of variation exceeding 20\%) suggests that parameter tuning could yield significant performance improvements. Confidence in these findings is high given the direct correspondence between CSV data and plotted values. Future analysis should consider incorporating error bars representing variance across multiple experimental runs to strengthen statistical validity.


\clearpage

\subsection{2000 nodes Swapping Time}

\begin{figure}[!htbp]
\centering
\includegraphics[width=0.95\linewidth]{images/2000_nodes_Swapping_Time/(a)_Travel_time_bar_chart.png}
\caption{(a) Travel time}
\label{fig:2000_nodes_Swapping_Time_a__Travel_time_bar_chart}
\end{figure}

This bar chart presents the relationship between swapping time and total travel time for the 2000 nodes Swapping Time experimental scenario. The x-axis displays swapping time values ranging from 1 to 4, while the y-axis quantifies total travel time. The single-series visualization facilitates analysis of how the dependent variable responds to changes in the independent parameter setting.

Analysis of the plotted data reveals that total travel time ranges from 13181.25 (at swapping time = 1) to 13540.28 (at swapping time = 4), representing a span of 359.03 units. The overall trend is increasing, with values rising from 13181.25 at the initial setting to 13540.28 at the final setting. 

These results indicate that swapping time configuration meaningfully impacts total travel time in this experimental context. The relatively modest variation suggests that this parameter has limited influence on the measured metric within the tested range. Confidence in these findings is high given the direct correspondence between CSV data and plotted values. Future analysis should consider incorporating error bars representing variance across multiple experimental runs to strengthen statistical validity.

\begin{figure}[!htbp]
\centering
\includegraphics[width=0.95\linewidth]{images/2000_nodes_Swapping_Time/(b)_Energy_consumed_bar_chart.png}
\caption{(b) Energy consumed}
\label{fig:2000_nodes_Swapping_Time_b__Energy_consumed_bar_chart}
\end{figure}

This bar chart presents the relationship between swapping time and total energy consumed for the 2000 nodes Swapping Time experimental scenario. The x-axis displays swapping time values ranging from 1 to 4, while the y-axis quantifies total energy consumed. The single-series visualization facilitates analysis of how the dependent variable responds to changes in the independent parameter setting.

Analysis of the plotted data reveals that total energy consumed ranges from 2057.20 (at swapping time = 1) to 2062.48 (at swapping time = 4), representing a span of 5.28 units. The overall trend is increasing, with values rising from 2057.20 at the initial setting to 2062.48 at the final setting. 

These results indicate that swapping time configuration meaningfully impacts total energy consumed in this experimental context. The relatively modest variation suggests that this parameter has limited influence on the measured metric within the tested range. Confidence in these findings is high given the direct correspondence between CSV data and plotted values. Future analysis should consider incorporating error bars representing variance across multiple experimental runs to strengthen statistical validity.

\begin{figure}[!htbp]
\centering
\includegraphics[width=0.95\linewidth]{images/2000_nodes_Swapping_Time/(c)_Distance_covered_bar_chart.png}
\caption{(c) Distance covered}
\label{fig:2000_nodes_Swapping_Time_c__Distance_covered_bar_chart}
\end{figure}

This bar chart presents the relationship between swapping time and total distance covered for the 2000 nodes Swapping Time experimental scenario. The x-axis displays swapping time values ranging from 1 to 4, while the y-axis quantifies total distance covered. The single-series visualization facilitates analysis of how the dependent variable responds to changes in the independent parameter setting.

Analysis of the plotted data reveals that total distance covered ranges from 8650.95 (at swapping time = 1) to 8701.15 (at swapping time = 4), representing a span of 50.20 units. The overall trend is increasing, with values rising from 8650.95 at the initial setting to 8701.15 at the final setting. 

These results indicate that swapping time configuration meaningfully impacts total distance covered in this experimental context. The relatively modest variation suggests that this parameter has limited influence on the measured metric within the tested range. Confidence in these findings is high given the direct correspondence between CSV data and plotted values. Future analysis should consider incorporating error bars representing variance across multiple experimental runs to strengthen statistical validity.

\begin{figure}[!htbp]
\centering
\includegraphics[width=0.95\linewidth]{images/2000_nodes_Swapping_Time/(d)_Run_time_bar_chart.png}
\caption{(d) Run time}
\label{fig:2000_nodes_Swapping_Time_d__Run_time_bar_chart}
\end{figure}

This bar chart presents the relationship between swapping time and total travel time for the 2000 nodes Swapping Time experimental scenario. The x-axis displays swapping time values ranging from 1 to 4, while the y-axis quantifies total travel time. The single-series visualization facilitates analysis of how the dependent variable responds to changes in the independent parameter setting.

Analysis of the plotted data reveals that total travel time ranges from 13181.25 (at swapping time = 1) to 13540.28 (at swapping time = 4), representing a span of 359.03 units. The overall trend is increasing, with values rising from 13181.25 at the initial setting to 13540.28 at the final setting. 

These results indicate that swapping time configuration meaningfully impacts total travel time in this experimental context. The relatively modest variation suggests that this parameter has limited influence on the measured metric within the tested range. Confidence in these findings is high given the direct correspondence between CSV data and plotted values. Future analysis should consider incorporating error bars representing variance across multiple experimental runs to strengthen statistical validity.

\begin{figure}[!htbp]
\centering
\includegraphics[width=0.95\linewidth]{images/2000_nodes_Swapping_Time/(e)_Module_swapped_bar_chart.png}
\caption{(e) Module swapped}
\label{fig:2000_nodes_Swapping_Time_e__Module_swapped_bar_chart}
\end{figure}

This bar chart presents the relationship between swapping time and total module swapped for the 2000 nodes Swapping Time experimental scenario. The x-axis displays swapping time values ranging from 1 to 4, while the y-axis quantifies total module swapped. The single-series visualization facilitates analysis of how the dependent variable responds to changes in the independent parameter setting.

Analysis of the plotted data reveals that total module swapped ranges from 100.00 (at swapping time = 1) to 100.00 (at swapping time = 1), representing a span of 0.00 units. The values remain relatively stable across the parameter range, with minimal net change between initial (100.00) and final (100.00) settings. 

These results indicate that swapping time configuration meaningfully impacts total module swapped in this experimental context. The relatively modest variation suggests that this parameter has limited influence on the measured metric within the tested range. Confidence in these findings is high given the direct correspondence between CSV data and plotted values. Future analysis should consider incorporating error bars representing variance across multiple experimental runs to strengthen statistical validity.


\clearpage

\subsection{2000 nodes Threshold}

\begin{figure}[!htbp]
\centering
\includegraphics[width=0.95\linewidth]{images/2000_nodes_Threshold/(a)_Travel_time_bar_chart.png}
\caption{(a) Travel time}
\label{fig:2000_nodes_Threshold_a__Travel_time_bar_chart}
\end{figure}

This bar chart presents the relationship between threshold and total travel time for the 2000 nodes Threshold experimental scenario. The x-axis displays threshold values ranging from 5 to 20, while the y-axis quantifies total travel time. The single-series visualization facilitates analysis of how the dependent variable responds to changes in the independent parameter setting.

Analysis of the plotted data reveals that total travel time ranges from 13059.88 (at threshold = 5) to 13381.98 (at threshold = 20), representing a span of 322.10 units. The overall trend is increasing, with values rising from 13059.88 at the initial setting to 13381.98 at the final setting. 

These results indicate that threshold configuration meaningfully impacts total travel time in this experimental context. The relatively modest variation suggests that this parameter has limited influence on the measured metric within the tested range. Confidence in these findings is high given the direct correspondence between CSV data and plotted values. Future analysis should consider incorporating error bars representing variance across multiple experimental runs to strengthen statistical validity.

\begin{figure}[!htbp]
\centering
\includegraphics[width=0.95\linewidth]{images/2000_nodes_Threshold/(b)_Energy_consumed_bar_chart.png}
\caption{(b) Energy consumed}
\label{fig:2000_nodes_Threshold_b__Energy_consumed_bar_chart}
\end{figure}

This bar chart presents the relationship between threshold and total energy consumed for the 2000 nodes Threshold experimental scenario. The x-axis displays threshold values ranging from 5 to 20, while the y-axis quantifies total energy consumed. The single-series visualization facilitates analysis of how the dependent variable responds to changes in the independent parameter setting.

Analysis of the plotted data reveals that total energy consumed ranges from 2020.24 (at threshold = 15) to 2057.78 (at threshold = 10), representing a span of 37.54 units. The overall trend is increasing, with values rising from 2027.67 at the initial setting to 2057.45 at the final setting. Notably, the minimum value occurs at an intermediate threshold setting (15), suggesting non-monotonic behavior that warrants further investigation.

These results indicate that threshold configuration meaningfully impacts total energy consumed in this experimental context. The relatively modest variation suggests that this parameter has limited influence on the measured metric within the tested range. Confidence in these findings is high given the direct correspondence between CSV data and plotted values. Future analysis should consider incorporating error bars representing variance across multiple experimental runs to strengthen statistical validity.

\begin{figure}[!htbp]
\centering
\includegraphics[width=0.95\linewidth]{images/2000_nodes_Threshold/(c)_Distance_covered_bar_chart.png}
\caption{(c) Distance covered}
\label{fig:2000_nodes_Threshold_c__Distance_covered_bar_chart}
\end{figure}

This bar chart presents the relationship between threshold and total distance covered for the 2000 nodes Threshold experimental scenario. The x-axis displays threshold values ranging from 5 to 20, while the y-axis quantifies total distance covered. The single-series visualization facilitates analysis of how the dependent variable responds to changes in the independent parameter setting.

Analysis of the plotted data reveals that total distance covered ranges from 8527.92 (at threshold = 5) to 8685.62 (at threshold = 10), representing a span of 157.70 units. The overall trend is increasing, with values rising from 8527.92 at the initial setting to 8652.17 at the final setting. 

These results indicate that threshold configuration meaningfully impacts total distance covered in this experimental context. The relatively modest variation suggests that this parameter has limited influence on the measured metric within the tested range. Confidence in these findings is high given the direct correspondence between CSV data and plotted values. Future analysis should consider incorporating error bars representing variance across multiple experimental runs to strengthen statistical validity.

\begin{figure}[!htbp]
\centering
\includegraphics[width=0.95\linewidth]{images/2000_nodes_Threshold/(d)_Run_time_bar_chart.png}
\caption{(d) Run time}
\label{fig:2000_nodes_Threshold_d__Run_time_bar_chart}
\end{figure}

This bar chart presents the relationship between threshold and total travel time for the 2000 nodes Threshold experimental scenario. The x-axis displays threshold values ranging from 5 to 20, while the y-axis quantifies total travel time. The single-series visualization facilitates analysis of how the dependent variable responds to changes in the independent parameter setting.

Analysis of the plotted data reveals that total travel time ranges from 13059.88 (at threshold = 5) to 13381.98 (at threshold = 20), representing a span of 322.10 units. The overall trend is increasing, with values rising from 13059.88 at the initial setting to 13381.98 at the final setting. 

These results indicate that threshold configuration meaningfully impacts total travel time in this experimental context. The relatively modest variation suggests that this parameter has limited influence on the measured metric within the tested range. Confidence in these findings is high given the direct correspondence between CSV data and plotted values. Future analysis should consider incorporating error bars representing variance across multiple experimental runs to strengthen statistical validity.

\begin{figure}[!htbp]
\centering
\includegraphics[width=0.95\linewidth]{images/2000_nodes_Threshold/(e)_Module_swapped_bar_chart.png}
\caption{(e) Module swapped}
\label{fig:2000_nodes_Threshold_e__Module_swapped_bar_chart}
\end{figure}

This bar chart presents the relationship between threshold and total module swapped for the 2000 nodes Threshold experimental scenario. The x-axis displays threshold values ranging from 5 to 20, while the y-axis quantifies total module swapped. The single-series visualization facilitates analysis of how the dependent variable responds to changes in the independent parameter setting.

Analysis of the plotted data reveals that total module swapped ranges from 97.00 (at threshold = 15) to 100.00 (at threshold = 20), representing a span of 3.00 units. The overall trend is increasing, with values rising from 98.00 at the initial setting to 100.00 at the final setting. Notably, the minimum value occurs at an intermediate threshold setting (15), suggesting non-monotonic behavior that warrants further investigation.

These results indicate that threshold configuration meaningfully impacts total module swapped in this experimental context. The relatively modest variation suggests that this parameter has limited influence on the measured metric within the tested range. Confidence in these findings is high given the direct correspondence between CSV data and plotted values. Future analysis should consider incorporating error bars representing variance across multiple experimental runs to strengthen statistical validity.


\clearpage

\subsection{2000 nodes Traffic}

\begin{figure}[!htbp]
\centering
\includegraphics[width=0.95\linewidth]{images/2000_nodes_Traffic/(a)_Travel_time_bar_chart.png}
\caption{(a) Travel time}
\label{fig:2000_nodes_Traffic_a__Travel_time_bar_chart}
\end{figure}

This bar chart presents the relationship between traffic and total travel time for the 2000 nodes Traffic experimental scenario. The x-axis displays traffic values ranging from High to Low, while the y-axis quantifies total travel time. The single-series visualization facilitates analysis of how the dependent variable responds to changes in the independent parameter setting.

Analysis of the plotted data reveals that total travel time ranges from 11773.91 (at traffic = Low) to 15961.57 (at traffic = High), representing a span of 4187.66 units. The overall trend is decreasing, with values declining from 15961.57 at the initial setting to 11773.91 at the final setting. 

These results indicate that traffic configuration meaningfully impacts total travel time in this experimental context. The substantial variation observed (coefficient of variation exceeding 20\%) suggests that parameter tuning could yield significant performance improvements. Confidence in these findings is high given the direct correspondence between CSV data and plotted values. Future analysis should consider incorporating error bars representing variance across multiple experimental runs to strengthen statistical validity.

\begin{figure}[!htbp]
\centering
\includegraphics[width=0.95\linewidth]{images/2000_nodes_Traffic/(b)_Energy_consumed_bar_chart.png}
\caption{(b) Energy consumed}
\label{fig:2000_nodes_Traffic_b__Energy_consumed_bar_chart}
\end{figure}

This bar chart presents the relationship between traffic and total energy consumed for the 2000 nodes Traffic experimental scenario. The x-axis displays traffic values ranging from High to Low, while the y-axis quantifies total energy consumed. The single-series visualization facilitates analysis of how the dependent variable responds to changes in the independent parameter setting.

Analysis of the plotted data reveals that total energy consumed ranges from 1974.45 (at traffic = Mid) to 2187.71 (at traffic = Low), representing a span of 213.26 units. The overall trend is increasing, with values rising from 1975.64 at the initial setting to 2187.71 at the final setting. Notably, the minimum value occurs at an intermediate traffic setting (Mid), suggesting non-monotonic behavior that warrants further investigation.

These results indicate that traffic configuration meaningfully impacts total energy consumed in this experimental context. The relatively modest variation suggests that this parameter has limited influence on the measured metric within the tested range. Confidence in these findings is high given the direct correspondence between CSV data and plotted values. Future analysis should consider incorporating error bars representing variance across multiple experimental runs to strengthen statistical validity.

\begin{figure}[!htbp]
\centering
\includegraphics[width=0.95\linewidth]{images/2000_nodes_Traffic/(c)_Distance_covered_bar_chart.png}
\caption{(c) Distance covered}
\label{fig:2000_nodes_Traffic_c__Distance_covered_bar_chart}
\end{figure}

This bar chart presents the relationship between traffic and total distance covered for the 2000 nodes Traffic experimental scenario. The x-axis displays traffic values ranging from High to Low, while the y-axis quantifies total distance covered. The single-series visualization facilitates analysis of how the dependent variable responds to changes in the independent parameter setting.

Analysis of the plotted data reveals that total distance covered ranges from 8689.81 (at traffic = High) to 8994.35 (at traffic = Low), representing a span of 304.54 units. The overall trend is increasing, with values rising from 8689.81 at the initial setting to 8994.35 at the final setting. 

These results indicate that traffic configuration meaningfully impacts total distance covered in this experimental context. The relatively modest variation suggests that this parameter has limited influence on the measured metric within the tested range. Confidence in these findings is high given the direct correspondence between CSV data and plotted values. Future analysis should consider incorporating error bars representing variance across multiple experimental runs to strengthen statistical validity.

\begin{figure}[!htbp]
\centering
\includegraphics[width=0.95\linewidth]{images/2000_nodes_Traffic/(d)_Run_time_bar_chart.png}
\caption{(d) Run time}
\label{fig:2000_nodes_Traffic_d__Run_time_bar_chart}
\end{figure}

This bar chart presents the relationship between traffic and total travel time for the 2000 nodes Traffic experimental scenario. The x-axis displays traffic values ranging from High to Low, while the y-axis quantifies total travel time. The single-series visualization facilitates analysis of how the dependent variable responds to changes in the independent parameter setting.

Analysis of the plotted data reveals that total travel time ranges from 11773.91 (at traffic = Low) to 15961.57 (at traffic = High), representing a span of 4187.66 units. The overall trend is decreasing, with values declining from 15961.57 at the initial setting to 11773.91 at the final setting. 

These results indicate that traffic configuration meaningfully impacts total travel time in this experimental context. The substantial variation observed (coefficient of variation exceeding 20\%) suggests that parameter tuning could yield significant performance improvements. Confidence in these findings is high given the direct correspondence between CSV data and plotted values. Future analysis should consider incorporating error bars representing variance across multiple experimental runs to strengthen statistical validity.

\begin{figure}[!htbp]
\centering
\includegraphics[width=0.95\linewidth]{images/2000_nodes_Traffic/(e)_Module_swapped_bar_chart.png}
\caption{(e) Module swapped}
\label{fig:2000_nodes_Traffic_e__Module_swapped_bar_chart}
\end{figure}

This bar chart presents the relationship between traffic and total module swapped for the 2000 nodes Traffic experimental scenario. The x-axis displays traffic values ranging from High to Low, while the y-axis quantifies total module swapped. The single-series visualization facilitates analysis of how the dependent variable responds to changes in the independent parameter setting.

Analysis of the plotted data reveals that total module swapped ranges from 96.00 (at traffic = High) to 107.00 (at traffic = Low), representing a span of 11.00 units. The overall trend is increasing, with values rising from 96.00 at the initial setting to 107.00 at the final setting. 

These results indicate that traffic configuration meaningfully impacts total module swapped in this experimental context. The relatively modest variation suggests that this parameter has limited influence on the measured metric within the tested range. Confidence in these findings is high given the direct correspondence between CSV data and plotted values. Future analysis should consider incorporating error bars representing variance across multiple experimental runs to strengthen statistical validity.


\clearpage

\subsection{500 nodes Module Change}

\begin{figure}[!htbp]
\centering
\includegraphics[width=0.95\linewidth]{images/500_nodes_Module_Change/(a)_Travel_Time_bar_chart.png}
\caption{(a) Travel Time}
\label{fig:500_nodes_Module_Change_a__Travel_Time_bar_chart}
\end{figure}

This bar chart presents the relationship between modules and total travel time for the 500 nodes Module Change experimental scenario. The x-axis displays modules values ranging from 4 to 7, while the y-axis quantifies total travel time. The single-series visualization facilitates analysis of how the dependent variable responds to changes in the independent parameter setting.

Analysis of the plotted data reveals that total travel time ranges from 3287.54 (at modules = 4) to 3287.54 (at modules = 4), representing a span of 0.00 units. The values remain relatively stable across the parameter range, with minimal net change between initial (3287.54) and final (3287.54) settings. 

These results indicate that modules configuration meaningfully impacts total travel time in this experimental context. The relatively modest variation suggests that this parameter has limited influence on the measured metric within the tested range. Confidence in these findings is high given the direct correspondence between CSV data and plotted values. Future analysis should consider incorporating error bars representing variance across multiple experimental runs to strengthen statistical validity.

\begin{figure}[!htbp]
\centering
\includegraphics[width=0.95\linewidth]{images/500_nodes_Module_Change/(b)_Energy_consumption_bar_chart.png}
\caption{(b) Energy consumption}
\label{fig:500_nodes_Module_Change_b__Energy_consumption_bar_chart}
\end{figure}

This bar chart presents the relationship between modules and total energy consumed for the 500 nodes Module Change experimental scenario. The x-axis displays modules values ranging from 4 to 7, while the y-axis quantifies total energy consumed. The single-series visualization facilitates analysis of how the dependent variable responds to changes in the independent parameter setting.

Analysis of the plotted data reveals that total energy consumed ranges from 323.50 (at modules = 4) to 323.50 (at modules = 4), representing a span of 0.00 units. The values remain relatively stable across the parameter range, with minimal net change between initial (323.50) and final (323.50) settings. 

These results indicate that modules configuration meaningfully impacts total energy consumed in this experimental context. The relatively modest variation suggests that this parameter has limited influence on the measured metric within the tested range. Confidence in these findings is high given the direct correspondence between CSV data and plotted values. Future analysis should consider incorporating error bars representing variance across multiple experimental runs to strengthen statistical validity.

\begin{figure}[!htbp]
\centering
\includegraphics[width=0.95\linewidth]{images/500_nodes_Module_Change/(c)_Distance_covered_bar_chart.png}
\caption{(c) Distance covered}
\label{fig:500_nodes_Module_Change_c__Distance_covered_bar_chart}
\end{figure}

This bar chart presents the relationship between modules and total distance covered for the 500 nodes Module Change experimental scenario. The x-axis displays modules values ranging from 4 to 7, while the y-axis quantifies total distance covered. The single-series visualization facilitates analysis of how the dependent variable responds to changes in the independent parameter setting.

Analysis of the plotted data reveals that total distance covered ranges from 2138.91 (at modules = 4) to 2138.91 (at modules = 4), representing a span of 0.00 units. The values remain relatively stable across the parameter range, with minimal net change between initial (2138.91) and final (2138.91) settings. 

These results indicate that modules configuration meaningfully impacts total distance covered in this experimental context. The relatively modest variation suggests that this parameter has limited influence on the measured metric within the tested range. Confidence in these findings is high given the direct correspondence between CSV data and plotted values. Future analysis should consider incorporating error bars representing variance across multiple experimental runs to strengthen statistical validity.

\begin{figure}[!htbp]
\centering
\includegraphics[width=0.95\linewidth]{images/500_nodes_Module_Change/(d)_Runtime_bar_chart.png}
\caption{(d) Runtime}
\label{fig:500_nodes_Module_Change_d__Runtime_bar_chart}
\end{figure}

This bar chart presents the relationship between modules and run time for the 500 nodes Module Change experimental scenario. The x-axis displays modules values ranging from 4 to 7, while the y-axis quantifies run time. The single-series visualization facilitates analysis of how the dependent variable responds to changes in the independent parameter setting.

Analysis of the plotted data reveals that run time ranges from 29.02 (at modules = 5) to 62.16 (at modules = 6), representing a span of 33.14 units. The overall trend is decreasing, with values declining from 58.53 at the initial setting to 36.14 at the final setting. Notably, the minimum value occurs at an intermediate modules setting (5), suggesting non-monotonic behavior that warrants further investigation.

These results indicate that modules configuration meaningfully impacts run time in this experimental context. The substantial variation observed (coefficient of variation exceeding 20\%) suggests that parameter tuning could yield significant performance improvements. Confidence in these findings is high given the direct correspondence between CSV data and plotted values. Future analysis should consider incorporating error bars representing variance across multiple experimental runs to strengthen statistical validity.

\begin{figure}[!htbp]
\centering
\includegraphics[width=0.95\linewidth]{images/500_nodes_Module_Change/(e)_Number_of_module_swapped_bar_chart.png}
\caption{(e) Number of module swapped}
\label{fig:500_nodes_Module_Change_e__Number_of_module_swapped_bar_chart}
\end{figure}

This bar chart presents the relationship between modules and modules for the 500 nodes Module Change experimental scenario. The x-axis displays modules values ranging from 4 to 7, while the y-axis quantifies modules. The single-series visualization facilitates analysis of how the dependent variable responds to changes in the independent parameter setting.

Analysis of the plotted data reveals that modules ranges from 4.00 (at modules = 4) to 7.00 (at modules = 7), representing a span of 3.00 units. The overall trend is increasing, with values rising from 4.00 at the initial setting to 7.00 at the final setting. 

These results indicate that modules configuration meaningfully impacts modules in this experimental context. The substantial variation observed (coefficient of variation exceeding 20\%) suggests that parameter tuning could yield significant performance improvements. Confidence in these findings is high given the direct correspondence between CSV data and plotted values. Future analysis should consider incorporating error bars representing variance across multiple experimental runs to strengthen statistical validity.


\clearpage

\subsection{500 nodes Swapping Time}

\begin{figure}[!htbp]
\centering
\includegraphics[width=0.95\linewidth]{images/500_nodes_Swapping_Time/(a)_Travel_Time_bar_chart.png}
\caption{(a) Travel Time}
\label{fig:500_nodes_Swapping_Time_a__Travel_Time_bar_chart}
\end{figure}

This bar chart presents the relationship between swapping time and total travel time for the 500 nodes Swapping Time experimental scenario. The x-axis displays swapping time values ranging from 1 to 4, while the y-axis quantifies total travel time. The single-series visualization facilitates analysis of how the dependent variable responds to changes in the independent parameter setting.

Analysis of the plotted data reveals that total travel time ranges from 3268.67 (at swapping time = 1) to 3317.54 (at swapping time = 4), representing a span of 48.87 units. The overall trend is increasing, with values rising from 3268.67 at the initial setting to 3317.54 at the final setting. 

These results indicate that swapping time configuration meaningfully impacts total travel time in this experimental context. The relatively modest variation suggests that this parameter has limited influence on the measured metric within the tested range. Confidence in these findings is high given the direct correspondence between CSV data and plotted values. Future analysis should consider incorporating error bars representing variance across multiple experimental runs to strengthen statistical validity.

\begin{figure}[!htbp]
\centering
\includegraphics[width=0.95\linewidth]{images/500_nodes_Swapping_Time/(b)_Energy_consumption_bar_chart.png}
\caption{(b) Energy consumption}
\label{fig:500_nodes_Swapping_Time_b__Energy_consumption_bar_chart}
\end{figure}

This bar chart presents the relationship between swapping time and total energy consumed for the 500 nodes Swapping Time experimental scenario. The x-axis displays swapping time values ranging from 1 to 4, while the y-axis quantifies total energy consumed. The single-series visualization facilitates analysis of how the dependent variable responds to changes in the independent parameter setting.

Analysis of the plotted data reveals that total energy consumed ranges from 323.44 (at swapping time = 1) to 323.50 (at swapping time = 2), representing a span of 0.06 units. The overall trend is increasing, with values rising from 323.44 at the initial setting to 323.50 at the final setting. 

These results indicate that swapping time configuration meaningfully impacts total energy consumed in this experimental context. The relatively modest variation suggests that this parameter has limited influence on the measured metric within the tested range. Confidence in these findings is high given the direct correspondence between CSV data and plotted values. Future analysis should consider incorporating error bars representing variance across multiple experimental runs to strengthen statistical validity.

\begin{figure}[!htbp]
\centering
\includegraphics[width=0.95\linewidth]{images/500_nodes_Swapping_Time/(c)_Distance_covered_bar_chart.png}
\caption{(c) Distance covered}
\label{fig:500_nodes_Swapping_Time_c__Distance_covered_bar_chart}
\end{figure}

This bar chart presents the relationship between swapping time and total distance covered for the 500 nodes Swapping Time experimental scenario. The x-axis displays swapping time values ranging from 1 to 4, while the y-axis quantifies total distance covered. The single-series visualization facilitates analysis of how the dependent variable responds to changes in the independent parameter setting.

Analysis of the plotted data reveals that total distance covered ranges from 2138.07 (at swapping time = 1) to 2138.91 (at swapping time = 2), representing a span of 0.84 units. The overall trend is increasing, with values rising from 2138.07 at the initial setting to 2138.91 at the final setting. 

These results indicate that swapping time configuration meaningfully impacts total distance covered in this experimental context. The relatively modest variation suggests that this parameter has limited influence on the measured metric within the tested range. Confidence in these findings is high given the direct correspondence between CSV data and plotted values. Future analysis should consider incorporating error bars representing variance across multiple experimental runs to strengthen statistical validity.

\begin{figure}[!htbp]
\centering
\includegraphics[width=0.95\linewidth]{images/500_nodes_Swapping_Time/(d)_Runtime_bar_chart.png}
\caption{(d) Runtime}
\label{fig:500_nodes_Swapping_Time_d__Runtime_bar_chart}
\end{figure}

This bar chart presents the relationship between swapping time and run time for the 500 nodes Swapping Time experimental scenario. The x-axis displays swapping time values ranging from 1 to 4, while the y-axis quantifies run time. The single-series visualization facilitates analysis of how the dependent variable responds to changes in the independent parameter setting.

Analysis of the plotted data reveals that run time ranges from 27.73 (at swapping time = 3) to 29.02 (at swapping time = 2), representing a span of 1.28 units. The overall trend is increasing, with values rising from 27.75 at the initial setting to 28.57 at the final setting. Notably, the minimum value occurs at an intermediate swapping time setting (3), suggesting non-monotonic behavior that warrants further investigation.

These results indicate that swapping time configuration meaningfully impacts run time in this experimental context. The relatively modest variation suggests that this parameter has limited influence on the measured metric within the tested range. Confidence in these findings is high given the direct correspondence between CSV data and plotted values. Future analysis should consider incorporating error bars representing variance across multiple experimental runs to strengthen statistical validity.

\begin{figure}[!htbp]
\centering
\includegraphics[width=0.95\linewidth]{images/500_nodes_Swapping_Time/(e)_Number_of_module_swapped_bar_chart.png}
\caption{(e) Number of module swapped}
\label{fig:500_nodes_Swapping_Time_e__Number_of_module_swapped_bar_chart}
\end{figure}

This bar chart presents the relationship between swapping time and total module swapped for the 500 nodes Swapping Time experimental scenario. The x-axis displays swapping time values ranging from 1 to 4, while the y-axis quantifies total module swapped. The single-series visualization facilitates analysis of how the dependent variable responds to changes in the independent parameter setting.

Analysis of the plotted data reveals that total module swapped ranges from 14.00 (at swapping time = 1) to 15.00 (at swapping time = 2), representing a span of 1.00 units. The overall trend is increasing, with values rising from 14.00 at the initial setting to 15.00 at the final setting. 

These results indicate that swapping time configuration meaningfully impacts total module swapped in this experimental context. The relatively modest variation suggests that this parameter has limited influence on the measured metric within the tested range. Confidence in these findings is high given the direct correspondence between CSV data and plotted values. Future analysis should consider incorporating error bars representing variance across multiple experimental runs to strengthen statistical validity.


\clearpage

\subsection{500 nodes Threshold}

\begin{figure}[!htbp]
\centering
\includegraphics[width=0.95\linewidth]{images/500_nodes_Threshold/(a)_Travel_Time_bar_chart.png}
\caption{(a) Travel Time}
\label{fig:500_nodes_Threshold_a__Travel_Time_bar_chart}
\end{figure}

This bar chart presents the relationship between threshold and total travel time for the 500 nodes Threshold experimental scenario. The x-axis displays threshold values ranging from 5 to 20, while the y-axis quantifies total travel time. The single-series visualization facilitates analysis of how the dependent variable responds to changes in the independent parameter setting.

Analysis of the plotted data reveals that total travel time ranges from 3287.54 (at threshold = 5) to 3287.54 (at threshold = 5), representing a span of 0.00 units. The values remain relatively stable across the parameter range, with minimal net change between initial (3287.54) and final (3287.54) settings. 

These results indicate that threshold configuration meaningfully impacts total travel time in this experimental context. The relatively modest variation suggests that this parameter has limited influence on the measured metric within the tested range. Confidence in these findings is high given the direct correspondence between CSV data and plotted values. Future analysis should consider incorporating error bars representing variance across multiple experimental runs to strengthen statistical validity.

\begin{figure}[!htbp]
\centering
\includegraphics[width=0.95\linewidth]{images/500_nodes_Threshold/(b)_Energy_consumption_bar_chart.png}
\caption{(b) Energy consumption}
\label{fig:500_nodes_Threshold_b__Energy_consumption_bar_chart}
\end{figure}

This bar chart presents the relationship between threshold and total energy consumed for the 500 nodes Threshold experimental scenario. The x-axis displays threshold values ranging from 5 to 20, while the y-axis quantifies total energy consumed. The single-series visualization facilitates analysis of how the dependent variable responds to changes in the independent parameter setting.

Analysis of the plotted data reveals that total energy consumed ranges from 323.50 (at threshold = 5) to 323.50 (at threshold = 5), representing a span of 0.00 units. The values remain relatively stable across the parameter range, with minimal net change between initial (323.50) and final (323.50) settings. 

These results indicate that threshold configuration meaningfully impacts total energy consumed in this experimental context. The relatively modest variation suggests that this parameter has limited influence on the measured metric within the tested range. Confidence in these findings is high given the direct correspondence between CSV data and plotted values. Future analysis should consider incorporating error bars representing variance across multiple experimental runs to strengthen statistical validity.

\begin{figure}[!htbp]
\centering
\includegraphics[width=0.95\linewidth]{images/500_nodes_Threshold/(c)_Distance_covered_bar_chart.png}
\caption{(c) Distance covered}
\label{fig:500_nodes_Threshold_c__Distance_covered_bar_chart}
\end{figure}

This bar chart presents the relationship between threshold and total distance covered for the 500 nodes Threshold experimental scenario. The x-axis displays threshold values ranging from 5 to 20, while the y-axis quantifies total distance covered. The single-series visualization facilitates analysis of how the dependent variable responds to changes in the independent parameter setting.

Analysis of the plotted data reveals that total distance covered ranges from 2138.91 (at threshold = 5) to 2138.91 (at threshold = 5), representing a span of 0.00 units. The values remain relatively stable across the parameter range, with minimal net change between initial (2138.91) and final (2138.91) settings. 

These results indicate that threshold configuration meaningfully impacts total distance covered in this experimental context. The relatively modest variation suggests that this parameter has limited influence on the measured metric within the tested range. Confidence in these findings is high given the direct correspondence between CSV data and plotted values. Future analysis should consider incorporating error bars representing variance across multiple experimental runs to strengthen statistical validity.

\begin{figure}[!htbp]
\centering
\includegraphics[width=0.95\linewidth]{images/500_nodes_Threshold/(d)_Runtime_bar_chart.png}
\caption{(d) Runtime}
\label{fig:500_nodes_Threshold_d__Runtime_bar_chart}
\end{figure}

This bar chart presents the relationship between threshold and run time for the 500 nodes Threshold experimental scenario. The x-axis displays threshold values ranging from 5 to 20, while the y-axis quantifies run time. The single-series visualization facilitates analysis of how the dependent variable responds to changes in the independent parameter setting.

Analysis of the plotted data reveals that run time ranges from 29.02 (at threshold = 20) to 66.47 (at threshold = 10), representing a span of 37.45 units. The overall trend is decreasing, with values declining from 58.34 at the initial setting to 29.02 at the final setting. 

These results indicate that threshold configuration meaningfully impacts run time in this experimental context. The substantial variation observed (coefficient of variation exceeding 20\%) suggests that parameter tuning could yield significant performance improvements. Confidence in these findings is high given the direct correspondence between CSV data and plotted values. Future analysis should consider incorporating error bars representing variance across multiple experimental runs to strengthen statistical validity.

\begin{figure}[!htbp]
\centering
\includegraphics[width=0.95\linewidth]{images/500_nodes_Threshold/(e)_Number_of_module_swapped_bar_chart.png}
\caption{(e) Number of module swapped}
\label{fig:500_nodes_Threshold_e__Number_of_module_swapped_bar_chart}
\end{figure}

This bar chart presents the relationship between threshold and total module swapped for the 500 nodes Threshold experimental scenario. The x-axis displays threshold values ranging from 5 to 20, while the y-axis quantifies total module swapped. The single-series visualization facilitates analysis of how the dependent variable responds to changes in the independent parameter setting.

Analysis of the plotted data reveals that total module swapped ranges from 15.00 (at threshold = 5) to 15.00 (at threshold = 5), representing a span of 0.00 units. The values remain relatively stable across the parameter range, with minimal net change between initial (15.00) and final (15.00) settings. 

These results indicate that threshold configuration meaningfully impacts total module swapped in this experimental context. The relatively modest variation suggests that this parameter has limited influence on the measured metric within the tested range. Confidence in these findings is high given the direct correspondence between CSV data and plotted values. Future analysis should consider incorporating error bars representing variance across multiple experimental runs to strengthen statistical validity.


\clearpage

\subsection{500 nodes Traffic}

\begin{figure}[!htbp]
\centering
\includegraphics[width=0.95\linewidth]{images/500_nodes_Traffic/(a)_Travel_Time_bar_chart.png}
\caption{(a) Travel Time}
\label{fig:500_nodes_Traffic_a__Travel_Time_bar_chart}
\end{figure}

This bar chart presents the relationship between traffic and total travel time for the 500 nodes Traffic experimental scenario. The x-axis displays traffic values ranging from High to Low, while the y-axis quantifies total travel time. The single-series visualization facilitates analysis of how the dependent variable responds to changes in the independent parameter setting.

Analysis of the plotted data reveals that total travel time ranges from 2915.88 (at traffic = Low) to 3618.28 (at traffic = High), representing a span of 702.40 units. The overall trend is decreasing, with values declining from 3618.28 at the initial setting to 2915.88 at the final setting. 

These results indicate that traffic configuration meaningfully impacts total travel time in this experimental context. The substantial variation observed (coefficient of variation exceeding 20\%) suggests that parameter tuning could yield significant performance improvements. Confidence in these findings is high given the direct correspondence between CSV data and plotted values. Future analysis should consider incorporating error bars representing variance across multiple experimental runs to strengthen statistical validity.

\begin{figure}[!htbp]
\centering
\includegraphics[width=0.95\linewidth]{images/500_nodes_Traffic/(b)_Energy_consumption_bar_chart.png}
\caption{(b) Energy consumption}
\label{fig:500_nodes_Traffic_b__Energy_consumption_bar_chart}
\end{figure}

This bar chart presents the relationship between traffic and total energy consumed for the 500 nodes Traffic experimental scenario. The x-axis displays traffic values ranging from High to Low, while the y-axis quantifies total energy consumed. The single-series visualization facilitates analysis of how the dependent variable responds to changes in the independent parameter setting.

Analysis of the plotted data reveals that total energy consumed ranges from 287.80 (at traffic = High) to 351.10 (at traffic = Low), representing a span of 63.29 units. The overall trend is increasing, with values rising from 287.80 at the initial setting to 351.10 at the final setting. 

These results indicate that traffic configuration meaningfully impacts total energy consumed in this experimental context. The substantial variation observed (coefficient of variation exceeding 20\%) suggests that parameter tuning could yield significant performance improvements. Confidence in these findings is high given the direct correspondence between CSV data and plotted values. Future analysis should consider incorporating error bars representing variance across multiple experimental runs to strengthen statistical validity.

\begin{figure}[!htbp]
\centering
\includegraphics[width=0.95\linewidth]{images/500_nodes_Traffic/(c)_Distance_covered_bar_chart.png}
\caption{(c) Distance covered}
\label{fig:500_nodes_Traffic_c__Distance_covered_bar_chart}
\end{figure}

This bar chart presents the relationship between traffic and total distance covered for the 500 nodes Traffic experimental scenario. The x-axis displays traffic values ranging from High to Low, while the y-axis quantifies total distance covered. The single-series visualization facilitates analysis of how the dependent variable responds to changes in the independent parameter setting.

Analysis of the plotted data reveals that total distance covered ranges from 1986.78 (at traffic = High) to 2241.45 (at traffic = Low), representing a span of 254.67 units. The overall trend is increasing, with values rising from 1986.78 at the initial setting to 2241.45 at the final setting. 

These results indicate that traffic configuration meaningfully impacts total distance covered in this experimental context. The relatively modest variation suggests that this parameter has limited influence on the measured metric within the tested range. Confidence in these findings is high given the direct correspondence between CSV data and plotted values. Future analysis should consider incorporating error bars representing variance across multiple experimental runs to strengthen statistical validity.

\begin{figure}[!htbp]
\centering
\includegraphics[width=0.95\linewidth]{images/500_nodes_Traffic/(d)_Runtime_bar_chart.png}
\caption{(d) Runtime}
\label{fig:500_nodes_Traffic_d__Runtime_bar_chart}
\end{figure}

This bar chart presents the relationship between traffic and run time for the 500 nodes Traffic experimental scenario. The x-axis displays traffic values ranging from High to Low, while the y-axis quantifies run time. The single-series visualization facilitates analysis of how the dependent variable responds to changes in the independent parameter setting.

Analysis of the plotted data reveals that run time ranges from 37.23 (at traffic = High) to 54.49 (at traffic = Low), representing a span of 17.26 units. The overall trend is increasing, with values rising from 37.23 at the initial setting to 54.49 at the final setting. 

These results indicate that traffic configuration meaningfully impacts run time in this experimental context. The substantial variation observed (coefficient of variation exceeding 20\%) suggests that parameter tuning could yield significant performance improvements. Confidence in these findings is high given the direct correspondence between CSV data and plotted values. Future analysis should consider incorporating error bars representing variance across multiple experimental runs to strengthen statistical validity.

\begin{figure}[!htbp]
\centering
\includegraphics[width=0.95\linewidth]{images/500_nodes_Traffic/(e)_Number_of_module_swapped_bar_chart.png}
\caption{(e) Number of module swapped}
\label{fig:500_nodes_Traffic_e__Number_of_module_swapped_bar_chart}
\end{figure}

This bar chart presents the relationship between traffic and total module swapped for the 500 nodes Traffic experimental scenario. The x-axis displays traffic values ranging from High to Low, while the y-axis quantifies total module swapped. The single-series visualization facilitates analysis of how the dependent variable responds to changes in the independent parameter setting.

Analysis of the plotted data reveals that total module swapped ranges from 12.00 (at traffic = High) to 16.00 (at traffic = Low), representing a span of 4.00 units. The overall trend is increasing, with values rising from 12.00 at the initial setting to 16.00 at the final setting. 

These results indicate that traffic configuration meaningfully impacts total module swapped in this experimental context. The substantial variation observed (coefficient of variation exceeding 20\%) suggests that parameter tuning could yield significant performance improvements. Confidence in these findings is high given the direct correspondence between CSV data and plotted values. Future analysis should consider incorporating error bars representing variance across multiple experimental runs to strengthen statistical validity.


\clearpage

\subsection{MT2TE and Other Program Module Change}

\begin{figure}[!htbp]
\centering
\includegraphics[width=0.95\linewidth]{images/MT2TE_and_Other_Program_Module_Change/(a)_Total_Travel_Time.png}
\caption{(a) Total Travel Time}
\label{fig:MT2TE_and_Other_Program_Module_Change_a__Total_Travel_Time}
\end{figure}

This grouped bar chart presents a comparative analysis of total travel time across multiple algorithms within the MT2TE and Other Program Module Change experimental configuration. The x-axis represents the number of nodes in the network, while the y-axis quantifies the total travel time metric. The legend identifies 4 distinct algorithms: Genetic Algorithm, EVRPBSS, Ant Colony, Clarke and Wright algorithm. This visualization enables direct comparison of algorithmic performance under identical network conditions.

Quantitative analysis reveals significant performance disparities among the evaluated algorithms. Clarke and Wright algorithm demonstrates superior performance with a mean total travel time of 688.78, while Genetic Algorithm exhibits the highest values averaging 1366.67. This represents an improvement of approximately 49.6\% when comparing the best to worst performing algorithms. The relative ranking of algorithms remains largely consistent across different node configurations, suggesting robust performance characteristics.

These findings support the hypothesis that algorithmic choice significantly impacts system performance metrics. Future work should incorporate statistical significance testing and confidence intervals to strengthen these comparative conclusions. Additionally, examining the computational complexity trade-offs between algorithms would provide valuable context for practical deployment decisions.

\begin{figure}[!htbp]
\centering
\includegraphics[width=0.95\linewidth]{images/MT2TE_and_Other_Program_Module_Change/(b)_Energy.png}
\caption{(b) Energy}
\label{fig:MT2TE_and_Other_Program_Module_Change_b__Energy}
\end{figure}

This grouped bar chart presents a comparative analysis of energy across multiple algorithms within the MT2TE and Other Program Module Change experimental configuration. The x-axis represents the number of nodes in the network, while the y-axis quantifies the energy metric. The legend identifies 4 distinct algorithms: Genetic Algorithm, EVRPBSS, Ant Colony, Clarke and Wright algorithm. This visualization enables direct comparison of algorithmic performance under identical network conditions.

Quantitative analysis reveals significant performance disparities among the evaluated algorithms. Clarke and Wright algorithm demonstrates superior performance with a mean energy of 688.78, while Genetic Algorithm exhibits the highest values averaging 1366.67. This represents an improvement of approximately 49.6\% when comparing the best to worst performing algorithms. The relative ranking of algorithms remains largely consistent across different node configurations, suggesting robust performance characteristics.

These findings support the hypothesis that algorithmic choice significantly impacts system performance metrics. Future work should incorporate statistical significance testing and confidence intervals to strengthen these comparative conclusions. Additionally, examining the computational complexity trade-offs between algorithms would provide valuable context for practical deployment decisions.

\begin{figure}[!htbp]
\centering
\includegraphics[width=0.95\linewidth]{images/MT2TE_and_Other_Program_Module_Change/(c)_Distance.png}
\caption{(c) Distance}
\label{fig:MT2TE_and_Other_Program_Module_Change_c__Distance}
\end{figure}

This grouped bar chart presents a comparative analysis of distance across multiple algorithms within the MT2TE and Other Program Module Change experimental configuration. The x-axis represents the number of nodes in the network, while the y-axis quantifies the distance metric. The legend identifies 4 distinct algorithms: Genetic Algorithm, EVRPBSS, Ant Colony, Clarke and Wright algorithm. This visualization enables direct comparison of algorithmic performance under identical network conditions.

Quantitative analysis reveals significant performance disparities among the evaluated algorithms. Clarke and Wright algorithm demonstrates superior performance with a mean distance of 688.78, while Genetic Algorithm exhibits the highest values averaging 1366.67. This represents an improvement of approximately 49.6\% when comparing the best to worst performing algorithms. The relative ranking of algorithms remains largely consistent across different node configurations, suggesting robust performance characteristics.

These findings support the hypothesis that algorithmic choice significantly impacts system performance metrics. Future work should incorporate statistical significance testing and confidence intervals to strengthen these comparative conclusions. Additionally, examining the computational complexity trade-offs between algorithms would provide valuable context for practical deployment decisions.

\begin{figure}[!htbp]
\centering
\includegraphics[width=0.95\linewidth]{images/MT2TE_and_Other_Program_Module_Change/(d)_Module_Swapped.png}
\caption{(d) Module Swapped}
\label{fig:MT2TE_and_Other_Program_Module_Change_d__Module_Swapped}
\end{figure}

This grouped bar chart presents a comparative analysis of module swapped across multiple algorithms within the MT2TE and Other Program Module Change experimental configuration. The x-axis represents the number of nodes in the network, while the y-axis quantifies the module swapped metric. The legend identifies 4 distinct algorithms: Genetic Algorithm, EVRPBSS, Ant Colony, Clarke and Wright algorithm. This visualization enables direct comparison of algorithmic performance under identical network conditions.

Quantitative analysis reveals significant performance disparities among the evaluated algorithms. Clarke and Wright algorithm demonstrates superior performance with a mean module swapped of 688.78, while Genetic Algorithm exhibits the highest values averaging 1366.67. This represents an improvement of approximately 49.6\% when comparing the best to worst performing algorithms. The relative ranking of algorithms remains largely consistent across different node configurations, suggesting robust performance characteristics.

These findings support the hypothesis that algorithmic choice significantly impacts system performance metrics. Future work should incorporate statistical significance testing and confidence intervals to strengthen these comparative conclusions. Additionally, examining the computational complexity trade-offs between algorithms would provide valuable context for practical deployment decisions.

\begin{figure}[!htbp]
\centering
\includegraphics[width=0.95\linewidth]{images/MT2TE_and_Other_Program_Module_Change/(e)_Execution_Time.png}
\caption{(e) Execution Time}
\label{fig:MT2TE_and_Other_Program_Module_Change_e__Execution_Time}
\end{figure}

This grouped bar chart presents a comparative analysis of execution time across multiple algorithms within the MT2TE and Other Program Module Change experimental configuration. The x-axis represents the number of nodes in the network, while the y-axis quantifies the execution time metric. The legend identifies 4 distinct algorithms: Genetic Algorithm, EVRPBSS, Ant Colony, Clarke and Wright algorithm. This visualization enables direct comparison of algorithmic performance under identical network conditions.

Quantitative analysis reveals significant performance disparities among the evaluated algorithms. Clarke and Wright algorithm demonstrates superior performance with a mean execution time of 688.78, while Genetic Algorithm exhibits the highest values averaging 1366.67. This represents an improvement of approximately 49.6\% when comparing the best to worst performing algorithms. The relative ranking of algorithms remains largely consistent across different node configurations, suggesting robust performance characteristics.

These findings support the hypothesis that algorithmic choice significantly impacts system performance metrics. Future work should incorporate statistical significance testing and confidence intervals to strengthen these comparative conclusions. Additionally, examining the computational complexity trade-offs between algorithms would provide valuable context for practical deployment decisions.

\begin{figure}[!htbp]
\centering
\includegraphics[width=0.95\linewidth]{images/MT2TE_and_Other_Program_Module_Change/(f)_Execution_Time.png}
\caption{(f) Execution Time}
\label{fig:MT2TE_and_Other_Program_Module_Change_f__Execution_Time}
\end{figure}

This grouped bar chart presents a comparative analysis of execution time across multiple algorithms within the MT2TE and Other Program Module Change experimental configuration. The x-axis represents the number of nodes in the network, while the y-axis quantifies the execution time metric. The legend identifies 3 distinct algorithms: EVRPBSS, Ant Colony, Clarke and Wright algorithm. This visualization enables direct comparison of algorithmic performance under identical network conditions.

Quantitative analysis reveals significant performance disparities among the evaluated algorithms. Clarke and Wright algorithm demonstrates superior performance with a mean execution time of 688.78, while Ant Colony exhibits the highest values averaging 1020.01. This represents an improvement of approximately 32.5\% when comparing the best to worst performing algorithms. The relative ranking of algorithms remains largely consistent across different node configurations, suggesting robust performance characteristics.

These findings support the hypothesis that algorithmic choice significantly impacts system performance metrics. Future work should incorporate statistical significance testing and confidence intervals to strengthen these comparative conclusions. Additionally, examining the computational complexity trade-offs between algorithms would provide valuable context for practical deployment decisions.


\clearpage

\subsection{MT2TE and Other Program Node Change}

\begin{figure}[!htbp]
\centering
\includegraphics[width=0.95\linewidth]{images/MT2TE_and_Other_Program_Node_Change/(a)_Total_Travel_Time.png}
\caption{(a) Total Travel Time}
\label{fig:MT2TE_and_Other_Program_Node_Change_a__Total_Travel_Time}
\end{figure}

This grouped bar chart presents a comparative analysis of total travel time across multiple algorithms within the MT2TE and Other Program Node Change experimental configuration. The x-axis represents the number of nodes in the network, while the y-axis quantifies the total travel time metric. The legend identifies 4 distinct algorithms: Genetic Algorithm, EVRPBSS, Ant Colony, Clarke and Wright algorithm. This visualization enables direct comparison of algorithmic performance under identical network conditions.

Quantitative analysis reveals significant performance disparities among the evaluated algorithms. Clarke and Wright algorithm demonstrates superior performance with a mean total travel time of 862.43, while Genetic Algorithm exhibits the highest values averaging 2037.20. This represents an improvement of approximately 57.7\% when comparing the best to worst performing algorithms. The relative ranking of algorithms remains largely consistent across different node configurations, suggesting robust performance characteristics.

These findings support the hypothesis that algorithmic choice significantly impacts system performance metrics. Future work should incorporate statistical significance testing and confidence intervals to strengthen these comparative conclusions. Additionally, examining the computational complexity trade-offs between algorithms would provide valuable context for practical deployment decisions.

\begin{figure}[!htbp]
\centering
\includegraphics[width=0.95\linewidth]{images/MT2TE_and_Other_Program_Node_Change/(b)_Energy.png}
\caption{(b) Energy}
\label{fig:MT2TE_and_Other_Program_Node_Change_b__Energy}
\end{figure}

This grouped bar chart presents a comparative analysis of energy across multiple algorithms within the MT2TE and Other Program Node Change experimental configuration. The x-axis represents the number of nodes in the network, while the y-axis quantifies the energy metric. The legend identifies 4 distinct algorithms: Genetic Algorithm, EVRPBSS, Ant Colony, Clarke and Wright algorithm. This visualization enables direct comparison of algorithmic performance under identical network conditions.

Quantitative analysis reveals significant performance disparities among the evaluated algorithms. Clarke and Wright algorithm demonstrates superior performance with a mean energy of 862.43, while Genetic Algorithm exhibits the highest values averaging 2037.20. This represents an improvement of approximately 57.7\% when comparing the best to worst performing algorithms. The relative ranking of algorithms remains largely consistent across different node configurations, suggesting robust performance characteristics.

These findings support the hypothesis that algorithmic choice significantly impacts system performance metrics. Future work should incorporate statistical significance testing and confidence intervals to strengthen these comparative conclusions. Additionally, examining the computational complexity trade-offs between algorithms would provide valuable context for practical deployment decisions.

\begin{figure}[!htbp]
\centering
\includegraphics[width=0.95\linewidth]{images/MT2TE_and_Other_Program_Node_Change/(c)_Distance.png}
\caption{(c) Distance}
\label{fig:MT2TE_and_Other_Program_Node_Change_c__Distance}
\end{figure}

This grouped bar chart presents a comparative analysis of distance across multiple algorithms within the MT2TE and Other Program Node Change experimental configuration. The x-axis represents the number of nodes in the network, while the y-axis quantifies the distance metric. The legend identifies 4 distinct algorithms: Genetic Algorithm, EVRPBSS, Ant Colony, Clarke and Wright algorithm. This visualization enables direct comparison of algorithmic performance under identical network conditions.

Quantitative analysis reveals significant performance disparities among the evaluated algorithms. Clarke and Wright algorithm demonstrates superior performance with a mean distance of 862.43, while Genetic Algorithm exhibits the highest values averaging 2037.20. This represents an improvement of approximately 57.7\% when comparing the best to worst performing algorithms. The relative ranking of algorithms remains largely consistent across different node configurations, suggesting robust performance characteristics.

These findings support the hypothesis that algorithmic choice significantly impacts system performance metrics. Future work should incorporate statistical significance testing and confidence intervals to strengthen these comparative conclusions. Additionally, examining the computational complexity trade-offs between algorithms would provide valuable context for practical deployment decisions.

\begin{figure}[!htbp]
\centering
\includegraphics[width=0.95\linewidth]{images/MT2TE_and_Other_Program_Node_Change/(d)_Execution_Time.png}
\caption{(d) Execution Time}
\label{fig:MT2TE_and_Other_Program_Node_Change_d__Execution_Time}
\end{figure}

This grouped bar chart presents a comparative analysis of execution time across multiple algorithms within the MT2TE and Other Program Node Change experimental configuration. The x-axis represents the number of nodes in the network, while the y-axis quantifies the execution time metric. The legend identifies 4 distinct algorithms: Genetic Algorithm, EVRPBSS, Ant Colony, Clarke and Wright algorithm. This visualization enables direct comparison of algorithmic performance under identical network conditions.

Quantitative analysis reveals significant performance disparities among the evaluated algorithms. Clarke and Wright algorithm demonstrates superior performance with a mean execution time of 862.43, while Genetic Algorithm exhibits the highest values averaging 2037.20. This represents an improvement of approximately 57.7\% when comparing the best to worst performing algorithms. The relative ranking of algorithms remains largely consistent across different node configurations, suggesting robust performance characteristics.

These findings support the hypothesis that algorithmic choice significantly impacts system performance metrics. Future work should incorporate statistical significance testing and confidence intervals to strengthen these comparative conclusions. Additionally, examining the computational complexity trade-offs between algorithms would provide valuable context for practical deployment decisions.

\begin{figure}[!htbp]
\centering
\includegraphics[width=0.95\linewidth]{images/MT2TE_and_Other_Program_Node_Change/(e)_Total_Module_Swapped.png}
\caption{(e) Total Module Swapped}
\label{fig:MT2TE_and_Other_Program_Node_Change_e__Total_Module_Swapped}
\end{figure}

This grouped bar chart presents a comparative analysis of total module swapped across multiple algorithms within the MT2TE and Other Program Node Change experimental configuration. The x-axis represents the number of nodes in the network, while the y-axis quantifies the total module swapped metric. The legend identifies 4 distinct algorithms: Genetic Algorithm, EVRPBSS, Ant Colony, Clarke and Wright algorithm. This visualization enables direct comparison of algorithmic performance under identical network conditions.

Quantitative analysis reveals significant performance disparities among the evaluated algorithms. Clarke and Wright algorithm demonstrates superior performance with a mean total module swapped of 862.43, while Genetic Algorithm exhibits the highest values averaging 2037.20. This represents an improvement of approximately 57.7\% when comparing the best to worst performing algorithms. The relative ranking of algorithms remains largely consistent across different node configurations, suggesting robust performance characteristics.

These findings support the hypothesis that algorithmic choice significantly impacts system performance metrics. Future work should incorporate statistical significance testing and confidence intervals to strengthen these comparative conclusions. Additionally, examining the computational complexity trade-offs between algorithms would provide valuable context for practical deployment decisions.

\begin{figure}[!htbp]
\centering
\includegraphics[width=0.95\linewidth]{images/MT2TE_and_Other_Program_Node_Change/(f)_Execution_Time.png}
\caption{(f) Execution Time}
\label{fig:MT2TE_and_Other_Program_Node_Change_f__Execution_Time}
\end{figure}

This grouped bar chart presents a comparative analysis of execution time across multiple algorithms within the MT2TE and Other Program Node Change experimental configuration. The x-axis represents the number of nodes in the network, while the y-axis quantifies the execution time metric. The legend identifies 4 distinct algorithms: Genetic Algorithm, EVRPBSS, Ant Colony, Clarke and Wright algorithm. This visualization enables direct comparison of algorithmic performance under identical network conditions.

Quantitative analysis reveals significant performance disparities among the evaluated algorithms. Clarke and Wright algorithm demonstrates superior performance with a mean execution time of 862.43, while Genetic Algorithm exhibits the highest values averaging 2037.20. This represents an improvement of approximately 57.7\% when comparing the best to worst performing algorithms. The relative ranking of algorithms remains largely consistent across different node configurations, suggesting robust performance characteristics.

These findings support the hypothesis that algorithmic choice significantly impacts system performance metrics. Future work should incorporate statistical significance testing and confidence intervals to strengthen these comparative conclusions. Additionally, examining the computational complexity trade-offs between algorithms would provide valuable context for practical deployment decisions.


\clearpage

\subsection{MT2TE and Other Program Swap Time Change}

\begin{figure}[!htbp]
\centering
\includegraphics[width=0.95\linewidth]{images/MT2TE_and_Other_Program_Swap_Time_Change/(a)_Total_Travel_Time.png}
\caption{(a) Total Travel Time}
\label{fig:MT2TE_and_Other_Program_Swap_Time_Change_a__Total_Travel_Time}
\end{figure}

This grouped bar chart presents a comparative analysis of total travel time across multiple algorithms within the MT2TE and Other Program Swap Time Change experimental configuration. The x-axis represents the number of nodes in the network, while the y-axis quantifies the total travel time metric. The legend identifies 4 distinct algorithms: Genetic Algorithm, EVRPBSS, Ant Colony, Clarke and Wright algorithm. This visualization enables direct comparison of algorithmic performance under identical network conditions.

Quantitative analysis reveals significant performance disparities among the evaluated algorithms. Clarke and Wright algorithm demonstrates superior performance with a mean total travel time of 697.62, while Genetic Algorithm exhibits the highest values averaging 1438.33. This represents an improvement of approximately 51.5\% when comparing the best to worst performing algorithms. The relative ranking of algorithms remains largely consistent across different node configurations, suggesting robust performance characteristics.

These findings support the hypothesis that algorithmic choice significantly impacts system performance metrics. Future work should incorporate statistical significance testing and confidence intervals to strengthen these comparative conclusions. Additionally, examining the computational complexity trade-offs between algorithms would provide valuable context for practical deployment decisions.

\begin{figure}[!htbp]
\centering
\includegraphics[width=0.95\linewidth]{images/MT2TE_and_Other_Program_Swap_Time_Change/(b)_Energy.png}
\caption{(b) Energy}
\label{fig:MT2TE_and_Other_Program_Swap_Time_Change_b__Energy}
\end{figure}

This grouped bar chart presents a comparative analysis of energy across multiple algorithms within the MT2TE and Other Program Swap Time Change experimental configuration. The x-axis represents the number of nodes in the network, while the y-axis quantifies the energy metric. The legend identifies 4 distinct algorithms: Genetic Algorithm, EVRPBSS, Ant Colony, Clarke and Wright algorithm. This visualization enables direct comparison of algorithmic performance under identical network conditions.

Quantitative analysis reveals significant performance disparities among the evaluated algorithms. Clarke and Wright algorithm demonstrates superior performance with a mean energy of 697.62, while Genetic Algorithm exhibits the highest values averaging 1438.33. This represents an improvement of approximately 51.5\% when comparing the best to worst performing algorithms. The relative ranking of algorithms remains largely consistent across different node configurations, suggesting robust performance characteristics.

These findings support the hypothesis that algorithmic choice significantly impacts system performance metrics. Future work should incorporate statistical significance testing and confidence intervals to strengthen these comparative conclusions. Additionally, examining the computational complexity trade-offs between algorithms would provide valuable context for practical deployment decisions.

\begin{figure}[!htbp]
\centering
\includegraphics[width=0.95\linewidth]{images/MT2TE_and_Other_Program_Swap_Time_Change/(c)_Distance.png}
\caption{(c) Distance}
\label{fig:MT2TE_and_Other_Program_Swap_Time_Change_c__Distance}
\end{figure}

This grouped bar chart presents a comparative analysis of distance across multiple algorithms within the MT2TE and Other Program Swap Time Change experimental configuration. The x-axis represents the number of nodes in the network, while the y-axis quantifies the distance metric. The legend identifies 4 distinct algorithms: Genetic Algorithm, EVRPBSS, Ant Colony, Clarke and Wright algorithm. This visualization enables direct comparison of algorithmic performance under identical network conditions.

Quantitative analysis reveals significant performance disparities among the evaluated algorithms. Clarke and Wright algorithm demonstrates superior performance with a mean distance of 697.62, while Genetic Algorithm exhibits the highest values averaging 1438.33. This represents an improvement of approximately 51.5\% when comparing the best to worst performing algorithms. The relative ranking of algorithms remains largely consistent across different node configurations, suggesting robust performance characteristics.

These findings support the hypothesis that algorithmic choice significantly impacts system performance metrics. Future work should incorporate statistical significance testing and confidence intervals to strengthen these comparative conclusions. Additionally, examining the computational complexity trade-offs between algorithms would provide valuable context for practical deployment decisions.

\begin{figure}[!htbp]
\centering
\includegraphics[width=0.95\linewidth]{images/MT2TE_and_Other_Program_Swap_Time_Change/(d)_Total_Module_Swapped.png}
\caption{(d) Total Module Swapped}
\label{fig:MT2TE_and_Other_Program_Swap_Time_Change_d__Total_Module_Swapped}
\end{figure}

This grouped bar chart presents a comparative analysis of total module swapped across multiple algorithms within the MT2TE and Other Program Swap Time Change experimental configuration. The x-axis represents the number of nodes in the network, while the y-axis quantifies the total module swapped metric. The legend identifies 4 distinct algorithms: Genetic Algorithm, EVRPBSS, Ant Colony, Clarke and Wright algorithm. This visualization enables direct comparison of algorithmic performance under identical network conditions.

Quantitative analysis reveals significant performance disparities among the evaluated algorithms. Clarke and Wright algorithm demonstrates superior performance with a mean total module swapped of 697.62, while Genetic Algorithm exhibits the highest values averaging 1438.33. This represents an improvement of approximately 51.5\% when comparing the best to worst performing algorithms. The relative ranking of algorithms remains largely consistent across different node configurations, suggesting robust performance characteristics.

These findings support the hypothesis that algorithmic choice significantly impacts system performance metrics. Future work should incorporate statistical significance testing and confidence intervals to strengthen these comparative conclusions. Additionally, examining the computational complexity trade-offs between algorithms would provide valuable context for practical deployment decisions.

\begin{figure}[!htbp]
\centering
\includegraphics[width=0.95\linewidth]{images/MT2TE_and_Other_Program_Swap_Time_Change/(e)_Execution_Time.png}
\caption{(e) Execution Time}
\label{fig:MT2TE_and_Other_Program_Swap_Time_Change_e__Execution_Time}
\end{figure}

This grouped bar chart presents a comparative analysis of execution time across multiple algorithms within the MT2TE and Other Program Swap Time Change experimental configuration. The x-axis represents the number of nodes in the network, while the y-axis quantifies the execution time metric. The legend identifies 4 distinct algorithms: Genetic Algorithm, EVRPBSS, Ant Colony, Clarke and Wright algorithm. This visualization enables direct comparison of algorithmic performance under identical network conditions.

Quantitative analysis reveals significant performance disparities among the evaluated algorithms. Clarke and Wright algorithm demonstrates superior performance with a mean execution time of 697.62, while Genetic Algorithm exhibits the highest values averaging 1438.33. This represents an improvement of approximately 51.5\% when comparing the best to worst performing algorithms. The relative ranking of algorithms remains largely consistent across different node configurations, suggesting robust performance characteristics.

These findings support the hypothesis that algorithmic choice significantly impacts system performance metrics. Future work should incorporate statistical significance testing and confidence intervals to strengthen these comparative conclusions. Additionally, examining the computational complexity trade-offs between algorithms would provide valuable context for practical deployment decisions.

\begin{figure}[!htbp]
\centering
\includegraphics[width=0.95\linewidth]{images/MT2TE_and_Other_Program_Swap_Time_Change/(f)_Execution_Time.png}
\caption{(f) Execution Time}
\label{fig:MT2TE_and_Other_Program_Swap_Time_Change_f__Execution_Time}
\end{figure}

This grouped bar chart presents a comparative analysis of execution time across multiple algorithms within the MT2TE and Other Program Swap Time Change experimental configuration. The x-axis represents the number of nodes in the network, while the y-axis quantifies the execution time metric. The legend identifies 3 distinct algorithms: EVRPBSS, Ant Colony, Clarke and Wright algorithm. This visualization enables direct comparison of algorithmic performance under identical network conditions.

Quantitative analysis reveals significant performance disparities among the evaluated algorithms. Clarke and Wright algorithm demonstrates superior performance with a mean execution time of 697.62, while Ant Colony exhibits the highest values averaging 1026.84. This represents an improvement of approximately 32.1\% when comparing the best to worst performing algorithms. The relative ranking of algorithms remains largely consistent across different node configurations, suggesting robust performance characteristics.

These findings support the hypothesis that algorithmic choice significantly impacts system performance metrics. Future work should incorporate statistical significance testing and confidence intervals to strengthen these comparative conclusions. Additionally, examining the computational complexity trade-offs between algorithms would provide valuable context for practical deployment decisions.


\clearpage

\subsection{MT2TE and Other Program Threshold Change}

\begin{figure}[!htbp]
\centering
\includegraphics[width=0.95\linewidth]{images/MT2TE_and_Other_Program_Threshold_Change/(a)_Total_Travel_Time.png}
\caption{(a) Total Travel Time}
\label{fig:MT2TE_and_Other_Program_Threshold_Change_a__Total_Travel_Time}
\end{figure}

This grouped bar chart presents a comparative analysis of total travel time across multiple algorithms within the MT2TE and Other Program Threshold Change experimental configuration. The x-axis represents the number of nodes in the network, while the y-axis quantifies the total travel time metric. The legend identifies 4 distinct algorithms: Genetic Algorithm, EVRPBSS, Ant Colony, Clarke and Wright algorithm. This visualization enables direct comparison of algorithmic performance under identical network conditions.

Quantitative analysis reveals significant performance disparities among the evaluated algorithms. Clarke and Wright algorithm demonstrates superior performance with a mean total travel time of 690.83, while Genetic Algorithm exhibits the highest values averaging 1400.93. This represents an improvement of approximately 50.7\% when comparing the best to worst performing algorithms. The relative ranking of algorithms remains largely consistent across different node configurations, suggesting robust performance characteristics.

These findings support the hypothesis that algorithmic choice significantly impacts system performance metrics. Future work should incorporate statistical significance testing and confidence intervals to strengthen these comparative conclusions. Additionally, examining the computational complexity trade-offs between algorithms would provide valuable context for practical deployment decisions.

\begin{figure}[!htbp]
\centering
\includegraphics[width=0.95\linewidth]{images/MT2TE_and_Other_Program_Threshold_Change/(b)_Energy.png}
\caption{(b) Energy}
\label{fig:MT2TE_and_Other_Program_Threshold_Change_b__Energy}
\end{figure}

This grouped bar chart presents a comparative analysis of energy across multiple algorithms within the MT2TE and Other Program Threshold Change experimental configuration. The x-axis represents the number of nodes in the network, while the y-axis quantifies the energy metric. The legend identifies 4 distinct algorithms: Genetic Algorithm, EVRPBSS, Ant Colony, Clarke and Wright algorithm. This visualization enables direct comparison of algorithmic performance under identical network conditions.

Quantitative analysis reveals significant performance disparities among the evaluated algorithms. Clarke and Wright algorithm demonstrates superior performance with a mean energy of 690.83, while Genetic Algorithm exhibits the highest values averaging 1400.93. This represents an improvement of approximately 50.7\% when comparing the best to worst performing algorithms. The relative ranking of algorithms remains largely consistent across different node configurations, suggesting robust performance characteristics.

These findings support the hypothesis that algorithmic choice significantly impacts system performance metrics. Future work should incorporate statistical significance testing and confidence intervals to strengthen these comparative conclusions. Additionally, examining the computational complexity trade-offs between algorithms would provide valuable context for practical deployment decisions.

\begin{figure}[!htbp]
\centering
\includegraphics[width=0.95\linewidth]{images/MT2TE_and_Other_Program_Threshold_Change/(c)_Distance.png}
\caption{(c) Distance}
\label{fig:MT2TE_and_Other_Program_Threshold_Change_c__Distance}
\end{figure}

This grouped bar chart presents a comparative analysis of distance across multiple algorithms within the MT2TE and Other Program Threshold Change experimental configuration. The x-axis represents the number of nodes in the network, while the y-axis quantifies the distance metric. The legend identifies 4 distinct algorithms: Genetic Algorithm, EVRPBSS, Ant Colony, Clarke and Wright algorithm. This visualization enables direct comparison of algorithmic performance under identical network conditions.

Quantitative analysis reveals significant performance disparities among the evaluated algorithms. Clarke and Wright algorithm demonstrates superior performance with a mean distance of 690.83, while Genetic Algorithm exhibits the highest values averaging 1400.93. This represents an improvement of approximately 50.7\% when comparing the best to worst performing algorithms. The relative ranking of algorithms remains largely consistent across different node configurations, suggesting robust performance characteristics.

These findings support the hypothesis that algorithmic choice significantly impacts system performance metrics. Future work should incorporate statistical significance testing and confidence intervals to strengthen these comparative conclusions. Additionally, examining the computational complexity trade-offs between algorithms would provide valuable context for practical deployment decisions.

\begin{figure}[!htbp]
\centering
\includegraphics[width=0.95\linewidth]{images/MT2TE_and_Other_Program_Threshold_Change/(d)_Module_Swapped.png}
\caption{(d) Module Swapped}
\label{fig:MT2TE_and_Other_Program_Threshold_Change_d__Module_Swapped}
\end{figure}

This grouped bar chart presents a comparative analysis of module swapped across multiple algorithms within the MT2TE and Other Program Threshold Change experimental configuration. The x-axis represents the number of nodes in the network, while the y-axis quantifies the module swapped metric. The legend identifies 4 distinct algorithms: Genetic Algorithm, EVRPBSS, Ant Colony, Clarke and Wright algorithm. This visualization enables direct comparison of algorithmic performance under identical network conditions.

Quantitative analysis reveals significant performance disparities among the evaluated algorithms. Clarke and Wright algorithm demonstrates superior performance with a mean module swapped of 690.83, while Genetic Algorithm exhibits the highest values averaging 1400.93. This represents an improvement of approximately 50.7\% when comparing the best to worst performing algorithms. The relative ranking of algorithms remains largely consistent across different node configurations, suggesting robust performance characteristics.

These findings support the hypothesis that algorithmic choice significantly impacts system performance metrics. Future work should incorporate statistical significance testing and confidence intervals to strengthen these comparative conclusions. Additionally, examining the computational complexity trade-offs between algorithms would provide valuable context for practical deployment decisions.

\begin{figure}[!htbp]
\centering
\includegraphics[width=0.95\linewidth]{images/MT2TE_and_Other_Program_Threshold_Change/(e)_Execution_Time.png}
\caption{(e) Execution Time}
\label{fig:MT2TE_and_Other_Program_Threshold_Change_e__Execution_Time}
\end{figure}

This grouped bar chart presents a comparative analysis of execution time across multiple algorithms within the MT2TE and Other Program Threshold Change experimental configuration. The x-axis represents the number of nodes in the network, while the y-axis quantifies the execution time metric. The legend identifies 4 distinct algorithms: Genetic Algorithm, EVRPBSS, Ant Colony, Clarke and Wright algorithm. This visualization enables direct comparison of algorithmic performance under identical network conditions.

Quantitative analysis reveals significant performance disparities among the evaluated algorithms. Clarke and Wright algorithm demonstrates superior performance with a mean execution time of 690.83, while Genetic Algorithm exhibits the highest values averaging 1400.93. This represents an improvement of approximately 50.7\% when comparing the best to worst performing algorithms. The relative ranking of algorithms remains largely consistent across different node configurations, suggesting robust performance characteristics.

These findings support the hypothesis that algorithmic choice significantly impacts system performance metrics. Future work should incorporate statistical significance testing and confidence intervals to strengthen these comparative conclusions. Additionally, examining the computational complexity trade-offs between algorithms would provide valuable context for practical deployment decisions.

\begin{figure}[!htbp]
\centering
\includegraphics[width=0.95\linewidth]{images/MT2TE_and_Other_Program_Threshold_Change/(f)_Execution_Time.png}
\caption{(f) Execution Time}
\label{fig:MT2TE_and_Other_Program_Threshold_Change_f__Execution_Time}
\end{figure}

This grouped bar chart presents a comparative analysis of execution time across multiple algorithms within the MT2TE and Other Program Threshold Change experimental configuration. The x-axis represents the number of nodes in the network, while the y-axis quantifies the execution time metric. The legend identifies 3 distinct algorithms: EVRPBSS, Ant Colony, Clarke and Wright algorithm. This visualization enables direct comparison of algorithmic performance under identical network conditions.

Quantitative analysis reveals significant performance disparities among the evaluated algorithms. Clarke and Wright algorithm demonstrates superior performance with a mean execution time of 690.83, while Ant Colony exhibits the highest values averaging 1020.01. This represents an improvement of approximately 32.3\% when comparing the best to worst performing algorithms. The relative ranking of algorithms remains largely consistent across different node configurations, suggesting robust performance characteristics.

These findings support the hypothesis that algorithmic choice significantly impacts system performance metrics. Future work should incorporate statistical significance testing and confidence intervals to strengthen these comparative conclusions. Additionally, examining the computational complexity trade-offs between algorithms would provide valuable context for practical deployment decisions.


\clearpage

\subsection{MT2TE and Other Program Traffic}

\begin{figure}[!htbp]
\centering
\includegraphics[width=0.95\linewidth]{images/MT2TE_and_Other_Program_Traffic/(a)_Travel_Time.png}
\caption{(a) Travel Time}
\label{fig:MT2TE_and_Other_Program_Traffic_a__Travel_Time}
\end{figure}

This grouped bar chart presents a comparative analysis of travel time across multiple algorithms within the MT2TE and Other Program Traffic experimental configuration. The x-axis represents the number of nodes in the network, while the y-axis quantifies the travel time metric. The legend identifies 4 distinct algorithms: Genetic Algorithm, EVRPBSS, Ant Colony, Clarke and Wright algorithm. This visualization enables direct comparison of algorithmic performance under identical network conditions.

Quantitative analysis reveals significant performance disparities among the evaluated algorithms. Clarke and Wright algorithm demonstrates superior performance with a mean travel time of 697.95, while Genetic Algorithm exhibits the highest values averaging 1420.64. This represents an improvement of approximately 50.9\% when comparing the best to worst performing algorithms. The relative ranking of algorithms remains largely consistent across different node configurations, suggesting robust performance characteristics.

These findings support the hypothesis that algorithmic choice significantly impacts system performance metrics. Future work should incorporate statistical significance testing and confidence intervals to strengthen these comparative conclusions. Additionally, examining the computational complexity trade-offs between algorithms would provide valuable context for practical deployment decisions.

\begin{figure}[!htbp]
\centering
\includegraphics[width=0.95\linewidth]{images/MT2TE_and_Other_Program_Traffic/(b)_Energy.png}
\caption{(b) Energy}
\label{fig:MT2TE_and_Other_Program_Traffic_b__Energy}
\end{figure}

This grouped bar chart presents a comparative analysis of energy across multiple algorithms within the MT2TE and Other Program Traffic experimental configuration. The x-axis represents the number of nodes in the network, while the y-axis quantifies the energy metric. The legend identifies 4 distinct algorithms: Genetic Algorithm, EVRPBSS, Ant Colony, Clarke and Wright algorithm. This visualization enables direct comparison of algorithmic performance under identical network conditions.

Quantitative analysis reveals significant performance disparities among the evaluated algorithms. Clarke and Wright algorithm demonstrates superior performance with a mean energy of 697.95, while Genetic Algorithm exhibits the highest values averaging 1420.64. This represents an improvement of approximately 50.9\% when comparing the best to worst performing algorithms. The relative ranking of algorithms remains largely consistent across different node configurations, suggesting robust performance characteristics.

These findings support the hypothesis that algorithmic choice significantly impacts system performance metrics. Future work should incorporate statistical significance testing and confidence intervals to strengthen these comparative conclusions. Additionally, examining the computational complexity trade-offs between algorithms would provide valuable context for practical deployment decisions.

\begin{figure}[!htbp]
\centering
\includegraphics[width=0.95\linewidth]{images/MT2TE_and_Other_Program_Traffic/(c)_Distance.png}
\caption{(c) Distance}
\label{fig:MT2TE_and_Other_Program_Traffic_c__Distance}
\end{figure}

This grouped bar chart presents a comparative analysis of distance across multiple algorithms within the MT2TE and Other Program Traffic experimental configuration. The x-axis represents the number of nodes in the network, while the y-axis quantifies the distance metric. The legend identifies 4 distinct algorithms: Genetic Algorithm, EVRPBSS, Ant Colony, Clarke and Wright algorithm. This visualization enables direct comparison of algorithmic performance under identical network conditions.

Quantitative analysis reveals significant performance disparities among the evaluated algorithms. Clarke and Wright algorithm demonstrates superior performance with a mean distance of 697.95, while Genetic Algorithm exhibits the highest values averaging 1420.64. This represents an improvement of approximately 50.9\% when comparing the best to worst performing algorithms. The relative ranking of algorithms remains largely consistent across different node configurations, suggesting robust performance characteristics.

These findings support the hypothesis that algorithmic choice significantly impacts system performance metrics. Future work should incorporate statistical significance testing and confidence intervals to strengthen these comparative conclusions. Additionally, examining the computational complexity trade-offs between algorithms would provide valuable context for practical deployment decisions.

\begin{figure}[!htbp]
\centering
\includegraphics[width=0.95\linewidth]{images/MT2TE_and_Other_Program_Traffic/(d)_Total_Module_Swapped.png}
\caption{(d) Total Module Swapped}
\label{fig:MT2TE_and_Other_Program_Traffic_d__Total_Module_Swapped}
\end{figure}

This grouped bar chart presents a comparative analysis of total module swapped across multiple algorithms within the MT2TE and Other Program Traffic experimental configuration. The x-axis represents the number of nodes in the network, while the y-axis quantifies the total module swapped metric. The legend identifies 4 distinct algorithms: Genetic Algorithm, EVRPBSS, Ant Colony, Clarke and Wright algorithm. This visualization enables direct comparison of algorithmic performance under identical network conditions.

Quantitative analysis reveals significant performance disparities among the evaluated algorithms. Clarke and Wright algorithm demonstrates superior performance with a mean total module swapped of 697.95, while Genetic Algorithm exhibits the highest values averaging 1420.64. This represents an improvement of approximately 50.9\% when comparing the best to worst performing algorithms. The relative ranking of algorithms remains largely consistent across different node configurations, suggesting robust performance characteristics.

These findings support the hypothesis that algorithmic choice significantly impacts system performance metrics. Future work should incorporate statistical significance testing and confidence intervals to strengthen these comparative conclusions. Additionally, examining the computational complexity trade-offs between algorithms would provide valuable context for practical deployment decisions.

\begin{figure}[!htbp]
\centering
\includegraphics[width=0.95\linewidth]{images/MT2TE_and_Other_Program_Traffic/(e)_Execution_Time.png}
\caption{(e) Execution Time}
\label{fig:MT2TE_and_Other_Program_Traffic_e__Execution_Time}
\end{figure}

This grouped bar chart presents a comparative analysis of execution time across multiple algorithms within the MT2TE and Other Program Traffic experimental configuration. The x-axis represents the number of nodes in the network, while the y-axis quantifies the execution time metric. The legend identifies 4 distinct algorithms: Genetic Algorithm, EVRPBSS, Ant Colony, Clarke and Wright algorithm. This visualization enables direct comparison of algorithmic performance under identical network conditions.

Quantitative analysis reveals significant performance disparities among the evaluated algorithms. Clarke and Wright algorithm demonstrates superior performance with a mean execution time of 697.95, while Genetic Algorithm exhibits the highest values averaging 1420.64. This represents an improvement of approximately 50.9\% when comparing the best to worst performing algorithms. The relative ranking of algorithms remains largely consistent across different node configurations, suggesting robust performance characteristics.

These findings support the hypothesis that algorithmic choice significantly impacts system performance metrics. Future work should incorporate statistical significance testing and confidence intervals to strengthen these comparative conclusions. Additionally, examining the computational complexity trade-offs between algorithms would provide valuable context for practical deployment decisions.

\begin{figure}[!htbp]
\centering
\includegraphics[width=0.95\linewidth]{images/MT2TE_and_Other_Program_Traffic/(f)_Execution_Time.png}
\caption{(f) Execution Time}
\label{fig:MT2TE_and_Other_Program_Traffic_f__Execution_Time}
\end{figure}

This grouped bar chart presents a comparative analysis of execution time across multiple algorithms within the MT2TE and Other Program Traffic experimental configuration. The x-axis represents the number of nodes in the network, while the y-axis quantifies the execution time metric. The legend identifies 4 distinct algorithms: Genetic Algorithm, EVRPBSS, Ant Colony, Clarke and Wright algorithm. This visualization enables direct comparison of algorithmic performance under identical network conditions.

Quantitative analysis reveals significant performance disparities among the evaluated algorithms. Clarke and Wright algorithm demonstrates superior performance with a mean execution time of 862.43, while Genetic Algorithm exhibits the highest values averaging 2037.20. This represents an improvement of approximately 57.7\% when comparing the best to worst performing algorithms. The relative ranking of algorithms remains largely consistent across different node configurations, suggesting robust performance characteristics.

These findings support the hypothesis that algorithmic choice significantly impacts system performance metrics. Future work should incorporate statistical significance testing and confidence intervals to strengthen these comparative conclusions. Additionally, examining the computational complexity trade-offs between algorithms would provide valuable context for practical deployment decisions.


\clearpage

\subsection{MT2TE Multi-Line Charts}

\begin{figure}[!htbp]
\centering
\includegraphics[width=0.95\linewidth]{images/MT2TE_Multi-Line_Charts/(a)_Travel_time_bar_chart.png}
\caption{(a) Travel time}
\label{fig:MT2TE_Multi_Line_Charts_a__Travel_time_bar_chart}
\end{figure}

This bar chart presents the relationship between 4 module and 5 module for the MT2TE Multi-Line Charts experimental scenario. The x-axis displays 4 module values ranging from 3287.54 to 14159.32, while the y-axis quantifies 5 module. The single-series visualization facilitates analysis of how the dependent variable responds to changes in the independent parameter setting.

Analysis of the plotted data reveals that 5 module ranges from 3287.54 (at 4 module = 3287.54) to 13381.98 (at 4 module = 14159.32), representing a span of 10094.44 units. The overall trend is increasing, with values rising from 3287.54 at the initial setting to 13381.98 at the final setting. 

These results indicate that 4 module configuration meaningfully impacts 5 module in this experimental context. The substantial variation observed (coefficient of variation exceeding 20\%) suggests that parameter tuning could yield significant performance improvements. Confidence in these findings is high given the direct correspondence between CSV data and plotted values. Future analysis should consider incorporating error bars representing variance across multiple experimental runs to strengthen statistical validity.

\begin{figure}[!htbp]
\centering
\includegraphics[width=0.95\linewidth]{images/MT2TE_Multi-Line_Charts/(b)_Energy_consumed_bar_chart.png}
\caption{(b) Energy consumed}
\label{fig:MT2TE_Multi_Line_Charts_b__Energy_consumed_bar_chart}
\end{figure}

This bar chart presents the relationship between 4 module and 5 module for the MT2TE Multi-Line Charts experimental scenario. The x-axis displays 4 module values ranging from 323.496 to 2154.006, while the y-axis quantifies 5 module. The single-series visualization facilitates analysis of how the dependent variable responds to changes in the independent parameter setting.

Analysis of the plotted data reveals that 5 module ranges from 323.50 (at 4 module = 323.496) to 2057.45 (at 4 module = 2154.006), representing a span of 1733.96 units. The overall trend is increasing, with values rising from 323.50 at the initial setting to 2057.45 at the final setting. 

These results indicate that 4 module configuration meaningfully impacts 5 module in this experimental context. The substantial variation observed (coefficient of variation exceeding 20\%) suggests that parameter tuning could yield significant performance improvements. Confidence in these findings is high given the direct correspondence between CSV data and plotted values. Future analysis should consider incorporating error bars representing variance across multiple experimental runs to strengthen statistical validity.

\begin{figure}[!htbp]
\centering
\includegraphics[width=0.95\linewidth]{images/MT2TE_Multi-Line_Charts/(c)_Distance_covered_bar_chart.png}
\caption{(c) Distance covered}
\label{fig:MT2TE_Multi_Line_Charts_c__Distance_covered_bar_chart}
\end{figure}

This bar chart presents the relationship between 4 module and 5 module for the MT2TE Multi-Line Charts experimental scenario. The x-axis displays 4 module values ranging from 2138.91 to 9234.69, while the y-axis quantifies 5 module. The single-series visualization facilitates analysis of how the dependent variable responds to changes in the independent parameter setting.

Analysis of the plotted data reveals that 5 module ranges from 2138.91 (at 4 module = 2138.91) to 8652.17 (at 4 module = 9234.69), representing a span of 6513.26 units. The overall trend is increasing, with values rising from 2138.91 at the initial setting to 8652.17 at the final setting. 

These results indicate that 4 module configuration meaningfully impacts 5 module in this experimental context. The substantial variation observed (coefficient of variation exceeding 20\%) suggests that parameter tuning could yield significant performance improvements. Confidence in these findings is high given the direct correspondence between CSV data and plotted values. Future analysis should consider incorporating error bars representing variance across multiple experimental runs to strengthen statistical validity.

\begin{figure}[!htbp]
\centering
\includegraphics[width=0.95\linewidth]{images/MT2TE_Multi-Line_Charts/(d)_Run_time_bar_chart.png}
\caption{(d) Run time}
\label{fig:MT2TE_Multi_Line_Charts_d__Run_time_bar_chart}
\end{figure}

This bar chart presents the relationship between 4 module and 5 module for the MT2TE Multi-Line Charts experimental scenario. The x-axis displays 4 module values ranging from 58.526 to 1545.515, while the y-axis quantifies 5 module. The single-series visualization facilitates analysis of how the dependent variable responds to changes in the independent parameter setting.

Analysis of the plotted data reveals that 5 module ranges from 29.02 (at 4 module = 58.526) to 1236.34 (at 4 module = 1545.515), representing a span of 1207.32 units. The overall trend is increasing, with values rising from 29.02 at the initial setting to 1236.34 at the final setting. 

These results indicate that 4 module configuration meaningfully impacts 5 module in this experimental context. The substantial variation observed (coefficient of variation exceeding 20\%) suggests that parameter tuning could yield significant performance improvements. Confidence in these findings is high given the direct correspondence between CSV data and plotted values. Future analysis should consider incorporating error bars representing variance across multiple experimental runs to strengthen statistical validity.

\begin{figure}[!htbp]
\centering
\includegraphics[width=0.95\linewidth]{images/MT2TE_Multi-Line_Charts/(e)_Module_swapped_bar_chart.png}
\caption{(e) Module swapped}
\label{fig:MT2TE_Multi_Line_Charts_e__Module_swapped_bar_chart}
\end{figure}

This bar chart presents the relationship between 4 module and 4 module for the MT2TE Multi-Line Charts experimental scenario. The x-axis displays 4 module values ranging from 15 to 106, while the y-axis quantifies 4 module. The single-series visualization facilitates analysis of how the dependent variable responds to changes in the independent parameter setting.

Analysis of the plotted data reveals that 4 module ranges from 15.00 (at 4 module = 15) to 106.00 (at 4 module = 106), representing a span of 91.00 units. The overall trend is increasing, with values rising from 15.00 at the initial setting to 106.00 at the final setting. 

These results indicate that 4 module configuration meaningfully impacts 4 module in this experimental context. The substantial variation observed (coefficient of variation exceeding 20\%) suggests that parameter tuning could yield significant performance improvements. Confidence in these findings is high given the direct correspondence between CSV data and plotted values. Future analysis should consider incorporating error bars representing variance across multiple experimental runs to strengthen statistical validity.


\clearpage

\subsection{MT2TE Node Change}

\begin{figure}[!htbp]
\centering
\includegraphics[width=0.95\linewidth]{images/MT2TE_Node_Change/(a)_Travel_time_bar_chart.png}
\caption{(a) Travel time}
\label{fig:MT2TE_Node_Change_a__Travel_time_bar_chart}
\end{figure}

This bar chart presents the relationship between number of nodes and total travel time for the MT2TE Node Change experimental scenario. The x-axis displays number of nodes values ranging from 500 to 2000, while the y-axis quantifies total travel time. The single-series visualization facilitates analysis of how the dependent variable responds to changes in the independent parameter setting.

Analysis of the plotted data reveals that total travel time ranges from 3228.14 (at number of nodes = 500) to 13338.35 (at number of nodes = 2000), representing a span of 10110.21 units. The overall trend is increasing, with values rising from 3228.14 at the initial setting to 13338.35 at the final setting. 

These results indicate that number of nodes configuration meaningfully impacts total travel time in this experimental context. The substantial variation observed (coefficient of variation exceeding 20\%) suggests that parameter tuning could yield significant performance improvements. Confidence in these findings is high given the direct correspondence between CSV data and plotted values. Future analysis should consider incorporating error bars representing variance across multiple experimental runs to strengthen statistical validity.

\begin{figure}[!htbp]
\centering
\includegraphics[width=0.95\linewidth]{images/MT2TE_Node_Change/(b)_Energy_consumed_bar_chart.png}
\caption{(b) Energy consumed}
\label{fig:MT2TE_Node_Change_b__Energy_consumed_bar_chart}
\end{figure}

This bar chart presents the relationship between number of nodes and total energy consumed for the MT2TE Node Change experimental scenario. The x-axis displays number of nodes values ranging from 500 to 2000, while the y-axis quantifies total energy consumed. The single-series visualization facilitates analysis of how the dependent variable responds to changes in the independent parameter setting.

Analysis of the plotted data reveals that total energy consumed ranges from 317.62 (at number of nodes = 500) to 2036.78 (at number of nodes = 2000), representing a span of 1719.17 units. The overall trend is increasing, with values rising from 317.62 at the initial setting to 2036.78 at the final setting. 

These results indicate that number of nodes configuration meaningfully impacts total energy consumed in this experimental context. The substantial variation observed (coefficient of variation exceeding 20\%) suggests that parameter tuning could yield significant performance improvements. Confidence in these findings is high given the direct correspondence between CSV data and plotted values. Future analysis should consider incorporating error bars representing variance across multiple experimental runs to strengthen statistical validity.

\begin{figure}[!htbp]
\centering
\includegraphics[width=0.95\linewidth]{images/MT2TE_Node_Change/(c)_Distance_covered_bar_chart.png}
\caption{(c) Distance covered}
\label{fig:MT2TE_Node_Change_c__Distance_covered_bar_chart}
\end{figure}

This bar chart presents the relationship between number of nodes and total distance covered for the MT2TE Node Change experimental scenario. The x-axis displays number of nodes values ranging from 500 to 2000, while the y-axis quantifies total distance covered. The single-series visualization facilitates analysis of how the dependent variable responds to changes in the independent parameter setting.

Analysis of the plotted data reveals that total distance covered ranges from 2105.32 (at number of nodes = 500) to 8672.23 (at number of nodes = 2000), representing a span of 6566.90 units. The overall trend is increasing, with values rising from 2105.32 at the initial setting to 8672.23 at the final setting. 

These results indicate that number of nodes configuration meaningfully impacts total distance covered in this experimental context. The substantial variation observed (coefficient of variation exceeding 20\%) suggests that parameter tuning could yield significant performance improvements. Confidence in these findings is high given the direct correspondence between CSV data and plotted values. Future analysis should consider incorporating error bars representing variance across multiple experimental runs to strengthen statistical validity.

\begin{figure}[!htbp]
\centering
\includegraphics[width=0.95\linewidth]{images/MT2TE_Node_Change/(d)_Run_time_bar_chart.png}
\caption{(d) Run time}
\label{fig:MT2TE_Node_Change_d__Run_time_bar_chart}
\end{figure}

This bar chart presents the relationship between number of nodes and run time for the MT2TE Node Change experimental scenario. The x-axis displays number of nodes values ranging from 500 to 2000, while the y-axis quantifies run time. The single-series visualization facilitates analysis of how the dependent variable responds to changes in the independent parameter setting.

Analysis of the plotted data reveals that run time ranges from 42.87 (at number of nodes = 500) to 1257.88 (at number of nodes = 2000), representing a span of 1215.01 units. The overall trend is increasing, with values rising from 42.87 at the initial setting to 1257.88 at the final setting. 

These results indicate that number of nodes configuration meaningfully impacts run time in this experimental context. The substantial variation observed (coefficient of variation exceeding 20\%) suggests that parameter tuning could yield significant performance improvements. Confidence in these findings is high given the direct correspondence between CSV data and plotted values. Future analysis should consider incorporating error bars representing variance across multiple experimental runs to strengthen statistical validity.

\begin{figure}[!htbp]
\centering
\includegraphics[width=0.95\linewidth]{images/MT2TE_Node_Change/(e)_Module_swapped_bar_chart.png}
\caption{(e) Module swapped}
\label{fig:MT2TE_Node_Change_e__Module_swapped_bar_chart}
\end{figure}

This bar chart presents the relationship between number of nodes and total module swapped for the MT2TE Node Change experimental scenario. The x-axis displays number of nodes values ranging from 500 to 2000, while the y-axis quantifies total module swapped. The single-series visualization facilitates analysis of how the dependent variable responds to changes in the independent parameter setting.

Analysis of the plotted data reveals that total module swapped ranges from 14.50 (at number of nodes = 500) to 98.75 (at number of nodes = 2000), representing a span of 84.25 units. The overall trend is increasing, with values rising from 14.50 at the initial setting to 98.75 at the final setting. 

These results indicate that number of nodes configuration meaningfully impacts total module swapped in this experimental context. The substantial variation observed (coefficient of variation exceeding 20\%) suggests that parameter tuning could yield significant performance improvements. Confidence in these findings is high given the direct correspondence between CSV data and plotted values. Future analysis should consider incorporating error bars representing variance across multiple experimental runs to strengthen statistical validity.


\clearpage

\subsection{Optimum and MT2TE Module Change}

\begin{figure}[!htbp]
\centering
\includegraphics[width=0.95\linewidth]{images/Optimum_and_MT2TE_Module_Change/(a)_Travel_Time.png}
\caption{(a) Travel Time}
\label{fig:Optimum_and_MT2TE_Module_Change_a__Travel_Time}
\end{figure}

This bar chart presents the relationship between module and optimum for the Optimum and MT2TE Module Change experimental scenario. The x-axis displays module values ranging from 3 to 6, while the y-axis quantifies optimum. The single-series visualization facilitates analysis of how the dependent variable responds to changes in the independent parameter setting.

Analysis of the plotted data reveals that optimum ranges from 111.20 (at module = 3) to 111.20 (at module = 3), representing a span of 0.00 units. The values remain relatively stable across the parameter range, with minimal net change between initial (111.20) and final (111.20) settings. 

These results indicate that module configuration meaningfully impacts optimum in this experimental context. The relatively modest variation suggests that this parameter has limited influence on the measured metric within the tested range. Confidence in these findings is high given the direct correspondence between CSV data and plotted values. Future analysis should consider incorporating error bars representing variance across multiple experimental runs to strengthen statistical validity.

\begin{figure}[!htbp]
\centering
\includegraphics[width=0.95\linewidth]{images/Optimum_and_MT2TE_Module_Change/(b)_Energy.png}
\caption{(b) Energy}
\label{fig:Optimum_and_MT2TE_Module_Change_b__Energy}
\end{figure}

This bar chart presents the relationship between module and optimum for the Optimum and MT2TE Module Change experimental scenario. The x-axis displays module values ranging from 3 to 6, while the y-axis quantifies optimum. The single-series visualization facilitates analysis of how the dependent variable responds to changes in the independent parameter setting.

Analysis of the plotted data reveals that optimum ranges from 9.01 (at module = 3) to 9.01 (at module = 3), representing a span of 0.00 units. The values remain relatively stable across the parameter range, with minimal net change between initial (9.01) and final (9.01) settings. 

These results indicate that module configuration meaningfully impacts optimum in this experimental context. The relatively modest variation suggests that this parameter has limited influence on the measured metric within the tested range. Confidence in these findings is high given the direct correspondence between CSV data and plotted values. Future analysis should consider incorporating error bars representing variance across multiple experimental runs to strengthen statistical validity.

\begin{figure}[!htbp]
\centering
\includegraphics[width=0.95\linewidth]{images/Optimum_and_MT2TE_Module_Change/(c)_Distance.png}
\caption{(c) Distance}
\label{fig:Optimum_and_MT2TE_Module_Change_c__Distance}
\end{figure}

This bar chart presents the relationship between module and optimum for the Optimum and MT2TE Module Change experimental scenario. The x-axis displays module values ranging from 3 to 6, while the y-axis quantifies optimum. The single-series visualization facilitates analysis of how the dependent variable responds to changes in the independent parameter setting.

Analysis of the plotted data reveals that optimum ranges from 74.38 (at module = 3) to 74.38 (at module = 3), representing a span of 0.00 units. The values remain relatively stable across the parameter range, with minimal net change between initial (74.38) and final (74.38) settings. 

These results indicate that module configuration meaningfully impacts optimum in this experimental context. The relatively modest variation suggests that this parameter has limited influence on the measured metric within the tested range. Confidence in these findings is high given the direct correspondence between CSV data and plotted values. Future analysis should consider incorporating error bars representing variance across multiple experimental runs to strengthen statistical validity.


\clearpage

\subsection{Optimum and MT2TE Node Change}

\begin{figure}[!htbp]
\centering
\includegraphics[width=0.95\linewidth]{images/Optimum_and_MT2TE_Node_Change/(a)_Total_Travel_Time_Optimum_and_MT2TE_Node_Change.png}
\caption{(a) Total Travel Time Optimum and MT2TE Node Change}
\label{fig:Optimum_and_MT2TE_Node_Change_a__Total_Travel_Time_Optimum_and_MT2TE_Node_Change}
\end{figure}

This bar chart presents the relationship between node and optimum for the Optimum and MT2TE Node Change experimental scenario. The x-axis displays node values ranging from 12 to 24, while the y-axis quantifies optimum. The single-series visualization facilitates analysis of how the dependent variable responds to changes in the independent parameter setting.

Analysis of the plotted data reveals that optimum ranges from 106.61 (at node = 12) to 153.26 (at node = 24), representing a span of 46.65 units. The overall trend is increasing, with values rising from 106.61 at the initial setting to 153.26 at the final setting. 

These results indicate that node configuration meaningfully impacts optimum in this experimental context. The substantial variation observed (coefficient of variation exceeding 20\%) suggests that parameter tuning could yield significant performance improvements. Confidence in these findings is high given the direct correspondence between CSV data and plotted values. Future analysis should consider incorporating error bars representing variance across multiple experimental runs to strengthen statistical validity.

\begin{figure}[!htbp]
\centering
\includegraphics[width=0.95\linewidth]{images/Optimum_and_MT2TE_Node_Change/(b)_Total_Energy_Consumed_Optimum_and_MT2TE_Node_Change.png}
\caption{(b) Total Energy Consumed Optimum and MT2TE Node Change}
\label{fig:Optimum_and_MT2TE_Node_Change_b__Total_Energy_Consumed_Optimum_and_MT2TE_Node_Change}
\end{figure}

This bar chart presents the relationship between node and optimum for the Optimum and MT2TE Node Change experimental scenario. The x-axis displays node values ranging from 12 to 24, while the y-axis quantifies optimum. The single-series visualization facilitates analysis of how the dependent variable responds to changes in the independent parameter setting.

Analysis of the plotted data reveals that optimum ranges from 8.72 (at node = 12) to 11.84 (at node = 24), representing a span of 3.12 units. The overall trend is increasing, with values rising from 8.72 at the initial setting to 11.84 at the final setting. 

These results indicate that node configuration meaningfully impacts optimum in this experimental context. The substantial variation observed (coefficient of variation exceeding 20\%) suggests that parameter tuning could yield significant performance improvements. Confidence in these findings is high given the direct correspondence between CSV data and plotted values. Future analysis should consider incorporating error bars representing variance across multiple experimental runs to strengthen statistical validity.

\begin{figure}[!htbp]
\centering
\includegraphics[width=0.95\linewidth]{images/Optimum_and_MT2TE_Node_Change/(c)_Total_Distance_Optimum_and_MT2TE_Node_Change.png}
\caption{(c) Total Distance Optimum and MT2TE Node Change}
\label{fig:Optimum_and_MT2TE_Node_Change_c__Total_Distance_Optimum_and_MT2TE_Node_Change}
\end{figure}

This bar chart presents the relationship between node and optimum for the Optimum and MT2TE Node Change experimental scenario. The x-axis displays node values ranging from 12 to 24, while the y-axis quantifies optimum. The single-series visualization facilitates analysis of how the dependent variable responds to changes in the independent parameter setting.

Analysis of the plotted data reveals that optimum ranges from 65.57 (at node = 12) to 98.75 (at node = 24), representing a span of 33.18 units. The overall trend is increasing, with values rising from 65.57 at the initial setting to 98.75 at the final setting. 

These results indicate that node configuration meaningfully impacts optimum in this experimental context. The substantial variation observed (coefficient of variation exceeding 20\%) suggests that parameter tuning could yield significant performance improvements. Confidence in these findings is high given the direct correspondence between CSV data and plotted values. Future analysis should consider incorporating error bars representing variance across multiple experimental runs to strengthen statistical validity.

\begin{figure}[!htbp]
\centering
\includegraphics[width=0.95\linewidth]{images/Optimum_and_MT2TE_Node_Change/(d)_Execution_Time_Optimum_and_MT2TE_Node_Change.png}
\caption{(d) Execution Time Optimum and MT2TE Node Change}
\label{fig:Optimum_and_MT2TE_Node_Change_d__Execution_Time_Optimum_and_MT2TE_Node_Change}
\end{figure}

This bar chart presents the relationship between node and optimum for the Optimum and MT2TE Node Change experimental scenario. The x-axis displays node values ranging from 12 to 28, while the y-axis quantifies optimum. The single-series visualization facilitates analysis of how the dependent variable responds to changes in the independent parameter setting.

Analysis of the plotted data reveals that optimum ranges from 12.00 (at node = 12) to 298800.00 (at node = 28), representing a span of 298788.00 units. The overall trend is increasing, with values rising from 12.00 at the initial setting to 298800.00 at the final setting. 

These results indicate that node configuration meaningfully impacts optimum in this experimental context. The substantial variation observed (coefficient of variation exceeding 20\%) suggests that parameter tuning could yield significant performance improvements. Confidence in these findings is high given the direct correspondence between CSV data and plotted values. Future analysis should consider incorporating error bars representing variance across multiple experimental runs to strengthen statistical validity.


\clearpage

\subsection{Optimum and MT2TE Swap Time Change}

\begin{figure}[!htbp]
\centering
\includegraphics[width=0.95\linewidth]{images/Optimum_and_MT2TE_Swap_Time_Change/(a)_Travel_Time.png}
\caption{(a) Travel Time}
\label{fig:Optimum_and_MT2TE_Swap_Time_Change_a__Travel_Time}
\end{figure}

This bar chart presents the relationship between swaptime and optimum for the Optimum and MT2TE Swap Time Change experimental scenario. The x-axis displays swaptime values ranging from 1 to 4, while the y-axis quantifies optimum. The single-series visualization facilitates analysis of how the dependent variable responds to changes in the independent parameter setting.

Analysis of the plotted data reveals that optimum ranges from 111.20 (at swaptime = 1) to 111.20 (at swaptime = 2), representing a span of 0.00 units. The overall trend is increasing, with values rising from 111.20 at the initial setting to 111.20 at the final setting. 

These results indicate that swaptime configuration meaningfully impacts optimum in this experimental context. The relatively modest variation suggests that this parameter has limited influence on the measured metric within the tested range. Confidence in these findings is high given the direct correspondence between CSV data and plotted values. Future analysis should consider incorporating error bars representing variance across multiple experimental runs to strengthen statistical validity.

\begin{figure}[!htbp]
\centering
\includegraphics[width=0.95\linewidth]{images/Optimum_and_MT2TE_Swap_Time_Change/(b)_Energy.png}
\caption{(b) Energy}
\label{fig:Optimum_and_MT2TE_Swap_Time_Change_b__Energy}
\end{figure}

This bar chart presents the relationship between swaptime and optimum for the Optimum and MT2TE Swap Time Change experimental scenario. The x-axis displays swaptime values ranging from 1 to 4, while the y-axis quantifies optimum. The single-series visualization facilitates analysis of how the dependent variable responds to changes in the independent parameter setting.

Analysis of the plotted data reveals that optimum ranges from 9.01 (at swaptime = 1) to 9.01 (at swaptime = 1), representing a span of 0.00 units. The values remain relatively stable across the parameter range, with minimal net change between initial (9.01) and final (9.01) settings. 

These results indicate that swaptime configuration meaningfully impacts optimum in this experimental context. The relatively modest variation suggests that this parameter has limited influence on the measured metric within the tested range. Confidence in these findings is high given the direct correspondence between CSV data and plotted values. Future analysis should consider incorporating error bars representing variance across multiple experimental runs to strengthen statistical validity.

\begin{figure}[!htbp]
\centering
\includegraphics[width=0.95\linewidth]{images/Optimum_and_MT2TE_Swap_Time_Change/(c)_Distance.png}
\caption{(c) Distance}
\label{fig:Optimum_and_MT2TE_Swap_Time_Change_c__Distance}
\end{figure}

This bar chart presents the relationship between swaptime and optimum for the Optimum and MT2TE Swap Time Change experimental scenario. The x-axis displays swaptime values ranging from 1 to 4, while the y-axis quantifies optimum. The single-series visualization facilitates analysis of how the dependent variable responds to changes in the independent parameter setting.

Analysis of the plotted data reveals that optimum ranges from 74.38 (at swaptime = 1) to 74.38 (at swaptime = 1), representing a span of 0.00 units. The values remain relatively stable across the parameter range, with minimal net change between initial (74.38) and final (74.38) settings. 

These results indicate that swaptime configuration meaningfully impacts optimum in this experimental context. The relatively modest variation suggests that this parameter has limited influence on the measured metric within the tested range. Confidence in these findings is high given the direct correspondence between CSV data and plotted values. Future analysis should consider incorporating error bars representing variance across multiple experimental runs to strengthen statistical validity.


\clearpage

\subsection{Optimum and MT2TE Threshold Change}

\begin{figure}[!htbp]
\centering
\includegraphics[width=0.95\linewidth]{images/Optimum_and_MT2TE_Threshold_Change/(a)_Travel_Time.png}
\caption{(a) Travel Time}
\label{fig:Optimum_and_MT2TE_Threshold_Change_a__Travel_Time}
\end{figure}

This bar chart presents the relationship between threshold and optimum for the Optimum and MT2TE Threshold Change experimental scenario. The x-axis displays threshold values ranging from 5 to 20, while the y-axis quantifies optimum. The single-series visualization facilitates analysis of how the dependent variable responds to changes in the independent parameter setting.

Analysis of the plotted data reveals that optimum ranges from 111.20 (at threshold = 5) to 111.20 (at threshold = 5), representing a span of 0.00 units. The values remain relatively stable across the parameter range, with minimal net change between initial (111.20) and final (111.20) settings. 

These results indicate that threshold configuration meaningfully impacts optimum in this experimental context. The relatively modest variation suggests that this parameter has limited influence on the measured metric within the tested range. Confidence in these findings is high given the direct correspondence between CSV data and plotted values. Future analysis should consider incorporating error bars representing variance across multiple experimental runs to strengthen statistical validity.

\begin{figure}[!htbp]
\centering
\includegraphics[width=0.95\linewidth]{images/Optimum_and_MT2TE_Threshold_Change/(b)_Energy.png}
\caption{(b) Energy}
\label{fig:Optimum_and_MT2TE_Threshold_Change_b__Energy}
\end{figure}

This bar chart presents the relationship between threshold and optimum for the Optimum and MT2TE Threshold Change experimental scenario. The x-axis displays threshold values ranging from 5 to 20, while the y-axis quantifies optimum. The single-series visualization facilitates analysis of how the dependent variable responds to changes in the independent parameter setting.

Analysis of the plotted data reveals that optimum ranges from 9.01 (at threshold = 5) to 9.01 (at threshold = 5), representing a span of 0.00 units. The values remain relatively stable across the parameter range, with minimal net change between initial (9.01) and final (9.01) settings. 

These results indicate that threshold configuration meaningfully impacts optimum in this experimental context. The relatively modest variation suggests that this parameter has limited influence on the measured metric within the tested range. Confidence in these findings is high given the direct correspondence between CSV data and plotted values. Future analysis should consider incorporating error bars representing variance across multiple experimental runs to strengthen statistical validity.

\begin{figure}[!htbp]
\centering
\includegraphics[width=0.95\linewidth]{images/Optimum_and_MT2TE_Threshold_Change/(c)_Distance.png}
\caption{(c) Distance}
\label{fig:Optimum_and_MT2TE_Threshold_Change_c__Distance}
\end{figure}

This bar chart presents the relationship between threshold and optimum for the Optimum and MT2TE Threshold Change experimental scenario. The x-axis displays threshold values ranging from 5 to 20, while the y-axis quantifies optimum. The single-series visualization facilitates analysis of how the dependent variable responds to changes in the independent parameter setting.

Analysis of the plotted data reveals that optimum ranges from 74.38 (at threshold = 5) to 74.38 (at threshold = 5), representing a span of 0.00 units. The values remain relatively stable across the parameter range, with minimal net change between initial (74.38) and final (74.38) settings. 

These results indicate that threshold configuration meaningfully impacts optimum in this experimental context. The relatively modest variation suggests that this parameter has limited influence on the measured metric within the tested range. Confidence in these findings is high given the direct correspondence between CSV data and plotted values. Future analysis should consider incorporating error bars representing variance across multiple experimental runs to strengthen statistical validity.


\clearpage

\subsection{Optimum and MT2TE Traffic Change}

\begin{figure}[!htbp]
\centering
\includegraphics[width=0.95\linewidth]{images/Optimum_and_MT2TE_Traffic_Change/(a)_Travel_Time.png}
\caption{(a) Travel Time}
\label{fig:Optimum_and_MT2TE_Traffic_Change_a__Travel_Time}
\end{figure}

This bar chart presents the relationship between trafficlevel and optimum for the Optimum and MT2TE Traffic Change experimental scenario. The x-axis displays trafficlevel values ranging from Low to High, while the y-axis quantifies optimum. The single-series visualization facilitates analysis of how the dependent variable responds to changes in the independent parameter setting.

Analysis of the plotted data reveals that optimum ranges from 94.72 (at trafficlevel = Low) to 133.80 (at trafficlevel = High), representing a span of 39.08 units. The overall trend is increasing, with values rising from 94.72 at the initial setting to 133.80 at the final setting. 

These results indicate that trafficlevel configuration meaningfully impacts optimum in this experimental context. The substantial variation observed (coefficient of variation exceeding 20\%) suggests that parameter tuning could yield significant performance improvements. Confidence in these findings is high given the direct correspondence between CSV data and plotted values. Future analysis should consider incorporating error bars representing variance across multiple experimental runs to strengthen statistical validity.

\begin{figure}[!htbp]
\centering
\includegraphics[width=0.95\linewidth]{images/Optimum_and_MT2TE_Traffic_Change/(b)_Energy.png}
\caption{(b) Energy}
\label{fig:Optimum_and_MT2TE_Traffic_Change_b__Energy}
\end{figure}

This bar chart presents the relationship between trafficlevel and optimum for the Optimum and MT2TE Traffic Change experimental scenario. The x-axis displays trafficlevel values ranging from Low to High, while the y-axis quantifies optimum. The single-series visualization facilitates analysis of how the dependent variable responds to changes in the independent parameter setting.

Analysis of the plotted data reveals that optimum ranges from 7.68 (at trafficlevel = Low) to 11.09 (at trafficlevel = High), representing a span of 3.41 units. The overall trend is increasing, with values rising from 7.68 at the initial setting to 11.09 at the final setting. 

These results indicate that trafficlevel configuration meaningfully impacts optimum in this experimental context. The substantial variation observed (coefficient of variation exceeding 20\%) suggests that parameter tuning could yield significant performance improvements. Confidence in these findings is high given the direct correspondence between CSV data and plotted values. Future analysis should consider incorporating error bars representing variance across multiple experimental runs to strengthen statistical validity.

\begin{figure}[!htbp]
\centering
\includegraphics[width=0.95\linewidth]{images/Optimum_and_MT2TE_Traffic_Change/(c)_Distance.png}
\caption{(c) Distance}
\label{fig:Optimum_and_MT2TE_Traffic_Change_c__Distance}
\end{figure}

This bar chart presents the relationship between traffic and total distance covered for the Optimum and MT2TE Traffic Change experimental scenario. The x-axis displays traffic values ranging from High to Low, while the y-axis quantifies total distance covered. The single-series visualization facilitates analysis of how the dependent variable responds to changes in the independent parameter setting.

Analysis of the plotted data reveals that total distance covered ranges from 8689.81 (at traffic = High) to 8994.35 (at traffic = Low), representing a span of 304.54 units. The overall trend is increasing, with values rising from 8689.81 at the initial setting to 8994.35 at the final setting. 

These results indicate that traffic configuration meaningfully impacts total distance covered in this experimental context. The relatively modest variation suggests that this parameter has limited influence on the measured metric within the tested range. Confidence in these findings is high given the direct correspondence between CSV data and plotted values. Future analysis should consider incorporating error bars representing variance across multiple experimental runs to strengthen statistical validity.


\clearpage

\subsection{Real Time Sumo Module Change}

\begin{figure}[!htbp]
\centering
\includegraphics[width=0.95\linewidth]{images/Real_Time_Sumo_Module_Change/(a)_Travel_Time.png}
\caption{(a) Travel Time}
\label{fig:Real_Time_Sumo_Module_Change_a__Travel_Time}
\end{figure}

This bar chart presents the relationship between modules and total travel time for the Real Time Sumo Module Change experimental scenario. The x-axis displays modules values ranging from 3 to 6, while the y-axis quantifies total travel time. The single-series visualization facilitates analysis of how the dependent variable responds to changes in the independent parameter setting.

Analysis of the plotted data reveals that total travel time ranges from 87.96 (at modules = 3) to 90.06 (at modules = 4), representing a span of 2.10 units. The overall trend is increasing, with values rising from 87.96 at the initial setting to 88.79 at the final setting. 

These results indicate that modules configuration meaningfully impacts total travel time in this experimental context. The relatively modest variation suggests that this parameter has limited influence on the measured metric within the tested range. Confidence in these findings is high given the direct correspondence between CSV data and plotted values. Future analysis should consider incorporating error bars representing variance across multiple experimental runs to strengthen statistical validity.

\begin{figure}[!htbp]
\centering
\includegraphics[width=0.95\linewidth]{images/Real_Time_Sumo_Module_Change/(b)_energy.png}
\caption{(b) energy}
\label{fig:Real_Time_Sumo_Module_Change_b__energy}
\end{figure}

This bar chart presents the relationship between modules and total energy consumed for the Real Time Sumo Module Change experimental scenario. The x-axis displays modules values ranging from 3 to 6, while the y-axis quantifies total energy consumed. The single-series visualization facilitates analysis of how the dependent variable responds to changes in the independent parameter setting.

Analysis of the plotted data reveals that total energy consumed ranges from 12.54 (at modules = 3) to 13.11 (at modules = 4), representing a span of 0.58 units. The overall trend is increasing, with values rising from 12.54 at the initial setting to 12.84 at the final setting. 

These results indicate that modules configuration meaningfully impacts total energy consumed in this experimental context. The relatively modest variation suggests that this parameter has limited influence on the measured metric within the tested range. Confidence in these findings is high given the direct correspondence between CSV data and plotted values. Future analysis should consider incorporating error bars representing variance across multiple experimental runs to strengthen statistical validity.

\begin{figure}[!htbp]
\centering
\includegraphics[width=0.95\linewidth]{images/Real_Time_Sumo_Module_Change/(c)_Distance.png}
\caption{(c) Distance}
\label{fig:Real_Time_Sumo_Module_Change_c__Distance}
\end{figure}

This bar chart presents the relationship between modules and total distance covered for the Real Time Sumo Module Change experimental scenario. The x-axis displays modules values ranging from 3 to 6, while the y-axis quantifies total distance covered. The single-series visualization facilitates analysis of how the dependent variable responds to changes in the independent parameter setting.

Analysis of the plotted data reveals that total distance covered ranges from 60.75 (at modules = 3) to 62.76 (at modules = 4), representing a span of 2.01 units. The overall trend is increasing, with values rising from 60.75 at the initial setting to 62.00 at the final setting. 

These results indicate that modules configuration meaningfully impacts total distance covered in this experimental context. The relatively modest variation suggests that this parameter has limited influence on the measured metric within the tested range. Confidence in these findings is high given the direct correspondence between CSV data and plotted values. Future analysis should consider incorporating error bars representing variance across multiple experimental runs to strengthen statistical validity.

\begin{figure}[!htbp]
\centering
\includegraphics[width=0.95\linewidth]{images/Real_Time_Sumo_Module_Change/(d)_Run_time.png}
\caption{(d) Run time}
\label{fig:Real_Time_Sumo_Module_Change_d__Run_time}
\end{figure}

This bar chart presents the relationship between modules and total travel time for the Real Time Sumo Module Change experimental scenario. The x-axis displays modules values ranging from 3 to 6, while the y-axis quantifies total travel time. The single-series visualization facilitates analysis of how the dependent variable responds to changes in the independent parameter setting.

Analysis of the plotted data reveals that total travel time ranges from 87.96 (at modules = 3) to 90.06 (at modules = 4), representing a span of 2.10 units. The overall trend is increasing, with values rising from 87.96 at the initial setting to 88.79 at the final setting. 

These results indicate that modules configuration meaningfully impacts total travel time in this experimental context. The relatively modest variation suggests that this parameter has limited influence on the measured metric within the tested range. Confidence in these findings is high given the direct correspondence between CSV data and plotted values. Future analysis should consider incorporating error bars representing variance across multiple experimental runs to strengthen statistical validity.


\clearpage

\subsection{Real Time Sumo Swap Time}

\begin{figure}[!htbp]
\centering
\includegraphics[width=0.95\linewidth]{images/Real_Time_Sumo_Swap_Time/(a)_Travel_Time.png}
\caption{(a) Travel Time}
\label{fig:Real_Time_Sumo_Swap_Time_a__Travel_Time}
\end{figure}

This bar chart presents the relationship between swap time (min) and total travel time for the Real Time Sumo Swap Time experimental scenario. The x-axis displays swap time (min) values ranging from 1 to 4, while the y-axis quantifies total travel time. The single-series visualization facilitates analysis of how the dependent variable responds to changes in the independent parameter setting.

Analysis of the plotted data reveals that total travel time ranges from 89.22 (at swap time (min) = 2) to 92.84 (at swap time (min) = 4), representing a span of 3.62 units. The overall trend is increasing, with values rising from 89.94 at the initial setting to 92.84 at the final setting. Notably, the minimum value occurs at an intermediate swap time (min) setting (2), suggesting non-monotonic behavior that warrants further investigation.

These results indicate that swap time (min) configuration meaningfully impacts total travel time in this experimental context. The relatively modest variation suggests that this parameter has limited influence on the measured metric within the tested range. Confidence in these findings is high given the direct correspondence between CSV data and plotted values. Future analysis should consider incorporating error bars representing variance across multiple experimental runs to strengthen statistical validity.

\begin{figure}[!htbp]
\centering
\includegraphics[width=0.95\linewidth]{images/Real_Time_Sumo_Swap_Time/(b)_energy.png}
\caption{(b) energy}
\label{fig:Real_Time_Sumo_Swap_Time_b__energy}
\end{figure}

This bar chart presents the relationship between swap time (min) and total energy consumed for the Real Time Sumo Swap Time experimental scenario. The x-axis displays swap time (min) values ranging from 1 to 4, while the y-axis quantifies total energy consumed. The single-series visualization facilitates analysis of how the dependent variable responds to changes in the independent parameter setting.

Analysis of the plotted data reveals that total energy consumed ranges from 12.57 (at swap time (min) = 3) to 13.18 (at swap time (min) = 4), representing a span of 0.60 units. The overall trend is increasing, with values rising from 12.92 at the initial setting to 13.18 at the final setting. Notably, the minimum value occurs at an intermediate swap time (min) setting (3), suggesting non-monotonic behavior that warrants further investigation.

These results indicate that swap time (min) configuration meaningfully impacts total energy consumed in this experimental context. The relatively modest variation suggests that this parameter has limited influence on the measured metric within the tested range. Confidence in these findings is high given the direct correspondence between CSV data and plotted values. Future analysis should consider incorporating error bars representing variance across multiple experimental runs to strengthen statistical validity.

\begin{figure}[!htbp]
\centering
\includegraphics[width=0.95\linewidth]{images/Real_Time_Sumo_Swap_Time/(c)_Distance.png}
\caption{(c) Distance}
\label{fig:Real_Time_Sumo_Swap_Time_c__Distance}
\end{figure}

This bar chart presents the relationship between swap time (min) and total distance covered for the Real Time Sumo Swap Time experimental scenario. The x-axis displays swap time (min) values ranging from 1 to 4, while the y-axis quantifies total distance covered. The single-series visualization facilitates analysis of how the dependent variable responds to changes in the independent parameter setting.

Analysis of the plotted data reveals that total distance covered ranges from 61.23 (at swap time (min) = 3) to 64.06 (at swap time (min) = 4), representing a span of 2.83 units. The overall trend is increasing, with values rising from 62.42 at the initial setting to 64.06 at the final setting. Notably, the minimum value occurs at an intermediate swap time (min) setting (3), suggesting non-monotonic behavior that warrants further investigation.

These results indicate that swap time (min) configuration meaningfully impacts total distance covered in this experimental context. The relatively modest variation suggests that this parameter has limited influence on the measured metric within the tested range. Confidence in these findings is high given the direct correspondence between CSV data and plotted values. Future analysis should consider incorporating error bars representing variance across multiple experimental runs to strengthen statistical validity.

\begin{figure}[!htbp]
\centering
\includegraphics[width=0.95\linewidth]{images/Real_Time_Sumo_Swap_Time/(d)_Runtime.png}
\caption{(d) Runtime}
\label{fig:Real_Time_Sumo_Swap_Time_d__Runtime}
\end{figure}

This bar chart presents the relationship between swap time (min) and run time for the Real Time Sumo Swap Time experimental scenario. The x-axis displays swap time (min) values ranging from 1 to 4, while the y-axis quantifies run time. The single-series visualization facilitates analysis of how the dependent variable responds to changes in the independent parameter setting.

Analysis of the plotted data reveals that run time ranges from 540.79 (at swap time (min) = 1) to 563.35 (at swap time (min) = 4), representing a span of 22.55 units. The overall trend is increasing, with values rising from 540.79 at the initial setting to 563.35 at the final setting. 

These results indicate that swap time (min) configuration meaningfully impacts run time in this experimental context. The relatively modest variation suggests that this parameter has limited influence on the measured metric within the tested range. Confidence in these findings is high given the direct correspondence between CSV data and plotted values. Future analysis should consider incorporating error bars representing variance across multiple experimental runs to strengthen statistical validity.


\clearpage

\subsection{Real Time Sumo Threshold}

\begin{figure}[!htbp]
\centering
\includegraphics[width=0.95\linewidth]{images/Real_Time_Sumo_Threshold/(a)_Travel_Time.png}
\caption{(a) Travel Time}
\label{fig:Real_Time_Sumo_Threshold_a__Travel_Time}
\end{figure}

This bar chart presents the relationship between threshold and total travel time for the Real Time Sumo Threshold experimental scenario. The x-axis displays threshold values ranging from 5 to 20, while the y-axis quantifies total travel time. The single-series visualization facilitates analysis of how the dependent variable responds to changes in the independent parameter setting.

Analysis of the plotted data reveals that total travel time ranges from 86.63 (at threshold = 5) to 92.25 (at threshold = 15), representing a span of 5.62 units. The overall trend is increasing, with values rising from 86.63 at the initial setting to 89.22 at the final setting. 

These results indicate that threshold configuration meaningfully impacts total travel time in this experimental context. The relatively modest variation suggests that this parameter has limited influence on the measured metric within the tested range. Confidence in these findings is high given the direct correspondence between CSV data and plotted values. Future analysis should consider incorporating error bars representing variance across multiple experimental runs to strengthen statistical validity.

\begin{figure}[!htbp]
\centering
\includegraphics[width=0.95\linewidth]{images/Real_Time_Sumo_Threshold/(b)_Energy.png}
\caption{(b) Energy}
\label{fig:Real_Time_Sumo_Threshold_b__Energy}
\end{figure}

This bar chart presents the relationship between threshold and total energy consumed for the Real Time Sumo Threshold experimental scenario. The x-axis displays threshold values ranging from 5 to 20, while the y-axis quantifies total energy consumed. The single-series visualization facilitates analysis of how the dependent variable responds to changes in the independent parameter setting.

Analysis of the plotted data reveals that total energy consumed ranges from 12.59 (at threshold = 5) to 13.17 (at threshold = 15), representing a span of 0.58 units. The overall trend is increasing, with values rising from 12.59 at the initial setting to 12.93 at the final setting. 

These results indicate that threshold configuration meaningfully impacts total energy consumed in this experimental context. The relatively modest variation suggests that this parameter has limited influence on the measured metric within the tested range. Confidence in these findings is high given the direct correspondence between CSV data and plotted values. Future analysis should consider incorporating error bars representing variance across multiple experimental runs to strengthen statistical validity.

\begin{figure}[!htbp]
\centering
\includegraphics[width=0.95\linewidth]{images/Real_Time_Sumo_Threshold/(c)_Distance.png}
\caption{(c) Distance}
\label{fig:Real_Time_Sumo_Threshold_c__Distance}
\end{figure}

This bar chart presents the relationship between threshold and total distance covered for the Real Time Sumo Threshold experimental scenario. The x-axis displays threshold values ranging from 5 to 20, while the y-axis quantifies total distance covered. The single-series visualization facilitates analysis of how the dependent variable responds to changes in the independent parameter setting.

Analysis of the plotted data reveals that total distance covered ranges from 60.06 (at threshold = 5) to 63.76 (at threshold = 15), representing a span of 3.70 units. The overall trend is increasing, with values rising from 60.06 at the initial setting to 61.70 at the final setting. 

These results indicate that threshold configuration meaningfully impacts total distance covered in this experimental context. The relatively modest variation suggests that this parameter has limited influence on the measured metric within the tested range. Confidence in these findings is high given the direct correspondence between CSV data and plotted values. Future analysis should consider incorporating error bars representing variance across multiple experimental runs to strengthen statistical validity.

\begin{figure}[!htbp]
\centering
\includegraphics[width=0.95\linewidth]{images/Real_Time_Sumo_Threshold/(d)_runtime.png}
\caption{(d) runtime}
\label{fig:Real_Time_Sumo_Threshold_d__runtime}
\end{figure}

This bar chart presents the relationship between threshold and run time for the Real Time Sumo Threshold experimental scenario. The x-axis displays threshold values ranging from 5 to 20, while the y-axis quantifies run time. The single-series visualization facilitates analysis of how the dependent variable responds to changes in the independent parameter setting.

Analysis of the plotted data reveals that run time ranges from 546.38 (at threshold = 10) to 568.65 (at threshold = 15), representing a span of 22.27 units. The overall trend is increasing, with values rising from 547.79 at the initial setting to 553.99 at the final setting. Notably, the minimum value occurs at an intermediate threshold setting (10), suggesting non-monotonic behavior that warrants further investigation.

These results indicate that threshold configuration meaningfully impacts run time in this experimental context. The relatively modest variation suggests that this parameter has limited influence on the measured metric within the tested range. Confidence in these findings is high given the direct correspondence between CSV data and plotted values. Future analysis should consider incorporating error bars representing variance across multiple experimental runs to strengthen statistical validity.


\clearpage

\subsection{Sumo Static Module Change}

\begin{figure}[!htbp]
\centering
\includegraphics[width=0.95\linewidth]{images/Sumo_Static_Module_Change/(a)_Travel_Time.png}
\caption{(a) Travel Time}
\label{fig:Sumo_Static_Module_Change_a__Travel_Time}
\end{figure}

This bar chart presents the relationship between modules and total travel time for the Sumo Static Module Change experimental scenario. The x-axis displays modules values ranging from 3 to 6, while the y-axis quantifies total travel time. The single-series visualization facilitates analysis of how the dependent variable responds to changes in the independent parameter setting.

Analysis of the plotted data reveals that total travel time ranges from 92.73 (at modules = 5) to 108.71 (at modules = 6), representing a span of 15.98 units. The overall trend is increasing, with values rising from 94.47 at the initial setting to 108.71 at the final setting. Notably, the minimum value occurs at an intermediate modules setting (5), suggesting non-monotonic behavior that warrants further investigation.

These results indicate that modules configuration meaningfully impacts total travel time in this experimental context. The relatively modest variation suggests that this parameter has limited influence on the measured metric within the tested range. Confidence in these findings is high given the direct correspondence between CSV data and plotted values. Future analysis should consider incorporating error bars representing variance across multiple experimental runs to strengthen statistical validity.

\begin{figure}[!htbp]
\centering
\includegraphics[width=0.95\linewidth]{images/Sumo_Static_Module_Change/(b)_Energy.png}
\caption{(b) Energy}
\label{fig:Sumo_Static_Module_Change_b__Energy}
\end{figure}

This bar chart presents the relationship between modules and total energy consumed for the Sumo Static Module Change experimental scenario. The x-axis displays modules values ranging from 3 to 6, while the y-axis quantifies total energy consumed. The single-series visualization facilitates analysis of how the dependent variable responds to changes in the independent parameter setting.

Analysis of the plotted data reveals that total energy consumed ranges from 9.05 (at modules = 5) to 11.61 (at modules = 6), representing a span of 2.55 units. The overall trend is increasing, with values rising from 9.29 at the initial setting to 11.61 at the final setting. Notably, the minimum value occurs at an intermediate modules setting (5), suggesting non-monotonic behavior that warrants further investigation.

These results indicate that modules configuration meaningfully impacts total energy consumed in this experimental context. The substantial variation observed (coefficient of variation exceeding 20\%) suggests that parameter tuning could yield significant performance improvements. Confidence in these findings is high given the direct correspondence between CSV data and plotted values. Future analysis should consider incorporating error bars representing variance across multiple experimental runs to strengthen statistical validity.

\begin{figure}[!htbp]
\centering
\includegraphics[width=0.95\linewidth]{images/Sumo_Static_Module_Change/(c)_Distance.png}
\caption{(c) Distance}
\label{fig:Sumo_Static_Module_Change_c__Distance}
\end{figure}

This bar chart presents the relationship between modules and total distance covered for the Sumo Static Module Change experimental scenario. The x-axis displays modules values ranging from 3 to 6, while the y-axis quantifies total distance covered. The single-series visualization facilitates analysis of how the dependent variable responds to changes in the independent parameter setting.

Analysis of the plotted data reveals that total distance covered ranges from 57.52 (at modules = 5) to 69.86 (at modules = 6), representing a span of 12.34 units. The overall trend is increasing, with values rising from 59.51 at the initial setting to 69.86 at the final setting. Notably, the minimum value occurs at an intermediate modules setting (5), suggesting non-monotonic behavior that warrants further investigation.

These results indicate that modules configuration meaningfully impacts total distance covered in this experimental context. The relatively modest variation suggests that this parameter has limited influence on the measured metric within the tested range. Confidence in these findings is high given the direct correspondence between CSV data and plotted values. Future analysis should consider incorporating error bars representing variance across multiple experimental runs to strengthen statistical validity.

\begin{figure}[!htbp]
\centering
\includegraphics[width=0.95\linewidth]{images/Sumo_Static_Module_Change/(d)_Module_Swapped.png}
\caption{(d) Module Swapped}
\label{fig:Sumo_Static_Module_Change_d__Module_Swapped}
\end{figure}

This bar chart presents the relationship between modules and modules for the Sumo Static Module Change experimental scenario. The x-axis displays modules values ranging from 3 to 6, while the y-axis quantifies modules. The single-series visualization facilitates analysis of how the dependent variable responds to changes in the independent parameter setting.

Analysis of the plotted data reveals that modules ranges from 3.00 (at modules = 3) to 6.00 (at modules = 6), representing a span of 3.00 units. The overall trend is increasing, with values rising from 3.00 at the initial setting to 6.00 at the final setting. 

These results indicate that modules configuration meaningfully impacts modules in this experimental context. The substantial variation observed (coefficient of variation exceeding 20\%) suggests that parameter tuning could yield significant performance improvements. Confidence in these findings is high given the direct correspondence between CSV data and plotted values. Future analysis should consider incorporating error bars representing variance across multiple experimental runs to strengthen statistical validity.

\begin{figure}[!htbp]
\centering
\includegraphics[width=0.95\linewidth]{images/Sumo_Static_Module_Change/(e)_Run_Time.png}
\caption{(e) Run Time}
\label{fig:Sumo_Static_Module_Change_e__Run_Time}
\end{figure}

This bar chart presents the relationship between modules and total travel time for the Sumo Static Module Change experimental scenario. The x-axis displays modules values ranging from 3 to 6, while the y-axis quantifies total travel time. The single-series visualization facilitates analysis of how the dependent variable responds to changes in the independent parameter setting.

Analysis of the plotted data reveals that total travel time ranges from 92.73 (at modules = 5) to 108.71 (at modules = 6), representing a span of 15.98 units. The overall trend is increasing, with values rising from 94.47 at the initial setting to 108.71 at the final setting. Notably, the minimum value occurs at an intermediate modules setting (5), suggesting non-monotonic behavior that warrants further investigation.

These results indicate that modules configuration meaningfully impacts total travel time in this experimental context. The relatively modest variation suggests that this parameter has limited influence on the measured metric within the tested range. Confidence in these findings is high given the direct correspondence between CSV data and plotted values. Future analysis should consider incorporating error bars representing variance across multiple experimental runs to strengthen statistical validity.


\clearpage

\subsection{Sumo Static Swap Time}

\begin{figure}[!htbp]
\centering
\includegraphics[width=0.95\linewidth]{images/Sumo_Static_Swap_Time/(a)_Travel_Time.png}
\caption{(a) Travel Time}
\label{fig:Sumo_Static_Swap_Time_a__Travel_Time}
\end{figure}

This bar chart presents the relationship between swap time (min) and total travel time for the Sumo Static Swap Time experimental scenario. The x-axis displays swap time (min) values ranging from 1 to 4, while the y-axis quantifies total travel time. The single-series visualization facilitates analysis of how the dependent variable responds to changes in the independent parameter setting.

Analysis of the plotted data reveals that total travel time ranges from 85.81 (at swap time (min) = 4) to 100.69 (at swap time (min) = 3), representing a span of 14.88 units. The overall trend is decreasing, with values declining from 100.19 at the initial setting to 85.81 at the final setting. 

These results indicate that swap time (min) configuration meaningfully impacts total travel time in this experimental context. The relatively modest variation suggests that this parameter has limited influence on the measured metric within the tested range. Confidence in these findings is high given the direct correspondence between CSV data and plotted values. Future analysis should consider incorporating error bars representing variance across multiple experimental runs to strengthen statistical validity.

\begin{figure}[!htbp]
\centering
\includegraphics[width=0.95\linewidth]{images/Sumo_Static_Swap_Time/(b)_Energy.png}
\caption{(b) Energy}
\label{fig:Sumo_Static_Swap_Time_b__Energy}
\end{figure}

This bar chart presents the relationship between swap time (min) and total energy consumed for the Sumo Static Swap Time experimental scenario. The x-axis displays swap time (min) values ranging from 1 to 4, while the y-axis quantifies total energy consumed. The single-series visualization facilitates analysis of how the dependent variable responds to changes in the independent parameter setting.

Analysis of the plotted data reveals that total energy consumed ranges from 9.05 (at swap time (min) = 2) to 11.51 (at swap time (min) = 3), representing a span of 2.46 units. The overall trend is decreasing, with values declining from 11.00 at the initial setting to 9.75 at the final setting. Notably, the minimum value occurs at an intermediate swap time (min) setting (2), suggesting non-monotonic behavior that warrants further investigation.

These results indicate that swap time (min) configuration meaningfully impacts total energy consumed in this experimental context. The substantial variation observed (coefficient of variation exceeding 20\%) suggests that parameter tuning could yield significant performance improvements. Confidence in these findings is high given the direct correspondence between CSV data and plotted values. Future analysis should consider incorporating error bars representing variance across multiple experimental runs to strengthen statistical validity.

\begin{figure}[!htbp]
\centering
\includegraphics[width=0.95\linewidth]{images/Sumo_Static_Swap_Time/(c)_Distance.png}
\caption{(c) Distance}
\label{fig:Sumo_Static_Swap_Time_c__Distance}
\end{figure}

This bar chart presents the relationship between swap time (min) and total distance covered for the Sumo Static Swap Time experimental scenario. The x-axis displays swap time (min) values ranging from 1 to 4, while the y-axis quantifies total distance covered. The single-series visualization facilitates analysis of how the dependent variable responds to changes in the independent parameter setting.

Analysis of the plotted data reveals that total distance covered ranges from 57.11 (at swap time (min) = 4) to 66.91 (at swap time (min) = 3), representing a span of 9.80 units. The overall trend is decreasing, with values declining from 64.69 at the initial setting to 57.11 at the final setting. 

These results indicate that swap time (min) configuration meaningfully impacts total distance covered in this experimental context. The relatively modest variation suggests that this parameter has limited influence on the measured metric within the tested range. Confidence in these findings is high given the direct correspondence between CSV data and plotted values. Future analysis should consider incorporating error bars representing variance across multiple experimental runs to strengthen statistical validity.

\begin{figure}[!htbp]
\centering
\includegraphics[width=0.95\linewidth]{images/Sumo_Static_Swap_Time/(d)_Module_Swapped.png}
\caption{(d) Module Swapped}
\label{fig:Sumo_Static_Swap_Time_d__Module_Swapped}
\end{figure}

This bar chart presents the relationship between swap time (min) and total module swapped for the Sumo Static Swap Time experimental scenario. The x-axis displays swap time (min) values ranging from 1 to 4, while the y-axis quantifies total module swapped. The single-series visualization facilitates analysis of how the dependent variable responds to changes in the independent parameter setting.

Analysis of the plotted data reveals that total module swapped ranges from 0.00 (at swap time (min) = 1) to 0.00 (at swap time (min) = 1), representing a span of 0.00 units. The values remain relatively stable across the parameter range, with minimal net change between initial (0.00) and final (0.00) settings. 

These results indicate that swap time (min) configuration meaningfully impacts total module swapped in this experimental context. The relatively modest variation suggests that this parameter has limited influence on the measured metric within the tested range. Confidence in these findings is high given the direct correspondence between CSV data and plotted values. Future analysis should consider incorporating error bars representing variance across multiple experimental runs to strengthen statistical validity.

\begin{figure}[!htbp]
\centering
\includegraphics[width=0.95\linewidth]{images/Sumo_Static_Swap_Time/(e)_Run_Time.png}
\caption{(e) Run Time}
\label{fig:Sumo_Static_Swap_Time_e__Run_Time}
\end{figure}

This bar chart presents the relationship between swap time (min) and total travel time for the Sumo Static Swap Time experimental scenario. The x-axis displays swap time (min) values ranging from 1 to 4, while the y-axis quantifies total travel time. The single-series visualization facilitates analysis of how the dependent variable responds to changes in the independent parameter setting.

Analysis of the plotted data reveals that total travel time ranges from 85.81 (at swap time (min) = 4) to 100.69 (at swap time (min) = 3), representing a span of 14.88 units. The overall trend is decreasing, with values declining from 100.19 at the initial setting to 85.81 at the final setting. 

These results indicate that swap time (min) configuration meaningfully impacts total travel time in this experimental context. The relatively modest variation suggests that this parameter has limited influence on the measured metric within the tested range. Confidence in these findings is high given the direct correspondence between CSV data and plotted values. Future analysis should consider incorporating error bars representing variance across multiple experimental runs to strengthen statistical validity.


\clearpage

\subsection{Sumo Static Threshold}

\begin{figure}[!htbp]
\centering
\includegraphics[width=0.95\linewidth]{images/Sumo_Static_Threshold/(a)_Travel_Time.png}
\caption{(a) Travel Time}
\label{fig:Sumo_Static_Threshold_a__Travel_Time}
\end{figure}

This bar chart presents the relationship between threshold and total travel time for the Sumo Static Threshold experimental scenario. The x-axis displays threshold values ranging from 5 to 20, while the y-axis quantifies total travel time. The single-series visualization facilitates analysis of how the dependent variable responds to changes in the independent parameter setting.

Analysis of the plotted data reveals that total travel time ranges from 86.54 (at threshold = 10) to 106.44 (at threshold = 5), representing a span of 19.90 units. The overall trend is decreasing, with values declining from 106.44 at the initial setting to 92.73 at the final setting. Notably, the minimum value occurs at an intermediate threshold setting (10), suggesting non-monotonic behavior that warrants further investigation.

These results indicate that threshold configuration meaningfully impacts total travel time in this experimental context. The substantial variation observed (coefficient of variation exceeding 20\%) suggests that parameter tuning could yield significant performance improvements. Confidence in these findings is high given the direct correspondence between CSV data and plotted values. Future analysis should consider incorporating error bars representing variance across multiple experimental runs to strengthen statistical validity.

\begin{figure}[!htbp]
\centering
\includegraphics[width=0.95\linewidth]{images/Sumo_Static_Threshold/(b)_Energy.png}
\caption{(b) Energy}
\label{fig:Sumo_Static_Threshold_b__Energy}
\end{figure}

This bar chart presents the relationship between threshold and total energy consumed for the Sumo Static Threshold experimental scenario. The x-axis displays threshold values ranging from 5 to 20, while the y-axis quantifies total energy consumed. The single-series visualization facilitates analysis of how the dependent variable responds to changes in the independent parameter setting.

Analysis of the plotted data reveals that total energy consumed ranges from 8.80 (at threshold = 10) to 11.34 (at threshold = 5), representing a span of 2.54 units. The overall trend is decreasing, with values declining from 11.34 at the initial setting to 9.05 at the final setting. Notably, the minimum value occurs at an intermediate threshold setting (10), suggesting non-monotonic behavior that warrants further investigation.

These results indicate that threshold configuration meaningfully impacts total energy consumed in this experimental context. The substantial variation observed (coefficient of variation exceeding 20\%) suggests that parameter tuning could yield significant performance improvements. Confidence in these findings is high given the direct correspondence between CSV data and plotted values. Future analysis should consider incorporating error bars representing variance across multiple experimental runs to strengthen statistical validity.

\begin{figure}[!htbp]
\centering
\includegraphics[width=0.95\linewidth]{images/Sumo_Static_Threshold/(c)_Distance.png}
\caption{(c) Distance}
\label{fig:Sumo_Static_Threshold_c__Distance}
\end{figure}

This bar chart presents the relationship between threshold and total distance covered for the Sumo Static Threshold experimental scenario. The x-axis displays threshold values ranging from 5 to 20, while the y-axis quantifies total distance covered. The single-series visualization facilitates analysis of how the dependent variable responds to changes in the independent parameter setting.

Analysis of the plotted data reveals that total distance covered ranges from 55.14 (at threshold = 10) to 69.18 (at threshold = 5), representing a span of 14.04 units. The overall trend is decreasing, with values declining from 69.18 at the initial setting to 57.52 at the final setting. Notably, the minimum value occurs at an intermediate threshold setting (10), suggesting non-monotonic behavior that warrants further investigation.

These results indicate that threshold configuration meaningfully impacts total distance covered in this experimental context. The substantial variation observed (coefficient of variation exceeding 20\%) suggests that parameter tuning could yield significant performance improvements. Confidence in these findings is high given the direct correspondence between CSV data and plotted values. Future analysis should consider incorporating error bars representing variance across multiple experimental runs to strengthen statistical validity.

\begin{figure}[!htbp]
\centering
\includegraphics[width=0.95\linewidth]{images/Sumo_Static_Threshold/(d)_Module_Swapped.png}
\caption{(d) Module Swapped}
\label{fig:Sumo_Static_Threshold_d__Module_Swapped}
\end{figure}

This bar chart presents the relationship between threshold and total module swapped for the Sumo Static Threshold experimental scenario. The x-axis displays threshold values ranging from 5 to 20, while the y-axis quantifies total module swapped. The single-series visualization facilitates analysis of how the dependent variable responds to changes in the independent parameter setting.

Analysis of the plotted data reveals that total module swapped ranges from 0.00 (at threshold = 5) to 0.00 (at threshold = 5), representing a span of 0.00 units. The values remain relatively stable across the parameter range, with minimal net change between initial (0.00) and final (0.00) settings. 

These results indicate that threshold configuration meaningfully impacts total module swapped in this experimental context. The relatively modest variation suggests that this parameter has limited influence on the measured metric within the tested range. Confidence in these findings is high given the direct correspondence between CSV data and plotted values. Future analysis should consider incorporating error bars representing variance across multiple experimental runs to strengthen statistical validity.

\begin{figure}[!htbp]
\centering
\includegraphics[width=0.95\linewidth]{images/Sumo_Static_Threshold/(e)_Run_Time.png}
\caption{(e) Run Time}
\label{fig:Sumo_Static_Threshold_e__Run_Time}
\end{figure}

This bar chart presents the relationship between threshold and total travel time for the Sumo Static Threshold experimental scenario. The x-axis displays threshold values ranging from 5 to 20, while the y-axis quantifies total travel time. The single-series visualization facilitates analysis of how the dependent variable responds to changes in the independent parameter setting.

Analysis of the plotted data reveals that total travel time ranges from 86.54 (at threshold = 10) to 106.44 (at threshold = 5), representing a span of 19.90 units. The overall trend is decreasing, with values declining from 106.44 at the initial setting to 92.73 at the final setting. Notably, the minimum value occurs at an intermediate threshold setting (10), suggesting non-monotonic behavior that warrants further investigation.

These results indicate that threshold configuration meaningfully impacts total travel time in this experimental context. The substantial variation observed (coefficient of variation exceeding 20\%) suggests that parameter tuning could yield significant performance improvements. Confidence in these findings is high given the direct correspondence between CSV data and plotted values. Future analysis should consider incorporating error bars representing variance across multiple experimental runs to strengthen statistical validity.


\clearpage

% Requires: \usepackage{graphicx}

\subsection{1000 nodes Module Change}

\begin{figure}[!htbp]
\centering
\includegraphics[width=0.95\linewidth]{images/1000_nodes_Module_Change/(a)_Travel_Time_bar_chart.png}
\caption{(a) Travel Time}
\label{fig:1000_nodes_Module_Change_a__Travel_Time_bar_chart}
\end{figure}

This bar chart presents the relationship between modules and total travel time for the 1000 nodes Module Change experimental scenario. The x-axis displays modules values ranging from 4 to 7, while the y-axis quantifies total travel time. The single-series visualization facilitates analysis of how the dependent variable responds to changes in the independent parameter setting.

Analysis of the plotted data reveals that total travel time ranges from 6779.83 (at modules = 4) to 6808.90 (at modules = 5), representing a span of 29.07 units. The overall trend is increasing, with values rising from 6779.83 at the initial setting to 6808.90 at the final setting. 

These results indicate that modules configuration meaningfully impacts total travel time in this experimental context. The relatively modest variation suggests that this parameter has limited influence on the measured metric within the tested range. Confidence in these findings is high given the direct correspondence between CSV data and plotted values. Future analysis should consider incorporating error bars representing variance across multiple experimental runs to strengthen statistical validity.

\begin{figure}[!htbp]
\centering
\includegraphics[width=0.95\linewidth]{images/1000_nodes_Module_Change/(b)_Energy_consumption_bar_chart.png}
\caption{(b) Energy consumption}
\label{fig:1000_nodes_Module_Change_b__Energy_consumption_bar_chart}
\end{figure}

This bar chart presents the relationship between modules and total energy consumed for the 1000 nodes Module Change experimental scenario. The x-axis displays modules values ranging from 4 to 7, while the y-axis quantifies total energy consumed. The single-series visualization facilitates analysis of how the dependent variable responds to changes in the independent parameter setting.

Analysis of the plotted data reveals that total energy consumed ranges from 793.93 (at modules = 4) to 795.16 (at modules = 5), representing a span of 1.22 units. The overall trend is increasing, with values rising from 793.93 at the initial setting to 795.16 at the final setting. 

These results indicate that modules configuration meaningfully impacts total energy consumed in this experimental context. The relatively modest variation suggests that this parameter has limited influence on the measured metric within the tested range. Confidence in these findings is high given the direct correspondence between CSV data and plotted values. Future analysis should consider incorporating error bars representing variance across multiple experimental runs to strengthen statistical validity.

\begin{figure}[!htbp]
\centering
\includegraphics[width=0.95\linewidth]{images/1000_nodes_Module_Change/(c)_Distance_covered_bar_chart.png}
\caption{(c) Distance covered}
\label{fig:1000_nodes_Module_Change_c__Distance_covered_bar_chart}
\end{figure}

This bar chart presents the relationship between modules and total distance covered for the 1000 nodes Module Change experimental scenario. The x-axis displays modules values ranging from 4 to 7, while the y-axis quantifies total distance covered. The single-series visualization facilitates analysis of how the dependent variable responds to changes in the independent parameter setting.

Analysis of the plotted data reveals that total distance covered ranges from 4422.50 (at modules = 4) to 4437.52 (at modules = 5), representing a span of 15.02 units. The overall trend is increasing, with values rising from 4422.50 at the initial setting to 4437.52 at the final setting. 

These results indicate that modules configuration meaningfully impacts total distance covered in this experimental context. The relatively modest variation suggests that this parameter has limited influence on the measured metric within the tested range. Confidence in these findings is high given the direct correspondence between CSV data and plotted values. Future analysis should consider incorporating error bars representing variance across multiple experimental runs to strengthen statistical validity.

\begin{figure}[!htbp]
\centering
\includegraphics[width=0.95\linewidth]{images/1000_nodes_Module_Change/(d)_Runtime_bar_chart.png}
\caption{(d) Runtime}
\label{fig:1000_nodes_Module_Change_d__Runtime_bar_chart}
\end{figure}

This bar chart presents the relationship between modules and run time for the 1000 nodes Module Change experimental scenario. The x-axis displays modules values ranging from 4 to 7, while the y-axis quantifies run time. The single-series visualization facilitates analysis of how the dependent variable responds to changes in the independent parameter setting.

Analysis of the plotted data reveals that run time ranges from 187.29 (at modules = 4) to 396.10 (at modules = 5), representing a span of 208.81 units. The overall trend is increasing, with values rising from 187.29 at the initial setting to 226.36 at the final setting. 

These results indicate that modules configuration meaningfully impacts run time in this experimental context. The substantial variation observed (coefficient of variation exceeding 20\%) suggests that parameter tuning could yield significant performance improvements. Confidence in these findings is high given the direct correspondence between CSV data and plotted values. Future analysis should consider incorporating error bars representing variance across multiple experimental runs to strengthen statistical validity.

\begin{figure}[!htbp]
\centering
\includegraphics[width=0.95\linewidth]{images/1000_nodes_Module_Change/(e)_Number_of_module_swapped_bar_chart.png}
\caption{(e) Number of module swapped}
\label{fig:1000_nodes_Module_Change_e__Number_of_module_swapped_bar_chart}
\end{figure}

This bar chart presents the relationship between modules and modules for the 1000 nodes Module Change experimental scenario. The x-axis displays modules values ranging from 4 to 7, while the y-axis quantifies modules. The single-series visualization facilitates analysis of how the dependent variable responds to changes in the independent parameter setting.

Analysis of the plotted data reveals that modules ranges from 4.00 (at modules = 4) to 7.00 (at modules = 7), representing a span of 3.00 units. The overall trend is increasing, with values rising from 4.00 at the initial setting to 7.00 at the final setting. 

These results indicate that modules configuration meaningfully impacts modules in this experimental context. The substantial variation observed (coefficient of variation exceeding 20\%) suggests that parameter tuning could yield significant performance improvements. Confidence in these findings is high given the direct correspondence between CSV data and plotted values. Future analysis should consider incorporating error bars representing variance across multiple experimental runs to strengthen statistical validity.


\clearpage

\subsection{1000 nodes Swapping Time}

\begin{figure}[!htbp]
\centering
\includegraphics[width=0.95\linewidth]{images/1000_nodes_Swapping_Time/(a)_Travel_Time_bar_chart.png}
\caption{(a) Travel Time}
\label{fig:1000_nodes_Swapping_Time_a__Travel_Time_bar_chart}
\end{figure}

This bar chart presents the relationship between swapping time and total travel time for the 1000 nodes Swapping Time experimental scenario. The x-axis displays swapping time values ranging from 1 to 4, while the y-axis quantifies total travel time. The single-series visualization facilitates analysis of how the dependent variable responds to changes in the independent parameter setting.

Analysis of the plotted data reveals that total travel time ranges from 6771.90 (at swapping time = 1) to 6882.78 (at swapping time = 4), representing a span of 110.88 units. The overall trend is increasing, with values rising from 6771.90 at the initial setting to 6882.78 at the final setting. 

These results indicate that swapping time configuration meaningfully impacts total travel time in this experimental context. The relatively modest variation suggests that this parameter has limited influence on the measured metric within the tested range. Confidence in these findings is high given the direct correspondence between CSV data and plotted values. Future analysis should consider incorporating error bars representing variance across multiple experimental runs to strengthen statistical validity.

\begin{figure}[!htbp]
\centering
\includegraphics[width=0.95\linewidth]{images/1000_nodes_Swapping_Time/(b)_Energy_consumption_bar_chart.png}
\caption{(b) Energy consumption}
\label{fig:1000_nodes_Swapping_Time_b__Energy_consumption_bar_chart}
\end{figure}

This bar chart presents the relationship between swapping time and total energy consumed for the 1000 nodes Swapping Time experimental scenario. The x-axis displays swapping time values ranging from 1 to 4, while the y-axis quantifies total energy consumed. The single-series visualization facilitates analysis of how the dependent variable responds to changes in the independent parameter setting.

Analysis of the plotted data reveals that total energy consumed ranges from 794.98 (at swapping time = 3) to 795.16 (at swapping time = 1), representing a span of 0.18 units. The overall trend is decreasing, with values declining from 795.16 at the initial setting to 794.98 at the final setting. Notably, the minimum value occurs at an intermediate swapping time setting (3), suggesting non-monotonic behavior that warrants further investigation.

These results indicate that swapping time configuration meaningfully impacts total energy consumed in this experimental context. The relatively modest variation suggests that this parameter has limited influence on the measured metric within the tested range. Confidence in these findings is high given the direct correspondence between CSV data and plotted values. Future analysis should consider incorporating error bars representing variance across multiple experimental runs to strengthen statistical validity.

\begin{figure}[!htbp]
\centering
\includegraphics[width=0.95\linewidth]{images/1000_nodes_Swapping_Time/(c)_Distance_covered_bar_chart.png}
\caption{(c) Distance covered}
\label{fig:1000_nodes_Swapping_Time_c__Distance_covered_bar_chart}
\end{figure}

This bar chart presents the relationship between swapping time and total distance covered for the 1000 nodes Swapping Time experimental scenario. The x-axis displays swapping time values ranging from 1 to 4, while the y-axis quantifies total distance covered. The single-series visualization facilitates analysis of how the dependent variable responds to changes in the independent parameter setting.

Analysis of the plotted data reveals that total distance covered ranges from 4436.55 (at swapping time = 3) to 4437.52 (at swapping time = 1), representing a span of 0.97 units. The overall trend is decreasing, with values declining from 4437.52 at the initial setting to 4436.55 at the final setting. Notably, the minimum value occurs at an intermediate swapping time setting (3), suggesting non-monotonic behavior that warrants further investigation.

These results indicate that swapping time configuration meaningfully impacts total distance covered in this experimental context. The relatively modest variation suggests that this parameter has limited influence on the measured metric within the tested range. Confidence in these findings is high given the direct correspondence between CSV data and plotted values. Future analysis should consider incorporating error bars representing variance across multiple experimental runs to strengthen statistical validity.

\begin{figure}[!htbp]
\centering
\includegraphics[width=0.95\linewidth]{images/1000_nodes_Swapping_Time/(d)_Runtime_bar_chart.png}
\caption{(d) Runtime}
\label{fig:1000_nodes_Swapping_Time_d__Runtime_bar_chart}
\end{figure}

This bar chart presents the relationship between swapping time and run time for the 1000 nodes Swapping Time experimental scenario. The x-axis displays swapping time values ranging from 1 to 4, while the y-axis quantifies run time. The single-series visualization facilitates analysis of how the dependent variable responds to changes in the independent parameter setting.

Analysis of the plotted data reveals that run time ranges from 190.96 (at swapping time = 3) to 399.38 (at swapping time = 4), representing a span of 208.42 units. The overall trend is increasing, with values rising from 363.28 at the initial setting to 399.38 at the final setting. Notably, the minimum value occurs at an intermediate swapping time setting (3), suggesting non-monotonic behavior that warrants further investigation.

These results indicate that swapping time configuration meaningfully impacts run time in this experimental context. The substantial variation observed (coefficient of variation exceeding 20\%) suggests that parameter tuning could yield significant performance improvements. Confidence in these findings is high given the direct correspondence between CSV data and plotted values. Future analysis should consider incorporating error bars representing variance across multiple experimental runs to strengthen statistical validity.

\begin{figure}[!htbp]
\centering
\includegraphics[width=0.95\linewidth]{images/1000_nodes_Swapping_Time/(e)_Number_of_module_swapped_bar_chart.png}
\caption{(e) Number of module swapped}
\label{fig:1000_nodes_Swapping_Time_e__Number_of_module_swapped_bar_chart}
\end{figure}

This bar chart presents the relationship between swapping time and total module swapped for the 1000 nodes Swapping Time experimental scenario. The x-axis displays swapping time values ranging from 1 to 4, while the y-axis quantifies total module swapped. The single-series visualization facilitates analysis of how the dependent variable responds to changes in the independent parameter setting.

Analysis of the plotted data reveals that total module swapped ranges from 37.00 (at swapping time = 1) to 37.00 (at swapping time = 1), representing a span of 0.00 units. The values remain relatively stable across the parameter range, with minimal net change between initial (37.00) and final (37.00) settings. 

These results indicate that swapping time configuration meaningfully impacts total module swapped in this experimental context. The relatively modest variation suggests that this parameter has limited influence on the measured metric within the tested range. Confidence in these findings is high given the direct correspondence between CSV data and plotted values. Future analysis should consider incorporating error bars representing variance across multiple experimental runs to strengthen statistical validity.


\clearpage

\subsection{1000 nodes Threshold}

\begin{figure}[!htbp]
\centering
\includegraphics[width=0.95\linewidth]{images/1000_nodes_Threshold/(a)_Travel_Time_bar_chart.png}
\caption{(a) Travel Time}
\label{fig:1000_nodes_Threshold_a__Travel_Time_bar_chart}
\end{figure}

This bar chart presents the relationship between threshold and total travel time for the 1000 nodes Threshold experimental scenario. The x-axis displays threshold values ranging from 5 to 20, while the y-axis quantifies total travel time. The single-series visualization facilitates analysis of how the dependent variable responds to changes in the independent parameter setting.

Analysis of the plotted data reveals that total travel time ranges from 6808.90 (at threshold = 5) to 6808.90 (at threshold = 5), representing a span of 0.00 units. The values remain relatively stable across the parameter range, with minimal net change between initial (6808.90) and final (6808.90) settings. 

These results indicate that threshold configuration meaningfully impacts total travel time in this experimental context. The relatively modest variation suggests that this parameter has limited influence on the measured metric within the tested range. Confidence in these findings is high given the direct correspondence between CSV data and plotted values. Future analysis should consider incorporating error bars representing variance across multiple experimental runs to strengthen statistical validity.

\begin{figure}[!htbp]
\centering
\includegraphics[width=0.95\linewidth]{images/1000_nodes_Threshold/(b)_Energy_consumption_bar_chart.png}
\caption{(b) Energy consumption}
\label{fig:1000_nodes_Threshold_b__Energy_consumption_bar_chart}
\end{figure}

This bar chart presents the relationship between threshold and total energy consumed for the 1000 nodes Threshold experimental scenario. The x-axis displays threshold values ranging from 5 to 20, while the y-axis quantifies total energy consumed. The single-series visualization facilitates analysis of how the dependent variable responds to changes in the independent parameter setting.

Analysis of the plotted data reveals that total energy consumed ranges from 795.16 (at threshold = 5) to 795.16 (at threshold = 5), representing a span of 0.00 units. The values remain relatively stable across the parameter range, with minimal net change between initial (795.16) and final (795.16) settings. 

These results indicate that threshold configuration meaningfully impacts total energy consumed in this experimental context. The relatively modest variation suggests that this parameter has limited influence on the measured metric within the tested range. Confidence in these findings is high given the direct correspondence between CSV data and plotted values. Future analysis should consider incorporating error bars representing variance across multiple experimental runs to strengthen statistical validity.

\begin{figure}[!htbp]
\centering
\includegraphics[width=0.95\linewidth]{images/1000_nodes_Threshold/(c)_Distance_covered_bar_chart.png}
\caption{(c) Distance covered}
\label{fig:1000_nodes_Threshold_c__Distance_covered_bar_chart}
\end{figure}

This bar chart presents the relationship between threshold and total distance covered for the 1000 nodes Threshold experimental scenario. The x-axis displays threshold values ranging from 5 to 20, while the y-axis quantifies total distance covered. The single-series visualization facilitates analysis of how the dependent variable responds to changes in the independent parameter setting.

Analysis of the plotted data reveals that total distance covered ranges from 4437.52 (at threshold = 5) to 4437.52 (at threshold = 5), representing a span of 0.00 units. The values remain relatively stable across the parameter range, with minimal net change between initial (4437.52) and final (4437.52) settings. 

These results indicate that threshold configuration meaningfully impacts total distance covered in this experimental context. The relatively modest variation suggests that this parameter has limited influence on the measured metric within the tested range. Confidence in these findings is high given the direct correspondence between CSV data and plotted values. Future analysis should consider incorporating error bars representing variance across multiple experimental runs to strengthen statistical validity.

\begin{figure}[!htbp]
\centering
\includegraphics[width=0.95\linewidth]{images/1000_nodes_Threshold/(d)_Runtime_bar_chart.png}
\caption{(d) Runtime}
\label{fig:1000_nodes_Threshold_d__Runtime_bar_chart}
\end{figure}

This bar chart presents the relationship between threshold and run time for the 1000 nodes Threshold experimental scenario. The x-axis displays threshold values ranging from 5 to 20, while the y-axis quantifies run time. The single-series visualization facilitates analysis of how the dependent variable responds to changes in the independent parameter setting.

Analysis of the plotted data reveals that run time ranges from 225.32 (at threshold = 15) to 396.10 (at threshold = 20), representing a span of 170.78 units. The overall trend is increasing, with values rising from 287.89 at the initial setting to 396.10 at the final setting. Notably, the minimum value occurs at an intermediate threshold setting (15), suggesting non-monotonic behavior that warrants further investigation.

These results indicate that threshold configuration meaningfully impacts run time in this experimental context. The substantial variation observed (coefficient of variation exceeding 20\%) suggests that parameter tuning could yield significant performance improvements. Confidence in these findings is high given the direct correspondence between CSV data and plotted values. Future analysis should consider incorporating error bars representing variance across multiple experimental runs to strengthen statistical validity.

\begin{figure}[!htbp]
\centering
\includegraphics[width=0.95\linewidth]{images/1000_nodes_Threshold/(e)_Number_of_module_swapped_bar_chart.png}
\caption{(e) Number of module swapped}
\label{fig:1000_nodes_Threshold_e__Number_of_module_swapped_bar_chart}
\end{figure}

This bar chart presents the relationship between threshold and total module swapped for the 1000 nodes Threshold experimental scenario. The x-axis displays threshold values ranging from 5 to 20, while the y-axis quantifies total module swapped. The single-series visualization facilitates analysis of how the dependent variable responds to changes in the independent parameter setting.

Analysis of the plotted data reveals that total module swapped ranges from 37.00 (at threshold = 5) to 37.00 (at threshold = 5), representing a span of 0.00 units. The values remain relatively stable across the parameter range, with minimal net change between initial (37.00) and final (37.00) settings. 

These results indicate that threshold configuration meaningfully impacts total module swapped in this experimental context. The relatively modest variation suggests that this parameter has limited influence on the measured metric within the tested range. Confidence in these findings is high given the direct correspondence between CSV data and plotted values. Future analysis should consider incorporating error bars representing variance across multiple experimental runs to strengthen statistical validity.


\clearpage

\subsection{1000 nodes Traffic}

\begin{figure}[!htbp]
\centering
\includegraphics[width=0.95\linewidth]{images/1000_nodes_Traffic/(a)_Travel_Time_bar_chart.png}
\caption{(a) Travel Time}
\label{fig:1000_nodes_Traffic_a__Travel_Time_bar_chart}
\end{figure}

This bar chart presents the relationship between traffic and total travel time for the 1000 nodes Traffic experimental scenario. The x-axis displays traffic values ranging from High to Low, while the y-axis quantifies total travel time. The single-series visualization facilitates analysis of how the dependent variable responds to changes in the independent parameter setting.

Analysis of the plotted data reveals that total travel time ranges from 6269.66 (at traffic = Low) to 8004.34 (at traffic = High), representing a span of 1734.68 units. The overall trend is decreasing, with values declining from 8004.34 at the initial setting to 6269.66 at the final setting. 

These results indicate that traffic configuration meaningfully impacts total travel time in this experimental context. The substantial variation observed (coefficient of variation exceeding 20\%) suggests that parameter tuning could yield significant performance improvements. Confidence in these findings is high given the direct correspondence between CSV data and plotted values. Future analysis should consider incorporating error bars representing variance across multiple experimental runs to strengthen statistical validity.

\begin{figure}[!htbp]
\centering
\includegraphics[width=0.95\linewidth]{images/1000_nodes_Traffic/(b)_Energy_consumption_bar_chart.png}
\caption{(b) Energy consumption}
\label{fig:1000_nodes_Traffic_b__Energy_consumption_bar_chart}
\end{figure}

This bar chart presents the relationship between traffic and total energy consumed for the 1000 nodes Traffic experimental scenario. The x-axis displays traffic values ranging from High to Low, while the y-axis quantifies total energy consumed. The single-series visualization facilitates analysis of how the dependent variable responds to changes in the independent parameter setting.

Analysis of the plotted data reveals that total energy consumed ranges from 757.04 (at traffic = High) to 883.10 (at traffic = Low), representing a span of 126.06 units. The overall trend is increasing, with values rising from 757.04 at the initial setting to 883.10 at the final setting. 

These results indicate that traffic configuration meaningfully impacts total energy consumed in this experimental context. The relatively modest variation suggests that this parameter has limited influence on the measured metric within the tested range. Confidence in these findings is high given the direct correspondence between CSV data and plotted values. Future analysis should consider incorporating error bars representing variance across multiple experimental runs to strengthen statistical validity.

\begin{figure}[!htbp]
\centering
\includegraphics[width=0.95\linewidth]{images/1000_nodes_Traffic/(c)_Distance_covered_bar_chart.png}
\caption{(c) Distance covered}
\label{fig:1000_nodes_Traffic_c__Distance_covered_bar_chart}
\end{figure}

This bar chart presents the relationship between traffic and total distance covered for the 1000 nodes Traffic experimental scenario. The x-axis displays traffic values ranging from High to Low, while the y-axis quantifies total distance covered. The single-series visualization facilitates analysis of how the dependent variable responds to changes in the independent parameter setting.

Analysis of the plotted data reveals that total distance covered ranges from 4374.17 (at traffic = High) to 4806.46 (at traffic = Low), representing a span of 432.29 units. The overall trend is increasing, with values rising from 4374.17 at the initial setting to 4806.46 at the final setting. 

These results indicate that traffic configuration meaningfully impacts total distance covered in this experimental context. The relatively modest variation suggests that this parameter has limited influence on the measured metric within the tested range. Confidence in these findings is high given the direct correspondence between CSV data and plotted values. Future analysis should consider incorporating error bars representing variance across multiple experimental runs to strengthen statistical validity.

\begin{figure}[!htbp]
\centering
\includegraphics[width=0.95\linewidth]{images/1000_nodes_Traffic/(d)_Runtime_bar_chart.png}
\caption{(d) Runtime}
\label{fig:1000_nodes_Traffic_d__Runtime_bar_chart}
\end{figure}

This bar chart presents the relationship between traffic and run time for the 1000 nodes Traffic experimental scenario. The x-axis displays traffic values ranging from High to Low, while the y-axis quantifies run time. The single-series visualization facilitates analysis of how the dependent variable responds to changes in the independent parameter setting.

Analysis of the plotted data reveals that run time ranges from 368.03 (at traffic = High) to 407.25 (at traffic = Mid), representing a span of 39.22 units. The overall trend is increasing, with values rising from 368.03 at the initial setting to 382.14 at the final setting. 

These results indicate that traffic configuration meaningfully impacts run time in this experimental context. The relatively modest variation suggests that this parameter has limited influence on the measured metric within the tested range. Confidence in these findings is high given the direct correspondence between CSV data and plotted values. Future analysis should consider incorporating error bars representing variance across multiple experimental runs to strengthen statistical validity.

\begin{figure}[!htbp]
\centering
\includegraphics[width=0.95\linewidth]{images/1000_nodes_Traffic/(e)_Number_of_module_swapped_bar_chart.png}
\caption{(e) Number of module swapped}
\label{fig:1000_nodes_Traffic_e__Number_of_module_swapped_bar_chart}
\end{figure}

This bar chart presents the relationship between traffic and total module swapped for the 1000 nodes Traffic experimental scenario. The x-axis displays traffic values ranging from High to Low, while the y-axis quantifies total module swapped. The single-series visualization facilitates analysis of how the dependent variable responds to changes in the independent parameter setting.

Analysis of the plotted data reveals that total module swapped ranges from 35.00 (at traffic = High) to 41.00 (at traffic = Low), representing a span of 6.00 units. The overall trend is increasing, with values rising from 35.00 at the initial setting to 41.00 at the final setting. 

These results indicate that traffic configuration meaningfully impacts total module swapped in this experimental context. The relatively modest variation suggests that this parameter has limited influence on the measured metric within the tested range. Confidence in these findings is high given the direct correspondence between CSV data and plotted values. Future analysis should consider incorporating error bars representing variance across multiple experimental runs to strengthen statistical validity.


\clearpage

\subsection{1500 nodes Module Change}

\begin{figure}[!htbp]
\centering
\includegraphics[width=0.95\linewidth]{images/1500_nodes_Module_Change/(a)_Travel_time_bar_chart.png}
\caption{(a) Travel time}
\label{fig:1500_nodes_Module_Change_a__Travel_time_bar_chart}
\end{figure}

This bar chart presents the relationship between modules and total travel time for the 1500 nodes Module Change experimental scenario. The x-axis displays modules values ranging from 4 to 7, while the y-axis quantifies total travel time. The single-series visualization facilitates analysis of how the dependent variable responds to changes in the independent parameter setting.

Analysis of the plotted data reveals that total travel time ranges from 10133.53 (at modules = 5) to 10275.22 (at modules = 4), representing a span of 141.69 units. The overall trend is decreasing, with values declining from 10275.22 at the initial setting to 10148.53 at the final setting. Notably, the minimum value occurs at an intermediate modules setting (5), suggesting non-monotonic behavior that warrants further investigation.

These results indicate that modules configuration meaningfully impacts total travel time in this experimental context. The relatively modest variation suggests that this parameter has limited influence on the measured metric within the tested range. Confidence in these findings is high given the direct correspondence between CSV data and plotted values. Future analysis should consider incorporating error bars representing variance across multiple experimental runs to strengthen statistical validity.

\begin{figure}[!htbp]
\centering
\includegraphics[width=0.95\linewidth]{images/1500_nodes_Module_Change/(b)_Energy_consumed_bar_chart.png}
\caption{(b) Energy consumed}
\label{fig:1500_nodes_Module_Change_b__Energy_consumed_bar_chart}
\end{figure}

This bar chart presents the relationship between modules and total energy consumed for the 1500 nodes Module Change experimental scenario. The x-axis displays modules values ranging from 4 to 7, while the y-axis quantifies total energy consumed. The single-series visualization facilitates analysis of how the dependent variable responds to changes in the independent parameter setting.

Analysis of the plotted data reveals that total energy consumed ranges from 1346.58 (at modules = 5) to 1351.12 (at modules = 4), representing a span of 4.55 units. The overall trend is decreasing, with values declining from 1351.12 at the initial setting to 1347.93 at the final setting. Notably, the minimum value occurs at an intermediate modules setting (5), suggesting non-monotonic behavior that warrants further investigation.

These results indicate that modules configuration meaningfully impacts total energy consumed in this experimental context. The relatively modest variation suggests that this parameter has limited influence on the measured metric within the tested range. Confidence in these findings is high given the direct correspondence between CSV data and plotted values. Future analysis should consider incorporating error bars representing variance across multiple experimental runs to strengthen statistical validity.

\begin{figure}[!htbp]
\centering
\includegraphics[width=0.95\linewidth]{images/1500_nodes_Module_Change/(c)_Distance_covered_bar_chart.png}
\caption{(c) Distance covered}
\label{fig:1500_nodes_Module_Change_c__Distance_covered_bar_chart}
\end{figure}

This bar chart presents the relationship between modules and total distance covered for the 1500 nodes Module Change experimental scenario. The x-axis displays modules values ranging from 4 to 7, while the y-axis quantifies total distance covered. The single-series visualization facilitates analysis of how the dependent variable responds to changes in the independent parameter setting.

Analysis of the plotted data reveals that total distance covered ranges from 6533.82 (at modules = 5) to 6628.04 (at modules = 4), representing a span of 94.22 units. The overall trend is decreasing, with values declining from 6628.04 at the initial setting to 6545.22 at the final setting. Notably, the minimum value occurs at an intermediate modules setting (5), suggesting non-monotonic behavior that warrants further investigation.

These results indicate that modules configuration meaningfully impacts total distance covered in this experimental context. The relatively modest variation suggests that this parameter has limited influence on the measured metric within the tested range. Confidence in these findings is high given the direct correspondence between CSV data and plotted values. Future analysis should consider incorporating error bars representing variance across multiple experimental runs to strengthen statistical validity.

\begin{figure}[!htbp]
\centering
\includegraphics[width=0.95\linewidth]{images/1500_nodes_Module_Change/(d)_Run_time_bar_chart.png}
\caption{(d) Run time}
\label{fig:1500_nodes_Module_Change_d__Run_time_bar_chart}
\end{figure}

This bar chart presents the relationship between modules and total travel time for the 1500 nodes Module Change experimental scenario. The x-axis displays modules values ranging from 4 to 7, while the y-axis quantifies total travel time. The single-series visualization facilitates analysis of how the dependent variable responds to changes in the independent parameter setting.

Analysis of the plotted data reveals that total travel time ranges from 10133.53 (at modules = 5) to 10275.22 (at modules = 4), representing a span of 141.69 units. The overall trend is decreasing, with values declining from 10275.22 at the initial setting to 10148.53 at the final setting. Notably, the minimum value occurs at an intermediate modules setting (5), suggesting non-monotonic behavior that warrants further investigation.

These results indicate that modules configuration meaningfully impacts total travel time in this experimental context. The relatively modest variation suggests that this parameter has limited influence on the measured metric within the tested range. Confidence in these findings is high given the direct correspondence between CSV data and plotted values. Future analysis should consider incorporating error bars representing variance across multiple experimental runs to strengthen statistical validity.

\begin{figure}[!htbp]
\centering
\includegraphics[width=0.95\linewidth]{images/1500_nodes_Module_Change/(e)_Module_swapped_bar_chart.png}
\caption{(e) Module swapped}
\label{fig:1500_nodes_Module_Change_e__Module_swapped_bar_chart}
\end{figure}

This bar chart presents the relationship between modules and modules for the 1500 nodes Module Change experimental scenario. The x-axis displays modules values ranging from 4 to 7, while the y-axis quantifies modules. The single-series visualization facilitates analysis of how the dependent variable responds to changes in the independent parameter setting.

Analysis of the plotted data reveals that modules ranges from 4.00 (at modules = 4) to 7.00 (at modules = 7), representing a span of 3.00 units. The overall trend is increasing, with values rising from 4.00 at the initial setting to 7.00 at the final setting. 

These results indicate that modules configuration meaningfully impacts modules in this experimental context. The substantial variation observed (coefficient of variation exceeding 20\%) suggests that parameter tuning could yield significant performance improvements. Confidence in these findings is high given the direct correspondence between CSV data and plotted values. Future analysis should consider incorporating error bars representing variance across multiple experimental runs to strengthen statistical validity.


\clearpage

\subsection{1500 nodes Swapping Time}

\begin{figure}[!htbp]
\centering
\includegraphics[width=0.95\linewidth]{images/1500_nodes_Swapping_Time/(a)_Travel_time_bar_chart.png}
\caption{(a) Travel time}
\label{fig:1500_nodes_Swapping_Time_a__Travel_time_bar_chart}
\end{figure}

This bar chart presents the relationship between swapping time and total travel time for the 1500 nodes Swapping Time experimental scenario. The x-axis displays swapping time values ranging from 1 to 4, while the y-axis quantifies total travel time. The single-series visualization facilitates analysis of how the dependent variable responds to changes in the independent parameter setting.

Analysis of the plotted data reveals that total travel time ranges from 10089.02 (at swapping time = 1) to 10262.19 (at swapping time = 4), representing a span of 173.17 units. The overall trend is increasing, with values rising from 10089.02 at the initial setting to 10262.19 at the final setting. 

These results indicate that swapping time configuration meaningfully impacts total travel time in this experimental context. The relatively modest variation suggests that this parameter has limited influence on the measured metric within the tested range. Confidence in these findings is high given the direct correspondence between CSV data and plotted values. Future analysis should consider incorporating error bars representing variance across multiple experimental runs to strengthen statistical validity.

\begin{figure}[!htbp]
\centering
\includegraphics[width=0.95\linewidth]{images/1500_nodes_Swapping_Time/(b)_Energy_consumed_bar_chart.png}
\caption{(b) Energy consumed}
\label{fig:1500_nodes_Swapping_Time_b__Energy_consumed_bar_chart}
\end{figure}

This bar chart presents the relationship between swapping time and total energy consumed for the 1500 nodes Swapping Time experimental scenario. The x-axis displays swapping time values ranging from 1 to 4, while the y-axis quantifies total energy consumed. The single-series visualization facilitates analysis of how the dependent variable responds to changes in the independent parameter setting.

Analysis of the plotted data reveals that total energy consumed ranges from 1346.58 (at swapping time = 2) to 1349.76 (at swapping time = 1), representing a span of 3.18 units. The overall trend is decreasing, with values declining from 1349.76 at the initial setting to 1346.59 at the final setting. Notably, the minimum value occurs at an intermediate swapping time setting (2), suggesting non-monotonic behavior that warrants further investigation.

These results indicate that swapping time configuration meaningfully impacts total energy consumed in this experimental context. The relatively modest variation suggests that this parameter has limited influence on the measured metric within the tested range. Confidence in these findings is high given the direct correspondence between CSV data and plotted values. Future analysis should consider incorporating error bars representing variance across multiple experimental runs to strengthen statistical validity.

\begin{figure}[!htbp]
\centering
\includegraphics[width=0.95\linewidth]{images/1500_nodes_Swapping_Time/(c)_Distance_covered_bar_chart.png}
\caption{(c) Distance covered}
\label{fig:1500_nodes_Swapping_Time_c__Distance_covered_bar_chart}
\end{figure}

This bar chart presents the relationship between swapping time and total distance covered for the 1500 nodes Swapping Time experimental scenario. The x-axis displays swapping time values ranging from 1 to 4, while the y-axis quantifies total distance covered. The single-series visualization facilitates analysis of how the dependent variable responds to changes in the independent parameter setting.

Analysis of the plotted data reveals that total distance covered ranges from 6533.82 (at swapping time = 2) to 6552.92 (at swapping time = 1), representing a span of 19.10 units. The overall trend is decreasing, with values declining from 6552.92 at the initial setting to 6534.02 at the final setting. Notably, the minimum value occurs at an intermediate swapping time setting (2), suggesting non-monotonic behavior that warrants further investigation.

These results indicate that swapping time configuration meaningfully impacts total distance covered in this experimental context. The relatively modest variation suggests that this parameter has limited influence on the measured metric within the tested range. Confidence in these findings is high given the direct correspondence between CSV data and plotted values. Future analysis should consider incorporating error bars representing variance across multiple experimental runs to strengthen statistical validity.

\begin{figure}[!htbp]
\centering
\includegraphics[width=0.95\linewidth]{images/1500_nodes_Swapping_Time/(d)_Run_time_bar_chart.png}
\caption{(d) Run time}
\label{fig:1500_nodes_Swapping_Time_d__Run_time_bar_chart}
\end{figure}

This bar chart presents the relationship between swapping time and total travel time for the 1500 nodes Swapping Time experimental scenario. The x-axis displays swapping time values ranging from 1 to 4, while the y-axis quantifies total travel time. The single-series visualization facilitates analysis of how the dependent variable responds to changes in the independent parameter setting.

Analysis of the plotted data reveals that total travel time ranges from 10089.02 (at swapping time = 1) to 10262.19 (at swapping time = 4), representing a span of 173.17 units. The overall trend is increasing, with values rising from 10089.02 at the initial setting to 10262.19 at the final setting. 

These results indicate that swapping time configuration meaningfully impacts total travel time in this experimental context. The relatively modest variation suggests that this parameter has limited influence on the measured metric within the tested range. Confidence in these findings is high given the direct correspondence between CSV data and plotted values. Future analysis should consider incorporating error bars representing variance across multiple experimental runs to strengthen statistical validity.

\begin{figure}[!htbp]
\centering
\includegraphics[width=0.95\linewidth]{images/1500_nodes_Swapping_Time/(e)_Module_swapped_bar_chart.png}
\caption{(e) Module swapped}
\label{fig:1500_nodes_Swapping_Time_e__Module_swapped_bar_chart}
\end{figure}

This bar chart presents the relationship between swapping time and total module swapped for the 1500 nodes Swapping Time experimental scenario. The x-axis displays swapping time values ranging from 1 to 4, while the y-axis quantifies total module swapped. The single-series visualization facilitates analysis of how the dependent variable responds to changes in the independent parameter setting.

Analysis of the plotted data reveals that total module swapped ranges from 64.00 (at swapping time = 1) to 64.00 (at swapping time = 1), representing a span of 0.00 units. The values remain relatively stable across the parameter range, with minimal net change between initial (64.00) and final (64.00) settings. 

These results indicate that swapping time configuration meaningfully impacts total module swapped in this experimental context. The relatively modest variation suggests that this parameter has limited influence on the measured metric within the tested range. Confidence in these findings is high given the direct correspondence between CSV data and plotted values. Future analysis should consider incorporating error bars representing variance across multiple experimental runs to strengthen statistical validity.


\clearpage

\subsection{1500 nodes Threshold}

\begin{figure}[!htbp]
\centering
\includegraphics[width=0.95\linewidth]{images/1500_nodes_Threshold/(a)_Travel_time_bar_chart.png}
\caption{(a) Travel time}
\label{fig:1500_nodes_Threshold_a__Travel_time_bar_chart}
\end{figure}

This bar chart presents the relationship between threshold and total travel time for the 1500 nodes Threshold experimental scenario. The x-axis displays threshold values ranging from 5 to 20, while the y-axis quantifies total travel time. The single-series visualization facilitates analysis of how the dependent variable responds to changes in the independent parameter setting.

Analysis of the plotted data reveals that total travel time ranges from 10133.53 (at threshold = 10) to 10148.53 (at threshold = 5), representing a span of 15.00 units. The overall trend is decreasing, with values declining from 10148.53 at the initial setting to 10133.53 at the final setting. Notably, the minimum value occurs at an intermediate threshold setting (10), suggesting non-monotonic behavior that warrants further investigation.

These results indicate that threshold configuration meaningfully impacts total travel time in this experimental context. The relatively modest variation suggests that this parameter has limited influence on the measured metric within the tested range. Confidence in these findings is high given the direct correspondence between CSV data and plotted values. Future analysis should consider incorporating error bars representing variance across multiple experimental runs to strengthen statistical validity.

\begin{figure}[!htbp]
\centering
\includegraphics[width=0.95\linewidth]{images/1500_nodes_Threshold/(b)_Energy_consumed_bar_chart.png}
\caption{(b) Energy consumed}
\label{fig:1500_nodes_Threshold_b__Energy_consumed_bar_chart}
\end{figure}

This bar chart presents the relationship between threshold and total energy consumed for the 1500 nodes Threshold experimental scenario. The x-axis displays threshold values ranging from 5 to 20, while the y-axis quantifies total energy consumed. The single-series visualization facilitates analysis of how the dependent variable responds to changes in the independent parameter setting.

Analysis of the plotted data reveals that total energy consumed ranges from 1346.58 (at threshold = 10) to 1347.93 (at threshold = 5), representing a span of 1.36 units. The overall trend is decreasing, with values declining from 1347.93 at the initial setting to 1346.58 at the final setting. Notably, the minimum value occurs at an intermediate threshold setting (10), suggesting non-monotonic behavior that warrants further investigation.

These results indicate that threshold configuration meaningfully impacts total energy consumed in this experimental context. The relatively modest variation suggests that this parameter has limited influence on the measured metric within the tested range. Confidence in these findings is high given the direct correspondence between CSV data and plotted values. Future analysis should consider incorporating error bars representing variance across multiple experimental runs to strengthen statistical validity.

\begin{figure}[!htbp]
\centering
\includegraphics[width=0.95\linewidth]{images/1500_nodes_Threshold/(c)_Distance_covered_bar_chart.png}
\caption{(c) Distance covered}
\label{fig:1500_nodes_Threshold_c__Distance_covered_bar_chart}
\end{figure}

This bar chart presents the relationship between threshold and total distance covered for the 1500 nodes Threshold experimental scenario. The x-axis displays threshold values ranging from 5 to 20, while the y-axis quantifies total distance covered. The single-series visualization facilitates analysis of how the dependent variable responds to changes in the independent parameter setting.

Analysis of the plotted data reveals that total distance covered ranges from 6533.82 (at threshold = 10) to 6545.22 (at threshold = 5), representing a span of 11.40 units. The overall trend is decreasing, with values declining from 6545.22 at the initial setting to 6533.82 at the final setting. Notably, the minimum value occurs at an intermediate threshold setting (10), suggesting non-monotonic behavior that warrants further investigation.

These results indicate that threshold configuration meaningfully impacts total distance covered in this experimental context. The relatively modest variation suggests that this parameter has limited influence on the measured metric within the tested range. Confidence in these findings is high given the direct correspondence between CSV data and plotted values. Future analysis should consider incorporating error bars representing variance across multiple experimental runs to strengthen statistical validity.

\begin{figure}[!htbp]
\centering
\includegraphics[width=0.95\linewidth]{images/1500_nodes_Threshold/(d)_Run_time_bar_chart.png}
\caption{(d) Run time}
\label{fig:1500_nodes_Threshold_d__Run_time_bar_chart}
\end{figure}

This bar chart presents the relationship between threshold and total travel time for the 1500 nodes Threshold experimental scenario. The x-axis displays threshold values ranging from 5 to 20, while the y-axis quantifies total travel time. The single-series visualization facilitates analysis of how the dependent variable responds to changes in the independent parameter setting.

Analysis of the plotted data reveals that total travel time ranges from 10133.53 (at threshold = 10) to 10148.53 (at threshold = 5), representing a span of 15.00 units. The overall trend is decreasing, with values declining from 10148.53 at the initial setting to 10133.53 at the final setting. Notably, the minimum value occurs at an intermediate threshold setting (10), suggesting non-monotonic behavior that warrants further investigation.

These results indicate that threshold configuration meaningfully impacts total travel time in this experimental context. The relatively modest variation suggests that this parameter has limited influence on the measured metric within the tested range. Confidence in these findings is high given the direct correspondence between CSV data and plotted values. Future analysis should consider incorporating error bars representing variance across multiple experimental runs to strengthen statistical validity.

\begin{figure}[!htbp]
\centering
\includegraphics[width=0.95\linewidth]{images/1500_nodes_Threshold/(e)_Module_swapped_bar_chart.png}
\caption{(e) Module swapped}
\label{fig:1500_nodes_Threshold_e__Module_swapped_bar_chart}
\end{figure}

This bar chart presents the relationship between threshold and total module swapped for the 1500 nodes Threshold experimental scenario. The x-axis displays threshold values ranging from 5 to 20, while the y-axis quantifies total module swapped. The single-series visualization facilitates analysis of how the dependent variable responds to changes in the independent parameter setting.

Analysis of the plotted data reveals that total module swapped ranges from 63.00 (at threshold = 5) to 64.00 (at threshold = 10), representing a span of 1.00 units. The overall trend is increasing, with values rising from 63.00 at the initial setting to 64.00 at the final setting. 

These results indicate that threshold configuration meaningfully impacts total module swapped in this experimental context. The relatively modest variation suggests that this parameter has limited influence on the measured metric within the tested range. Confidence in these findings is high given the direct correspondence between CSV data and plotted values. Future analysis should consider incorporating error bars representing variance across multiple experimental runs to strengthen statistical validity.


\clearpage

\subsection{1500 nodes Traffic}

\begin{figure}[!htbp]
\centering
\includegraphics[width=0.95\linewidth]{images/1500_nodes_Traffic/(a)_Travel_time_bar_chart.png}
\caption{(a) Travel time}
\label{fig:1500_nodes_Traffic_a__Travel_time_bar_chart}
\end{figure}

This bar chart presents the relationship between traffic and total travel time for the 1500 nodes Traffic experimental scenario. The x-axis displays traffic values ranging from High to Low, while the y-axis quantifies total travel time. The single-series visualization facilitates analysis of how the dependent variable responds to changes in the independent parameter setting.

Analysis of the plotted data reveals that total travel time ranges from 8682.70 (at traffic = Low) to 11868.31 (at traffic = High), representing a span of 3185.61 units. The overall trend is decreasing, with values declining from 11868.31 at the initial setting to 8682.70 at the final setting. 

These results indicate that traffic configuration meaningfully impacts total travel time in this experimental context. The substantial variation observed (coefficient of variation exceeding 20\%) suggests that parameter tuning could yield significant performance improvements. Confidence in these findings is high given the direct correspondence between CSV data and plotted values. Future analysis should consider incorporating error bars representing variance across multiple experimental runs to strengthen statistical validity.

\begin{figure}[!htbp]
\centering
\includegraphics[width=0.95\linewidth]{images/1500_nodes_Traffic/(b)_Energy_consumed_bar_chart.png}
\caption{(b) Energy consumed}
\label{fig:1500_nodes_Traffic_b__Energy_consumed_bar_chart}
\end{figure}

This bar chart presents the relationship between traffic and total energy consumed for the 1500 nodes Traffic experimental scenario. The x-axis displays traffic values ranging from High to Low, while the y-axis quantifies total energy consumed. The single-series visualization facilitates analysis of how the dependent variable responds to changes in the independent parameter setting.

Analysis of the plotted data reveals that total energy consumed ranges from 1303.59 (at traffic = High) to 1443.98 (at traffic = Low), representing a span of 140.40 units. The overall trend is increasing, with values rising from 1303.59 at the initial setting to 1443.98 at the final setting. 

These results indicate that traffic configuration meaningfully impacts total energy consumed in this experimental context. The relatively modest variation suggests that this parameter has limited influence on the measured metric within the tested range. Confidence in these findings is high given the direct correspondence between CSV data and plotted values. Future analysis should consider incorporating error bars representing variance across multiple experimental runs to strengthen statistical validity.

\begin{figure}[!htbp]
\centering
\includegraphics[width=0.95\linewidth]{images/1500_nodes_Traffic/(c)_Distance_covered_bar_chart.png}
\caption{(c) Distance covered}
\label{fig:1500_nodes_Traffic_c__Distance_covered_bar_chart}
\end{figure}

This bar chart presents the relationship between traffic and total distance covered for the 1500 nodes Traffic experimental scenario. The x-axis displays traffic values ranging from High to Low, while the y-axis quantifies total distance covered. The single-series visualization facilitates analysis of how the dependent variable responds to changes in the independent parameter setting.

Analysis of the plotted data reveals that total distance covered ranges from 6463.22 (at traffic = Mid) to 6661.07 (at traffic = Low), representing a span of 197.85 units. The overall trend is increasing, with values rising from 6493.82 at the initial setting to 6661.07 at the final setting. Notably, the minimum value occurs at an intermediate traffic setting (Mid), suggesting non-monotonic behavior that warrants further investigation.

These results indicate that traffic configuration meaningfully impacts total distance covered in this experimental context. The relatively modest variation suggests that this parameter has limited influence on the measured metric within the tested range. Confidence in these findings is high given the direct correspondence between CSV data and plotted values. Future analysis should consider incorporating error bars representing variance across multiple experimental runs to strengthen statistical validity.

\begin{figure}[!htbp]
\centering
\includegraphics[width=0.95\linewidth]{images/1500_nodes_Traffic/(d)_Run_time_bar_chart.png}
\caption{(d) Run time}
\label{fig:1500_nodes_Traffic_d__Run_time_bar_chart}
\end{figure}

This bar chart presents the relationship between traffic and total travel time for the 1500 nodes Traffic experimental scenario. The x-axis displays traffic values ranging from High to Low, while the y-axis quantifies total travel time. The single-series visualization facilitates analysis of how the dependent variable responds to changes in the independent parameter setting.

Analysis of the plotted data reveals that total travel time ranges from 8682.70 (at traffic = Low) to 11868.31 (at traffic = High), representing a span of 3185.61 units. The overall trend is decreasing, with values declining from 11868.31 at the initial setting to 8682.70 at the final setting. 

These results indicate that traffic configuration meaningfully impacts total travel time in this experimental context. The substantial variation observed (coefficient of variation exceeding 20\%) suggests that parameter tuning could yield significant performance improvements. Confidence in these findings is high given the direct correspondence between CSV data and plotted values. Future analysis should consider incorporating error bars representing variance across multiple experimental runs to strengthen statistical validity.

\begin{figure}[!htbp]
\centering
\includegraphics[width=0.95\linewidth]{images/1500_nodes_Traffic/(e)_Module_swapped_bar_chart.png}
\caption{(e) Module swapped}
\label{fig:1500_nodes_Traffic_e__Module_swapped_bar_chart}
\end{figure}

This bar chart presents the relationship between traffic and total module swapped for the 1500 nodes Traffic experimental scenario. The x-axis displays traffic values ranging from High to Low, while the y-axis quantifies total module swapped. The single-series visualization facilitates analysis of how the dependent variable responds to changes in the independent parameter setting.

Analysis of the plotted data reveals that total module swapped ranges from 63.00 (at traffic = High) to 69.00 (at traffic = Low), representing a span of 6.00 units. The overall trend is increasing, with values rising from 63.00 at the initial setting to 69.00 at the final setting. 

These results indicate that traffic configuration meaningfully impacts total module swapped in this experimental context. The relatively modest variation suggests that this parameter has limited influence on the measured metric within the tested range. Confidence in these findings is high given the direct correspondence between CSV data and plotted values. Future analysis should consider incorporating error bars representing variance across multiple experimental runs to strengthen statistical validity.


\clearpage

\subsection{2000 nodes Module Change}

\begin{figure}[!htbp]
\centering
\includegraphics[width=0.95\linewidth]{images/2000_nodes_Module_Change/(a)_Travel_time_bar_chart.png}
\caption{(a) Travel time}
\label{fig:2000_nodes_Module_Change_a__Travel_time_bar_chart}
\end{figure}

This bar chart presents the relationship between modules and total travel time for the 2000 nodes Module Change experimental scenario. The x-axis displays modules values ranging from 4 to 7, while the y-axis quantifies total travel time. The single-series visualization facilitates analysis of how the dependent variable responds to changes in the independent parameter setting.

Analysis of the plotted data reveals that total travel time ranges from 13169.12 (at modules = 7) to 14159.32 (at modules = 4), representing a span of 990.20 units. The overall trend is decreasing, with values declining from 14159.32 at the initial setting to 13169.12 at the final setting. 

These results indicate that modules configuration meaningfully impacts total travel time in this experimental context. The relatively modest variation suggests that this parameter has limited influence on the measured metric within the tested range. Confidence in these findings is high given the direct correspondence between CSV data and plotted values. Future analysis should consider incorporating error bars representing variance across multiple experimental runs to strengthen statistical validity.

\begin{figure}[!htbp]
\centering
\includegraphics[width=0.95\linewidth]{images/2000_nodes_Module_Change/(b)_Energy_consumed_bar_chart.png}
\caption{(b) Energy consumed}
\label{fig:2000_nodes_Module_Change_b__Energy_consumed_bar_chart}
\end{figure}

This bar chart presents the relationship between modules and total energy consumed for the 2000 nodes Module Change experimental scenario. The x-axis displays modules values ranging from 4 to 7, while the y-axis quantifies total energy consumed. The single-series visualization facilitates analysis of how the dependent variable responds to changes in the independent parameter setting.

Analysis of the plotted data reveals that total energy consumed ranges from 2030.72 (at modules = 7) to 2154.01 (at modules = 4), representing a span of 123.28 units. The overall trend is decreasing, with values declining from 2154.01 at the initial setting to 2030.72 at the final setting. 

These results indicate that modules configuration meaningfully impacts total energy consumed in this experimental context. The relatively modest variation suggests that this parameter has limited influence on the measured metric within the tested range. Confidence in these findings is high given the direct correspondence between CSV data and plotted values. Future analysis should consider incorporating error bars representing variance across multiple experimental runs to strengthen statistical validity.

\begin{figure}[!htbp]
\centering
\includegraphics[width=0.95\linewidth]{images/2000_nodes_Module_Change/(c)_Distance_covered_bar_chart.png}
\caption{(c) Distance covered}
\label{fig:2000_nodes_Module_Change_c__Distance_covered_bar_chart}
\end{figure}

This bar chart presents the relationship between modules and total distance covered for the 2000 nodes Module Change experimental scenario. The x-axis displays modules values ranging from 4 to 7, while the y-axis quantifies total distance covered. The single-series visualization facilitates analysis of how the dependent variable responds to changes in the independent parameter setting.

Analysis of the plotted data reveals that total distance covered ranges from 8588.97 (at modules = 7) to 9234.69 (at modules = 4), representing a span of 645.72 units. The overall trend is decreasing, with values declining from 9234.69 at the initial setting to 8588.97 at the final setting. 

These results indicate that modules configuration meaningfully impacts total distance covered in this experimental context. The relatively modest variation suggests that this parameter has limited influence on the measured metric within the tested range. Confidence in these findings is high given the direct correspondence between CSV data and plotted values. Future analysis should consider incorporating error bars representing variance across multiple experimental runs to strengthen statistical validity.

\begin{figure}[!htbp]
\centering
\includegraphics[width=0.95\linewidth]{images/2000_nodes_Module_Change/(d)_Run_time_bar_chart.png}
\caption{(d) Run time}
\label{fig:2000_nodes_Module_Change_d__Run_time_bar_chart}
\end{figure}

This bar chart presents the relationship between modules and total travel time for the 2000 nodes Module Change experimental scenario. The x-axis displays modules values ranging from 4 to 7, while the y-axis quantifies total travel time. The single-series visualization facilitates analysis of how the dependent variable responds to changes in the independent parameter setting.

Analysis of the plotted data reveals that total travel time ranges from 13169.12 (at modules = 7) to 14159.32 (at modules = 4), representing a span of 990.20 units. The overall trend is decreasing, with values declining from 14159.32 at the initial setting to 13169.12 at the final setting. 

These results indicate that modules configuration meaningfully impacts total travel time in this experimental context. The relatively modest variation suggests that this parameter has limited influence on the measured metric within the tested range. Confidence in these findings is high given the direct correspondence between CSV data and plotted values. Future analysis should consider incorporating error bars representing variance across multiple experimental runs to strengthen statistical validity.

\begin{figure}[!htbp]
\centering
\includegraphics[width=0.95\linewidth]{images/2000_nodes_Module_Change/(e)_Module_swapped_bar_chart.png}
\caption{(e) Module swapped}
\label{fig:2000_nodes_Module_Change_e__Module_swapped_bar_chart}
\end{figure}

This bar chart presents the relationship between modules and modules for the 2000 nodes Module Change experimental scenario. The x-axis displays modules values ranging from 4 to 7, while the y-axis quantifies modules. The single-series visualization facilitates analysis of how the dependent variable responds to changes in the independent parameter setting.

Analysis of the plotted data reveals that modules ranges from 4.00 (at modules = 4) to 7.00 (at modules = 7), representing a span of 3.00 units. The overall trend is increasing, with values rising from 4.00 at the initial setting to 7.00 at the final setting. 

These results indicate that modules configuration meaningfully impacts modules in this experimental context. The substantial variation observed (coefficient of variation exceeding 20\%) suggests that parameter tuning could yield significant performance improvements. Confidence in these findings is high given the direct correspondence between CSV data and plotted values. Future analysis should consider incorporating error bars representing variance across multiple experimental runs to strengthen statistical validity.


\clearpage

\subsection{2000 nodes Swapping Time}

\begin{figure}[!htbp]
\centering
\includegraphics[width=0.95\linewidth]{images/2000_nodes_Swapping_Time/(a)_Travel_time_bar_chart.png}
\caption{(a) Travel time}
\label{fig:2000_nodes_Swapping_Time_a__Travel_time_bar_chart}
\end{figure}

This bar chart presents the relationship between swapping time and total travel time for the 2000 nodes Swapping Time experimental scenario. The x-axis displays swapping time values ranging from 1 to 4, while the y-axis quantifies total travel time. The single-series visualization facilitates analysis of how the dependent variable responds to changes in the independent parameter setting.

Analysis of the plotted data reveals that total travel time ranges from 13181.25 (at swapping time = 1) to 13540.28 (at swapping time = 4), representing a span of 359.03 units. The overall trend is increasing, with values rising from 13181.25 at the initial setting to 13540.28 at the final setting. 

These results indicate that swapping time configuration meaningfully impacts total travel time in this experimental context. The relatively modest variation suggests that this parameter has limited influence on the measured metric within the tested range. Confidence in these findings is high given the direct correspondence between CSV data and plotted values. Future analysis should consider incorporating error bars representing variance across multiple experimental runs to strengthen statistical validity.

\begin{figure}[!htbp]
\centering
\includegraphics[width=0.95\linewidth]{images/2000_nodes_Swapping_Time/(b)_Energy_consumed_bar_chart.png}
\caption{(b) Energy consumed}
\label{fig:2000_nodes_Swapping_Time_b__Energy_consumed_bar_chart}
\end{figure}

This bar chart presents the relationship between swapping time and total energy consumed for the 2000 nodes Swapping Time experimental scenario. The x-axis displays swapping time values ranging from 1 to 4, while the y-axis quantifies total energy consumed. The single-series visualization facilitates analysis of how the dependent variable responds to changes in the independent parameter setting.

Analysis of the plotted data reveals that total energy consumed ranges from 2057.20 (at swapping time = 1) to 2062.48 (at swapping time = 4), representing a span of 5.28 units. The overall trend is increasing, with values rising from 2057.20 at the initial setting to 2062.48 at the final setting. 

These results indicate that swapping time configuration meaningfully impacts total energy consumed in this experimental context. The relatively modest variation suggests that this parameter has limited influence on the measured metric within the tested range. Confidence in these findings is high given the direct correspondence between CSV data and plotted values. Future analysis should consider incorporating error bars representing variance across multiple experimental runs to strengthen statistical validity.

\begin{figure}[!htbp]
\centering
\includegraphics[width=0.95\linewidth]{images/2000_nodes_Swapping_Time/(c)_Distance_covered_bar_chart.png}
\caption{(c) Distance covered}
\label{fig:2000_nodes_Swapping_Time_c__Distance_covered_bar_chart}
\end{figure}

This bar chart presents the relationship between swapping time and total distance covered for the 2000 nodes Swapping Time experimental scenario. The x-axis displays swapping time values ranging from 1 to 4, while the y-axis quantifies total distance covered. The single-series visualization facilitates analysis of how the dependent variable responds to changes in the independent parameter setting.

Analysis of the plotted data reveals that total distance covered ranges from 8650.95 (at swapping time = 1) to 8701.15 (at swapping time = 4), representing a span of 50.20 units. The overall trend is increasing, with values rising from 8650.95 at the initial setting to 8701.15 at the final setting. 

These results indicate that swapping time configuration meaningfully impacts total distance covered in this experimental context. The relatively modest variation suggests that this parameter has limited influence on the measured metric within the tested range. Confidence in these findings is high given the direct correspondence between CSV data and plotted values. Future analysis should consider incorporating error bars representing variance across multiple experimental runs to strengthen statistical validity.

\begin{figure}[!htbp]
\centering
\includegraphics[width=0.95\linewidth]{images/2000_nodes_Swapping_Time/(d)_Run_time_bar_chart.png}
\caption{(d) Run time}
\label{fig:2000_nodes_Swapping_Time_d__Run_time_bar_chart}
\end{figure}

This bar chart presents the relationship between swapping time and total travel time for the 2000 nodes Swapping Time experimental scenario. The x-axis displays swapping time values ranging from 1 to 4, while the y-axis quantifies total travel time. The single-series visualization facilitates analysis of how the dependent variable responds to changes in the independent parameter setting.

Analysis of the plotted data reveals that total travel time ranges from 13181.25 (at swapping time = 1) to 13540.28 (at swapping time = 4), representing a span of 359.03 units. The overall trend is increasing, with values rising from 13181.25 at the initial setting to 13540.28 at the final setting. 

These results indicate that swapping time configuration meaningfully impacts total travel time in this experimental context. The relatively modest variation suggests that this parameter has limited influence on the measured metric within the tested range. Confidence in these findings is high given the direct correspondence between CSV data and plotted values. Future analysis should consider incorporating error bars representing variance across multiple experimental runs to strengthen statistical validity.

\begin{figure}[!htbp]
\centering
\includegraphics[width=0.95\linewidth]{images/2000_nodes_Swapping_Time/(e)_Module_swapped_bar_chart.png}
\caption{(e) Module swapped}
\label{fig:2000_nodes_Swapping_Time_e__Module_swapped_bar_chart}
\end{figure}

This bar chart presents the relationship between swapping time and total module swapped for the 2000 nodes Swapping Time experimental scenario. The x-axis displays swapping time values ranging from 1 to 4, while the y-axis quantifies total module swapped. The single-series visualization facilitates analysis of how the dependent variable responds to changes in the independent parameter setting.

Analysis of the plotted data reveals that total module swapped ranges from 100.00 (at swapping time = 1) to 100.00 (at swapping time = 1), representing a span of 0.00 units. The values remain relatively stable across the parameter range, with minimal net change between initial (100.00) and final (100.00) settings. 

These results indicate that swapping time configuration meaningfully impacts total module swapped in this experimental context. The relatively modest variation suggests that this parameter has limited influence on the measured metric within the tested range. Confidence in these findings is high given the direct correspondence between CSV data and plotted values. Future analysis should consider incorporating error bars representing variance across multiple experimental runs to strengthen statistical validity.


\clearpage

\subsection{2000 nodes Threshold}

\begin{figure}[!htbp]
\centering
\includegraphics[width=0.95\linewidth]{images/2000_nodes_Threshold/(a)_Travel_time_bar_chart.png}
\caption{(a) Travel time}
\label{fig:2000_nodes_Threshold_a__Travel_time_bar_chart}
\end{figure}

This bar chart presents the relationship between threshold and total travel time for the 2000 nodes Threshold experimental scenario. The x-axis displays threshold values ranging from 5 to 20, while the y-axis quantifies total travel time. The single-series visualization facilitates analysis of how the dependent variable responds to changes in the independent parameter setting.

Analysis of the plotted data reveals that total travel time ranges from 13059.88 (at threshold = 5) to 13381.98 (at threshold = 20), representing a span of 322.10 units. The overall trend is increasing, with values rising from 13059.88 at the initial setting to 13381.98 at the final setting. 

These results indicate that threshold configuration meaningfully impacts total travel time in this experimental context. The relatively modest variation suggests that this parameter has limited influence on the measured metric within the tested range. Confidence in these findings is high given the direct correspondence between CSV data and plotted values. Future analysis should consider incorporating error bars representing variance across multiple experimental runs to strengthen statistical validity.

\begin{figure}[!htbp]
\centering
\includegraphics[width=0.95\linewidth]{images/2000_nodes_Threshold/(b)_Energy_consumed_bar_chart.png}
\caption{(b) Energy consumed}
\label{fig:2000_nodes_Threshold_b__Energy_consumed_bar_chart}
\end{figure}

This bar chart presents the relationship between threshold and total energy consumed for the 2000 nodes Threshold experimental scenario. The x-axis displays threshold values ranging from 5 to 20, while the y-axis quantifies total energy consumed. The single-series visualization facilitates analysis of how the dependent variable responds to changes in the independent parameter setting.

Analysis of the plotted data reveals that total energy consumed ranges from 2020.24 (at threshold = 15) to 2057.78 (at threshold = 10), representing a span of 37.54 units. The overall trend is increasing, with values rising from 2027.67 at the initial setting to 2057.45 at the final setting. Notably, the minimum value occurs at an intermediate threshold setting (15), suggesting non-monotonic behavior that warrants further investigation.

These results indicate that threshold configuration meaningfully impacts total energy consumed in this experimental context. The relatively modest variation suggests that this parameter has limited influence on the measured metric within the tested range. Confidence in these findings is high given the direct correspondence between CSV data and plotted values. Future analysis should consider incorporating error bars representing variance across multiple experimental runs to strengthen statistical validity.

\begin{figure}[!htbp]
\centering
\includegraphics[width=0.95\linewidth]{images/2000_nodes_Threshold/(c)_Distance_covered_bar_chart.png}
\caption{(c) Distance covered}
\label{fig:2000_nodes_Threshold_c__Distance_covered_bar_chart}
\end{figure}

This bar chart presents the relationship between threshold and total distance covered for the 2000 nodes Threshold experimental scenario. The x-axis displays threshold values ranging from 5 to 20, while the y-axis quantifies total distance covered. The single-series visualization facilitates analysis of how the dependent variable responds to changes in the independent parameter setting.

Analysis of the plotted data reveals that total distance covered ranges from 8527.92 (at threshold = 5) to 8685.62 (at threshold = 10), representing a span of 157.70 units. The overall trend is increasing, with values rising from 8527.92 at the initial setting to 8652.17 at the final setting. 

These results indicate that threshold configuration meaningfully impacts total distance covered in this experimental context. The relatively modest variation suggests that this parameter has limited influence on the measured metric within the tested range. Confidence in these findings is high given the direct correspondence between CSV data and plotted values. Future analysis should consider incorporating error bars representing variance across multiple experimental runs to strengthen statistical validity.

\begin{figure}[!htbp]
\centering
\includegraphics[width=0.95\linewidth]{images/2000_nodes_Threshold/(d)_Run_time_bar_chart.png}
\caption{(d) Run time}
\label{fig:2000_nodes_Threshold_d__Run_time_bar_chart}
\end{figure}

This bar chart presents the relationship between threshold and total travel time for the 2000 nodes Threshold experimental scenario. The x-axis displays threshold values ranging from 5 to 20, while the y-axis quantifies total travel time. The single-series visualization facilitates analysis of how the dependent variable responds to changes in the independent parameter setting.

Analysis of the plotted data reveals that total travel time ranges from 13059.88 (at threshold = 5) to 13381.98 (at threshold = 20), representing a span of 322.10 units. The overall trend is increasing, with values rising from 13059.88 at the initial setting to 13381.98 at the final setting. 

These results indicate that threshold configuration meaningfully impacts total travel time in this experimental context. The relatively modest variation suggests that this parameter has limited influence on the measured metric within the tested range. Confidence in these findings is high given the direct correspondence between CSV data and plotted values. Future analysis should consider incorporating error bars representing variance across multiple experimental runs to strengthen statistical validity.

\begin{figure}[!htbp]
\centering
\includegraphics[width=0.95\linewidth]{images/2000_nodes_Threshold/(e)_Module_swapped_bar_chart.png}
\caption{(e) Module swapped}
\label{fig:2000_nodes_Threshold_e__Module_swapped_bar_chart}
\end{figure}

This bar chart presents the relationship between threshold and total module swapped for the 2000 nodes Threshold experimental scenario. The x-axis displays threshold values ranging from 5 to 20, while the y-axis quantifies total module swapped. The single-series visualization facilitates analysis of how the dependent variable responds to changes in the independent parameter setting.

Analysis of the plotted data reveals that total module swapped ranges from 97.00 (at threshold = 15) to 100.00 (at threshold = 20), representing a span of 3.00 units. The overall trend is increasing, with values rising from 98.00 at the initial setting to 100.00 at the final setting. Notably, the minimum value occurs at an intermediate threshold setting (15), suggesting non-monotonic behavior that warrants further investigation.

These results indicate that threshold configuration meaningfully impacts total module swapped in this experimental context. The relatively modest variation suggests that this parameter has limited influence on the measured metric within the tested range. Confidence in these findings is high given the direct correspondence between CSV data and plotted values. Future analysis should consider incorporating error bars representing variance across multiple experimental runs to strengthen statistical validity.


\clearpage

\subsection{2000 nodes Traffic}

\begin{figure}[!htbp]
\centering
\includegraphics[width=0.95\linewidth]{images/2000_nodes_Traffic/(a)_Travel_time_bar_chart.png}
\caption{(a) Travel time}
\label{fig:2000_nodes_Traffic_a__Travel_time_bar_chart}
\end{figure}

This bar chart presents the relationship between traffic and total travel time for the 2000 nodes Traffic experimental scenario. The x-axis displays traffic values ranging from High to Low, while the y-axis quantifies total travel time. The single-series visualization facilitates analysis of how the dependent variable responds to changes in the independent parameter setting.

Analysis of the plotted data reveals that total travel time ranges from 11773.91 (at traffic = Low) to 15961.57 (at traffic = High), representing a span of 4187.66 units. The overall trend is decreasing, with values declining from 15961.57 at the initial setting to 11773.91 at the final setting. 

These results indicate that traffic configuration meaningfully impacts total travel time in this experimental context. The substantial variation observed (coefficient of variation exceeding 20\%) suggests that parameter tuning could yield significant performance improvements. Confidence in these findings is high given the direct correspondence between CSV data and plotted values. Future analysis should consider incorporating error bars representing variance across multiple experimental runs to strengthen statistical validity.

\begin{figure}[!htbp]
\centering
\includegraphics[width=0.95\linewidth]{images/2000_nodes_Traffic/(b)_Energy_consumed_bar_chart.png}
\caption{(b) Energy consumed}
\label{fig:2000_nodes_Traffic_b__Energy_consumed_bar_chart}
\end{figure}

This bar chart presents the relationship between traffic and total energy consumed for the 2000 nodes Traffic experimental scenario. The x-axis displays traffic values ranging from High to Low, while the y-axis quantifies total energy consumed. The single-series visualization facilitates analysis of how the dependent variable responds to changes in the independent parameter setting.

Analysis of the plotted data reveals that total energy consumed ranges from 1974.45 (at traffic = Mid) to 2187.71 (at traffic = Low), representing a span of 213.26 units. The overall trend is increasing, with values rising from 1975.64 at the initial setting to 2187.71 at the final setting. Notably, the minimum value occurs at an intermediate traffic setting (Mid), suggesting non-monotonic behavior that warrants further investigation.

These results indicate that traffic configuration meaningfully impacts total energy consumed in this experimental context. The relatively modest variation suggests that this parameter has limited influence on the measured metric within the tested range. Confidence in these findings is high given the direct correspondence between CSV data and plotted values. Future analysis should consider incorporating error bars representing variance across multiple experimental runs to strengthen statistical validity.

\begin{figure}[!htbp]
\centering
\includegraphics[width=0.95\linewidth]{images/2000_nodes_Traffic/(c)_Distance_covered_bar_chart.png}
\caption{(c) Distance covered}
\label{fig:2000_nodes_Traffic_c__Distance_covered_bar_chart}
\end{figure}

This bar chart presents the relationship between traffic and total distance covered for the 2000 nodes Traffic experimental scenario. The x-axis displays traffic values ranging from High to Low, while the y-axis quantifies total distance covered. The single-series visualization facilitates analysis of how the dependent variable responds to changes in the independent parameter setting.

Analysis of the plotted data reveals that total distance covered ranges from 8689.81 (at traffic = High) to 8994.35 (at traffic = Low), representing a span of 304.54 units. The overall trend is increasing, with values rising from 8689.81 at the initial setting to 8994.35 at the final setting. 

These results indicate that traffic configuration meaningfully impacts total distance covered in this experimental context. The relatively modest variation suggests that this parameter has limited influence on the measured metric within the tested range. Confidence in these findings is high given the direct correspondence between CSV data and plotted values. Future analysis should consider incorporating error bars representing variance across multiple experimental runs to strengthen statistical validity.

\begin{figure}[!htbp]
\centering
\includegraphics[width=0.95\linewidth]{images/2000_nodes_Traffic/(d)_Run_time_bar_chart.png}
\caption{(d) Run time}
\label{fig:2000_nodes_Traffic_d__Run_time_bar_chart}
\end{figure}

This bar chart presents the relationship between traffic and total travel time for the 2000 nodes Traffic experimental scenario. The x-axis displays traffic values ranging from High to Low, while the y-axis quantifies total travel time. The single-series visualization facilitates analysis of how the dependent variable responds to changes in the independent parameter setting.

Analysis of the plotted data reveals that total travel time ranges from 11773.91 (at traffic = Low) to 15961.57 (at traffic = High), representing a span of 4187.66 units. The overall trend is decreasing, with values declining from 15961.57 at the initial setting to 11773.91 at the final setting. 

These results indicate that traffic configuration meaningfully impacts total travel time in this experimental context. The substantial variation observed (coefficient of variation exceeding 20\%) suggests that parameter tuning could yield significant performance improvements. Confidence in these findings is high given the direct correspondence between CSV data and plotted values. Future analysis should consider incorporating error bars representing variance across multiple experimental runs to strengthen statistical validity.

\begin{figure}[!htbp]
\centering
\includegraphics[width=0.95\linewidth]{images/2000_nodes_Traffic/(e)_Module_swapped_bar_chart.png}
\caption{(e) Module swapped}
\label{fig:2000_nodes_Traffic_e__Module_swapped_bar_chart}
\end{figure}

This bar chart presents the relationship between traffic and total module swapped for the 2000 nodes Traffic experimental scenario. The x-axis displays traffic values ranging from High to Low, while the y-axis quantifies total module swapped. The single-series visualization facilitates analysis of how the dependent variable responds to changes in the independent parameter setting.

Analysis of the plotted data reveals that total module swapped ranges from 96.00 (at traffic = High) to 107.00 (at traffic = Low), representing a span of 11.00 units. The overall trend is increasing, with values rising from 96.00 at the initial setting to 107.00 at the final setting. 

These results indicate that traffic configuration meaningfully impacts total module swapped in this experimental context. The relatively modest variation suggests that this parameter has limited influence on the measured metric within the tested range. Confidence in these findings is high given the direct correspondence between CSV data and plotted values. Future analysis should consider incorporating error bars representing variance across multiple experimental runs to strengthen statistical validity.


\clearpage

\subsection{500 nodes Module Change}

\begin{figure}[!htbp]
\centering
\includegraphics[width=0.95\linewidth]{images/500_nodes_Module_Change/(a)_Travel_Time_bar_chart.png}
\caption{(a) Travel Time}
\label{fig:500_nodes_Module_Change_a__Travel_Time_bar_chart}
\end{figure}

This bar chart presents the relationship between modules and total travel time for the 500 nodes Module Change experimental scenario. The x-axis displays modules values ranging from 4 to 7, while the y-axis quantifies total travel time. The single-series visualization facilitates analysis of how the dependent variable responds to changes in the independent parameter setting.

Analysis of the plotted data reveals that total travel time ranges from 3287.54 (at modules = 4) to 3287.54 (at modules = 4), representing a span of 0.00 units. The values remain relatively stable across the parameter range, with minimal net change between initial (3287.54) and final (3287.54) settings. 

These results indicate that modules configuration meaningfully impacts total travel time in this experimental context. The relatively modest variation suggests that this parameter has limited influence on the measured metric within the tested range. Confidence in these findings is high given the direct correspondence between CSV data and plotted values. Future analysis should consider incorporating error bars representing variance across multiple experimental runs to strengthen statistical validity.

\begin{figure}[!htbp]
\centering
\includegraphics[width=0.95\linewidth]{images/500_nodes_Module_Change/(b)_Energy_consumption_bar_chart.png}
\caption{(b) Energy consumption}
\label{fig:500_nodes_Module_Change_b__Energy_consumption_bar_chart}
\end{figure}

This bar chart presents the relationship between modules and total energy consumed for the 500 nodes Module Change experimental scenario. The x-axis displays modules values ranging from 4 to 7, while the y-axis quantifies total energy consumed. The single-series visualization facilitates analysis of how the dependent variable responds to changes in the independent parameter setting.

Analysis of the plotted data reveals that total energy consumed ranges from 323.50 (at modules = 4) to 323.50 (at modules = 4), representing a span of 0.00 units. The values remain relatively stable across the parameter range, with minimal net change between initial (323.50) and final (323.50) settings. 

These results indicate that modules configuration meaningfully impacts total energy consumed in this experimental context. The relatively modest variation suggests that this parameter has limited influence on the measured metric within the tested range. Confidence in these findings is high given the direct correspondence between CSV data and plotted values. Future analysis should consider incorporating error bars representing variance across multiple experimental runs to strengthen statistical validity.

\begin{figure}[!htbp]
\centering
\includegraphics[width=0.95\linewidth]{images/500_nodes_Module_Change/(c)_Distance_covered_bar_chart.png}
\caption{(c) Distance covered}
\label{fig:500_nodes_Module_Change_c__Distance_covered_bar_chart}
\end{figure}

This bar chart presents the relationship between modules and total distance covered for the 500 nodes Module Change experimental scenario. The x-axis displays modules values ranging from 4 to 7, while the y-axis quantifies total distance covered. The single-series visualization facilitates analysis of how the dependent variable responds to changes in the independent parameter setting.

Analysis of the plotted data reveals that total distance covered ranges from 2138.91 (at modules = 4) to 2138.91 (at modules = 4), representing a span of 0.00 units. The values remain relatively stable across the parameter range, with minimal net change between initial (2138.91) and final (2138.91) settings. 

These results indicate that modules configuration meaningfully impacts total distance covered in this experimental context. The relatively modest variation suggests that this parameter has limited influence on the measured metric within the tested range. Confidence in these findings is high given the direct correspondence between CSV data and plotted values. Future analysis should consider incorporating error bars representing variance across multiple experimental runs to strengthen statistical validity.

\begin{figure}[!htbp]
\centering
\includegraphics[width=0.95\linewidth]{images/500_nodes_Module_Change/(d)_Runtime_bar_chart.png}
\caption{(d) Runtime}
\label{fig:500_nodes_Module_Change_d__Runtime_bar_chart}
\end{figure}

This bar chart presents the relationship between modules and run time for the 500 nodes Module Change experimental scenario. The x-axis displays modules values ranging from 4 to 7, while the y-axis quantifies run time. The single-series visualization facilitates analysis of how the dependent variable responds to changes in the independent parameter setting.

Analysis of the plotted data reveals that run time ranges from 29.02 (at modules = 5) to 62.16 (at modules = 6), representing a span of 33.14 units. The overall trend is decreasing, with values declining from 58.53 at the initial setting to 36.14 at the final setting. Notably, the minimum value occurs at an intermediate modules setting (5), suggesting non-monotonic behavior that warrants further investigation.

These results indicate that modules configuration meaningfully impacts run time in this experimental context. The substantial variation observed (coefficient of variation exceeding 20\%) suggests that parameter tuning could yield significant performance improvements. Confidence in these findings is high given the direct correspondence between CSV data and plotted values. Future analysis should consider incorporating error bars representing variance across multiple experimental runs to strengthen statistical validity.

\begin{figure}[!htbp]
\centering
\includegraphics[width=0.95\linewidth]{images/500_nodes_Module_Change/(e)_Number_of_module_swapped_bar_chart.png}
\caption{(e) Number of module swapped}
\label{fig:500_nodes_Module_Change_e__Number_of_module_swapped_bar_chart}
\end{figure}

This bar chart presents the relationship between modules and modules for the 500 nodes Module Change experimental scenario. The x-axis displays modules values ranging from 4 to 7, while the y-axis quantifies modules. The single-series visualization facilitates analysis of how the dependent variable responds to changes in the independent parameter setting.

Analysis of the plotted data reveals that modules ranges from 4.00 (at modules = 4) to 7.00 (at modules = 7), representing a span of 3.00 units. The overall trend is increasing, with values rising from 4.00 at the initial setting to 7.00 at the final setting. 

These results indicate that modules configuration meaningfully impacts modules in this experimental context. The substantial variation observed (coefficient of variation exceeding 20\%) suggests that parameter tuning could yield significant performance improvements. Confidence in these findings is high given the direct correspondence between CSV data and plotted values. Future analysis should consider incorporating error bars representing variance across multiple experimental runs to strengthen statistical validity.


\clearpage

\subsection{500 nodes Swapping Time}

\begin{figure}[!htbp]
\centering
\includegraphics[width=0.95\linewidth]{images/500_nodes_Swapping_Time/(a)_Travel_Time_bar_chart.png}
\caption{(a) Travel Time}
\label{fig:500_nodes_Swapping_Time_a__Travel_Time_bar_chart}
\end{figure}

This bar chart presents the relationship between swapping time and total travel time for the 500 nodes Swapping Time experimental scenario. The x-axis displays swapping time values ranging from 1 to 4, while the y-axis quantifies total travel time. The single-series visualization facilitates analysis of how the dependent variable responds to changes in the independent parameter setting.

Analysis of the plotted data reveals that total travel time ranges from 3268.67 (at swapping time = 1) to 3317.54 (at swapping time = 4), representing a span of 48.87 units. The overall trend is increasing, with values rising from 3268.67 at the initial setting to 3317.54 at the final setting. 

These results indicate that swapping time configuration meaningfully impacts total travel time in this experimental context. The relatively modest variation suggests that this parameter has limited influence on the measured metric within the tested range. Confidence in these findings is high given the direct correspondence between CSV data and plotted values. Future analysis should consider incorporating error bars representing variance across multiple experimental runs to strengthen statistical validity.

\begin{figure}[!htbp]
\centering
\includegraphics[width=0.95\linewidth]{images/500_nodes_Swapping_Time/(b)_Energy_consumption_bar_chart.png}
\caption{(b) Energy consumption}
\label{fig:500_nodes_Swapping_Time_b__Energy_consumption_bar_chart}
\end{figure}

This bar chart presents the relationship between swapping time and total energy consumed for the 500 nodes Swapping Time experimental scenario. The x-axis displays swapping time values ranging from 1 to 4, while the y-axis quantifies total energy consumed. The single-series visualization facilitates analysis of how the dependent variable responds to changes in the independent parameter setting.

Analysis of the plotted data reveals that total energy consumed ranges from 323.44 (at swapping time = 1) to 323.50 (at swapping time = 2), representing a span of 0.06 units. The overall trend is increasing, with values rising from 323.44 at the initial setting to 323.50 at the final setting. 

These results indicate that swapping time configuration meaningfully impacts total energy consumed in this experimental context. The relatively modest variation suggests that this parameter has limited influence on the measured metric within the tested range. Confidence in these findings is high given the direct correspondence between CSV data and plotted values. Future analysis should consider incorporating error bars representing variance across multiple experimental runs to strengthen statistical validity.

\begin{figure}[!htbp]
\centering
\includegraphics[width=0.95\linewidth]{images/500_nodes_Swapping_Time/(c)_Distance_covered_bar_chart.png}
\caption{(c) Distance covered}
\label{fig:500_nodes_Swapping_Time_c__Distance_covered_bar_chart}
\end{figure}

This bar chart presents the relationship between swapping time and total distance covered for the 500 nodes Swapping Time experimental scenario. The x-axis displays swapping time values ranging from 1 to 4, while the y-axis quantifies total distance covered. The single-series visualization facilitates analysis of how the dependent variable responds to changes in the independent parameter setting.

Analysis of the plotted data reveals that total distance covered ranges from 2138.07 (at swapping time = 1) to 2138.91 (at swapping time = 2), representing a span of 0.84 units. The overall trend is increasing, with values rising from 2138.07 at the initial setting to 2138.91 at the final setting. 

These results indicate that swapping time configuration meaningfully impacts total distance covered in this experimental context. The relatively modest variation suggests that this parameter has limited influence on the measured metric within the tested range. Confidence in these findings is high given the direct correspondence between CSV data and plotted values. Future analysis should consider incorporating error bars representing variance across multiple experimental runs to strengthen statistical validity.

\begin{figure}[!htbp]
\centering
\includegraphics[width=0.95\linewidth]{images/500_nodes_Swapping_Time/(d)_Runtime_bar_chart.png}
\caption{(d) Runtime}
\label{fig:500_nodes_Swapping_Time_d__Runtime_bar_chart}
\end{figure}

This bar chart presents the relationship between swapping time and run time for the 500 nodes Swapping Time experimental scenario. The x-axis displays swapping time values ranging from 1 to 4, while the y-axis quantifies run time. The single-series visualization facilitates analysis of how the dependent variable responds to changes in the independent parameter setting.

Analysis of the plotted data reveals that run time ranges from 27.73 (at swapping time = 3) to 29.02 (at swapping time = 2), representing a span of 1.28 units. The overall trend is increasing, with values rising from 27.75 at the initial setting to 28.57 at the final setting. Notably, the minimum value occurs at an intermediate swapping time setting (3), suggesting non-monotonic behavior that warrants further investigation.

These results indicate that swapping time configuration meaningfully impacts run time in this experimental context. The relatively modest variation suggests that this parameter has limited influence on the measured metric within the tested range. Confidence in these findings is high given the direct correspondence between CSV data and plotted values. Future analysis should consider incorporating error bars representing variance across multiple experimental runs to strengthen statistical validity.

\begin{figure}[!htbp]
\centering
\includegraphics[width=0.95\linewidth]{images/500_nodes_Swapping_Time/(e)_Number_of_module_swapped_bar_chart.png}
\caption{(e) Number of module swapped}
\label{fig:500_nodes_Swapping_Time_e__Number_of_module_swapped_bar_chart}
\end{figure}

This bar chart presents the relationship between swapping time and total module swapped for the 500 nodes Swapping Time experimental scenario. The x-axis displays swapping time values ranging from 1 to 4, while the y-axis quantifies total module swapped. The single-series visualization facilitates analysis of how the dependent variable responds to changes in the independent parameter setting.

Analysis of the plotted data reveals that total module swapped ranges from 14.00 (at swapping time = 1) to 15.00 (at swapping time = 2), representing a span of 1.00 units. The overall trend is increasing, with values rising from 14.00 at the initial setting to 15.00 at the final setting. 

These results indicate that swapping time configuration meaningfully impacts total module swapped in this experimental context. The relatively modest variation suggests that this parameter has limited influence on the measured metric within the tested range. Confidence in these findings is high given the direct correspondence between CSV data and plotted values. Future analysis should consider incorporating error bars representing variance across multiple experimental runs to strengthen statistical validity.


\clearpage

\subsection{500 nodes Threshold}

\begin{figure}[!htbp]
\centering
\includegraphics[width=0.95\linewidth]{images/500_nodes_Threshold/(a)_Travel_Time_bar_chart.png}
\caption{(a) Travel Time}
\label{fig:500_nodes_Threshold_a__Travel_Time_bar_chart}
\end{figure}

This bar chart presents the relationship between threshold and total travel time for the 500 nodes Threshold experimental scenario. The x-axis displays threshold values ranging from 5 to 20, while the y-axis quantifies total travel time. The single-series visualization facilitates analysis of how the dependent variable responds to changes in the independent parameter setting.

Analysis of the plotted data reveals that total travel time ranges from 3287.54 (at threshold = 5) to 3287.54 (at threshold = 5), representing a span of 0.00 units. The values remain relatively stable across the parameter range, with minimal net change between initial (3287.54) and final (3287.54) settings. 

These results indicate that threshold configuration meaningfully impacts total travel time in this experimental context. The relatively modest variation suggests that this parameter has limited influence on the measured metric within the tested range. Confidence in these findings is high given the direct correspondence between CSV data and plotted values. Future analysis should consider incorporating error bars representing variance across multiple experimental runs to strengthen statistical validity.

\begin{figure}[!htbp]
\centering
\includegraphics[width=0.95\linewidth]{images/500_nodes_Threshold/(b)_Energy_consumption_bar_chart.png}
\caption{(b) Energy consumption}
\label{fig:500_nodes_Threshold_b__Energy_consumption_bar_chart}
\end{figure}

This bar chart presents the relationship between threshold and total energy consumed for the 500 nodes Threshold experimental scenario. The x-axis displays threshold values ranging from 5 to 20, while the y-axis quantifies total energy consumed. The single-series visualization facilitates analysis of how the dependent variable responds to changes in the independent parameter setting.

Analysis of the plotted data reveals that total energy consumed ranges from 323.50 (at threshold = 5) to 323.50 (at threshold = 5), representing a span of 0.00 units. The values remain relatively stable across the parameter range, with minimal net change between initial (323.50) and final (323.50) settings. 

These results indicate that threshold configuration meaningfully impacts total energy consumed in this experimental context. The relatively modest variation suggests that this parameter has limited influence on the measured metric within the tested range. Confidence in these findings is high given the direct correspondence between CSV data and plotted values. Future analysis should consider incorporating error bars representing variance across multiple experimental runs to strengthen statistical validity.

\begin{figure}[!htbp]
\centering
\includegraphics[width=0.95\linewidth]{images/500_nodes_Threshold/(c)_Distance_covered_bar_chart.png}
\caption{(c) Distance covered}
\label{fig:500_nodes_Threshold_c__Distance_covered_bar_chart}
\end{figure}

This bar chart presents the relationship between threshold and total distance covered for the 500 nodes Threshold experimental scenario. The x-axis displays threshold values ranging from 5 to 20, while the y-axis quantifies total distance covered. The single-series visualization facilitates analysis of how the dependent variable responds to changes in the independent parameter setting.

Analysis of the plotted data reveals that total distance covered ranges from 2138.91 (at threshold = 5) to 2138.91 (at threshold = 5), representing a span of 0.00 units. The values remain relatively stable across the parameter range, with minimal net change between initial (2138.91) and final (2138.91) settings. 

These results indicate that threshold configuration meaningfully impacts total distance covered in this experimental context. The relatively modest variation suggests that this parameter has limited influence on the measured metric within the tested range. Confidence in these findings is high given the direct correspondence between CSV data and plotted values. Future analysis should consider incorporating error bars representing variance across multiple experimental runs to strengthen statistical validity.

\begin{figure}[!htbp]
\centering
\includegraphics[width=0.95\linewidth]{images/500_nodes_Threshold/(d)_Runtime_bar_chart.png}
\caption{(d) Runtime}
\label{fig:500_nodes_Threshold_d__Runtime_bar_chart}
\end{figure}

This bar chart presents the relationship between threshold and run time for the 500 nodes Threshold experimental scenario. The x-axis displays threshold values ranging from 5 to 20, while the y-axis quantifies run time. The single-series visualization facilitates analysis of how the dependent variable responds to changes in the independent parameter setting.

Analysis of the plotted data reveals that run time ranges from 29.02 (at threshold = 20) to 66.47 (at threshold = 10), representing a span of 37.45 units. The overall trend is decreasing, with values declining from 58.34 at the initial setting to 29.02 at the final setting. 

These results indicate that threshold configuration meaningfully impacts run time in this experimental context. The substantial variation observed (coefficient of variation exceeding 20\%) suggests that parameter tuning could yield significant performance improvements. Confidence in these findings is high given the direct correspondence between CSV data and plotted values. Future analysis should consider incorporating error bars representing variance across multiple experimental runs to strengthen statistical validity.

\begin{figure}[!htbp]
\centering
\includegraphics[width=0.95\linewidth]{images/500_nodes_Threshold/(e)_Number_of_module_swapped_bar_chart.png}
\caption{(e) Number of module swapped}
\label{fig:500_nodes_Threshold_e__Number_of_module_swapped_bar_chart}
\end{figure}

This bar chart presents the relationship between threshold and total module swapped for the 500 nodes Threshold experimental scenario. The x-axis displays threshold values ranging from 5 to 20, while the y-axis quantifies total module swapped. The single-series visualization facilitates analysis of how the dependent variable responds to changes in the independent parameter setting.

Analysis of the plotted data reveals that total module swapped ranges from 15.00 (at threshold = 5) to 15.00 (at threshold = 5), representing a span of 0.00 units. The values remain relatively stable across the parameter range, with minimal net change between initial (15.00) and final (15.00) settings. 

These results indicate that threshold configuration meaningfully impacts total module swapped in this experimental context. The relatively modest variation suggests that this parameter has limited influence on the measured metric within the tested range. Confidence in these findings is high given the direct correspondence between CSV data and plotted values. Future analysis should consider incorporating error bars representing variance across multiple experimental runs to strengthen statistical validity.


\clearpage

\subsection{500 nodes Traffic}

\begin{figure}[!htbp]
\centering
\includegraphics[width=0.95\linewidth]{images/500_nodes_Traffic/(a)_Travel_Time_bar_chart.png}
\caption{(a) Travel Time}
\label{fig:500_nodes_Traffic_a__Travel_Time_bar_chart}
\end{figure}

This bar chart presents the relationship between traffic and total travel time for the 500 nodes Traffic experimental scenario. The x-axis displays traffic values ranging from High to Low, while the y-axis quantifies total travel time. The single-series visualization facilitates analysis of how the dependent variable responds to changes in the independent parameter setting.

Analysis of the plotted data reveals that total travel time ranges from 2915.88 (at traffic = Low) to 3618.28 (at traffic = High), representing a span of 702.40 units. The overall trend is decreasing, with values declining from 3618.28 at the initial setting to 2915.88 at the final setting. 

These results indicate that traffic configuration meaningfully impacts total travel time in this experimental context. The substantial variation observed (coefficient of variation exceeding 20\%) suggests that parameter tuning could yield significant performance improvements. Confidence in these findings is high given the direct correspondence between CSV data and plotted values. Future analysis should consider incorporating error bars representing variance across multiple experimental runs to strengthen statistical validity.

\begin{figure}[!htbp]
\centering
\includegraphics[width=0.95\linewidth]{images/500_nodes_Traffic/(b)_Energy_consumption_bar_chart.png}
\caption{(b) Energy consumption}
\label{fig:500_nodes_Traffic_b__Energy_consumption_bar_chart}
\end{figure}

This bar chart presents the relationship between traffic and total energy consumed for the 500 nodes Traffic experimental scenario. The x-axis displays traffic values ranging from High to Low, while the y-axis quantifies total energy consumed. The single-series visualization facilitates analysis of how the dependent variable responds to changes in the independent parameter setting.

Analysis of the plotted data reveals that total energy consumed ranges from 287.80 (at traffic = High) to 351.10 (at traffic = Low), representing a span of 63.29 units. The overall trend is increasing, with values rising from 287.80 at the initial setting to 351.10 at the final setting. 

These results indicate that traffic configuration meaningfully impacts total energy consumed in this experimental context. The substantial variation observed (coefficient of variation exceeding 20\%) suggests that parameter tuning could yield significant performance improvements. Confidence in these findings is high given the direct correspondence between CSV data and plotted values. Future analysis should consider incorporating error bars representing variance across multiple experimental runs to strengthen statistical validity.

\begin{figure}[!htbp]
\centering
\includegraphics[width=0.95\linewidth]{images/500_nodes_Traffic/(c)_Distance_covered_bar_chart.png}
\caption{(c) Distance covered}
\label{fig:500_nodes_Traffic_c__Distance_covered_bar_chart}
\end{figure}

This bar chart presents the relationship between traffic and total distance covered for the 500 nodes Traffic experimental scenario. The x-axis displays traffic values ranging from High to Low, while the y-axis quantifies total distance covered. The single-series visualization facilitates analysis of how the dependent variable responds to changes in the independent parameter setting.

Analysis of the plotted data reveals that total distance covered ranges from 1986.78 (at traffic = High) to 2241.45 (at traffic = Low), representing a span of 254.67 units. The overall trend is increasing, with values rising from 1986.78 at the initial setting to 2241.45 at the final setting. 

These results indicate that traffic configuration meaningfully impacts total distance covered in this experimental context. The relatively modest variation suggests that this parameter has limited influence on the measured metric within the tested range. Confidence in these findings is high given the direct correspondence between CSV data and plotted values. Future analysis should consider incorporating error bars representing variance across multiple experimental runs to strengthen statistical validity.

\begin{figure}[!htbp]
\centering
\includegraphics[width=0.95\linewidth]{images/500_nodes_Traffic/(d)_Runtime_bar_chart.png}
\caption{(d) Runtime}
\label{fig:500_nodes_Traffic_d__Runtime_bar_chart}
\end{figure}

This bar chart presents the relationship between traffic and run time for the 500 nodes Traffic experimental scenario. The x-axis displays traffic values ranging from High to Low, while the y-axis quantifies run time. The single-series visualization facilitates analysis of how the dependent variable responds to changes in the independent parameter setting.

Analysis of the plotted data reveals that run time ranges from 37.23 (at traffic = High) to 54.49 (at traffic = Low), representing a span of 17.26 units. The overall trend is increasing, with values rising from 37.23 at the initial setting to 54.49 at the final setting. 

These results indicate that traffic configuration meaningfully impacts run time in this experimental context. The substantial variation observed (coefficient of variation exceeding 20\%) suggests that parameter tuning could yield significant performance improvements. Confidence in these findings is high given the direct correspondence between CSV data and plotted values. Future analysis should consider incorporating error bars representing variance across multiple experimental runs to strengthen statistical validity.

\begin{figure}[!htbp]
\centering
\includegraphics[width=0.95\linewidth]{images/500_nodes_Traffic/(e)_Number_of_module_swapped_bar_chart.png}
\caption{(e) Number of module swapped}
\label{fig:500_nodes_Traffic_e__Number_of_module_swapped_bar_chart}
\end{figure}

This bar chart presents the relationship between traffic and total module swapped for the 500 nodes Traffic experimental scenario. The x-axis displays traffic values ranging from High to Low, while the y-axis quantifies total module swapped. The single-series visualization facilitates analysis of how the dependent variable responds to changes in the independent parameter setting.

Analysis of the plotted data reveals that total module swapped ranges from 12.00 (at traffic = High) to 16.00 (at traffic = Low), representing a span of 4.00 units. The overall trend is increasing, with values rising from 12.00 at the initial setting to 16.00 at the final setting. 

These results indicate that traffic configuration meaningfully impacts total module swapped in this experimental context. The substantial variation observed (coefficient of variation exceeding 20\%) suggests that parameter tuning could yield significant performance improvements. Confidence in these findings is high given the direct correspondence between CSV data and plotted values. Future analysis should consider incorporating error bars representing variance across multiple experimental runs to strengthen statistical validity.


\clearpage

\subsection{MT2TE and Other Program Module Change}

\begin{figure}[!htbp]
\centering
\includegraphics[width=0.95\linewidth]{images/MT2TE_and_Other_Program_Module_Change/(a)_Total_Travel_Time.png}
\caption{(a) Total Travel Time}
\label{fig:MT2TE_and_Other_Program_Module_Change_a__Total_Travel_Time}
\end{figure}

This grouped bar chart presents a comparative analysis of total travel time across multiple algorithms within the MT2TE and Other Program Module Change experimental configuration. The x-axis represents the number of nodes in the network, while the y-axis quantifies the total travel time metric. The legend identifies 4 distinct algorithms: Genetic Algorithm, EVRPBSS, Ant Colony, Clarke and Wright algorithm. This visualization enables direct comparison of algorithmic performance under identical network conditions.

Quantitative analysis reveals significant performance disparities among the evaluated algorithms. Clarke and Wright algorithm demonstrates superior performance with a mean total travel time of 688.78, while Genetic Algorithm exhibits the highest values averaging 1366.67. This represents an improvement of approximately 49.6\% when comparing the best to worst performing algorithms. The relative ranking of algorithms remains largely consistent across different node configurations, suggesting robust performance characteristics.

These findings support the hypothesis that algorithmic choice significantly impacts system performance metrics. Future work should incorporate statistical significance testing and confidence intervals to strengthen these comparative conclusions. Additionally, examining the computational complexity trade-offs between algorithms would provide valuable context for practical deployment decisions.

\begin{figure}[!htbp]
\centering
\includegraphics[width=0.95\linewidth]{images/MT2TE_and_Other_Program_Module_Change/(b)_Energy.png}
\caption{(b) Energy}
\label{fig:MT2TE_and_Other_Program_Module_Change_b__Energy}
\end{figure}

This grouped bar chart presents a comparative analysis of energy across multiple algorithms within the MT2TE and Other Program Module Change experimental configuration. The x-axis represents the number of nodes in the network, while the y-axis quantifies the energy metric. The legend identifies 4 distinct algorithms: Genetic Algorithm, EVRPBSS, Ant Colony, Clarke and Wright algorithm. This visualization enables direct comparison of algorithmic performance under identical network conditions.

Quantitative analysis reveals significant performance disparities among the evaluated algorithms. Clarke and Wright algorithm demonstrates superior performance with a mean energy of 688.78, while Genetic Algorithm exhibits the highest values averaging 1366.67. This represents an improvement of approximately 49.6\% when comparing the best to worst performing algorithms. The relative ranking of algorithms remains largely consistent across different node configurations, suggesting robust performance characteristics.

These findings support the hypothesis that algorithmic choice significantly impacts system performance metrics. Future work should incorporate statistical significance testing and confidence intervals to strengthen these comparative conclusions. Additionally, examining the computational complexity trade-offs between algorithms would provide valuable context for practical deployment decisions.

\begin{figure}[!htbp]
\centering
\includegraphics[width=0.95\linewidth]{images/MT2TE_and_Other_Program_Module_Change/(c)_Distance.png}
\caption{(c) Distance}
\label{fig:MT2TE_and_Other_Program_Module_Change_c__Distance}
\end{figure}

This grouped bar chart presents a comparative analysis of distance across multiple algorithms within the MT2TE and Other Program Module Change experimental configuration. The x-axis represents the number of nodes in the network, while the y-axis quantifies the distance metric. The legend identifies 4 distinct algorithms: Genetic Algorithm, EVRPBSS, Ant Colony, Clarke and Wright algorithm. This visualization enables direct comparison of algorithmic performance under identical network conditions.

Quantitative analysis reveals significant performance disparities among the evaluated algorithms. Clarke and Wright algorithm demonstrates superior performance with a mean distance of 688.78, while Genetic Algorithm exhibits the highest values averaging 1366.67. This represents an improvement of approximately 49.6\% when comparing the best to worst performing algorithms. The relative ranking of algorithms remains largely consistent across different node configurations, suggesting robust performance characteristics.

These findings support the hypothesis that algorithmic choice significantly impacts system performance metrics. Future work should incorporate statistical significance testing and confidence intervals to strengthen these comparative conclusions. Additionally, examining the computational complexity trade-offs between algorithms would provide valuable context for practical deployment decisions.

\begin{figure}[!htbp]
\centering
\includegraphics[width=0.95\linewidth]{images/MT2TE_and_Other_Program_Module_Change/(d)_Module_Swapped.png}
\caption{(d) Module Swapped}
\label{fig:MT2TE_and_Other_Program_Module_Change_d__Module_Swapped}
\end{figure}

This grouped bar chart presents a comparative analysis of module swapped across multiple algorithms within the MT2TE and Other Program Module Change experimental configuration. The x-axis represents the number of nodes in the network, while the y-axis quantifies the module swapped metric. The legend identifies 4 distinct algorithms: Genetic Algorithm, EVRPBSS, Ant Colony, Clarke and Wright algorithm. This visualization enables direct comparison of algorithmic performance under identical network conditions.

Quantitative analysis reveals significant performance disparities among the evaluated algorithms. Clarke and Wright algorithm demonstrates superior performance with a mean module swapped of 688.78, while Genetic Algorithm exhibits the highest values averaging 1366.67. This represents an improvement of approximately 49.6\% when comparing the best to worst performing algorithms. The relative ranking of algorithms remains largely consistent across different node configurations, suggesting robust performance characteristics.

These findings support the hypothesis that algorithmic choice significantly impacts system performance metrics. Future work should incorporate statistical significance testing and confidence intervals to strengthen these comparative conclusions. Additionally, examining the computational complexity trade-offs between algorithms would provide valuable context for practical deployment decisions.

\begin{figure}[!htbp]
\centering
\includegraphics[width=0.95\linewidth]{images/MT2TE_and_Other_Program_Module_Change/(e)_Execution_Time.png}
\caption{(e) Execution Time}
\label{fig:MT2TE_and_Other_Program_Module_Change_e__Execution_Time}
\end{figure}

This grouped bar chart presents a comparative analysis of execution time across multiple algorithms within the MT2TE and Other Program Module Change experimental configuration. The x-axis represents the number of nodes in the network, while the y-axis quantifies the execution time metric. The legend identifies 4 distinct algorithms: Genetic Algorithm, EVRPBSS, Ant Colony, Clarke and Wright algorithm. This visualization enables direct comparison of algorithmic performance under identical network conditions.

Quantitative analysis reveals significant performance disparities among the evaluated algorithms. Clarke and Wright algorithm demonstrates superior performance with a mean execution time of 688.78, while Genetic Algorithm exhibits the highest values averaging 1366.67. This represents an improvement of approximately 49.6\% when comparing the best to worst performing algorithms. The relative ranking of algorithms remains largely consistent across different node configurations, suggesting robust performance characteristics.

These findings support the hypothesis that algorithmic choice significantly impacts system performance metrics. Future work should incorporate statistical significance testing and confidence intervals to strengthen these comparative conclusions. Additionally, examining the computational complexity trade-offs between algorithms would provide valuable context for practical deployment decisions.

\begin{figure}[!htbp]
\centering
\includegraphics[width=0.95\linewidth]{images/MT2TE_and_Other_Program_Module_Change/(f)_Execution_Time.png}
\caption{(f) Execution Time}
\label{fig:MT2TE_and_Other_Program_Module_Change_f__Execution_Time}
\end{figure}

This grouped bar chart presents a comparative analysis of execution time across multiple algorithms within the MT2TE and Other Program Module Change experimental configuration. The x-axis represents the number of nodes in the network, while the y-axis quantifies the execution time metric. The legend identifies 3 distinct algorithms: EVRPBSS, Ant Colony, Clarke and Wright algorithm. This visualization enables direct comparison of algorithmic performance under identical network conditions.

Quantitative analysis reveals significant performance disparities among the evaluated algorithms. Clarke and Wright algorithm demonstrates superior performance with a mean execution time of 688.78, while Ant Colony exhibits the highest values averaging 1020.01. This represents an improvement of approximately 32.5\% when comparing the best to worst performing algorithms. The relative ranking of algorithms remains largely consistent across different node configurations, suggesting robust performance characteristics.

These findings support the hypothesis that algorithmic choice significantly impacts system performance metrics. Future work should incorporate statistical significance testing and confidence intervals to strengthen these comparative conclusions. Additionally, examining the computational complexity trade-offs between algorithms would provide valuable context for practical deployment decisions.


\clearpage

\subsection{MT2TE and Other Program Node Change}

\begin{figure}[!htbp]
\centering
\includegraphics[width=0.95\linewidth]{images/MT2TE_and_Other_Program_Node_Change/(a)_Total_Travel_Time.png}
\caption{(a) Total Travel Time}
\label{fig:MT2TE_and_Other_Program_Node_Change_a__Total_Travel_Time}
\end{figure}

This grouped bar chart presents a comparative analysis of total travel time across multiple algorithms within the MT2TE and Other Program Node Change experimental configuration. The x-axis represents the number of nodes in the network, while the y-axis quantifies the total travel time metric. The legend identifies 4 distinct algorithms: Genetic Algorithm, EVRPBSS, Ant Colony, Clarke and Wright algorithm. This visualization enables direct comparison of algorithmic performance under identical network conditions.

Quantitative analysis reveals significant performance disparities among the evaluated algorithms. Clarke and Wright algorithm demonstrates superior performance with a mean total travel time of 862.43, while Genetic Algorithm exhibits the highest values averaging 2037.20. This represents an improvement of approximately 57.7\% when comparing the best to worst performing algorithms. The relative ranking of algorithms remains largely consistent across different node configurations, suggesting robust performance characteristics.

These findings support the hypothesis that algorithmic choice significantly impacts system performance metrics. Future work should incorporate statistical significance testing and confidence intervals to strengthen these comparative conclusions. Additionally, examining the computational complexity trade-offs between algorithms would provide valuable context for practical deployment decisions.

\begin{figure}[!htbp]
\centering
\includegraphics[width=0.95\linewidth]{images/MT2TE_and_Other_Program_Node_Change/(b)_Energy.png}
\caption{(b) Energy}
\label{fig:MT2TE_and_Other_Program_Node_Change_b__Energy}
\end{figure}

This grouped bar chart presents a comparative analysis of energy across multiple algorithms within the MT2TE and Other Program Node Change experimental configuration. The x-axis represents the number of nodes in the network, while the y-axis quantifies the energy metric. The legend identifies 4 distinct algorithms: Genetic Algorithm, EVRPBSS, Ant Colony, Clarke and Wright algorithm. This visualization enables direct comparison of algorithmic performance under identical network conditions.

Quantitative analysis reveals significant performance disparities among the evaluated algorithms. Clarke and Wright algorithm demonstrates superior performance with a mean energy of 862.43, while Genetic Algorithm exhibits the highest values averaging 2037.20. This represents an improvement of approximately 57.7\% when comparing the best to worst performing algorithms. The relative ranking of algorithms remains largely consistent across different node configurations, suggesting robust performance characteristics.

These findings support the hypothesis that algorithmic choice significantly impacts system performance metrics. Future work should incorporate statistical significance testing and confidence intervals to strengthen these comparative conclusions. Additionally, examining the computational complexity trade-offs between algorithms would provide valuable context for practical deployment decisions.

\begin{figure}[!htbp]
\centering
\includegraphics[width=0.95\linewidth]{images/MT2TE_and_Other_Program_Node_Change/(c)_Distance.png}
\caption{(c) Distance}
\label{fig:MT2TE_and_Other_Program_Node_Change_c__Distance}
\end{figure}

This grouped bar chart presents a comparative analysis of distance across multiple algorithms within the MT2TE and Other Program Node Change experimental configuration. The x-axis represents the number of nodes in the network, while the y-axis quantifies the distance metric. The legend identifies 4 distinct algorithms: Genetic Algorithm, EVRPBSS, Ant Colony, Clarke and Wright algorithm. This visualization enables direct comparison of algorithmic performance under identical network conditions.

Quantitative analysis reveals significant performance disparities among the evaluated algorithms. Clarke and Wright algorithm demonstrates superior performance with a mean distance of 862.43, while Genetic Algorithm exhibits the highest values averaging 2037.20. This represents an improvement of approximately 57.7\% when comparing the best to worst performing algorithms. The relative ranking of algorithms remains largely consistent across different node configurations, suggesting robust performance characteristics.

These findings support the hypothesis that algorithmic choice significantly impacts system performance metrics. Future work should incorporate statistical significance testing and confidence intervals to strengthen these comparative conclusions. Additionally, examining the computational complexity trade-offs between algorithms would provide valuable context for practical deployment decisions.

\begin{figure}[!htbp]
\centering
\includegraphics[width=0.95\linewidth]{images/MT2TE_and_Other_Program_Node_Change/(d)_Execution_Time.png}
\caption{(d) Execution Time}
\label{fig:MT2TE_and_Other_Program_Node_Change_d__Execution_Time}
\end{figure}

This grouped bar chart presents a comparative analysis of execution time across multiple algorithms within the MT2TE and Other Program Node Change experimental configuration. The x-axis represents the number of nodes in the network, while the y-axis quantifies the execution time metric. The legend identifies 4 distinct algorithms: Genetic Algorithm, EVRPBSS, Ant Colony, Clarke and Wright algorithm. This visualization enables direct comparison of algorithmic performance under identical network conditions.

Quantitative analysis reveals significant performance disparities among the evaluated algorithms. Clarke and Wright algorithm demonstrates superior performance with a mean execution time of 862.43, while Genetic Algorithm exhibits the highest values averaging 2037.20. This represents an improvement of approximately 57.7\% when comparing the best to worst performing algorithms. The relative ranking of algorithms remains largely consistent across different node configurations, suggesting robust performance characteristics.

These findings support the hypothesis that algorithmic choice significantly impacts system performance metrics. Future work should incorporate statistical significance testing and confidence intervals to strengthen these comparative conclusions. Additionally, examining the computational complexity trade-offs between algorithms would provide valuable context for practical deployment decisions.

\begin{figure}[!htbp]
\centering
\includegraphics[width=0.95\linewidth]{images/MT2TE_and_Other_Program_Node_Change/(e)_Total_Module_Swapped.png}
\caption{(e) Total Module Swapped}
\label{fig:MT2TE_and_Other_Program_Node_Change_e__Total_Module_Swapped}
\end{figure}

This grouped bar chart presents a comparative analysis of total module swapped across multiple algorithms within the MT2TE and Other Program Node Change experimental configuration. The x-axis represents the number of nodes in the network, while the y-axis quantifies the total module swapped metric. The legend identifies 4 distinct algorithms: Genetic Algorithm, EVRPBSS, Ant Colony, Clarke and Wright algorithm. This visualization enables direct comparison of algorithmic performance under identical network conditions.

Quantitative analysis reveals significant performance disparities among the evaluated algorithms. Clarke and Wright algorithm demonstrates superior performance with a mean total module swapped of 862.43, while Genetic Algorithm exhibits the highest values averaging 2037.20. This represents an improvement of approximately 57.7\% when comparing the best to worst performing algorithms. The relative ranking of algorithms remains largely consistent across different node configurations, suggesting robust performance characteristics.

These findings support the hypothesis that algorithmic choice significantly impacts system performance metrics. Future work should incorporate statistical significance testing and confidence intervals to strengthen these comparative conclusions. Additionally, examining the computational complexity trade-offs between algorithms would provide valuable context for practical deployment decisions.

\begin{figure}[!htbp]
\centering
\includegraphics[width=0.95\linewidth]{images/MT2TE_and_Other_Program_Node_Change/(f)_Execution_Time.png}
\caption{(f) Execution Time}
\label{fig:MT2TE_and_Other_Program_Node_Change_f__Execution_Time}
\end{figure}

This grouped bar chart presents a comparative analysis of execution time across multiple algorithms within the MT2TE and Other Program Node Change experimental configuration. The x-axis represents the number of nodes in the network, while the y-axis quantifies the execution time metric. The legend identifies 4 distinct algorithms: Genetic Algorithm, EVRPBSS, Ant Colony, Clarke and Wright algorithm. This visualization enables direct comparison of algorithmic performance under identical network conditions.

Quantitative analysis reveals significant performance disparities among the evaluated algorithms. Clarke and Wright algorithm demonstrates superior performance with a mean execution time of 862.43, while Genetic Algorithm exhibits the highest values averaging 2037.20. This represents an improvement of approximately 57.7\% when comparing the best to worst performing algorithms. The relative ranking of algorithms remains largely consistent across different node configurations, suggesting robust performance characteristics.

These findings support the hypothesis that algorithmic choice significantly impacts system performance metrics. Future work should incorporate statistical significance testing and confidence intervals to strengthen these comparative conclusions. Additionally, examining the computational complexity trade-offs between algorithms would provide valuable context for practical deployment decisions.


\clearpage

\subsection{MT2TE and Other Program Swap Time Change}

\begin{figure}[!htbp]
\centering
\includegraphics[width=0.95\linewidth]{images/MT2TE_and_Other_Program_Swap_Time_Change/(a)_Total_Travel_Time.png}
\caption{(a) Total Travel Time}
\label{fig:MT2TE_and_Other_Program_Swap_Time_Change_a__Total_Travel_Time}
\end{figure}

This grouped bar chart presents a comparative analysis of total travel time across multiple algorithms within the MT2TE and Other Program Swap Time Change experimental configuration. The x-axis represents the number of nodes in the network, while the y-axis quantifies the total travel time metric. The legend identifies 4 distinct algorithms: Genetic Algorithm, EVRPBSS, Ant Colony, Clarke and Wright algorithm. This visualization enables direct comparison of algorithmic performance under identical network conditions.

Quantitative analysis reveals significant performance disparities among the evaluated algorithms. Clarke and Wright algorithm demonstrates superior performance with a mean total travel time of 697.62, while Genetic Algorithm exhibits the highest values averaging 1438.33. This represents an improvement of approximately 51.5\% when comparing the best to worst performing algorithms. The relative ranking of algorithms remains largely consistent across different node configurations, suggesting robust performance characteristics.

These findings support the hypothesis that algorithmic choice significantly impacts system performance metrics. Future work should incorporate statistical significance testing and confidence intervals to strengthen these comparative conclusions. Additionally, examining the computational complexity trade-offs between algorithms would provide valuable context for practical deployment decisions.

\begin{figure}[!htbp]
\centering
\includegraphics[width=0.95\linewidth]{images/MT2TE_and_Other_Program_Swap_Time_Change/(b)_Energy.png}
\caption{(b) Energy}
\label{fig:MT2TE_and_Other_Program_Swap_Time_Change_b__Energy}
\end{figure}

This grouped bar chart presents a comparative analysis of energy across multiple algorithms within the MT2TE and Other Program Swap Time Change experimental configuration. The x-axis represents the number of nodes in the network, while the y-axis quantifies the energy metric. The legend identifies 4 distinct algorithms: Genetic Algorithm, EVRPBSS, Ant Colony, Clarke and Wright algorithm. This visualization enables direct comparison of algorithmic performance under identical network conditions.

Quantitative analysis reveals significant performance disparities among the evaluated algorithms. Clarke and Wright algorithm demonstrates superior performance with a mean energy of 697.62, while Genetic Algorithm exhibits the highest values averaging 1438.33. This represents an improvement of approximately 51.5\% when comparing the best to worst performing algorithms. The relative ranking of algorithms remains largely consistent across different node configurations, suggesting robust performance characteristics.

These findings support the hypothesis that algorithmic choice significantly impacts system performance metrics. Future work should incorporate statistical significance testing and confidence intervals to strengthen these comparative conclusions. Additionally, examining the computational complexity trade-offs between algorithms would provide valuable context for practical deployment decisions.

\begin{figure}[!htbp]
\centering
\includegraphics[width=0.95\linewidth]{images/MT2TE_and_Other_Program_Swap_Time_Change/(c)_Distance.png}
\caption{(c) Distance}
\label{fig:MT2TE_and_Other_Program_Swap_Time_Change_c__Distance}
\end{figure}

This grouped bar chart presents a comparative analysis of distance across multiple algorithms within the MT2TE and Other Program Swap Time Change experimental configuration. The x-axis represents the number of nodes in the network, while the y-axis quantifies the distance metric. The legend identifies 4 distinct algorithms: Genetic Algorithm, EVRPBSS, Ant Colony, Clarke and Wright algorithm. This visualization enables direct comparison of algorithmic performance under identical network conditions.

Quantitative analysis reveals significant performance disparities among the evaluated algorithms. Clarke and Wright algorithm demonstrates superior performance with a mean distance of 697.62, while Genetic Algorithm exhibits the highest values averaging 1438.33. This represents an improvement of approximately 51.5\% when comparing the best to worst performing algorithms. The relative ranking of algorithms remains largely consistent across different node configurations, suggesting robust performance characteristics.

These findings support the hypothesis that algorithmic choice significantly impacts system performance metrics. Future work should incorporate statistical significance testing and confidence intervals to strengthen these comparative conclusions. Additionally, examining the computational complexity trade-offs between algorithms would provide valuable context for practical deployment decisions.

\begin{figure}[!htbp]
\centering
\includegraphics[width=0.95\linewidth]{images/MT2TE_and_Other_Program_Swap_Time_Change/(d)_Total_Module_Swapped.png}
\caption{(d) Total Module Swapped}
\label{fig:MT2TE_and_Other_Program_Swap_Time_Change_d__Total_Module_Swapped}
\end{figure}

This grouped bar chart presents a comparative analysis of total module swapped across multiple algorithms within the MT2TE and Other Program Swap Time Change experimental configuration. The x-axis represents the number of nodes in the network, while the y-axis quantifies the total module swapped metric. The legend identifies 4 distinct algorithms: Genetic Algorithm, EVRPBSS, Ant Colony, Clarke and Wright algorithm. This visualization enables direct comparison of algorithmic performance under identical network conditions.

Quantitative analysis reveals significant performance disparities among the evaluated algorithms. Clarke and Wright algorithm demonstrates superior performance with a mean total module swapped of 697.62, while Genetic Algorithm exhibits the highest values averaging 1438.33. This represents an improvement of approximately 51.5\% when comparing the best to worst performing algorithms. The relative ranking of algorithms remains largely consistent across different node configurations, suggesting robust performance characteristics.

These findings support the hypothesis that algorithmic choice significantly impacts system performance metrics. Future work should incorporate statistical significance testing and confidence intervals to strengthen these comparative conclusions. Additionally, examining the computational complexity trade-offs between algorithms would provide valuable context for practical deployment decisions.

\begin{figure}[!htbp]
\centering
\includegraphics[width=0.95\linewidth]{images/MT2TE_and_Other_Program_Swap_Time_Change/(e)_Execution_Time.png}
\caption{(e) Execution Time}
\label{fig:MT2TE_and_Other_Program_Swap_Time_Change_e__Execution_Time}
\end{figure}

This grouped bar chart presents a comparative analysis of execution time across multiple algorithms within the MT2TE and Other Program Swap Time Change experimental configuration. The x-axis represents the number of nodes in the network, while the y-axis quantifies the execution time metric. The legend identifies 4 distinct algorithms: Genetic Algorithm, EVRPBSS, Ant Colony, Clarke and Wright algorithm. This visualization enables direct comparison of algorithmic performance under identical network conditions.

Quantitative analysis reveals significant performance disparities among the evaluated algorithms. Clarke and Wright algorithm demonstrates superior performance with a mean execution time of 697.62, while Genetic Algorithm exhibits the highest values averaging 1438.33. This represents an improvement of approximately 51.5\% when comparing the best to worst performing algorithms. The relative ranking of algorithms remains largely consistent across different node configurations, suggesting robust performance characteristics.

These findings support the hypothesis that algorithmic choice significantly impacts system performance metrics. Future work should incorporate statistical significance testing and confidence intervals to strengthen these comparative conclusions. Additionally, examining the computational complexity trade-offs between algorithms would provide valuable context for practical deployment decisions.

\begin{figure}[!htbp]
\centering
\includegraphics[width=0.95\linewidth]{images/MT2TE_and_Other_Program_Swap_Time_Change/(f)_Execution_Time.png}
\caption{(f) Execution Time}
\label{fig:MT2TE_and_Other_Program_Swap_Time_Change_f__Execution_Time}
\end{figure}

This grouped bar chart presents a comparative analysis of execution time across multiple algorithms within the MT2TE and Other Program Swap Time Change experimental configuration. The x-axis represents the number of nodes in the network, while the y-axis quantifies the execution time metric. The legend identifies 3 distinct algorithms: EVRPBSS, Ant Colony, Clarke and Wright algorithm. This visualization enables direct comparison of algorithmic performance under identical network conditions.

Quantitative analysis reveals significant performance disparities among the evaluated algorithms. Clarke and Wright algorithm demonstrates superior performance with a mean execution time of 697.62, while Ant Colony exhibits the highest values averaging 1026.84. This represents an improvement of approximately 32.1\% when comparing the best to worst performing algorithms. The relative ranking of algorithms remains largely consistent across different node configurations, suggesting robust performance characteristics.

These findings support the hypothesis that algorithmic choice significantly impacts system performance metrics. Future work should incorporate statistical significance testing and confidence intervals to strengthen these comparative conclusions. Additionally, examining the computational complexity trade-offs between algorithms would provide valuable context for practical deployment decisions.


\clearpage

\subsection{MT2TE and Other Program Threshold Change}

\begin{figure}[!htbp]
\centering
\includegraphics[width=0.95\linewidth]{images/MT2TE_and_Other_Program_Threshold_Change/(a)_Total_Travel_Time.png}
\caption{(a) Total Travel Time}
\label{fig:MT2TE_and_Other_Program_Threshold_Change_a__Total_Travel_Time}
\end{figure}

This grouped bar chart presents a comparative analysis of total travel time across multiple algorithms within the MT2TE and Other Program Threshold Change experimental configuration. The x-axis represents the number of nodes in the network, while the y-axis quantifies the total travel time metric. The legend identifies 4 distinct algorithms: Genetic Algorithm, EVRPBSS, Ant Colony, Clarke and Wright algorithm. This visualization enables direct comparison of algorithmic performance under identical network conditions.

Quantitative analysis reveals significant performance disparities among the evaluated algorithms. Clarke and Wright algorithm demonstrates superior performance with a mean total travel time of 690.83, while Genetic Algorithm exhibits the highest values averaging 1400.93. This represents an improvement of approximately 50.7\% when comparing the best to worst performing algorithms. The relative ranking of algorithms remains largely consistent across different node configurations, suggesting robust performance characteristics.

These findings support the hypothesis that algorithmic choice significantly impacts system performance metrics. Future work should incorporate statistical significance testing and confidence intervals to strengthen these comparative conclusions. Additionally, examining the computational complexity trade-offs between algorithms would provide valuable context for practical deployment decisions.

\begin{figure}[!htbp]
\centering
\includegraphics[width=0.95\linewidth]{images/MT2TE_and_Other_Program_Threshold_Change/(b)_Energy.png}
\caption{(b) Energy}
\label{fig:MT2TE_and_Other_Program_Threshold_Change_b__Energy}
\end{figure}

This grouped bar chart presents a comparative analysis of energy across multiple algorithms within the MT2TE and Other Program Threshold Change experimental configuration. The x-axis represents the number of nodes in the network, while the y-axis quantifies the energy metric. The legend identifies 4 distinct algorithms: Genetic Algorithm, EVRPBSS, Ant Colony, Clarke and Wright algorithm. This visualization enables direct comparison of algorithmic performance under identical network conditions.

Quantitative analysis reveals significant performance disparities among the evaluated algorithms. Clarke and Wright algorithm demonstrates superior performance with a mean energy of 690.83, while Genetic Algorithm exhibits the highest values averaging 1400.93. This represents an improvement of approximately 50.7\% when comparing the best to worst performing algorithms. The relative ranking of algorithms remains largely consistent across different node configurations, suggesting robust performance characteristics.

These findings support the hypothesis that algorithmic choice significantly impacts system performance metrics. Future work should incorporate statistical significance testing and confidence intervals to strengthen these comparative conclusions. Additionally, examining the computational complexity trade-offs between algorithms would provide valuable context for practical deployment decisions.

\begin{figure}[!htbp]
\centering
\includegraphics[width=0.95\linewidth]{images/MT2TE_and_Other_Program_Threshold_Change/(c)_Distance.png}
\caption{(c) Distance}
\label{fig:MT2TE_and_Other_Program_Threshold_Change_c__Distance}
\end{figure}

This grouped bar chart presents a comparative analysis of distance across multiple algorithms within the MT2TE and Other Program Threshold Change experimental configuration. The x-axis represents the number of nodes in the network, while the y-axis quantifies the distance metric. The legend identifies 4 distinct algorithms: Genetic Algorithm, EVRPBSS, Ant Colony, Clarke and Wright algorithm. This visualization enables direct comparison of algorithmic performance under identical network conditions.

Quantitative analysis reveals significant performance disparities among the evaluated algorithms. Clarke and Wright algorithm demonstrates superior performance with a mean distance of 690.83, while Genetic Algorithm exhibits the highest values averaging 1400.93. This represents an improvement of approximately 50.7\% when comparing the best to worst performing algorithms. The relative ranking of algorithms remains largely consistent across different node configurations, suggesting robust performance characteristics.

These findings support the hypothesis that algorithmic choice significantly impacts system performance metrics. Future work should incorporate statistical significance testing and confidence intervals to strengthen these comparative conclusions. Additionally, examining the computational complexity trade-offs between algorithms would provide valuable context for practical deployment decisions.

\begin{figure}[!htbp]
\centering
\includegraphics[width=0.95\linewidth]{images/MT2TE_and_Other_Program_Threshold_Change/(d)_Module_Swapped.png}
\caption{(d) Module Swapped}
\label{fig:MT2TE_and_Other_Program_Threshold_Change_d__Module_Swapped}
\end{figure}

This grouped bar chart presents a comparative analysis of module swapped across multiple algorithms within the MT2TE and Other Program Threshold Change experimental configuration. The x-axis represents the number of nodes in the network, while the y-axis quantifies the module swapped metric. The legend identifies 4 distinct algorithms: Genetic Algorithm, EVRPBSS, Ant Colony, Clarke and Wright algorithm. This visualization enables direct comparison of algorithmic performance under identical network conditions.

Quantitative analysis reveals significant performance disparities among the evaluated algorithms. Clarke and Wright algorithm demonstrates superior performance with a mean module swapped of 690.83, while Genetic Algorithm exhibits the highest values averaging 1400.93. This represents an improvement of approximately 50.7\% when comparing the best to worst performing algorithms. The relative ranking of algorithms remains largely consistent across different node configurations, suggesting robust performance characteristics.

These findings support the hypothesis that algorithmic choice significantly impacts system performance metrics. Future work should incorporate statistical significance testing and confidence intervals to strengthen these comparative conclusions. Additionally, examining the computational complexity trade-offs between algorithms would provide valuable context for practical deployment decisions.

\begin{figure}[!htbp]
\centering
\includegraphics[width=0.95\linewidth]{images/MT2TE_and_Other_Program_Threshold_Change/(e)_Execution_Time.png}
\caption{(e) Execution Time}
\label{fig:MT2TE_and_Other_Program_Threshold_Change_e__Execution_Time}
\end{figure}

This grouped bar chart presents a comparative analysis of execution time across multiple algorithms within the MT2TE and Other Program Threshold Change experimental configuration. The x-axis represents the number of nodes in the network, while the y-axis quantifies the execution time metric. The legend identifies 4 distinct algorithms: Genetic Algorithm, EVRPBSS, Ant Colony, Clarke and Wright algorithm. This visualization enables direct comparison of algorithmic performance under identical network conditions.

Quantitative analysis reveals significant performance disparities among the evaluated algorithms. Clarke and Wright algorithm demonstrates superior performance with a mean execution time of 690.83, while Genetic Algorithm exhibits the highest values averaging 1400.93. This represents an improvement of approximately 50.7\% when comparing the best to worst performing algorithms. The relative ranking of algorithms remains largely consistent across different node configurations, suggesting robust performance characteristics.

These findings support the hypothesis that algorithmic choice significantly impacts system performance metrics. Future work should incorporate statistical significance testing and confidence intervals to strengthen these comparative conclusions. Additionally, examining the computational complexity trade-offs between algorithms would provide valuable context for practical deployment decisions.

\begin{figure}[!htbp]
\centering
\includegraphics[width=0.95\linewidth]{images/MT2TE_and_Other_Program_Threshold_Change/(f)_Execution_Time.png}
\caption{(f) Execution Time}
\label{fig:MT2TE_and_Other_Program_Threshold_Change_f__Execution_Time}
\end{figure}

This grouped bar chart presents a comparative analysis of execution time across multiple algorithms within the MT2TE and Other Program Threshold Change experimental configuration. The x-axis represents the number of nodes in the network, while the y-axis quantifies the execution time metric. The legend identifies 3 distinct algorithms: EVRPBSS, Ant Colony, Clarke and Wright algorithm. This visualization enables direct comparison of algorithmic performance under identical network conditions.

Quantitative analysis reveals significant performance disparities among the evaluated algorithms. Clarke and Wright algorithm demonstrates superior performance with a mean execution time of 690.83, while Ant Colony exhibits the highest values averaging 1020.01. This represents an improvement of approximately 32.3\% when comparing the best to worst performing algorithms. The relative ranking of algorithms remains largely consistent across different node configurations, suggesting robust performance characteristics.

These findings support the hypothesis that algorithmic choice significantly impacts system performance metrics. Future work should incorporate statistical significance testing and confidence intervals to strengthen these comparative conclusions. Additionally, examining the computational complexity trade-offs between algorithms would provide valuable context for practical deployment decisions.


\clearpage

\subsection{MT2TE and Other Program Traffic}

\begin{figure}[!htbp]
\centering
\includegraphics[width=0.95\linewidth]{images/MT2TE_and_Other_Program_Traffic/(a)_Travel_Time.png}
\caption{(a) Travel Time}
\label{fig:MT2TE_and_Other_Program_Traffic_a__Travel_Time}
\end{figure}

This grouped bar chart presents a comparative analysis of travel time across multiple algorithms within the MT2TE and Other Program Traffic experimental configuration. The x-axis represents the number of nodes in the network, while the y-axis quantifies the travel time metric. The legend identifies 4 distinct algorithms: Genetic Algorithm, EVRPBSS, Ant Colony, Clarke and Wright algorithm. This visualization enables direct comparison of algorithmic performance under identical network conditions.

Quantitative analysis reveals significant performance disparities among the evaluated algorithms. Clarke and Wright algorithm demonstrates superior performance with a mean travel time of 697.95, while Genetic Algorithm exhibits the highest values averaging 1420.64. This represents an improvement of approximately 50.9\% when comparing the best to worst performing algorithms. The relative ranking of algorithms remains largely consistent across different node configurations, suggesting robust performance characteristics.

These findings support the hypothesis that algorithmic choice significantly impacts system performance metrics. Future work should incorporate statistical significance testing and confidence intervals to strengthen these comparative conclusions. Additionally, examining the computational complexity trade-offs between algorithms would provide valuable context for practical deployment decisions.

\begin{figure}[!htbp]
\centering
\includegraphics[width=0.95\linewidth]{images/MT2TE_and_Other_Program_Traffic/(b)_Energy.png}
\caption{(b) Energy}
\label{fig:MT2TE_and_Other_Program_Traffic_b__Energy}
\end{figure}

This grouped bar chart presents a comparative analysis of energy across multiple algorithms within the MT2TE and Other Program Traffic experimental configuration. The x-axis represents the number of nodes in the network, while the y-axis quantifies the energy metric. The legend identifies 4 distinct algorithms: Genetic Algorithm, EVRPBSS, Ant Colony, Clarke and Wright algorithm. This visualization enables direct comparison of algorithmic performance under identical network conditions.

Quantitative analysis reveals significant performance disparities among the evaluated algorithms. Clarke and Wright algorithm demonstrates superior performance with a mean energy of 697.95, while Genetic Algorithm exhibits the highest values averaging 1420.64. This represents an improvement of approximately 50.9\% when comparing the best to worst performing algorithms. The relative ranking of algorithms remains largely consistent across different node configurations, suggesting robust performance characteristics.

These findings support the hypothesis that algorithmic choice significantly impacts system performance metrics. Future work should incorporate statistical significance testing and confidence intervals to strengthen these comparative conclusions. Additionally, examining the computational complexity trade-offs between algorithms would provide valuable context for practical deployment decisions.

\begin{figure}[!htbp]
\centering
\includegraphics[width=0.95\linewidth]{images/MT2TE_and_Other_Program_Traffic/(c)_Distance.png}
\caption{(c) Distance}
\label{fig:MT2TE_and_Other_Program_Traffic_c__Distance}
\end{figure}

This grouped bar chart presents a comparative analysis of distance across multiple algorithms within the MT2TE and Other Program Traffic experimental configuration. The x-axis represents the number of nodes in the network, while the y-axis quantifies the distance metric. The legend identifies 4 distinct algorithms: Genetic Algorithm, EVRPBSS, Ant Colony, Clarke and Wright algorithm. This visualization enables direct comparison of algorithmic performance under identical network conditions.

Quantitative analysis reveals significant performance disparities among the evaluated algorithms. Clarke and Wright algorithm demonstrates superior performance with a mean distance of 697.95, while Genetic Algorithm exhibits the highest values averaging 1420.64. This represents an improvement of approximately 50.9\% when comparing the best to worst performing algorithms. The relative ranking of algorithms remains largely consistent across different node configurations, suggesting robust performance characteristics.

These findings support the hypothesis that algorithmic choice significantly impacts system performance metrics. Future work should incorporate statistical significance testing and confidence intervals to strengthen these comparative conclusions. Additionally, examining the computational complexity trade-offs between algorithms would provide valuable context for practical deployment decisions.

\begin{figure}[!htbp]
\centering
\includegraphics[width=0.95\linewidth]{images/MT2TE_and_Other_Program_Traffic/(d)_Total_Module_Swapped.png}
\caption{(d) Total Module Swapped}
\label{fig:MT2TE_and_Other_Program_Traffic_d__Total_Module_Swapped}
\end{figure}

This grouped bar chart presents a comparative analysis of total module swapped across multiple algorithms within the MT2TE and Other Program Traffic experimental configuration. The x-axis represents the number of nodes in the network, while the y-axis quantifies the total module swapped metric. The legend identifies 4 distinct algorithms: Genetic Algorithm, EVRPBSS, Ant Colony, Clarke and Wright algorithm. This visualization enables direct comparison of algorithmic performance under identical network conditions.

Quantitative analysis reveals significant performance disparities among the evaluated algorithms. Clarke and Wright algorithm demonstrates superior performance with a mean total module swapped of 697.95, while Genetic Algorithm exhibits the highest values averaging 1420.64. This represents an improvement of approximately 50.9\% when comparing the best to worst performing algorithms. The relative ranking of algorithms remains largely consistent across different node configurations, suggesting robust performance characteristics.

These findings support the hypothesis that algorithmic choice significantly impacts system performance metrics. Future work should incorporate statistical significance testing and confidence intervals to strengthen these comparative conclusions. Additionally, examining the computational complexity trade-offs between algorithms would provide valuable context for practical deployment decisions.

\begin{figure}[!htbp]
\centering
\includegraphics[width=0.95\linewidth]{images/MT2TE_and_Other_Program_Traffic/(e)_Execution_Time.png}
\caption{(e) Execution Time}
\label{fig:MT2TE_and_Other_Program_Traffic_e__Execution_Time}
\end{figure}

This grouped bar chart presents a comparative analysis of execution time across multiple algorithms within the MT2TE and Other Program Traffic experimental configuration. The x-axis represents the number of nodes in the network, while the y-axis quantifies the execution time metric. The legend identifies 4 distinct algorithms: Genetic Algorithm, EVRPBSS, Ant Colony, Clarke and Wright algorithm. This visualization enables direct comparison of algorithmic performance under identical network conditions.

Quantitative analysis reveals significant performance disparities among the evaluated algorithms. Clarke and Wright algorithm demonstrates superior performance with a mean execution time of 697.95, while Genetic Algorithm exhibits the highest values averaging 1420.64. This represents an improvement of approximately 50.9\% when comparing the best to worst performing algorithms. The relative ranking of algorithms remains largely consistent across different node configurations, suggesting robust performance characteristics.

These findings support the hypothesis that algorithmic choice significantly impacts system performance metrics. Future work should incorporate statistical significance testing and confidence intervals to strengthen these comparative conclusions. Additionally, examining the computational complexity trade-offs between algorithms would provide valuable context for practical deployment decisions.

\begin{figure}[!htbp]
\centering
\includegraphics[width=0.95\linewidth]{images/MT2TE_and_Other_Program_Traffic/(f)_Execution_Time.png}
\caption{(f) Execution Time}
\label{fig:MT2TE_and_Other_Program_Traffic_f__Execution_Time}
\end{figure}

This grouped bar chart presents a comparative analysis of execution time across multiple algorithms within the MT2TE and Other Program Traffic experimental configuration. The x-axis represents the number of nodes in the network, while the y-axis quantifies the execution time metric. The legend identifies 4 distinct algorithms: Genetic Algorithm, EVRPBSS, Ant Colony, Clarke and Wright algorithm. This visualization enables direct comparison of algorithmic performance under identical network conditions.

Quantitative analysis reveals significant performance disparities among the evaluated algorithms. Clarke and Wright algorithm demonstrates superior performance with a mean execution time of 862.43, while Genetic Algorithm exhibits the highest values averaging 2037.20. This represents an improvement of approximately 57.7\% when comparing the best to worst performing algorithms. The relative ranking of algorithms remains largely consistent across different node configurations, suggesting robust performance characteristics.

These findings support the hypothesis that algorithmic choice significantly impacts system performance metrics. Future work should incorporate statistical significance testing and confidence intervals to strengthen these comparative conclusions. Additionally, examining the computational complexity trade-offs between algorithms would provide valuable context for practical deployment decisions.


\clearpage

\subsection{MT2TE Multi-Line Charts}

\begin{figure}[!htbp]
\centering
\includegraphics[width=0.95\linewidth]{images/MT2TE_Multi-Line_Charts/(a)_Travel_time_bar_chart.png}
\caption{(a) Travel time}
\label{fig:MT2TE_Multi_Line_Charts_a__Travel_time_bar_chart}
\end{figure}

This bar chart presents the relationship between 4 module and 5 module for the MT2TE Multi-Line Charts experimental scenario. The x-axis displays 4 module values ranging from 3287.54 to 14159.32, while the y-axis quantifies 5 module. The single-series visualization facilitates analysis of how the dependent variable responds to changes in the independent parameter setting.

Analysis of the plotted data reveals that 5 module ranges from 3287.54 (at 4 module = 3287.54) to 13381.98 (at 4 module = 14159.32), representing a span of 10094.44 units. The overall trend is increasing, with values rising from 3287.54 at the initial setting to 13381.98 at the final setting. 

These results indicate that 4 module configuration meaningfully impacts 5 module in this experimental context. The substantial variation observed (coefficient of variation exceeding 20\%) suggests that parameter tuning could yield significant performance improvements. Confidence in these findings is high given the direct correspondence between CSV data and plotted values. Future analysis should consider incorporating error bars representing variance across multiple experimental runs to strengthen statistical validity.

\begin{figure}[!htbp]
\centering
\includegraphics[width=0.95\linewidth]{images/MT2TE_Multi-Line_Charts/(b)_Energy_consumed_bar_chart.png}
\caption{(b) Energy consumed}
\label{fig:MT2TE_Multi_Line_Charts_b__Energy_consumed_bar_chart}
\end{figure}

This bar chart presents the relationship between 4 module and 5 module for the MT2TE Multi-Line Charts experimental scenario. The x-axis displays 4 module values ranging from 323.496 to 2154.006, while the y-axis quantifies 5 module. The single-series visualization facilitates analysis of how the dependent variable responds to changes in the independent parameter setting.

Analysis of the plotted data reveals that 5 module ranges from 323.50 (at 4 module = 323.496) to 2057.45 (at 4 module = 2154.006), representing a span of 1733.96 units. The overall trend is increasing, with values rising from 323.50 at the initial setting to 2057.45 at the final setting. 

These results indicate that 4 module configuration meaningfully impacts 5 module in this experimental context. The substantial variation observed (coefficient of variation exceeding 20\%) suggests that parameter tuning could yield significant performance improvements. Confidence in these findings is high given the direct correspondence between CSV data and plotted values. Future analysis should consider incorporating error bars representing variance across multiple experimental runs to strengthen statistical validity.

\begin{figure}[!htbp]
\centering
\includegraphics[width=0.95\linewidth]{images/MT2TE_Multi-Line_Charts/(c)_Distance_covered_bar_chart.png}
\caption{(c) Distance covered}
\label{fig:MT2TE_Multi_Line_Charts_c__Distance_covered_bar_chart}
\end{figure}

This bar chart presents the relationship between 4 module and 5 module for the MT2TE Multi-Line Charts experimental scenario. The x-axis displays 4 module values ranging from 2138.91 to 9234.69, while the y-axis quantifies 5 module. The single-series visualization facilitates analysis of how the dependent variable responds to changes in the independent parameter setting.

Analysis of the plotted data reveals that 5 module ranges from 2138.91 (at 4 module = 2138.91) to 8652.17 (at 4 module = 9234.69), representing a span of 6513.26 units. The overall trend is increasing, with values rising from 2138.91 at the initial setting to 8652.17 at the final setting. 

These results indicate that 4 module configuration meaningfully impacts 5 module in this experimental context. The substantial variation observed (coefficient of variation exceeding 20\%) suggests that parameter tuning could yield significant performance improvements. Confidence in these findings is high given the direct correspondence between CSV data and plotted values. Future analysis should consider incorporating error bars representing variance across multiple experimental runs to strengthen statistical validity.

\begin{figure}[!htbp]
\centering
\includegraphics[width=0.95\linewidth]{images/MT2TE_Multi-Line_Charts/(d)_Run_time_bar_chart.png}
\caption{(d) Run time}
\label{fig:MT2TE_Multi_Line_Charts_d__Run_time_bar_chart}
\end{figure}

This bar chart presents the relationship between 4 module and 5 module for the MT2TE Multi-Line Charts experimental scenario. The x-axis displays 4 module values ranging from 58.526 to 1545.515, while the y-axis quantifies 5 module. The single-series visualization facilitates analysis of how the dependent variable responds to changes in the independent parameter setting.

Analysis of the plotted data reveals that 5 module ranges from 29.02 (at 4 module = 58.526) to 1236.34 (at 4 module = 1545.515), representing a span of 1207.32 units. The overall trend is increasing, with values rising from 29.02 at the initial setting to 1236.34 at the final setting. 

These results indicate that 4 module configuration meaningfully impacts 5 module in this experimental context. The substantial variation observed (coefficient of variation exceeding 20\%) suggests that parameter tuning could yield significant performance improvements. Confidence in these findings is high given the direct correspondence between CSV data and plotted values. Future analysis should consider incorporating error bars representing variance across multiple experimental runs to strengthen statistical validity.

\begin{figure}[!htbp]
\centering
\includegraphics[width=0.95\linewidth]{images/MT2TE_Multi-Line_Charts/(e)_Module_swapped_bar_chart.png}
\caption{(e) Module swapped}
\label{fig:MT2TE_Multi_Line_Charts_e__Module_swapped_bar_chart}
\end{figure}

This bar chart presents the relationship between 4 module and 4 module for the MT2TE Multi-Line Charts experimental scenario. The x-axis displays 4 module values ranging from 15 to 106, while the y-axis quantifies 4 module. The single-series visualization facilitates analysis of how the dependent variable responds to changes in the independent parameter setting.

Analysis of the plotted data reveals that 4 module ranges from 15.00 (at 4 module = 15) to 106.00 (at 4 module = 106), representing a span of 91.00 units. The overall trend is increasing, with values rising from 15.00 at the initial setting to 106.00 at the final setting. 

These results indicate that 4 module configuration meaningfully impacts 4 module in this experimental context. The substantial variation observed (coefficient of variation exceeding 20\%) suggests that parameter tuning could yield significant performance improvements. Confidence in these findings is high given the direct correspondence between CSV data and plotted values. Future analysis should consider incorporating error bars representing variance across multiple experimental runs to strengthen statistical validity.


\clearpage

\subsection{MT2TE Node Change}

\begin{figure}[!htbp]
\centering
\includegraphics[width=0.95\linewidth]{images/MT2TE_Node_Change/(a)_Travel_time_bar_chart.png}
\caption{(a) Travel time}
\label{fig:MT2TE_Node_Change_a__Travel_time_bar_chart}
\end{figure}

This bar chart presents the relationship between number of nodes and total travel time for the MT2TE Node Change experimental scenario. The x-axis displays number of nodes values ranging from 500 to 2000, while the y-axis quantifies total travel time. The single-series visualization facilitates analysis of how the dependent variable responds to changes in the independent parameter setting.

Analysis of the plotted data reveals that total travel time ranges from 3228.14 (at number of nodes = 500) to 13338.35 (at number of nodes = 2000), representing a span of 10110.21 units. The overall trend is increasing, with values rising from 3228.14 at the initial setting to 13338.35 at the final setting. 

These results indicate that number of nodes configuration meaningfully impacts total travel time in this experimental context. The substantial variation observed (coefficient of variation exceeding 20\%) suggests that parameter tuning could yield significant performance improvements. Confidence in these findings is high given the direct correspondence between CSV data and plotted values. Future analysis should consider incorporating error bars representing variance across multiple experimental runs to strengthen statistical validity.

\begin{figure}[!htbp]
\centering
\includegraphics[width=0.95\linewidth]{images/MT2TE_Node_Change/(b)_Energy_consumed_bar_chart.png}
\caption{(b) Energy consumed}
\label{fig:MT2TE_Node_Change_b__Energy_consumed_bar_chart}
\end{figure}

This bar chart presents the relationship between number of nodes and total energy consumed for the MT2TE Node Change experimental scenario. The x-axis displays number of nodes values ranging from 500 to 2000, while the y-axis quantifies total energy consumed. The single-series visualization facilitates analysis of how the dependent variable responds to changes in the independent parameter setting.

Analysis of the plotted data reveals that total energy consumed ranges from 317.62 (at number of nodes = 500) to 2036.78 (at number of nodes = 2000), representing a span of 1719.17 units. The overall trend is increasing, with values rising from 317.62 at the initial setting to 2036.78 at the final setting. 

These results indicate that number of nodes configuration meaningfully impacts total energy consumed in this experimental context. The substantial variation observed (coefficient of variation exceeding 20\%) suggests that parameter tuning could yield significant performance improvements. Confidence in these findings is high given the direct correspondence between CSV data and plotted values. Future analysis should consider incorporating error bars representing variance across multiple experimental runs to strengthen statistical validity.

\begin{figure}[!htbp]
\centering
\includegraphics[width=0.95\linewidth]{images/MT2TE_Node_Change/(c)_Distance_covered_bar_chart.png}
\caption{(c) Distance covered}
\label{fig:MT2TE_Node_Change_c__Distance_covered_bar_chart}
\end{figure}

This bar chart presents the relationship between number of nodes and total distance covered for the MT2TE Node Change experimental scenario. The x-axis displays number of nodes values ranging from 500 to 2000, while the y-axis quantifies total distance covered. The single-series visualization facilitates analysis of how the dependent variable responds to changes in the independent parameter setting.

Analysis of the plotted data reveals that total distance covered ranges from 2105.32 (at number of nodes = 500) to 8672.23 (at number of nodes = 2000), representing a span of 6566.90 units. The overall trend is increasing, with values rising from 2105.32 at the initial setting to 8672.23 at the final setting. 

These results indicate that number of nodes configuration meaningfully impacts total distance covered in this experimental context. The substantial variation observed (coefficient of variation exceeding 20\%) suggests that parameter tuning could yield significant performance improvements. Confidence in these findings is high given the direct correspondence between CSV data and plotted values. Future analysis should consider incorporating error bars representing variance across multiple experimental runs to strengthen statistical validity.

\begin{figure}[!htbp]
\centering
\includegraphics[width=0.95\linewidth]{images/MT2TE_Node_Change/(d)_Run_time_bar_chart.png}
\caption{(d) Run time}
\label{fig:MT2TE_Node_Change_d__Run_time_bar_chart}
\end{figure}

This bar chart presents the relationship between number of nodes and run time for the MT2TE Node Change experimental scenario. The x-axis displays number of nodes values ranging from 500 to 2000, while the y-axis quantifies run time. The single-series visualization facilitates analysis of how the dependent variable responds to changes in the independent parameter setting.

Analysis of the plotted data reveals that run time ranges from 42.87 (at number of nodes = 500) to 1257.88 (at number of nodes = 2000), representing a span of 1215.01 units. The overall trend is increasing, with values rising from 42.87 at the initial setting to 1257.88 at the final setting. 

These results indicate that number of nodes configuration meaningfully impacts run time in this experimental context. The substantial variation observed (coefficient of variation exceeding 20\%) suggests that parameter tuning could yield significant performance improvements. Confidence in these findings is high given the direct correspondence between CSV data and plotted values. Future analysis should consider incorporating error bars representing variance across multiple experimental runs to strengthen statistical validity.

\begin{figure}[!htbp]
\centering
\includegraphics[width=0.95\linewidth]{images/MT2TE_Node_Change/(e)_Module_swapped_bar_chart.png}
\caption{(e) Module swapped}
\label{fig:MT2TE_Node_Change_e__Module_swapped_bar_chart}
\end{figure}

This bar chart presents the relationship between number of nodes and total module swapped for the MT2TE Node Change experimental scenario. The x-axis displays number of nodes values ranging from 500 to 2000, while the y-axis quantifies total module swapped. The single-series visualization facilitates analysis of how the dependent variable responds to changes in the independent parameter setting.

Analysis of the plotted data reveals that total module swapped ranges from 14.50 (at number of nodes = 500) to 98.75 (at number of nodes = 2000), representing a span of 84.25 units. The overall trend is increasing, with values rising from 14.50 at the initial setting to 98.75 at the final setting. 

These results indicate that number of nodes configuration meaningfully impacts total module swapped in this experimental context. The substantial variation observed (coefficient of variation exceeding 20\%) suggests that parameter tuning could yield significant performance improvements. Confidence in these findings is high given the direct correspondence between CSV data and plotted values. Future analysis should consider incorporating error bars representing variance across multiple experimental runs to strengthen statistical validity.


\clearpage

\subsection{Optimum and MT2TE Module Change}

\begin{figure}[!htbp]
\centering
\includegraphics[width=0.95\linewidth]{images/Optimum_and_MT2TE_Module_Change/(a)_Travel_Time.png}
\caption{(a) Travel Time}
\label{fig:Optimum_and_MT2TE_Module_Change_a__Travel_Time}
\end{figure}

This bar chart presents the relationship between module and optimum for the Optimum and MT2TE Module Change experimental scenario. The x-axis displays module values ranging from 3 to 6, while the y-axis quantifies optimum. The single-series visualization facilitates analysis of how the dependent variable responds to changes in the independent parameter setting.

Analysis of the plotted data reveals that optimum ranges from 111.20 (at module = 3) to 111.20 (at module = 3), representing a span of 0.00 units. The values remain relatively stable across the parameter range, with minimal net change between initial (111.20) and final (111.20) settings. 

These results indicate that module configuration meaningfully impacts optimum in this experimental context. The relatively modest variation suggests that this parameter has limited influence on the measured metric within the tested range. Confidence in these findings is high given the direct correspondence between CSV data and plotted values. Future analysis should consider incorporating error bars representing variance across multiple experimental runs to strengthen statistical validity.

\begin{figure}[!htbp]
\centering
\includegraphics[width=0.95\linewidth]{images/Optimum_and_MT2TE_Module_Change/(b)_Energy.png}
\caption{(b) Energy}
\label{fig:Optimum_and_MT2TE_Module_Change_b__Energy}
\end{figure}

This bar chart presents the relationship between module and optimum for the Optimum and MT2TE Module Change experimental scenario. The x-axis displays module values ranging from 3 to 6, while the y-axis quantifies optimum. The single-series visualization facilitates analysis of how the dependent variable responds to changes in the independent parameter setting.

Analysis of the plotted data reveals that optimum ranges from 9.01 (at module = 3) to 9.01 (at module = 3), representing a span of 0.00 units. The values remain relatively stable across the parameter range, with minimal net change between initial (9.01) and final (9.01) settings. 

These results indicate that module configuration meaningfully impacts optimum in this experimental context. The relatively modest variation suggests that this parameter has limited influence on the measured metric within the tested range. Confidence in these findings is high given the direct correspondence between CSV data and plotted values. Future analysis should consider incorporating error bars representing variance across multiple experimental runs to strengthen statistical validity.

\begin{figure}[!htbp]
\centering
\includegraphics[width=0.95\linewidth]{images/Optimum_and_MT2TE_Module_Change/(c)_Distance.png}
\caption{(c) Distance}
\label{fig:Optimum_and_MT2TE_Module_Change_c__Distance}
\end{figure}

This bar chart presents the relationship between module and optimum for the Optimum and MT2TE Module Change experimental scenario. The x-axis displays module values ranging from 3 to 6, while the y-axis quantifies optimum. The single-series visualization facilitates analysis of how the dependent variable responds to changes in the independent parameter setting.

Analysis of the plotted data reveals that optimum ranges from 74.38 (at module = 3) to 74.38 (at module = 3), representing a span of 0.00 units. The values remain relatively stable across the parameter range, with minimal net change between initial (74.38) and final (74.38) settings. 

These results indicate that module configuration meaningfully impacts optimum in this experimental context. The relatively modest variation suggests that this parameter has limited influence on the measured metric within the tested range. Confidence in these findings is high given the direct correspondence between CSV data and plotted values. Future analysis should consider incorporating error bars representing variance across multiple experimental runs to strengthen statistical validity.


\clearpage

\subsection{Optimum and MT2TE Node Change}

\begin{figure}[!htbp]
\centering
\includegraphics[width=0.95\linewidth]{images/Optimum_and_MT2TE_Node_Change/(a)_Total_Travel_Time_Optimum_and_MT2TE_Node_Change.png}
\caption{(a) Total Travel Time Optimum and MT2TE Node Change}
\label{fig:Optimum_and_MT2TE_Node_Change_a__Total_Travel_Time_Optimum_and_MT2TE_Node_Change}
\end{figure}

This bar chart presents the relationship between node and optimum for the Optimum and MT2TE Node Change experimental scenario. The x-axis displays node values ranging from 12 to 24, while the y-axis quantifies optimum. The single-series visualization facilitates analysis of how the dependent variable responds to changes in the independent parameter setting.

Analysis of the plotted data reveals that optimum ranges from 106.61 (at node = 12) to 153.26 (at node = 24), representing a span of 46.65 units. The overall trend is increasing, with values rising from 106.61 at the initial setting to 153.26 at the final setting. 

These results indicate that node configuration meaningfully impacts optimum in this experimental context. The substantial variation observed (coefficient of variation exceeding 20\%) suggests that parameter tuning could yield significant performance improvements. Confidence in these findings is high given the direct correspondence between CSV data and plotted values. Future analysis should consider incorporating error bars representing variance across multiple experimental runs to strengthen statistical validity.

\begin{figure}[!htbp]
\centering
\includegraphics[width=0.95\linewidth]{images/Optimum_and_MT2TE_Node_Change/(b)_Total_Energy_Consumed_Optimum_and_MT2TE_Node_Change.png}
\caption{(b) Total Energy Consumed Optimum and MT2TE Node Change}
\label{fig:Optimum_and_MT2TE_Node_Change_b__Total_Energy_Consumed_Optimum_and_MT2TE_Node_Change}
\end{figure}

This bar chart presents the relationship between node and optimum for the Optimum and MT2TE Node Change experimental scenario. The x-axis displays node values ranging from 12 to 24, while the y-axis quantifies optimum. The single-series visualization facilitates analysis of how the dependent variable responds to changes in the independent parameter setting.

Analysis of the plotted data reveals that optimum ranges from 8.72 (at node = 12) to 11.84 (at node = 24), representing a span of 3.12 units. The overall trend is increasing, with values rising from 8.72 at the initial setting to 11.84 at the final setting. 

These results indicate that node configuration meaningfully impacts optimum in this experimental context. The substantial variation observed (coefficient of variation exceeding 20\%) suggests that parameter tuning could yield significant performance improvements. Confidence in these findings is high given the direct correspondence between CSV data and plotted values. Future analysis should consider incorporating error bars representing variance across multiple experimental runs to strengthen statistical validity.

\begin{figure}[!htbp]
\centering
\includegraphics[width=0.95\linewidth]{images/Optimum_and_MT2TE_Node_Change/(c)_Total_Distance_Optimum_and_MT2TE_Node_Change.png}
\caption{(c) Total Distance Optimum and MT2TE Node Change}
\label{fig:Optimum_and_MT2TE_Node_Change_c__Total_Distance_Optimum_and_MT2TE_Node_Change}
\end{figure}

This bar chart presents the relationship between node and optimum for the Optimum and MT2TE Node Change experimental scenario. The x-axis displays node values ranging from 12 to 24, while the y-axis quantifies optimum. The single-series visualization facilitates analysis of how the dependent variable responds to changes in the independent parameter setting.

Analysis of the plotted data reveals that optimum ranges from 65.57 (at node = 12) to 98.75 (at node = 24), representing a span of 33.18 units. The overall trend is increasing, with values rising from 65.57 at the initial setting to 98.75 at the final setting. 

These results indicate that node configuration meaningfully impacts optimum in this experimental context. The substantial variation observed (coefficient of variation exceeding 20\%) suggests that parameter tuning could yield significant performance improvements. Confidence in these findings is high given the direct correspondence between CSV data and plotted values. Future analysis should consider incorporating error bars representing variance across multiple experimental runs to strengthen statistical validity.

\begin{figure}[!htbp]
\centering
\includegraphics[width=0.95\linewidth]{images/Optimum_and_MT2TE_Node_Change/(d)_Execution_Time_Optimum_and_MT2TE_Node_Change.png}
\caption{(d) Execution Time Optimum and MT2TE Node Change}
\label{fig:Optimum_and_MT2TE_Node_Change_d__Execution_Time_Optimum_and_MT2TE_Node_Change}
\end{figure}

This bar chart presents the relationship between node and optimum for the Optimum and MT2TE Node Change experimental scenario. The x-axis displays node values ranging from 12 to 28, while the y-axis quantifies optimum. The single-series visualization facilitates analysis of how the dependent variable responds to changes in the independent parameter setting.

Analysis of the plotted data reveals that optimum ranges from 12.00 (at node = 12) to 298800.00 (at node = 28), representing a span of 298788.00 units. The overall trend is increasing, with values rising from 12.00 at the initial setting to 298800.00 at the final setting. 

These results indicate that node configuration meaningfully impacts optimum in this experimental context. The substantial variation observed (coefficient of variation exceeding 20\%) suggests that parameter tuning could yield significant performance improvements. Confidence in these findings is high given the direct correspondence between CSV data and plotted values. Future analysis should consider incorporating error bars representing variance across multiple experimental runs to strengthen statistical validity.


\clearpage

\subsection{Optimum and MT2TE Swap Time Change}

\begin{figure}[!htbp]
\centering
\includegraphics[width=0.95\linewidth]{images/Optimum_and_MT2TE_Swap_Time_Change/(a)_Travel_Time.png}
\caption{(a) Travel Time}
\label{fig:Optimum_and_MT2TE_Swap_Time_Change_a__Travel_Time}
\end{figure}

This bar chart presents the relationship between swaptime and optimum for the Optimum and MT2TE Swap Time Change experimental scenario. The x-axis displays swaptime values ranging from 1 to 4, while the y-axis quantifies optimum. The single-series visualization facilitates analysis of how the dependent variable responds to changes in the independent parameter setting.

Analysis of the plotted data reveals that optimum ranges from 111.20 (at swaptime = 1) to 111.20 (at swaptime = 2), representing a span of 0.00 units. The overall trend is increasing, with values rising from 111.20 at the initial setting to 111.20 at the final setting. 

These results indicate that swaptime configuration meaningfully impacts optimum in this experimental context. The relatively modest variation suggests that this parameter has limited influence on the measured metric within the tested range. Confidence in these findings is high given the direct correspondence between CSV data and plotted values. Future analysis should consider incorporating error bars representing variance across multiple experimental runs to strengthen statistical validity.

\begin{figure}[!htbp]
\centering
\includegraphics[width=0.95\linewidth]{images/Optimum_and_MT2TE_Swap_Time_Change/(b)_Energy.png}
\caption{(b) Energy}
\label{fig:Optimum_and_MT2TE_Swap_Time_Change_b__Energy}
\end{figure}

This bar chart presents the relationship between swaptime and optimum for the Optimum and MT2TE Swap Time Change experimental scenario. The x-axis displays swaptime values ranging from 1 to 4, while the y-axis quantifies optimum. The single-series visualization facilitates analysis of how the dependent variable responds to changes in the independent parameter setting.

Analysis of the plotted data reveals that optimum ranges from 9.01 (at swaptime = 1) to 9.01 (at swaptime = 1), representing a span of 0.00 units. The values remain relatively stable across the parameter range, with minimal net change between initial (9.01) and final (9.01) settings. 

These results indicate that swaptime configuration meaningfully impacts optimum in this experimental context. The relatively modest variation suggests that this parameter has limited influence on the measured metric within the tested range. Confidence in these findings is high given the direct correspondence between CSV data and plotted values. Future analysis should consider incorporating error bars representing variance across multiple experimental runs to strengthen statistical validity.

\begin{figure}[!htbp]
\centering
\includegraphics[width=0.95\linewidth]{images/Optimum_and_MT2TE_Swap_Time_Change/(c)_Distance.png}
\caption{(c) Distance}
\label{fig:Optimum_and_MT2TE_Swap_Time_Change_c__Distance}
\end{figure}

This bar chart presents the relationship between swaptime and optimum for the Optimum and MT2TE Swap Time Change experimental scenario. The x-axis displays swaptime values ranging from 1 to 4, while the y-axis quantifies optimum. The single-series visualization facilitates analysis of how the dependent variable responds to changes in the independent parameter setting.

Analysis of the plotted data reveals that optimum ranges from 74.38 (at swaptime = 1) to 74.38 (at swaptime = 1), representing a span of 0.00 units. The values remain relatively stable across the parameter range, with minimal net change between initial (74.38) and final (74.38) settings. 

These results indicate that swaptime configuration meaningfully impacts optimum in this experimental context. The relatively modest variation suggests that this parameter has limited influence on the measured metric within the tested range. Confidence in these findings is high given the direct correspondence between CSV data and plotted values. Future analysis should consider incorporating error bars representing variance across multiple experimental runs to strengthen statistical validity.


\clearpage

\subsection{Optimum and MT2TE Threshold Change}

\begin{figure}[!htbp]
\centering
\includegraphics[width=0.95\linewidth]{images/Optimum_and_MT2TE_Threshold_Change/(a)_Travel_Time.png}
\caption{(a) Travel Time}
\label{fig:Optimum_and_MT2TE_Threshold_Change_a__Travel_Time}
\end{figure}

This bar chart presents the relationship between threshold and optimum for the Optimum and MT2TE Threshold Change experimental scenario. The x-axis displays threshold values ranging from 5 to 20, while the y-axis quantifies optimum. The single-series visualization facilitates analysis of how the dependent variable responds to changes in the independent parameter setting.

Analysis of the plotted data reveals that optimum ranges from 111.20 (at threshold = 5) to 111.20 (at threshold = 5), representing a span of 0.00 units. The values remain relatively stable across the parameter range, with minimal net change between initial (111.20) and final (111.20) settings. 

These results indicate that threshold configuration meaningfully impacts optimum in this experimental context. The relatively modest variation suggests that this parameter has limited influence on the measured metric within the tested range. Confidence in these findings is high given the direct correspondence between CSV data and plotted values. Future analysis should consider incorporating error bars representing variance across multiple experimental runs to strengthen statistical validity.

\begin{figure}[!htbp]
\centering
\includegraphics[width=0.95\linewidth]{images/Optimum_and_MT2TE_Threshold_Change/(b)_Energy.png}
\caption{(b) Energy}
\label{fig:Optimum_and_MT2TE_Threshold_Change_b__Energy}
\end{figure}

This bar chart presents the relationship between threshold and optimum for the Optimum and MT2TE Threshold Change experimental scenario. The x-axis displays threshold values ranging from 5 to 20, while the y-axis quantifies optimum. The single-series visualization facilitates analysis of how the dependent variable responds to changes in the independent parameter setting.

Analysis of the plotted data reveals that optimum ranges from 9.01 (at threshold = 5) to 9.01 (at threshold = 5), representing a span of 0.00 units. The values remain relatively stable across the parameter range, with minimal net change between initial (9.01) and final (9.01) settings. 

These results indicate that threshold configuration meaningfully impacts optimum in this experimental context. The relatively modest variation suggests that this parameter has limited influence on the measured metric within the tested range. Confidence in these findings is high given the direct correspondence between CSV data and plotted values. Future analysis should consider incorporating error bars representing variance across multiple experimental runs to strengthen statistical validity.

\begin{figure}[!htbp]
\centering
\includegraphics[width=0.95\linewidth]{images/Optimum_and_MT2TE_Threshold_Change/(c)_Distance.png}
\caption{(c) Distance}
\label{fig:Optimum_and_MT2TE_Threshold_Change_c__Distance}
\end{figure}

This bar chart presents the relationship between threshold and optimum for the Optimum and MT2TE Threshold Change experimental scenario. The x-axis displays threshold values ranging from 5 to 20, while the y-axis quantifies optimum. The single-series visualization facilitates analysis of how the dependent variable responds to changes in the independent parameter setting.

Analysis of the plotted data reveals that optimum ranges from 74.38 (at threshold = 5) to 74.38 (at threshold = 5), representing a span of 0.00 units. The values remain relatively stable across the parameter range, with minimal net change between initial (74.38) and final (74.38) settings. 

These results indicate that threshold configuration meaningfully impacts optimum in this experimental context. The relatively modest variation suggests that this parameter has limited influence on the measured metric within the tested range. Confidence in these findings is high given the direct correspondence between CSV data and plotted values. Future analysis should consider incorporating error bars representing variance across multiple experimental runs to strengthen statistical validity.


\clearpage

\subsection{Optimum and MT2TE Traffic Change}

\begin{figure}[!htbp]
\centering
\includegraphics[width=0.95\linewidth]{images/Optimum_and_MT2TE_Traffic_Change/(a)_Travel_Time.png}
\caption{(a) Travel Time}
\label{fig:Optimum_and_MT2TE_Traffic_Change_a__Travel_Time}
\end{figure}

This bar chart presents the relationship between trafficlevel and optimum for the Optimum and MT2TE Traffic Change experimental scenario. The x-axis displays trafficlevel values ranging from Low to High, while the y-axis quantifies optimum. The single-series visualization facilitates analysis of how the dependent variable responds to changes in the independent parameter setting.

Analysis of the plotted data reveals that optimum ranges from 94.72 (at trafficlevel = Low) to 133.80 (at trafficlevel = High), representing a span of 39.08 units. The overall trend is increasing, with values rising from 94.72 at the initial setting to 133.80 at the final setting. 

These results indicate that trafficlevel configuration meaningfully impacts optimum in this experimental context. The substantial variation observed (coefficient of variation exceeding 20\%) suggests that parameter tuning could yield significant performance improvements. Confidence in these findings is high given the direct correspondence between CSV data and plotted values. Future analysis should consider incorporating error bars representing variance across multiple experimental runs to strengthen statistical validity.

\begin{figure}[!htbp]
\centering
\includegraphics[width=0.95\linewidth]{images/Optimum_and_MT2TE_Traffic_Change/(b)_Energy.png}
\caption{(b) Energy}
\label{fig:Optimum_and_MT2TE_Traffic_Change_b__Energy}
\end{figure}

This bar chart presents the relationship between trafficlevel and optimum for the Optimum and MT2TE Traffic Change experimental scenario. The x-axis displays trafficlevel values ranging from Low to High, while the y-axis quantifies optimum. The single-series visualization facilitates analysis of how the dependent variable responds to changes in the independent parameter setting.

Analysis of the plotted data reveals that optimum ranges from 7.68 (at trafficlevel = Low) to 11.09 (at trafficlevel = High), representing a span of 3.41 units. The overall trend is increasing, with values rising from 7.68 at the initial setting to 11.09 at the final setting. 

These results indicate that trafficlevel configuration meaningfully impacts optimum in this experimental context. The substantial variation observed (coefficient of variation exceeding 20\%) suggests that parameter tuning could yield significant performance improvements. Confidence in these findings is high given the direct correspondence between CSV data and plotted values. Future analysis should consider incorporating error bars representing variance across multiple experimental runs to strengthen statistical validity.

\begin{figure}[!htbp]
\centering
\includegraphics[width=0.95\linewidth]{images/Optimum_and_MT2TE_Traffic_Change/(c)_Distance.png}
\caption{(c) Distance}
\label{fig:Optimum_and_MT2TE_Traffic_Change_c__Distance}
\end{figure}

This bar chart presents the relationship between traffic and total distance covered for the Optimum and MT2TE Traffic Change experimental scenario. The x-axis displays traffic values ranging from High to Low, while the y-axis quantifies total distance covered. The single-series visualization facilitates analysis of how the dependent variable responds to changes in the independent parameter setting.

Analysis of the plotted data reveals that total distance covered ranges from 8689.81 (at traffic = High) to 8994.35 (at traffic = Low), representing a span of 304.54 units. The overall trend is increasing, with values rising from 8689.81 at the initial setting to 8994.35 at the final setting. 

These results indicate that traffic configuration meaningfully impacts total distance covered in this experimental context. The relatively modest variation suggests that this parameter has limited influence on the measured metric within the tested range. Confidence in these findings is high given the direct correspondence between CSV data and plotted values. Future analysis should consider incorporating error bars representing variance across multiple experimental runs to strengthen statistical validity.


\clearpage

\subsection{Real Time Sumo Module Change}

\begin{figure}[!htbp]
\centering
\includegraphics[width=0.95\linewidth]{images/Real_Time_Sumo_Module_Change/(a)_Travel_Time.png}
\caption{(a) Travel Time}
\label{fig:Real_Time_Sumo_Module_Change_a__Travel_Time}
\end{figure}

This bar chart presents the relationship between modules and total travel time for the Real Time Sumo Module Change experimental scenario. The x-axis displays modules values ranging from 3 to 6, while the y-axis quantifies total travel time. The single-series visualization facilitates analysis of how the dependent variable responds to changes in the independent parameter setting.

Analysis of the plotted data reveals that total travel time ranges from 87.96 (at modules = 3) to 90.06 (at modules = 4), representing a span of 2.10 units. The overall trend is increasing, with values rising from 87.96 at the initial setting to 88.79 at the final setting. 

These results indicate that modules configuration meaningfully impacts total travel time in this experimental context. The relatively modest variation suggests that this parameter has limited influence on the measured metric within the tested range. Confidence in these findings is high given the direct correspondence between CSV data and plotted values. Future analysis should consider incorporating error bars representing variance across multiple experimental runs to strengthen statistical validity.

\begin{figure}[!htbp]
\centering
\includegraphics[width=0.95\linewidth]{images/Real_Time_Sumo_Module_Change/(b)_energy.png}
\caption{(b) energy}
\label{fig:Real_Time_Sumo_Module_Change_b__energy}
\end{figure}

This bar chart presents the relationship between modules and total energy consumed for the Real Time Sumo Module Change experimental scenario. The x-axis displays modules values ranging from 3 to 6, while the y-axis quantifies total energy consumed. The single-series visualization facilitates analysis of how the dependent variable responds to changes in the independent parameter setting.

Analysis of the plotted data reveals that total energy consumed ranges from 12.54 (at modules = 3) to 13.11 (at modules = 4), representing a span of 0.58 units. The overall trend is increasing, with values rising from 12.54 at the initial setting to 12.84 at the final setting. 

These results indicate that modules configuration meaningfully impacts total energy consumed in this experimental context. The relatively modest variation suggests that this parameter has limited influence on the measured metric within the tested range. Confidence in these findings is high given the direct correspondence between CSV data and plotted values. Future analysis should consider incorporating error bars representing variance across multiple experimental runs to strengthen statistical validity.

\begin{figure}[!htbp]
\centering
\includegraphics[width=0.95\linewidth]{images/Real_Time_Sumo_Module_Change/(c)_Distance.png}
\caption{(c) Distance}
\label{fig:Real_Time_Sumo_Module_Change_c__Distance}
\end{figure}

This bar chart presents the relationship between modules and total distance covered for the Real Time Sumo Module Change experimental scenario. The x-axis displays modules values ranging from 3 to 6, while the y-axis quantifies total distance covered. The single-series visualization facilitates analysis of how the dependent variable responds to changes in the independent parameter setting.

Analysis of the plotted data reveals that total distance covered ranges from 60.75 (at modules = 3) to 62.76 (at modules = 4), representing a span of 2.01 units. The overall trend is increasing, with values rising from 60.75 at the initial setting to 62.00 at the final setting. 

These results indicate that modules configuration meaningfully impacts total distance covered in this experimental context. The relatively modest variation suggests that this parameter has limited influence on the measured metric within the tested range. Confidence in these findings is high given the direct correspondence between CSV data and plotted values. Future analysis should consider incorporating error bars representing variance across multiple experimental runs to strengthen statistical validity.

\begin{figure}[!htbp]
\centering
\includegraphics[width=0.95\linewidth]{images/Real_Time_Sumo_Module_Change/(d)_Run_time.png}
\caption{(d) Run time}
\label{fig:Real_Time_Sumo_Module_Change_d__Run_time}
\end{figure}

This bar chart presents the relationship between modules and total travel time for the Real Time Sumo Module Change experimental scenario. The x-axis displays modules values ranging from 3 to 6, while the y-axis quantifies total travel time. The single-series visualization facilitates analysis of how the dependent variable responds to changes in the independent parameter setting.

Analysis of the plotted data reveals that total travel time ranges from 87.96 (at modules = 3) to 90.06 (at modules = 4), representing a span of 2.10 units. The overall trend is increasing, with values rising from 87.96 at the initial setting to 88.79 at the final setting. 

These results indicate that modules configuration meaningfully impacts total travel time in this experimental context. The relatively modest variation suggests that this parameter has limited influence on the measured metric within the tested range. Confidence in these findings is high given the direct correspondence between CSV data and plotted values. Future analysis should consider incorporating error bars representing variance across multiple experimental runs to strengthen statistical validity.


\clearpage

\subsection{Real Time Sumo Swap Time}

\begin{figure}[!htbp]
\centering
\includegraphics[width=0.95\linewidth]{images/Real_Time_Sumo_Swap_Time/(a)_Travel_Time.png}
\caption{(a) Travel Time}
\label{fig:Real_Time_Sumo_Swap_Time_a__Travel_Time}
\end{figure}

This bar chart presents the relationship between swap time (min) and total travel time for the Real Time Sumo Swap Time experimental scenario. The x-axis displays swap time (min) values ranging from 1 to 4, while the y-axis quantifies total travel time. The single-series visualization facilitates analysis of how the dependent variable responds to changes in the independent parameter setting.

Analysis of the plotted data reveals that total travel time ranges from 89.22 (at swap time (min) = 2) to 92.84 (at swap time (min) = 4), representing a span of 3.62 units. The overall trend is increasing, with values rising from 89.94 at the initial setting to 92.84 at the final setting. Notably, the minimum value occurs at an intermediate swap time (min) setting (2), suggesting non-monotonic behavior that warrants further investigation.

These results indicate that swap time (min) configuration meaningfully impacts total travel time in this experimental context. The relatively modest variation suggests that this parameter has limited influence on the measured metric within the tested range. Confidence in these findings is high given the direct correspondence between CSV data and plotted values. Future analysis should consider incorporating error bars representing variance across multiple experimental runs to strengthen statistical validity.

\begin{figure}[!htbp]
\centering
\includegraphics[width=0.95\linewidth]{images/Real_Time_Sumo_Swap_Time/(b)_energy.png}
\caption{(b) energy}
\label{fig:Real_Time_Sumo_Swap_Time_b__energy}
\end{figure}

This bar chart presents the relationship between swap time (min) and total energy consumed for the Real Time Sumo Swap Time experimental scenario. The x-axis displays swap time (min) values ranging from 1 to 4, while the y-axis quantifies total energy consumed. The single-series visualization facilitates analysis of how the dependent variable responds to changes in the independent parameter setting.

Analysis of the plotted data reveals that total energy consumed ranges from 12.57 (at swap time (min) = 3) to 13.18 (at swap time (min) = 4), representing a span of 0.60 units. The overall trend is increasing, with values rising from 12.92 at the initial setting to 13.18 at the final setting. Notably, the minimum value occurs at an intermediate swap time (min) setting (3), suggesting non-monotonic behavior that warrants further investigation.

These results indicate that swap time (min) configuration meaningfully impacts total energy consumed in this experimental context. The relatively modest variation suggests that this parameter has limited influence on the measured metric within the tested range. Confidence in these findings is high given the direct correspondence between CSV data and plotted values. Future analysis should consider incorporating error bars representing variance across multiple experimental runs to strengthen statistical validity.

\begin{figure}[!htbp]
\centering
\includegraphics[width=0.95\linewidth]{images/Real_Time_Sumo_Swap_Time/(c)_Distance.png}
\caption{(c) Distance}
\label{fig:Real_Time_Sumo_Swap_Time_c__Distance}
\end{figure}

This bar chart presents the relationship between swap time (min) and total distance covered for the Real Time Sumo Swap Time experimental scenario. The x-axis displays swap time (min) values ranging from 1 to 4, while the y-axis quantifies total distance covered. The single-series visualization facilitates analysis of how the dependent variable responds to changes in the independent parameter setting.

Analysis of the plotted data reveals that total distance covered ranges from 61.23 (at swap time (min) = 3) to 64.06 (at swap time (min) = 4), representing a span of 2.83 units. The overall trend is increasing, with values rising from 62.42 at the initial setting to 64.06 at the final setting. Notably, the minimum value occurs at an intermediate swap time (min) setting (3), suggesting non-monotonic behavior that warrants further investigation.

These results indicate that swap time (min) configuration meaningfully impacts total distance covered in this experimental context. The relatively modest variation suggests that this parameter has limited influence on the measured metric within the tested range. Confidence in these findings is high given the direct correspondence between CSV data and plotted values. Future analysis should consider incorporating error bars representing variance across multiple experimental runs to strengthen statistical validity.

\begin{figure}[!htbp]
\centering
\includegraphics[width=0.95\linewidth]{images/Real_Time_Sumo_Swap_Time/(d)_Runtime.png}
\caption{(d) Runtime}
\label{fig:Real_Time_Sumo_Swap_Time_d__Runtime}
\end{figure}

This bar chart presents the relationship between swap time (min) and run time for the Real Time Sumo Swap Time experimental scenario. The x-axis displays swap time (min) values ranging from 1 to 4, while the y-axis quantifies run time. The single-series visualization facilitates analysis of how the dependent variable responds to changes in the independent parameter setting.

Analysis of the plotted data reveals that run time ranges from 540.79 (at swap time (min) = 1) to 563.35 (at swap time (min) = 4), representing a span of 22.55 units. The overall trend is increasing, with values rising from 540.79 at the initial setting to 563.35 at the final setting. 

These results indicate that swap time (min) configuration meaningfully impacts run time in this experimental context. The relatively modest variation suggests that this parameter has limited influence on the measured metric within the tested range. Confidence in these findings is high given the direct correspondence between CSV data and plotted values. Future analysis should consider incorporating error bars representing variance across multiple experimental runs to strengthen statistical validity.


\clearpage

\subsection{Real Time Sumo Threshold}

\begin{figure}[!htbp]
\centering
\includegraphics[width=0.95\linewidth]{images/Real_Time_Sumo_Threshold/(a)_Travel_Time.png}
\caption{(a) Travel Time}
\label{fig:Real_Time_Sumo_Threshold_a__Travel_Time}
\end{figure}

This bar chart presents the relationship between threshold and total travel time for the Real Time Sumo Threshold experimental scenario. The x-axis displays threshold values ranging from 5 to 20, while the y-axis quantifies total travel time. The single-series visualization facilitates analysis of how the dependent variable responds to changes in the independent parameter setting.

Analysis of the plotted data reveals that total travel time ranges from 86.63 (at threshold = 5) to 92.25 (at threshold = 15), representing a span of 5.62 units. The overall trend is increasing, with values rising from 86.63 at the initial setting to 89.22 at the final setting. 

These results indicate that threshold configuration meaningfully impacts total travel time in this experimental context. The relatively modest variation suggests that this parameter has limited influence on the measured metric within the tested range. Confidence in these findings is high given the direct correspondence between CSV data and plotted values. Future analysis should consider incorporating error bars representing variance across multiple experimental runs to strengthen statistical validity.

\begin{figure}[!htbp]
\centering
\includegraphics[width=0.95\linewidth]{images/Real_Time_Sumo_Threshold/(b)_Energy.png}
\caption{(b) Energy}
\label{fig:Real_Time_Sumo_Threshold_b__Energy}
\end{figure}

This bar chart presents the relationship between threshold and total energy consumed for the Real Time Sumo Threshold experimental scenario. The x-axis displays threshold values ranging from 5 to 20, while the y-axis quantifies total energy consumed. The single-series visualization facilitates analysis of how the dependent variable responds to changes in the independent parameter setting.

Analysis of the plotted data reveals that total energy consumed ranges from 12.59 (at threshold = 5) to 13.17 (at threshold = 15), representing a span of 0.58 units. The overall trend is increasing, with values rising from 12.59 at the initial setting to 12.93 at the final setting. 

These results indicate that threshold configuration meaningfully impacts total energy consumed in this experimental context. The relatively modest variation suggests that this parameter has limited influence on the measured metric within the tested range. Confidence in these findings is high given the direct correspondence between CSV data and plotted values. Future analysis should consider incorporating error bars representing variance across multiple experimental runs to strengthen statistical validity.

\begin{figure}[!htbp]
\centering
\includegraphics[width=0.95\linewidth]{images/Real_Time_Sumo_Threshold/(c)_Distance.png}
\caption{(c) Distance}
\label{fig:Real_Time_Sumo_Threshold_c__Distance}
\end{figure}

This bar chart presents the relationship between threshold and total distance covered for the Real Time Sumo Threshold experimental scenario. The x-axis displays threshold values ranging from 5 to 20, while the y-axis quantifies total distance covered. The single-series visualization facilitates analysis of how the dependent variable responds to changes in the independent parameter setting.

Analysis of the plotted data reveals that total distance covered ranges from 60.06 (at threshold = 5) to 63.76 (at threshold = 15), representing a span of 3.70 units. The overall trend is increasing, with values rising from 60.06 at the initial setting to 61.70 at the final setting. 

These results indicate that threshold configuration meaningfully impacts total distance covered in this experimental context. The relatively modest variation suggests that this parameter has limited influence on the measured metric within the tested range. Confidence in these findings is high given the direct correspondence between CSV data and plotted values. Future analysis should consider incorporating error bars representing variance across multiple experimental runs to strengthen statistical validity.

\begin{figure}[!htbp]
\centering
\includegraphics[width=0.95\linewidth]{images/Real_Time_Sumo_Threshold/(d)_runtime.png}
\caption{(d) runtime}
\label{fig:Real_Time_Sumo_Threshold_d__runtime}
\end{figure}

This bar chart presents the relationship between threshold and run time for the Real Time Sumo Threshold experimental scenario. The x-axis displays threshold values ranging from 5 to 20, while the y-axis quantifies run time. The single-series visualization facilitates analysis of how the dependent variable responds to changes in the independent parameter setting.

Analysis of the plotted data reveals that run time ranges from 546.38 (at threshold = 10) to 568.65 (at threshold = 15), representing a span of 22.27 units. The overall trend is increasing, with values rising from 547.79 at the initial setting to 553.99 at the final setting. Notably, the minimum value occurs at an intermediate threshold setting (10), suggesting non-monotonic behavior that warrants further investigation.

These results indicate that threshold configuration meaningfully impacts run time in this experimental context. The relatively modest variation suggests that this parameter has limited influence on the measured metric within the tested range. Confidence in these findings is high given the direct correspondence between CSV data and plotted values. Future analysis should consider incorporating error bars representing variance across multiple experimental runs to strengthen statistical validity.


\clearpage

\subsection{Sumo Static Module Change}

\begin{figure}[!htbp]
\centering
\includegraphics[width=0.95\linewidth]{images/Sumo_Static_Module_Change/(a)_Travel_Time.png}
\caption{(a) Travel Time}
\label{fig:Sumo_Static_Module_Change_a__Travel_Time}
\end{figure}

This bar chart presents the relationship between modules and total travel time for the Sumo Static Module Change experimental scenario. The x-axis displays modules values ranging from 3 to 6, while the y-axis quantifies total travel time. The single-series visualization facilitates analysis of how the dependent variable responds to changes in the independent parameter setting.

Analysis of the plotted data reveals that total travel time ranges from 92.73 (at modules = 5) to 108.71 (at modules = 6), representing a span of 15.98 units. The overall trend is increasing, with values rising from 94.47 at the initial setting to 108.71 at the final setting. Notably, the minimum value occurs at an intermediate modules setting (5), suggesting non-monotonic behavior that warrants further investigation.

These results indicate that modules configuration meaningfully impacts total travel time in this experimental context. The relatively modest variation suggests that this parameter has limited influence on the measured metric within the tested range. Confidence in these findings is high given the direct correspondence between CSV data and plotted values. Future analysis should consider incorporating error bars representing variance across multiple experimental runs to strengthen statistical validity.

\begin{figure}[!htbp]
\centering
\includegraphics[width=0.95\linewidth]{images/Sumo_Static_Module_Change/(b)_Energy.png}
\caption{(b) Energy}
\label{fig:Sumo_Static_Module_Change_b__Energy}
\end{figure}

This bar chart presents the relationship between modules and total energy consumed for the Sumo Static Module Change experimental scenario. The x-axis displays modules values ranging from 3 to 6, while the y-axis quantifies total energy consumed. The single-series visualization facilitates analysis of how the dependent variable responds to changes in the independent parameter setting.

Analysis of the plotted data reveals that total energy consumed ranges from 9.05 (at modules = 5) to 11.61 (at modules = 6), representing a span of 2.55 units. The overall trend is increasing, with values rising from 9.29 at the initial setting to 11.61 at the final setting. Notably, the minimum value occurs at an intermediate modules setting (5), suggesting non-monotonic behavior that warrants further investigation.

These results indicate that modules configuration meaningfully impacts total energy consumed in this experimental context. The substantial variation observed (coefficient of variation exceeding 20\%) suggests that parameter tuning could yield significant performance improvements. Confidence in these findings is high given the direct correspondence between CSV data and plotted values. Future analysis should consider incorporating error bars representing variance across multiple experimental runs to strengthen statistical validity.

\begin{figure}[!htbp]
\centering
\includegraphics[width=0.95\linewidth]{images/Sumo_Static_Module_Change/(c)_Distance.png}
\caption{(c) Distance}
\label{fig:Sumo_Static_Module_Change_c__Distance}
\end{figure}

This bar chart presents the relationship between modules and total distance covered for the Sumo Static Module Change experimental scenario. The x-axis displays modules values ranging from 3 to 6, while the y-axis quantifies total distance covered. The single-series visualization facilitates analysis of how the dependent variable responds to changes in the independent parameter setting.

Analysis of the plotted data reveals that total distance covered ranges from 57.52 (at modules = 5) to 69.86 (at modules = 6), representing a span of 12.34 units. The overall trend is increasing, with values rising from 59.51 at the initial setting to 69.86 at the final setting. Notably, the minimum value occurs at an intermediate modules setting (5), suggesting non-monotonic behavior that warrants further investigation.

These results indicate that modules configuration meaningfully impacts total distance covered in this experimental context. The relatively modest variation suggests that this parameter has limited influence on the measured metric within the tested range. Confidence in these findings is high given the direct correspondence between CSV data and plotted values. Future analysis should consider incorporating error bars representing variance across multiple experimental runs to strengthen statistical validity.

\begin{figure}[!htbp]
\centering
\includegraphics[width=0.95\linewidth]{images/Sumo_Static_Module_Change/(d)_Module_Swapped.png}
\caption{(d) Module Swapped}
\label{fig:Sumo_Static_Module_Change_d__Module_Swapped}
\end{figure}

This bar chart presents the relationship between modules and modules for the Sumo Static Module Change experimental scenario. The x-axis displays modules values ranging from 3 to 6, while the y-axis quantifies modules. The single-series visualization facilitates analysis of how the dependent variable responds to changes in the independent parameter setting.

Analysis of the plotted data reveals that modules ranges from 3.00 (at modules = 3) to 6.00 (at modules = 6), representing a span of 3.00 units. The overall trend is increasing, with values rising from 3.00 at the initial setting to 6.00 at the final setting. 

These results indicate that modules configuration meaningfully impacts modules in this experimental context. The substantial variation observed (coefficient of variation exceeding 20\%) suggests that parameter tuning could yield significant performance improvements. Confidence in these findings is high given the direct correspondence between CSV data and plotted values. Future analysis should consider incorporating error bars representing variance across multiple experimental runs to strengthen statistical validity.

\begin{figure}[!htbp]
\centering
\includegraphics[width=0.95\linewidth]{images/Sumo_Static_Module_Change/(e)_Run_Time.png}
\caption{(e) Run Time}
\label{fig:Sumo_Static_Module_Change_e__Run_Time}
\end{figure}

This bar chart presents the relationship between modules and total travel time for the Sumo Static Module Change experimental scenario. The x-axis displays modules values ranging from 3 to 6, while the y-axis quantifies total travel time. The single-series visualization facilitates analysis of how the dependent variable responds to changes in the independent parameter setting.

Analysis of the plotted data reveals that total travel time ranges from 92.73 (at modules = 5) to 108.71 (at modules = 6), representing a span of 15.98 units. The overall trend is increasing, with values rising from 94.47 at the initial setting to 108.71 at the final setting. Notably, the minimum value occurs at an intermediate modules setting (5), suggesting non-monotonic behavior that warrants further investigation.

These results indicate that modules configuration meaningfully impacts total travel time in this experimental context. The relatively modest variation suggests that this parameter has limited influence on the measured metric within the tested range. Confidence in these findings is high given the direct correspondence between CSV data and plotted values. Future analysis should consider incorporating error bars representing variance across multiple experimental runs to strengthen statistical validity.


\clearpage

\subsection{Sumo Static Swap Time}

\begin{figure}[!htbp]
\centering
\includegraphics[width=0.95\linewidth]{images/Sumo_Static_Swap_Time/(a)_Travel_Time.png}
\caption{(a) Travel Time}
\label{fig:Sumo_Static_Swap_Time_a__Travel_Time}
\end{figure}

This bar chart presents the relationship between swap time (min) and total travel time for the Sumo Static Swap Time experimental scenario. The x-axis displays swap time (min) values ranging from 1 to 4, while the y-axis quantifies total travel time. The single-series visualization facilitates analysis of how the dependent variable responds to changes in the independent parameter setting.

Analysis of the plotted data reveals that total travel time ranges from 85.81 (at swap time (min) = 4) to 100.69 (at swap time (min) = 3), representing a span of 14.88 units. The overall trend is decreasing, with values declining from 100.19 at the initial setting to 85.81 at the final setting. 

These results indicate that swap time (min) configuration meaningfully impacts total travel time in this experimental context. The relatively modest variation suggests that this parameter has limited influence on the measured metric within the tested range. Confidence in these findings is high given the direct correspondence between CSV data and plotted values. Future analysis should consider incorporating error bars representing variance across multiple experimental runs to strengthen statistical validity.

\begin{figure}[!htbp]
\centering
\includegraphics[width=0.95\linewidth]{images/Sumo_Static_Swap_Time/(b)_Energy.png}
\caption{(b) Energy}
\label{fig:Sumo_Static_Swap_Time_b__Energy}
\end{figure}

This bar chart presents the relationship between swap time (min) and total energy consumed for the Sumo Static Swap Time experimental scenario. The x-axis displays swap time (min) values ranging from 1 to 4, while the y-axis quantifies total energy consumed. The single-series visualization facilitates analysis of how the dependent variable responds to changes in the independent parameter setting.

Analysis of the plotted data reveals that total energy consumed ranges from 9.05 (at swap time (min) = 2) to 11.51 (at swap time (min) = 3), representing a span of 2.46 units. The overall trend is decreasing, with values declining from 11.00 at the initial setting to 9.75 at the final setting. Notably, the minimum value occurs at an intermediate swap time (min) setting (2), suggesting non-monotonic behavior that warrants further investigation.

These results indicate that swap time (min) configuration meaningfully impacts total energy consumed in this experimental context. The substantial variation observed (coefficient of variation exceeding 20\%) suggests that parameter tuning could yield significant performance improvements. Confidence in these findings is high given the direct correspondence between CSV data and plotted values. Future analysis should consider incorporating error bars representing variance across multiple experimental runs to strengthen statistical validity.

\begin{figure}[!htbp]
\centering
\includegraphics[width=0.95\linewidth]{images/Sumo_Static_Swap_Time/(c)_Distance.png}
\caption{(c) Distance}
\label{fig:Sumo_Static_Swap_Time_c__Distance}
\end{figure}

This bar chart presents the relationship between swap time (min) and total distance covered for the Sumo Static Swap Time experimental scenario. The x-axis displays swap time (min) values ranging from 1 to 4, while the y-axis quantifies total distance covered. The single-series visualization facilitates analysis of how the dependent variable responds to changes in the independent parameter setting.

Analysis of the plotted data reveals that total distance covered ranges from 57.11 (at swap time (min) = 4) to 66.91 (at swap time (min) = 3), representing a span of 9.80 units. The overall trend is decreasing, with values declining from 64.69 at the initial setting to 57.11 at the final setting. 

These results indicate that swap time (min) configuration meaningfully impacts total distance covered in this experimental context. The relatively modest variation suggests that this parameter has limited influence on the measured metric within the tested range. Confidence in these findings is high given the direct correspondence between CSV data and plotted values. Future analysis should consider incorporating error bars representing variance across multiple experimental runs to strengthen statistical validity.

\begin{figure}[!htbp]
\centering
\includegraphics[width=0.95\linewidth]{images/Sumo_Static_Swap_Time/(d)_Module_Swapped.png}
\caption{(d) Module Swapped}
\label{fig:Sumo_Static_Swap_Time_d__Module_Swapped}
\end{figure}

This bar chart presents the relationship between swap time (min) and total module swapped for the Sumo Static Swap Time experimental scenario. The x-axis displays swap time (min) values ranging from 1 to 4, while the y-axis quantifies total module swapped. The single-series visualization facilitates analysis of how the dependent variable responds to changes in the independent parameter setting.

Analysis of the plotted data reveals that total module swapped ranges from 0.00 (at swap time (min) = 1) to 0.00 (at swap time (min) = 1), representing a span of 0.00 units. The values remain relatively stable across the parameter range, with minimal net change between initial (0.00) and final (0.00) settings. 

These results indicate that swap time (min) configuration meaningfully impacts total module swapped in this experimental context. The relatively modest variation suggests that this parameter has limited influence on the measured metric within the tested range. Confidence in these findings is high given the direct correspondence between CSV data and plotted values. Future analysis should consider incorporating error bars representing variance across multiple experimental runs to strengthen statistical validity.

\begin{figure}[!htbp]
\centering
\includegraphics[width=0.95\linewidth]{images/Sumo_Static_Swap_Time/(e)_Run_Time.png}
\caption{(e) Run Time}
\label{fig:Sumo_Static_Swap_Time_e__Run_Time}
\end{figure}

This bar chart presents the relationship between swap time (min) and total travel time for the Sumo Static Swap Time experimental scenario. The x-axis displays swap time (min) values ranging from 1 to 4, while the y-axis quantifies total travel time. The single-series visualization facilitates analysis of how the dependent variable responds to changes in the independent parameter setting.

Analysis of the plotted data reveals that total travel time ranges from 85.81 (at swap time (min) = 4) to 100.69 (at swap time (min) = 3), representing a span of 14.88 units. The overall trend is decreasing, with values declining from 100.19 at the initial setting to 85.81 at the final setting. 

These results indicate that swap time (min) configuration meaningfully impacts total travel time in this experimental context. The relatively modest variation suggests that this parameter has limited influence on the measured metric within the tested range. Confidence in these findings is high given the direct correspondence between CSV data and plotted values. Future analysis should consider incorporating error bars representing variance across multiple experimental runs to strengthen statistical validity.


\clearpage

\subsection{Sumo Static Threshold}

\begin{figure}[!htbp]
\centering
\includegraphics[width=0.95\linewidth]{images/Sumo_Static_Threshold/(a)_Travel_Time.png}
\caption{(a) Travel Time}
\label{fig:Sumo_Static_Threshold_a__Travel_Time}
\end{figure}

This bar chart presents the relationship between threshold and total travel time for the Sumo Static Threshold experimental scenario. The x-axis displays threshold values ranging from 5 to 20, while the y-axis quantifies total travel time. The single-series visualization facilitates analysis of how the dependent variable responds to changes in the independent parameter setting.

Analysis of the plotted data reveals that total travel time ranges from 86.54 (at threshold = 10) to 106.44 (at threshold = 5), representing a span of 19.90 units. The overall trend is decreasing, with values declining from 106.44 at the initial setting to 92.73 at the final setting. Notably, the minimum value occurs at an intermediate threshold setting (10), suggesting non-monotonic behavior that warrants further investigation.

These results indicate that threshold configuration meaningfully impacts total travel time in this experimental context. The substantial variation observed (coefficient of variation exceeding 20\%) suggests that parameter tuning could yield significant performance improvements. Confidence in these findings is high given the direct correspondence between CSV data and plotted values. Future analysis should consider incorporating error bars representing variance across multiple experimental runs to strengthen statistical validity.

\begin{figure}[!htbp]
\centering
\includegraphics[width=0.95\linewidth]{images/Sumo_Static_Threshold/(b)_Energy.png}
\caption{(b) Energy}
\label{fig:Sumo_Static_Threshold_b__Energy}
\end{figure}

This bar chart presents the relationship between threshold and total energy consumed for the Sumo Static Threshold experimental scenario. The x-axis displays threshold values ranging from 5 to 20, while the y-axis quantifies total energy consumed. The single-series visualization facilitates analysis of how the dependent variable responds to changes in the independent parameter setting.

Analysis of the plotted data reveals that total energy consumed ranges from 8.80 (at threshold = 10) to 11.34 (at threshold = 5), representing a span of 2.54 units. The overall trend is decreasing, with values declining from 11.34 at the initial setting to 9.05 at the final setting. Notably, the minimum value occurs at an intermediate threshold setting (10), suggesting non-monotonic behavior that warrants further investigation.

These results indicate that threshold configuration meaningfully impacts total energy consumed in this experimental context. The substantial variation observed (coefficient of variation exceeding 20\%) suggests that parameter tuning could yield significant performance improvements. Confidence in these findings is high given the direct correspondence between CSV data and plotted values. Future analysis should consider incorporating error bars representing variance across multiple experimental runs to strengthen statistical validity.

\begin{figure}[!htbp]
\centering
\includegraphics[width=0.95\linewidth]{images/Sumo_Static_Threshold/(c)_Distance.png}
\caption{(c) Distance}
\label{fig:Sumo_Static_Threshold_c__Distance}
\end{figure}

This bar chart presents the relationship between threshold and total distance covered for the Sumo Static Threshold experimental scenario. The x-axis displays threshold values ranging from 5 to 20, while the y-axis quantifies total distance covered. The single-series visualization facilitates analysis of how the dependent variable responds to changes in the independent parameter setting.

Analysis of the plotted data reveals that total distance covered ranges from 55.14 (at threshold = 10) to 69.18 (at threshold = 5), representing a span of 14.04 units. The overall trend is decreasing, with values declining from 69.18 at the initial setting to 57.52 at the final setting. Notably, the minimum value occurs at an intermediate threshold setting (10), suggesting non-monotonic behavior that warrants further investigation.

These results indicate that threshold configuration meaningfully impacts total distance covered in this experimental context. The substantial variation observed (coefficient of variation exceeding 20\%) suggests that parameter tuning could yield significant performance improvements. Confidence in these findings is high given the direct correspondence between CSV data and plotted values. Future analysis should consider incorporating error bars representing variance across multiple experimental runs to strengthen statistical validity.

\begin{figure}[!htbp]
\centering
\includegraphics[width=0.95\linewidth]{images/Sumo_Static_Threshold/(d)_Module_Swapped.png}
\caption{(d) Module Swapped}
\label{fig:Sumo_Static_Threshold_d__Module_Swapped}
\end{figure}

This bar chart presents the relationship between threshold and total module swapped for the Sumo Static Threshold experimental scenario. The x-axis displays threshold values ranging from 5 to 20, while the y-axis quantifies total module swapped. The single-series visualization facilitates analysis of how the dependent variable responds to changes in the independent parameter setting.

Analysis of the plotted data reveals that total module swapped ranges from 0.00 (at threshold = 5) to 0.00 (at threshold = 5), representing a span of 0.00 units. The values remain relatively stable across the parameter range, with minimal net change between initial (0.00) and final (0.00) settings. 

These results indicate that threshold configuration meaningfully impacts total module swapped in this experimental context. The relatively modest variation suggests that this parameter has limited influence on the measured metric within the tested range. Confidence in these findings is high given the direct correspondence between CSV data and plotted values. Future analysis should consider incorporating error bars representing variance across multiple experimental runs to strengthen statistical validity.

\begin{figure}[!htbp]
\centering
\includegraphics[width=0.95\linewidth]{images/Sumo_Static_Threshold/(e)_Run_Time.png}
\caption{(e) Run Time}
\label{fig:Sumo_Static_Threshold_e__Run_Time}
\end{figure}

This bar chart presents the relationship between threshold and total travel time for the Sumo Static Threshold experimental scenario. The x-axis displays threshold values ranging from 5 to 20, while the y-axis quantifies total travel time. The single-series visualization facilitates analysis of how the dependent variable responds to changes in the independent parameter setting.

Analysis of the plotted data reveals that total travel time ranges from 86.54 (at threshold = 10) to 106.44 (at threshold = 5), representing a span of 19.90 units. The overall trend is decreasing, with values declining from 106.44 at the initial setting to 92.73 at the final setting. Notably, the minimum value occurs at an intermediate threshold setting (10), suggesting non-monotonic behavior that warrants further investigation.

These results indicate that threshold configuration meaningfully impacts total travel time in this experimental context. The substantial variation observed (coefficient of variation exceeding 20\%) suggests that parameter tuning could yield significant performance improvements. Confidence in these findings is high given the direct correspondence between CSV data and plotted values. Future analysis should consider incorporating error bars representing variance across multiple experimental runs to strengthen statistical validity.


\clearpage

% Requires: \usepackage{graphicx}

\subsection{1000 nodes Module Change}

\begin{figure}[!htbp]
\centering
\includegraphics[width=0.95\linewidth]{images/1000_nodes_Module_Change/(a)_Travel_Time_bar_chart.png}
\caption{(a) Travel Time}
\label{fig:1000_nodes_Module_Change_a__Travel_Time_bar_chart}
\end{figure}

This bar chart presents the relationship between modules and total travel time for the 1000 nodes Module Change experimental scenario. The x-axis displays modules values ranging from 4 to 7, while the y-axis quantifies total travel time. The single-series visualization facilitates analysis of how the dependent variable responds to changes in the independent parameter setting.

Analysis of the plotted data reveals that total travel time ranges from 6779.83 (at modules = 4) to 6808.90 (at modules = 5), representing a span of 29.07 units. The overall trend is increasing, with values rising from 6779.83 at the initial setting to 6808.90 at the final setting. 

These results indicate that modules configuration meaningfully impacts total travel time in this experimental context. The relatively modest variation suggests that this parameter has limited influence on the measured metric within the tested range. Confidence in these findings is high given the direct correspondence between CSV data and plotted values. Future analysis should consider incorporating error bars representing variance across multiple experimental runs to strengthen statistical validity.

\begin{figure}[!htbp]
\centering
\includegraphics[width=0.95\linewidth]{images/1000_nodes_Module_Change/(b)_Energy_consumption_bar_chart.png}
\caption{(b) Energy consumption}
\label{fig:1000_nodes_Module_Change_b__Energy_consumption_bar_chart}
\end{figure}

This bar chart presents the relationship between modules and total energy consumed for the 1000 nodes Module Change experimental scenario. The x-axis displays modules values ranging from 4 to 7, while the y-axis quantifies total energy consumed. The single-series visualization facilitates analysis of how the dependent variable responds to changes in the independent parameter setting.

Analysis of the plotted data reveals that total energy consumed ranges from 793.93 (at modules = 4) to 795.16 (at modules = 5), representing a span of 1.22 units. The overall trend is increasing, with values rising from 793.93 at the initial setting to 795.16 at the final setting. 

These results indicate that modules configuration meaningfully impacts total energy consumed in this experimental context. The relatively modest variation suggests that this parameter has limited influence on the measured metric within the tested range. Confidence in these findings is high given the direct correspondence between CSV data and plotted values. Future analysis should consider incorporating error bars representing variance across multiple experimental runs to strengthen statistical validity.

\begin{figure}[!htbp]
\centering
\includegraphics[width=0.95\linewidth]{images/1000_nodes_Module_Change/(c)_Distance_covered_bar_chart.png}
\caption{(c) Distance covered}
\label{fig:1000_nodes_Module_Change_c__Distance_covered_bar_chart}
\end{figure}

This bar chart presents the relationship between modules and total distance covered for the 1000 nodes Module Change experimental scenario. The x-axis displays modules values ranging from 4 to 7, while the y-axis quantifies total distance covered. The single-series visualization facilitates analysis of how the dependent variable responds to changes in the independent parameter setting.

Analysis of the plotted data reveals that total distance covered ranges from 4422.50 (at modules = 4) to 4437.52 (at modules = 5), representing a span of 15.02 units. The overall trend is increasing, with values rising from 4422.50 at the initial setting to 4437.52 at the final setting. 

These results indicate that modules configuration meaningfully impacts total distance covered in this experimental context. The relatively modest variation suggests that this parameter has limited influence on the measured metric within the tested range. Confidence in these findings is high given the direct correspondence between CSV data and plotted values. Future analysis should consider incorporating error bars representing variance across multiple experimental runs to strengthen statistical validity.

\begin{figure}[!htbp]
\centering
\includegraphics[width=0.95\linewidth]{images/1000_nodes_Module_Change/(d)_Runtime_bar_chart.png}
\caption{(d) Runtime}
\label{fig:1000_nodes_Module_Change_d__Runtime_bar_chart}
\end{figure}

This bar chart presents the relationship between modules and run time for the 1000 nodes Module Change experimental scenario. The x-axis displays modules values ranging from 4 to 7, while the y-axis quantifies run time. The single-series visualization facilitates analysis of how the dependent variable responds to changes in the independent parameter setting.

Analysis of the plotted data reveals that run time ranges from 187.29 (at modules = 4) to 396.10 (at modules = 5), representing a span of 208.81 units. The overall trend is increasing, with values rising from 187.29 at the initial setting to 226.36 at the final setting. 

These results indicate that modules configuration meaningfully impacts run time in this experimental context. The substantial variation observed (coefficient of variation exceeding 20\%) suggests that parameter tuning could yield significant performance improvements. Confidence in these findings is high given the direct correspondence between CSV data and plotted values. Future analysis should consider incorporating error bars representing variance across multiple experimental runs to strengthen statistical validity.

\begin{figure}[!htbp]
\centering
\includegraphics[width=0.95\linewidth]{images/1000_nodes_Module_Change/(e)_Number_of_module_swapped_bar_chart.png}
\caption{(e) Number of module swapped}
\label{fig:1000_nodes_Module_Change_e__Number_of_module_swapped_bar_chart}
\end{figure}

This bar chart presents the relationship between modules and modules for the 1000 nodes Module Change experimental scenario. The x-axis displays modules values ranging from 4 to 7, while the y-axis quantifies modules. The single-series visualization facilitates analysis of how the dependent variable responds to changes in the independent parameter setting.

Analysis of the plotted data reveals that modules ranges from 4.00 (at modules = 4) to 7.00 (at modules = 7), representing a span of 3.00 units. The overall trend is increasing, with values rising from 4.00 at the initial setting to 7.00 at the final setting. 

These results indicate that modules configuration meaningfully impacts modules in this experimental context. The substantial variation observed (coefficient of variation exceeding 20\%) suggests that parameter tuning could yield significant performance improvements. Confidence in these findings is high given the direct correspondence between CSV data and plotted values. Future analysis should consider incorporating error bars representing variance across multiple experimental runs to strengthen statistical validity.


\clearpage

\subsection{1000 nodes Swapping Time}

\begin{figure}[!htbp]
\centering
\includegraphics[width=0.95\linewidth]{images/1000_nodes_Swapping_Time/(a)_Travel_Time_bar_chart.png}
\caption{(a) Travel Time}
\label{fig:1000_nodes_Swapping_Time_a__Travel_Time_bar_chart}
\end{figure}

This bar chart presents the relationship between swapping time and total travel time for the 1000 nodes Swapping Time experimental scenario. The x-axis displays swapping time values ranging from 1 to 4, while the y-axis quantifies total travel time. The single-series visualization facilitates analysis of how the dependent variable responds to changes in the independent parameter setting.

Analysis of the plotted data reveals that total travel time ranges from 6771.90 (at swapping time = 1) to 6882.78 (at swapping time = 4), representing a span of 110.88 units. The overall trend is increasing, with values rising from 6771.90 at the initial setting to 6882.78 at the final setting. 

These results indicate that swapping time configuration meaningfully impacts total travel time in this experimental context. The relatively modest variation suggests that this parameter has limited influence on the measured metric within the tested range. Confidence in these findings is high given the direct correspondence between CSV data and plotted values. Future analysis should consider incorporating error bars representing variance across multiple experimental runs to strengthen statistical validity.

\begin{figure}[!htbp]
\centering
\includegraphics[width=0.95\linewidth]{images/1000_nodes_Swapping_Time/(b)_Energy_consumption_bar_chart.png}
\caption{(b) Energy consumption}
\label{fig:1000_nodes_Swapping_Time_b__Energy_consumption_bar_chart}
\end{figure}

This bar chart presents the relationship between swapping time and total energy consumed for the 1000 nodes Swapping Time experimental scenario. The x-axis displays swapping time values ranging from 1 to 4, while the y-axis quantifies total energy consumed. The single-series visualization facilitates analysis of how the dependent variable responds to changes in the independent parameter setting.

Analysis of the plotted data reveals that total energy consumed ranges from 794.98 (at swapping time = 3) to 795.16 (at swapping time = 1), representing a span of 0.18 units. The overall trend is decreasing, with values declining from 795.16 at the initial setting to 794.98 at the final setting. Notably, the minimum value occurs at an intermediate swapping time setting (3), suggesting non-monotonic behavior that warrants further investigation.

These results indicate that swapping time configuration meaningfully impacts total energy consumed in this experimental context. The relatively modest variation suggests that this parameter has limited influence on the measured metric within the tested range. Confidence in these findings is high given the direct correspondence between CSV data and plotted values. Future analysis should consider incorporating error bars representing variance across multiple experimental runs to strengthen statistical validity.

\begin{figure}[!htbp]
\centering
\includegraphics[width=0.95\linewidth]{images/1000_nodes_Swapping_Time/(c)_Distance_covered_bar_chart.png}
\caption{(c) Distance covered}
\label{fig:1000_nodes_Swapping_Time_c__Distance_covered_bar_chart}
\end{figure}

This bar chart presents the relationship between swapping time and total distance covered for the 1000 nodes Swapping Time experimental scenario. The x-axis displays swapping time values ranging from 1 to 4, while the y-axis quantifies total distance covered. The single-series visualization facilitates analysis of how the dependent variable responds to changes in the independent parameter setting.

Analysis of the plotted data reveals that total distance covered ranges from 4436.55 (at swapping time = 3) to 4437.52 (at swapping time = 1), representing a span of 0.97 units. The overall trend is decreasing, with values declining from 4437.52 at the initial setting to 4436.55 at the final setting. Notably, the minimum value occurs at an intermediate swapping time setting (3), suggesting non-monotonic behavior that warrants further investigation.

These results indicate that swapping time configuration meaningfully impacts total distance covered in this experimental context. The relatively modest variation suggests that this parameter has limited influence on the measured metric within the tested range. Confidence in these findings is high given the direct correspondence between CSV data and plotted values. Future analysis should consider incorporating error bars representing variance across multiple experimental runs to strengthen statistical validity.

\begin{figure}[!htbp]
\centering
\includegraphics[width=0.95\linewidth]{images/1000_nodes_Swapping_Time/(d)_Runtime_bar_chart.png}
\caption{(d) Runtime}
\label{fig:1000_nodes_Swapping_Time_d__Runtime_bar_chart}
\end{figure}

This bar chart presents the relationship between swapping time and run time for the 1000 nodes Swapping Time experimental scenario. The x-axis displays swapping time values ranging from 1 to 4, while the y-axis quantifies run time. The single-series visualization facilitates analysis of how the dependent variable responds to changes in the independent parameter setting.

Analysis of the plotted data reveals that run time ranges from 190.96 (at swapping time = 3) to 399.38 (at swapping time = 4), representing a span of 208.42 units. The overall trend is increasing, with values rising from 363.28 at the initial setting to 399.38 at the final setting. Notably, the minimum value occurs at an intermediate swapping time setting (3), suggesting non-monotonic behavior that warrants further investigation.

These results indicate that swapping time configuration meaningfully impacts run time in this experimental context. The substantial variation observed (coefficient of variation exceeding 20\%) suggests that parameter tuning could yield significant performance improvements. Confidence in these findings is high given the direct correspondence between CSV data and plotted values. Future analysis should consider incorporating error bars representing variance across multiple experimental runs to strengthen statistical validity.

\begin{figure}[!htbp]
\centering
\includegraphics[width=0.95\linewidth]{images/1000_nodes_Swapping_Time/(e)_Number_of_module_swapped_bar_chart.png}
\caption{(e) Number of module swapped}
\label{fig:1000_nodes_Swapping_Time_e__Number_of_module_swapped_bar_chart}
\end{figure}

This bar chart presents the relationship between swapping time and total module swapped for the 1000 nodes Swapping Time experimental scenario. The x-axis displays swapping time values ranging from 1 to 4, while the y-axis quantifies total module swapped. The single-series visualization facilitates analysis of how the dependent variable responds to changes in the independent parameter setting.

Analysis of the plotted data reveals that total module swapped ranges from 37.00 (at swapping time = 1) to 37.00 (at swapping time = 1), representing a span of 0.00 units. The values remain relatively stable across the parameter range, with minimal net change between initial (37.00) and final (37.00) settings. 

These results indicate that swapping time configuration meaningfully impacts total module swapped in this experimental context. The relatively modest variation suggests that this parameter has limited influence on the measured metric within the tested range. Confidence in these findings is high given the direct correspondence between CSV data and plotted values. Future analysis should consider incorporating error bars representing variance across multiple experimental runs to strengthen statistical validity.


\clearpage

\subsection{1000 nodes Threshold}

\begin{figure}[!htbp]
\centering
\includegraphics[width=0.95\linewidth]{images/1000_nodes_Threshold/(a)_Travel_Time_bar_chart.png}
\caption{(a) Travel Time}
\label{fig:1000_nodes_Threshold_a__Travel_Time_bar_chart}
\end{figure}

This bar chart presents the relationship between threshold and total travel time for the 1000 nodes Threshold experimental scenario. The x-axis displays threshold values ranging from 5 to 20, while the y-axis quantifies total travel time. The single-series visualization facilitates analysis of how the dependent variable responds to changes in the independent parameter setting.

Analysis of the plotted data reveals that total travel time ranges from 6808.90 (at threshold = 5) to 6808.90 (at threshold = 5), representing a span of 0.00 units. The values remain relatively stable across the parameter range, with minimal net change between initial (6808.90) and final (6808.90) settings. 

These results indicate that threshold configuration meaningfully impacts total travel time in this experimental context. The relatively modest variation suggests that this parameter has limited influence on the measured metric within the tested range. Confidence in these findings is high given the direct correspondence between CSV data and plotted values. Future analysis should consider incorporating error bars representing variance across multiple experimental runs to strengthen statistical validity.

\begin{figure}[!htbp]
\centering
\includegraphics[width=0.95\linewidth]{images/1000_nodes_Threshold/(b)_Energy_consumption_bar_chart.png}
\caption{(b) Energy consumption}
\label{fig:1000_nodes_Threshold_b__Energy_consumption_bar_chart}
\end{figure}

This bar chart presents the relationship between threshold and total energy consumed for the 1000 nodes Threshold experimental scenario. The x-axis displays threshold values ranging from 5 to 20, while the y-axis quantifies total energy consumed. The single-series visualization facilitates analysis of how the dependent variable responds to changes in the independent parameter setting.

Analysis of the plotted data reveals that total energy consumed ranges from 795.16 (at threshold = 5) to 795.16 (at threshold = 5), representing a span of 0.00 units. The values remain relatively stable across the parameter range, with minimal net change between initial (795.16) and final (795.16) settings. 

These results indicate that threshold configuration meaningfully impacts total energy consumed in this experimental context. The relatively modest variation suggests that this parameter has limited influence on the measured metric within the tested range. Confidence in these findings is high given the direct correspondence between CSV data and plotted values. Future analysis should consider incorporating error bars representing variance across multiple experimental runs to strengthen statistical validity.

\begin{figure}[!htbp]
\centering
\includegraphics[width=0.95\linewidth]{images/1000_nodes_Threshold/(c)_Distance_covered_bar_chart.png}
\caption{(c) Distance covered}
\label{fig:1000_nodes_Threshold_c__Distance_covered_bar_chart}
\end{figure}

This bar chart presents the relationship between threshold and total distance covered for the 1000 nodes Threshold experimental scenario. The x-axis displays threshold values ranging from 5 to 20, while the y-axis quantifies total distance covered. The single-series visualization facilitates analysis of how the dependent variable responds to changes in the independent parameter setting.

Analysis of the plotted data reveals that total distance covered ranges from 4437.52 (at threshold = 5) to 4437.52 (at threshold = 5), representing a span of 0.00 units. The values remain relatively stable across the parameter range, with minimal net change between initial (4437.52) and final (4437.52) settings. 

These results indicate that threshold configuration meaningfully impacts total distance covered in this experimental context. The relatively modest variation suggests that this parameter has limited influence on the measured metric within the tested range. Confidence in these findings is high given the direct correspondence between CSV data and plotted values. Future analysis should consider incorporating error bars representing variance across multiple experimental runs to strengthen statistical validity.

\begin{figure}[!htbp]
\centering
\includegraphics[width=0.95\linewidth]{images/1000_nodes_Threshold/(d)_Runtime_bar_chart.png}
\caption{(d) Runtime}
\label{fig:1000_nodes_Threshold_d__Runtime_bar_chart}
\end{figure}

This bar chart presents the relationship between threshold and run time for the 1000 nodes Threshold experimental scenario. The x-axis displays threshold values ranging from 5 to 20, while the y-axis quantifies run time. The single-series visualization facilitates analysis of how the dependent variable responds to changes in the independent parameter setting.

Analysis of the plotted data reveals that run time ranges from 225.32 (at threshold = 15) to 396.10 (at threshold = 20), representing a span of 170.78 units. The overall trend is increasing, with values rising from 287.89 at the initial setting to 396.10 at the final setting. Notably, the minimum value occurs at an intermediate threshold setting (15), suggesting non-monotonic behavior that warrants further investigation.

These results indicate that threshold configuration meaningfully impacts run time in this experimental context. The substantial variation observed (coefficient of variation exceeding 20\%) suggests that parameter tuning could yield significant performance improvements. Confidence in these findings is high given the direct correspondence between CSV data and plotted values. Future analysis should consider incorporating error bars representing variance across multiple experimental runs to strengthen statistical validity.

\begin{figure}[!htbp]
\centering
\includegraphics[width=0.95\linewidth]{images/1000_nodes_Threshold/(e)_Number_of_module_swapped_bar_chart.png}
\caption{(e) Number of module swapped}
\label{fig:1000_nodes_Threshold_e__Number_of_module_swapped_bar_chart}
\end{figure}

This bar chart presents the relationship between threshold and total module swapped for the 1000 nodes Threshold experimental scenario. The x-axis displays threshold values ranging from 5 to 20, while the y-axis quantifies total module swapped. The single-series visualization facilitates analysis of how the dependent variable responds to changes in the independent parameter setting.

Analysis of the plotted data reveals that total module swapped ranges from 37.00 (at threshold = 5) to 37.00 (at threshold = 5), representing a span of 0.00 units. The values remain relatively stable across the parameter range, with minimal net change between initial (37.00) and final (37.00) settings. 

These results indicate that threshold configuration meaningfully impacts total module swapped in this experimental context. The relatively modest variation suggests that this parameter has limited influence on the measured metric within the tested range. Confidence in these findings is high given the direct correspondence between CSV data and plotted values. Future analysis should consider incorporating error bars representing variance across multiple experimental runs to strengthen statistical validity.


\clearpage

\subsection{1000 nodes Traffic}

\begin{figure}[!htbp]
\centering
\includegraphics[width=0.95\linewidth]{images/1000_nodes_Traffic/(a)_Travel_Time_bar_chart.png}
\caption{(a) Travel Time}
\label{fig:1000_nodes_Traffic_a__Travel_Time_bar_chart}
\end{figure}

This bar chart presents the relationship between traffic and total travel time for the 1000 nodes Traffic experimental scenario. The x-axis displays traffic values ranging from High to Low, while the y-axis quantifies total travel time. The single-series visualization facilitates analysis of how the dependent variable responds to changes in the independent parameter setting.

Analysis of the plotted data reveals that total travel time ranges from 6269.66 (at traffic = Low) to 8004.34 (at traffic = High), representing a span of 1734.68 units. The overall trend is decreasing, with values declining from 8004.34 at the initial setting to 6269.66 at the final setting. 

These results indicate that traffic configuration meaningfully impacts total travel time in this experimental context. The substantial variation observed (coefficient of variation exceeding 20\%) suggests that parameter tuning could yield significant performance improvements. Confidence in these findings is high given the direct correspondence between CSV data and plotted values. Future analysis should consider incorporating error bars representing variance across multiple experimental runs to strengthen statistical validity.

\begin{figure}[!htbp]
\centering
\includegraphics[width=0.95\linewidth]{images/1000_nodes_Traffic/(b)_Energy_consumption_bar_chart.png}
\caption{(b) Energy consumption}
\label{fig:1000_nodes_Traffic_b__Energy_consumption_bar_chart}
\end{figure}

This bar chart presents the relationship between traffic and total energy consumed for the 1000 nodes Traffic experimental scenario. The x-axis displays traffic values ranging from High to Low, while the y-axis quantifies total energy consumed. The single-series visualization facilitates analysis of how the dependent variable responds to changes in the independent parameter setting.

Analysis of the plotted data reveals that total energy consumed ranges from 757.04 (at traffic = High) to 883.10 (at traffic = Low), representing a span of 126.06 units. The overall trend is increasing, with values rising from 757.04 at the initial setting to 883.10 at the final setting. 

These results indicate that traffic configuration meaningfully impacts total energy consumed in this experimental context. The relatively modest variation suggests that this parameter has limited influence on the measured metric within the tested range. Confidence in these findings is high given the direct correspondence between CSV data and plotted values. Future analysis should consider incorporating error bars representing variance across multiple experimental runs to strengthen statistical validity.

\begin{figure}[!htbp]
\centering
\includegraphics[width=0.95\linewidth]{images/1000_nodes_Traffic/(c)_Distance_covered_bar_chart.png}
\caption{(c) Distance covered}
\label{fig:1000_nodes_Traffic_c__Distance_covered_bar_chart}
\end{figure}

This bar chart presents the relationship between traffic and total distance covered for the 1000 nodes Traffic experimental scenario. The x-axis displays traffic values ranging from High to Low, while the y-axis quantifies total distance covered. The single-series visualization facilitates analysis of how the dependent variable responds to changes in the independent parameter setting.

Analysis of the plotted data reveals that total distance covered ranges from 4374.17 (at traffic = High) to 4806.46 (at traffic = Low), representing a span of 432.29 units. The overall trend is increasing, with values rising from 4374.17 at the initial setting to 4806.46 at the final setting. 

These results indicate that traffic configuration meaningfully impacts total distance covered in this experimental context. The relatively modest variation suggests that this parameter has limited influence on the measured metric within the tested range. Confidence in these findings is high given the direct correspondence between CSV data and plotted values. Future analysis should consider incorporating error bars representing variance across multiple experimental runs to strengthen statistical validity.

\begin{figure}[!htbp]
\centering
\includegraphics[width=0.95\linewidth]{images/1000_nodes_Traffic/(d)_Runtime_bar_chart.png}
\caption{(d) Runtime}
\label{fig:1000_nodes_Traffic_d__Runtime_bar_chart}
\end{figure}

This bar chart presents the relationship between traffic and run time for the 1000 nodes Traffic experimental scenario. The x-axis displays traffic values ranging from High to Low, while the y-axis quantifies run time. The single-series visualization facilitates analysis of how the dependent variable responds to changes in the independent parameter setting.

Analysis of the plotted data reveals that run time ranges from 368.03 (at traffic = High) to 407.25 (at traffic = Mid), representing a span of 39.22 units. The overall trend is increasing, with values rising from 368.03 at the initial setting to 382.14 at the final setting. 

These results indicate that traffic configuration meaningfully impacts run time in this experimental context. The relatively modest variation suggests that this parameter has limited influence on the measured metric within the tested range. Confidence in these findings is high given the direct correspondence between CSV data and plotted values. Future analysis should consider incorporating error bars representing variance across multiple experimental runs to strengthen statistical validity.

\begin{figure}[!htbp]
\centering
\includegraphics[width=0.95\linewidth]{images/1000_nodes_Traffic/(e)_Number_of_module_swapped_bar_chart.png}
\caption{(e) Number of module swapped}
\label{fig:1000_nodes_Traffic_e__Number_of_module_swapped_bar_chart}
\end{figure}

This bar chart presents the relationship between traffic and total module swapped for the 1000 nodes Traffic experimental scenario. The x-axis displays traffic values ranging from High to Low, while the y-axis quantifies total module swapped. The single-series visualization facilitates analysis of how the dependent variable responds to changes in the independent parameter setting.

Analysis of the plotted data reveals that total module swapped ranges from 35.00 (at traffic = High) to 41.00 (at traffic = Low), representing a span of 6.00 units. The overall trend is increasing, with values rising from 35.00 at the initial setting to 41.00 at the final setting. 

These results indicate that traffic configuration meaningfully impacts total module swapped in this experimental context. The relatively modest variation suggests that this parameter has limited influence on the measured metric within the tested range. Confidence in these findings is high given the direct correspondence between CSV data and plotted values. Future analysis should consider incorporating error bars representing variance across multiple experimental runs to strengthen statistical validity.


\clearpage

\subsection{1500 nodes Module Change}

\begin{figure}[!htbp]
\centering
\includegraphics[width=0.95\linewidth]{images/1500_nodes_Module_Change/(a)_Travel_time_bar_chart.png}
\caption{(a) Travel time}
\label{fig:1500_nodes_Module_Change_a__Travel_time_bar_chart}
\end{figure}

This bar chart presents the relationship between modules and total travel time for the 1500 nodes Module Change experimental scenario. The x-axis displays modules values ranging from 4 to 7, while the y-axis quantifies total travel time. The single-series visualization facilitates analysis of how the dependent variable responds to changes in the independent parameter setting.

Analysis of the plotted data reveals that total travel time ranges from 10133.53 (at modules = 5) to 10275.22 (at modules = 4), representing a span of 141.69 units. The overall trend is decreasing, with values declining from 10275.22 at the initial setting to 10148.53 at the final setting. Notably, the minimum value occurs at an intermediate modules setting (5), suggesting non-monotonic behavior that warrants further investigation.

These results indicate that modules configuration meaningfully impacts total travel time in this experimental context. The relatively modest variation suggests that this parameter has limited influence on the measured metric within the tested range. Confidence in these findings is high given the direct correspondence between CSV data and plotted values. Future analysis should consider incorporating error bars representing variance across multiple experimental runs to strengthen statistical validity.

\begin{figure}[!htbp]
\centering
\includegraphics[width=0.95\linewidth]{images/1500_nodes_Module_Change/(b)_Energy_consumed_bar_chart.png}
\caption{(b) Energy consumed}
\label{fig:1500_nodes_Module_Change_b__Energy_consumed_bar_chart}
\end{figure}

This bar chart presents the relationship between modules and total energy consumed for the 1500 nodes Module Change experimental scenario. The x-axis displays modules values ranging from 4 to 7, while the y-axis quantifies total energy consumed. The single-series visualization facilitates analysis of how the dependent variable responds to changes in the independent parameter setting.

Analysis of the plotted data reveals that total energy consumed ranges from 1346.58 (at modules = 5) to 1351.12 (at modules = 4), representing a span of 4.55 units. The overall trend is decreasing, with values declining from 1351.12 at the initial setting to 1347.93 at the final setting. Notably, the minimum value occurs at an intermediate modules setting (5), suggesting non-monotonic behavior that warrants further investigation.

These results indicate that modules configuration meaningfully impacts total energy consumed in this experimental context. The relatively modest variation suggests that this parameter has limited influence on the measured metric within the tested range. Confidence in these findings is high given the direct correspondence between CSV data and plotted values. Future analysis should consider incorporating error bars representing variance across multiple experimental runs to strengthen statistical validity.

\begin{figure}[!htbp]
\centering
\includegraphics[width=0.95\linewidth]{images/1500_nodes_Module_Change/(c)_Distance_covered_bar_chart.png}
\caption{(c) Distance covered}
\label{fig:1500_nodes_Module_Change_c__Distance_covered_bar_chart}
\end{figure}

This bar chart presents the relationship between modules and total distance covered for the 1500 nodes Module Change experimental scenario. The x-axis displays modules values ranging from 4 to 7, while the y-axis quantifies total distance covered. The single-series visualization facilitates analysis of how the dependent variable responds to changes in the independent parameter setting.

Analysis of the plotted data reveals that total distance covered ranges from 6533.82 (at modules = 5) to 6628.04 (at modules = 4), representing a span of 94.22 units. The overall trend is decreasing, with values declining from 6628.04 at the initial setting to 6545.22 at the final setting. Notably, the minimum value occurs at an intermediate modules setting (5), suggesting non-monotonic behavior that warrants further investigation.

These results indicate that modules configuration meaningfully impacts total distance covered in this experimental context. The relatively modest variation suggests that this parameter has limited influence on the measured metric within the tested range. Confidence in these findings is high given the direct correspondence between CSV data and plotted values. Future analysis should consider incorporating error bars representing variance across multiple experimental runs to strengthen statistical validity.

\begin{figure}[!htbp]
\centering
\includegraphics[width=0.95\linewidth]{images/1500_nodes_Module_Change/(d)_Run_time_bar_chart.png}
\caption{(d) Run time}
\label{fig:1500_nodes_Module_Change_d__Run_time_bar_chart}
\end{figure}

This bar chart presents the relationship between modules and total travel time for the 1500 nodes Module Change experimental scenario. The x-axis displays modules values ranging from 4 to 7, while the y-axis quantifies total travel time. The single-series visualization facilitates analysis of how the dependent variable responds to changes in the independent parameter setting.

Analysis of the plotted data reveals that total travel time ranges from 10133.53 (at modules = 5) to 10275.22 (at modules = 4), representing a span of 141.69 units. The overall trend is decreasing, with values declining from 10275.22 at the initial setting to 10148.53 at the final setting. Notably, the minimum value occurs at an intermediate modules setting (5), suggesting non-monotonic behavior that warrants further investigation.

These results indicate that modules configuration meaningfully impacts total travel time in this experimental context. The relatively modest variation suggests that this parameter has limited influence on the measured metric within the tested range. Confidence in these findings is high given the direct correspondence between CSV data and plotted values. Future analysis should consider incorporating error bars representing variance across multiple experimental runs to strengthen statistical validity.

\begin{figure}[!htbp]
\centering
\includegraphics[width=0.95\linewidth]{images/1500_nodes_Module_Change/(e)_Module_swapped_bar_chart.png}
\caption{(e) Module swapped}
\label{fig:1500_nodes_Module_Change_e__Module_swapped_bar_chart}
\end{figure}

This bar chart presents the relationship between modules and modules for the 1500 nodes Module Change experimental scenario. The x-axis displays modules values ranging from 4 to 7, while the y-axis quantifies modules. The single-series visualization facilitates analysis of how the dependent variable responds to changes in the independent parameter setting.

Analysis of the plotted data reveals that modules ranges from 4.00 (at modules = 4) to 7.00 (at modules = 7), representing a span of 3.00 units. The overall trend is increasing, with values rising from 4.00 at the initial setting to 7.00 at the final setting. 

These results indicate that modules configuration meaningfully impacts modules in this experimental context. The substantial variation observed (coefficient of variation exceeding 20\%) suggests that parameter tuning could yield significant performance improvements. Confidence in these findings is high given the direct correspondence between CSV data and plotted values. Future analysis should consider incorporating error bars representing variance across multiple experimental runs to strengthen statistical validity.


\clearpage

\subsection{1500 nodes Swapping Time}

\begin{figure}[!htbp]
\centering
\includegraphics[width=0.95\linewidth]{images/1500_nodes_Swapping_Time/(a)_Travel_time_bar_chart.png}
\caption{(a) Travel time}
\label{fig:1500_nodes_Swapping_Time_a__Travel_time_bar_chart}
\end{figure}

This bar chart presents the relationship between swapping time and total travel time for the 1500 nodes Swapping Time experimental scenario. The x-axis displays swapping time values ranging from 1 to 4, while the y-axis quantifies total travel time. The single-series visualization facilitates analysis of how the dependent variable responds to changes in the independent parameter setting.

Analysis of the plotted data reveals that total travel time ranges from 10089.02 (at swapping time = 1) to 10262.19 (at swapping time = 4), representing a span of 173.17 units. The overall trend is increasing, with values rising from 10089.02 at the initial setting to 10262.19 at the final setting. 

These results indicate that swapping time configuration meaningfully impacts total travel time in this experimental context. The relatively modest variation suggests that this parameter has limited influence on the measured metric within the tested range. Confidence in these findings is high given the direct correspondence between CSV data and plotted values. Future analysis should consider incorporating error bars representing variance across multiple experimental runs to strengthen statistical validity.

\begin{figure}[!htbp]
\centering
\includegraphics[width=0.95\linewidth]{images/1500_nodes_Swapping_Time/(b)_Energy_consumed_bar_chart.png}
\caption{(b) Energy consumed}
\label{fig:1500_nodes_Swapping_Time_b__Energy_consumed_bar_chart}
\end{figure}

This bar chart presents the relationship between swapping time and total energy consumed for the 1500 nodes Swapping Time experimental scenario. The x-axis displays swapping time values ranging from 1 to 4, while the y-axis quantifies total energy consumed. The single-series visualization facilitates analysis of how the dependent variable responds to changes in the independent parameter setting.

Analysis of the plotted data reveals that total energy consumed ranges from 1346.58 (at swapping time = 2) to 1349.76 (at swapping time = 1), representing a span of 3.18 units. The overall trend is decreasing, with values declining from 1349.76 at the initial setting to 1346.59 at the final setting. Notably, the minimum value occurs at an intermediate swapping time setting (2), suggesting non-monotonic behavior that warrants further investigation.

These results indicate that swapping time configuration meaningfully impacts total energy consumed in this experimental context. The relatively modest variation suggests that this parameter has limited influence on the measured metric within the tested range. Confidence in these findings is high given the direct correspondence between CSV data and plotted values. Future analysis should consider incorporating error bars representing variance across multiple experimental runs to strengthen statistical validity.

\begin{figure}[!htbp]
\centering
\includegraphics[width=0.95\linewidth]{images/1500_nodes_Swapping_Time/(c)_Distance_covered_bar_chart.png}
\caption{(c) Distance covered}
\label{fig:1500_nodes_Swapping_Time_c__Distance_covered_bar_chart}
\end{figure}

This bar chart presents the relationship between swapping time and total distance covered for the 1500 nodes Swapping Time experimental scenario. The x-axis displays swapping time values ranging from 1 to 4, while the y-axis quantifies total distance covered. The single-series visualization facilitates analysis of how the dependent variable responds to changes in the independent parameter setting.

Analysis of the plotted data reveals that total distance covered ranges from 6533.82 (at swapping time = 2) to 6552.92 (at swapping time = 1), representing a span of 19.10 units. The overall trend is decreasing, with values declining from 6552.92 at the initial setting to 6534.02 at the final setting. Notably, the minimum value occurs at an intermediate swapping time setting (2), suggesting non-monotonic behavior that warrants further investigation.

These results indicate that swapping time configuration meaningfully impacts total distance covered in this experimental context. The relatively modest variation suggests that this parameter has limited influence on the measured metric within the tested range. Confidence in these findings is high given the direct correspondence between CSV data and plotted values. Future analysis should consider incorporating error bars representing variance across multiple experimental runs to strengthen statistical validity.

\begin{figure}[!htbp]
\centering
\includegraphics[width=0.95\linewidth]{images/1500_nodes_Swapping_Time/(d)_Run_time_bar_chart.png}
\caption{(d) Run time}
\label{fig:1500_nodes_Swapping_Time_d__Run_time_bar_chart}
\end{figure}

This bar chart presents the relationship between swapping time and total travel time for the 1500 nodes Swapping Time experimental scenario. The x-axis displays swapping time values ranging from 1 to 4, while the y-axis quantifies total travel time. The single-series visualization facilitates analysis of how the dependent variable responds to changes in the independent parameter setting.

Analysis of the plotted data reveals that total travel time ranges from 10089.02 (at swapping time = 1) to 10262.19 (at swapping time = 4), representing a span of 173.17 units. The overall trend is increasing, with values rising from 10089.02 at the initial setting to 10262.19 at the final setting. 

These results indicate that swapping time configuration meaningfully impacts total travel time in this experimental context. The relatively modest variation suggests that this parameter has limited influence on the measured metric within the tested range. Confidence in these findings is high given the direct correspondence between CSV data and plotted values. Future analysis should consider incorporating error bars representing variance across multiple experimental runs to strengthen statistical validity.

\begin{figure}[!htbp]
\centering
\includegraphics[width=0.95\linewidth]{images/1500_nodes_Swapping_Time/(e)_Module_swapped_bar_chart.png}
\caption{(e) Module swapped}
\label{fig:1500_nodes_Swapping_Time_e__Module_swapped_bar_chart}
\end{figure}

This bar chart presents the relationship between swapping time and total module swapped for the 1500 nodes Swapping Time experimental scenario. The x-axis displays swapping time values ranging from 1 to 4, while the y-axis quantifies total module swapped. The single-series visualization facilitates analysis of how the dependent variable responds to changes in the independent parameter setting.

Analysis of the plotted data reveals that total module swapped ranges from 64.00 (at swapping time = 1) to 64.00 (at swapping time = 1), representing a span of 0.00 units. The values remain relatively stable across the parameter range, with minimal net change between initial (64.00) and final (64.00) settings. 

These results indicate that swapping time configuration meaningfully impacts total module swapped in this experimental context. The relatively modest variation suggests that this parameter has limited influence on the measured metric within the tested range. Confidence in these findings is high given the direct correspondence between CSV data and plotted values. Future analysis should consider incorporating error bars representing variance across multiple experimental runs to strengthen statistical validity.


\clearpage

\subsection{1500 nodes Threshold}

\begin{figure}[!htbp]
\centering
\includegraphics[width=0.95\linewidth]{images/1500_nodes_Threshold/(a)_Travel_time_bar_chart.png}
\caption{(a) Travel time}
\label{fig:1500_nodes_Threshold_a__Travel_time_bar_chart}
\end{figure}

This bar chart presents the relationship between threshold and total travel time for the 1500 nodes Threshold experimental scenario. The x-axis displays threshold values ranging from 5 to 20, while the y-axis quantifies total travel time. The single-series visualization facilitates analysis of how the dependent variable responds to changes in the independent parameter setting.

Analysis of the plotted data reveals that total travel time ranges from 10133.53 (at threshold = 10) to 10148.53 (at threshold = 5), representing a span of 15.00 units. The overall trend is decreasing, with values declining from 10148.53 at the initial setting to 10133.53 at the final setting. Notably, the minimum value occurs at an intermediate threshold setting (10), suggesting non-monotonic behavior that warrants further investigation.

These results indicate that threshold configuration meaningfully impacts total travel time in this experimental context. The relatively modest variation suggests that this parameter has limited influence on the measured metric within the tested range. Confidence in these findings is high given the direct correspondence between CSV data and plotted values. Future analysis should consider incorporating error bars representing variance across multiple experimental runs to strengthen statistical validity.

\begin{figure}[!htbp]
\centering
\includegraphics[width=0.95\linewidth]{images/1500_nodes_Threshold/(b)_Energy_consumed_bar_chart.png}
\caption{(b) Energy consumed}
\label{fig:1500_nodes_Threshold_b__Energy_consumed_bar_chart}
\end{figure}

This bar chart presents the relationship between threshold and total energy consumed for the 1500 nodes Threshold experimental scenario. The x-axis displays threshold values ranging from 5 to 20, while the y-axis quantifies total energy consumed. The single-series visualization facilitates analysis of how the dependent variable responds to changes in the independent parameter setting.

Analysis of the plotted data reveals that total energy consumed ranges from 1346.58 (at threshold = 10) to 1347.93 (at threshold = 5), representing a span of 1.36 units. The overall trend is decreasing, with values declining from 1347.93 at the initial setting to 1346.58 at the final setting. Notably, the minimum value occurs at an intermediate threshold setting (10), suggesting non-monotonic behavior that warrants further investigation.

These results indicate that threshold configuration meaningfully impacts total energy consumed in this experimental context. The relatively modest variation suggests that this parameter has limited influence on the measured metric within the tested range. Confidence in these findings is high given the direct correspondence between CSV data and plotted values. Future analysis should consider incorporating error bars representing variance across multiple experimental runs to strengthen statistical validity.

\begin{figure}[!htbp]
\centering
\includegraphics[width=0.95\linewidth]{images/1500_nodes_Threshold/(c)_Distance_covered_bar_chart.png}
\caption{(c) Distance covered}
\label{fig:1500_nodes_Threshold_c__Distance_covered_bar_chart}
\end{figure}

This bar chart presents the relationship between threshold and total distance covered for the 1500 nodes Threshold experimental scenario. The x-axis displays threshold values ranging from 5 to 20, while the y-axis quantifies total distance covered. The single-series visualization facilitates analysis of how the dependent variable responds to changes in the independent parameter setting.

Analysis of the plotted data reveals that total distance covered ranges from 6533.82 (at threshold = 10) to 6545.22 (at threshold = 5), representing a span of 11.40 units. The overall trend is decreasing, with values declining from 6545.22 at the initial setting to 6533.82 at the final setting. Notably, the minimum value occurs at an intermediate threshold setting (10), suggesting non-monotonic behavior that warrants further investigation.

These results indicate that threshold configuration meaningfully impacts total distance covered in this experimental context. The relatively modest variation suggests that this parameter has limited influence on the measured metric within the tested range. Confidence in these findings is high given the direct correspondence between CSV data and plotted values. Future analysis should consider incorporating error bars representing variance across multiple experimental runs to strengthen statistical validity.

\begin{figure}[!htbp]
\centering
\includegraphics[width=0.95\linewidth]{images/1500_nodes_Threshold/(d)_Run_time_bar_chart.png}
\caption{(d) Run time}
\label{fig:1500_nodes_Threshold_d__Run_time_bar_chart}
\end{figure}

This bar chart presents the relationship between threshold and total travel time for the 1500 nodes Threshold experimental scenario. The x-axis displays threshold values ranging from 5 to 20, while the y-axis quantifies total travel time. The single-series visualization facilitates analysis of how the dependent variable responds to changes in the independent parameter setting.

Analysis of the plotted data reveals that total travel time ranges from 10133.53 (at threshold = 10) to 10148.53 (at threshold = 5), representing a span of 15.00 units. The overall trend is decreasing, with values declining from 10148.53 at the initial setting to 10133.53 at the final setting. Notably, the minimum value occurs at an intermediate threshold setting (10), suggesting non-monotonic behavior that warrants further investigation.

These results indicate that threshold configuration meaningfully impacts total travel time in this experimental context. The relatively modest variation suggests that this parameter has limited influence on the measured metric within the tested range. Confidence in these findings is high given the direct correspondence between CSV data and plotted values. Future analysis should consider incorporating error bars representing variance across multiple experimental runs to strengthen statistical validity.

\begin{figure}[!htbp]
\centering
\includegraphics[width=0.95\linewidth]{images/1500_nodes_Threshold/(e)_Module_swapped_bar_chart.png}
\caption{(e) Module swapped}
\label{fig:1500_nodes_Threshold_e__Module_swapped_bar_chart}
\end{figure}

This bar chart presents the relationship between threshold and total module swapped for the 1500 nodes Threshold experimental scenario. The x-axis displays threshold values ranging from 5 to 20, while the y-axis quantifies total module swapped. The single-series visualization facilitates analysis of how the dependent variable responds to changes in the independent parameter setting.

Analysis of the plotted data reveals that total module swapped ranges from 63.00 (at threshold = 5) to 64.00 (at threshold = 10), representing a span of 1.00 units. The overall trend is increasing, with values rising from 63.00 at the initial setting to 64.00 at the final setting. 

These results indicate that threshold configuration meaningfully impacts total module swapped in this experimental context. The relatively modest variation suggests that this parameter has limited influence on the measured metric within the tested range. Confidence in these findings is high given the direct correspondence between CSV data and plotted values. Future analysis should consider incorporating error bars representing variance across multiple experimental runs to strengthen statistical validity.


\clearpage

\subsection{1500 nodes Traffic}

\begin{figure}[!htbp]
\centering
\includegraphics[width=0.95\linewidth]{images/1500_nodes_Traffic/(a)_Travel_time_bar_chart.png}
\caption{(a) Travel time}
\label{fig:1500_nodes_Traffic_a__Travel_time_bar_chart}
\end{figure}

This bar chart presents the relationship between traffic and total travel time for the 1500 nodes Traffic experimental scenario. The x-axis displays traffic values ranging from High to Low, while the y-axis quantifies total travel time. The single-series visualization facilitates analysis of how the dependent variable responds to changes in the independent parameter setting.

Analysis of the plotted data reveals that total travel time ranges from 8682.70 (at traffic = Low) to 11868.31 (at traffic = High), representing a span of 3185.61 units. The overall trend is decreasing, with values declining from 11868.31 at the initial setting to 8682.70 at the final setting. 

These results indicate that traffic configuration meaningfully impacts total travel time in this experimental context. The substantial variation observed (coefficient of variation exceeding 20\%) suggests that parameter tuning could yield significant performance improvements. Confidence in these findings is high given the direct correspondence between CSV data and plotted values. Future analysis should consider incorporating error bars representing variance across multiple experimental runs to strengthen statistical validity.

\begin{figure}[!htbp]
\centering
\includegraphics[width=0.95\linewidth]{images/1500_nodes_Traffic/(b)_Energy_consumed_bar_chart.png}
\caption{(b) Energy consumed}
\label{fig:1500_nodes_Traffic_b__Energy_consumed_bar_chart}
\end{figure}

This bar chart presents the relationship between traffic and total energy consumed for the 1500 nodes Traffic experimental scenario. The x-axis displays traffic values ranging from High to Low, while the y-axis quantifies total energy consumed. The single-series visualization facilitates analysis of how the dependent variable responds to changes in the independent parameter setting.

Analysis of the plotted data reveals that total energy consumed ranges from 1303.59 (at traffic = High) to 1443.98 (at traffic = Low), representing a span of 140.40 units. The overall trend is increasing, with values rising from 1303.59 at the initial setting to 1443.98 at the final setting. 

These results indicate that traffic configuration meaningfully impacts total energy consumed in this experimental context. The relatively modest variation suggests that this parameter has limited influence on the measured metric within the tested range. Confidence in these findings is high given the direct correspondence between CSV data and plotted values. Future analysis should consider incorporating error bars representing variance across multiple experimental runs to strengthen statistical validity.

\begin{figure}[!htbp]
\centering
\includegraphics[width=0.95\linewidth]{images/1500_nodes_Traffic/(c)_Distance_covered_bar_chart.png}
\caption{(c) Distance covered}
\label{fig:1500_nodes_Traffic_c__Distance_covered_bar_chart}
\end{figure}

This bar chart presents the relationship between traffic and total distance covered for the 1500 nodes Traffic experimental scenario. The x-axis displays traffic values ranging from High to Low, while the y-axis quantifies total distance covered. The single-series visualization facilitates analysis of how the dependent variable responds to changes in the independent parameter setting.

Analysis of the plotted data reveals that total distance covered ranges from 6463.22 (at traffic = Mid) to 6661.07 (at traffic = Low), representing a span of 197.85 units. The overall trend is increasing, with values rising from 6493.82 at the initial setting to 6661.07 at the final setting. Notably, the minimum value occurs at an intermediate traffic setting (Mid), suggesting non-monotonic behavior that warrants further investigation.

These results indicate that traffic configuration meaningfully impacts total distance covered in this experimental context. The relatively modest variation suggests that this parameter has limited influence on the measured metric within the tested range. Confidence in these findings is high given the direct correspondence between CSV data and plotted values. Future analysis should consider incorporating error bars representing variance across multiple experimental runs to strengthen statistical validity.

\begin{figure}[!htbp]
\centering
\includegraphics[width=0.95\linewidth]{images/1500_nodes_Traffic/(d)_Run_time_bar_chart.png}
\caption{(d) Run time}
\label{fig:1500_nodes_Traffic_d__Run_time_bar_chart}
\end{figure}

This bar chart presents the relationship between traffic and total travel time for the 1500 nodes Traffic experimental scenario. The x-axis displays traffic values ranging from High to Low, while the y-axis quantifies total travel time. The single-series visualization facilitates analysis of how the dependent variable responds to changes in the independent parameter setting.

Analysis of the plotted data reveals that total travel time ranges from 8682.70 (at traffic = Low) to 11868.31 (at traffic = High), representing a span of 3185.61 units. The overall trend is decreasing, with values declining from 11868.31 at the initial setting to 8682.70 at the final setting. 

These results indicate that traffic configuration meaningfully impacts total travel time in this experimental context. The substantial variation observed (coefficient of variation exceeding 20\%) suggests that parameter tuning could yield significant performance improvements. Confidence in these findings is high given the direct correspondence between CSV data and plotted values. Future analysis should consider incorporating error bars representing variance across multiple experimental runs to strengthen statistical validity.

\begin{figure}[!htbp]
\centering
\includegraphics[width=0.95\linewidth]{images/1500_nodes_Traffic/(e)_Module_swapped_bar_chart.png}
\caption{(e) Module swapped}
\label{fig:1500_nodes_Traffic_e__Module_swapped_bar_chart}
\end{figure}

This bar chart presents the relationship between traffic and total module swapped for the 1500 nodes Traffic experimental scenario. The x-axis displays traffic values ranging from High to Low, while the y-axis quantifies total module swapped. The single-series visualization facilitates analysis of how the dependent variable responds to changes in the independent parameter setting.

Analysis of the plotted data reveals that total module swapped ranges from 63.00 (at traffic = High) to 69.00 (at traffic = Low), representing a span of 6.00 units. The overall trend is increasing, with values rising from 63.00 at the initial setting to 69.00 at the final setting. 

These results indicate that traffic configuration meaningfully impacts total module swapped in this experimental context. The relatively modest variation suggests that this parameter has limited influence on the measured metric within the tested range. Confidence in these findings is high given the direct correspondence between CSV data and plotted values. Future analysis should consider incorporating error bars representing variance across multiple experimental runs to strengthen statistical validity.


\clearpage

\subsection{2000 nodes Module Change}

\begin{figure}[!htbp]
\centering
\includegraphics[width=0.95\linewidth]{images/2000_nodes_Module_Change/(a)_Travel_time_bar_chart.png}
\caption{(a) Travel time}
\label{fig:2000_nodes_Module_Change_a__Travel_time_bar_chart}
\end{figure}

This bar chart presents the relationship between modules and total travel time for the 2000 nodes Module Change experimental scenario. The x-axis displays modules values ranging from 4 to 7, while the y-axis quantifies total travel time. The single-series visualization facilitates analysis of how the dependent variable responds to changes in the independent parameter setting.

Analysis of the plotted data reveals that total travel time ranges from 13169.12 (at modules = 7) to 14159.32 (at modules = 4), representing a span of 990.20 units. The overall trend is decreasing, with values declining from 14159.32 at the initial setting to 13169.12 at the final setting. 

These results indicate that modules configuration meaningfully impacts total travel time in this experimental context. The relatively modest variation suggests that this parameter has limited influence on the measured metric within the tested range. Confidence in these findings is high given the direct correspondence between CSV data and plotted values. Future analysis should consider incorporating error bars representing variance across multiple experimental runs to strengthen statistical validity.

\begin{figure}[!htbp]
\centering
\includegraphics[width=0.95\linewidth]{images/2000_nodes_Module_Change/(b)_Energy_consumed_bar_chart.png}
\caption{(b) Energy consumed}
\label{fig:2000_nodes_Module_Change_b__Energy_consumed_bar_chart}
\end{figure}

This bar chart presents the relationship between modules and total energy consumed for the 2000 nodes Module Change experimental scenario. The x-axis displays modules values ranging from 4 to 7, while the y-axis quantifies total energy consumed. The single-series visualization facilitates analysis of how the dependent variable responds to changes in the independent parameter setting.

Analysis of the plotted data reveals that total energy consumed ranges from 2030.72 (at modules = 7) to 2154.01 (at modules = 4), representing a span of 123.28 units. The overall trend is decreasing, with values declining from 2154.01 at the initial setting to 2030.72 at the final setting. 

These results indicate that modules configuration meaningfully impacts total energy consumed in this experimental context. The relatively modest variation suggests that this parameter has limited influence on the measured metric within the tested range. Confidence in these findings is high given the direct correspondence between CSV data and plotted values. Future analysis should consider incorporating error bars representing variance across multiple experimental runs to strengthen statistical validity.

\begin{figure}[!htbp]
\centering
\includegraphics[width=0.95\linewidth]{images/2000_nodes_Module_Change/(c)_Distance_covered_bar_chart.png}
\caption{(c) Distance covered}
\label{fig:2000_nodes_Module_Change_c__Distance_covered_bar_chart}
\end{figure}

This bar chart presents the relationship between modules and total distance covered for the 2000 nodes Module Change experimental scenario. The x-axis displays modules values ranging from 4 to 7, while the y-axis quantifies total distance covered. The single-series visualization facilitates analysis of how the dependent variable responds to changes in the independent parameter setting.

Analysis of the plotted data reveals that total distance covered ranges from 8588.97 (at modules = 7) to 9234.69 (at modules = 4), representing a span of 645.72 units. The overall trend is decreasing, with values declining from 9234.69 at the initial setting to 8588.97 at the final setting. 

These results indicate that modules configuration meaningfully impacts total distance covered in this experimental context. The relatively modest variation suggests that this parameter has limited influence on the measured metric within the tested range. Confidence in these findings is high given the direct correspondence between CSV data and plotted values. Future analysis should consider incorporating error bars representing variance across multiple experimental runs to strengthen statistical validity.

\begin{figure}[!htbp]
\centering
\includegraphics[width=0.95\linewidth]{images/2000_nodes_Module_Change/(d)_Run_time_bar_chart.png}
\caption{(d) Run time}
\label{fig:2000_nodes_Module_Change_d__Run_time_bar_chart}
\end{figure}

This bar chart presents the relationship between modules and total travel time for the 2000 nodes Module Change experimental scenario. The x-axis displays modules values ranging from 4 to 7, while the y-axis quantifies total travel time. The single-series visualization facilitates analysis of how the dependent variable responds to changes in the independent parameter setting.

Analysis of the plotted data reveals that total travel time ranges from 13169.12 (at modules = 7) to 14159.32 (at modules = 4), representing a span of 990.20 units. The overall trend is decreasing, with values declining from 14159.32 at the initial setting to 13169.12 at the final setting. 

These results indicate that modules configuration meaningfully impacts total travel time in this experimental context. The relatively modest variation suggests that this parameter has limited influence on the measured metric within the tested range. Confidence in these findings is high given the direct correspondence between CSV data and plotted values. Future analysis should consider incorporating error bars representing variance across multiple experimental runs to strengthen statistical validity.

\begin{figure}[!htbp]
\centering
\includegraphics[width=0.95\linewidth]{images/2000_nodes_Module_Change/(e)_Module_swapped_bar_chart.png}
\caption{(e) Module swapped}
\label{fig:2000_nodes_Module_Change_e__Module_swapped_bar_chart}
\end{figure}

This bar chart presents the relationship between modules and modules for the 2000 nodes Module Change experimental scenario. The x-axis displays modules values ranging from 4 to 7, while the y-axis quantifies modules. The single-series visualization facilitates analysis of how the dependent variable responds to changes in the independent parameter setting.

Analysis of the plotted data reveals that modules ranges from 4.00 (at modules = 4) to 7.00 (at modules = 7), representing a span of 3.00 units. The overall trend is increasing, with values rising from 4.00 at the initial setting to 7.00 at the final setting. 

These results indicate that modules configuration meaningfully impacts modules in this experimental context. The substantial variation observed (coefficient of variation exceeding 20\%) suggests that parameter tuning could yield significant performance improvements. Confidence in these findings is high given the direct correspondence between CSV data and plotted values. Future analysis should consider incorporating error bars representing variance across multiple experimental runs to strengthen statistical validity.


\clearpage

\subsection{2000 nodes Swapping Time}

\begin{figure}[!htbp]
\centering
\includegraphics[width=0.95\linewidth]{images/2000_nodes_Swapping_Time/(a)_Travel_time_bar_chart.png}
\caption{(a) Travel time}
\label{fig:2000_nodes_Swapping_Time_a__Travel_time_bar_chart}
\end{figure}

This bar chart presents the relationship between swapping time and total travel time for the 2000 nodes Swapping Time experimental scenario. The x-axis displays swapping time values ranging from 1 to 4, while the y-axis quantifies total travel time. The single-series visualization facilitates analysis of how the dependent variable responds to changes in the independent parameter setting.

Analysis of the plotted data reveals that total travel time ranges from 13181.25 (at swapping time = 1) to 13540.28 (at swapping time = 4), representing a span of 359.03 units. The overall trend is increasing, with values rising from 13181.25 at the initial setting to 13540.28 at the final setting. 

These results indicate that swapping time configuration meaningfully impacts total travel time in this experimental context. The relatively modest variation suggests that this parameter has limited influence on the measured metric within the tested range. Confidence in these findings is high given the direct correspondence between CSV data and plotted values. Future analysis should consider incorporating error bars representing variance across multiple experimental runs to strengthen statistical validity.

\begin{figure}[!htbp]
\centering
\includegraphics[width=0.95\linewidth]{images/2000_nodes_Swapping_Time/(b)_Energy_consumed_bar_chart.png}
\caption{(b) Energy consumed}
\label{fig:2000_nodes_Swapping_Time_b__Energy_consumed_bar_chart}
\end{figure}

This bar chart presents the relationship between swapping time and total energy consumed for the 2000 nodes Swapping Time experimental scenario. The x-axis displays swapping time values ranging from 1 to 4, while the y-axis quantifies total energy consumed. The single-series visualization facilitates analysis of how the dependent variable responds to changes in the independent parameter setting.

Analysis of the plotted data reveals that total energy consumed ranges from 2057.20 (at swapping time = 1) to 2062.48 (at swapping time = 4), representing a span of 5.28 units. The overall trend is increasing, with values rising from 2057.20 at the initial setting to 2062.48 at the final setting. 

These results indicate that swapping time configuration meaningfully impacts total energy consumed in this experimental context. The relatively modest variation suggests that this parameter has limited influence on the measured metric within the tested range. Confidence in these findings is high given the direct correspondence between CSV data and plotted values. Future analysis should consider incorporating error bars representing variance across multiple experimental runs to strengthen statistical validity.

\begin{figure}[!htbp]
\centering
\includegraphics[width=0.95\linewidth]{images/2000_nodes_Swapping_Time/(c)_Distance_covered_bar_chart.png}
\caption{(c) Distance covered}
\label{fig:2000_nodes_Swapping_Time_c__Distance_covered_bar_chart}
\end{figure}

This bar chart presents the relationship between swapping time and total distance covered for the 2000 nodes Swapping Time experimental scenario. The x-axis displays swapping time values ranging from 1 to 4, while the y-axis quantifies total distance covered. The single-series visualization facilitates analysis of how the dependent variable responds to changes in the independent parameter setting.

Analysis of the plotted data reveals that total distance covered ranges from 8650.95 (at swapping time = 1) to 8701.15 (at swapping time = 4), representing a span of 50.20 units. The overall trend is increasing, with values rising from 8650.95 at the initial setting to 8701.15 at the final setting. 

These results indicate that swapping time configuration meaningfully impacts total distance covered in this experimental context. The relatively modest variation suggests that this parameter has limited influence on the measured metric within the tested range. Confidence in these findings is high given the direct correspondence between CSV data and plotted values. Future analysis should consider incorporating error bars representing variance across multiple experimental runs to strengthen statistical validity.

\begin{figure}[!htbp]
\centering
\includegraphics[width=0.95\linewidth]{images/2000_nodes_Swapping_Time/(d)_Run_time_bar_chart.png}
\caption{(d) Run time}
\label{fig:2000_nodes_Swapping_Time_d__Run_time_bar_chart}
\end{figure}

This bar chart presents the relationship between swapping time and total travel time for the 2000 nodes Swapping Time experimental scenario. The x-axis displays swapping time values ranging from 1 to 4, while the y-axis quantifies total travel time. The single-series visualization facilitates analysis of how the dependent variable responds to changes in the independent parameter setting.

Analysis of the plotted data reveals that total travel time ranges from 13181.25 (at swapping time = 1) to 13540.28 (at swapping time = 4), representing a span of 359.03 units. The overall trend is increasing, with values rising from 13181.25 at the initial setting to 13540.28 at the final setting. 

These results indicate that swapping time configuration meaningfully impacts total travel time in this experimental context. The relatively modest variation suggests that this parameter has limited influence on the measured metric within the tested range. Confidence in these findings is high given the direct correspondence between CSV data and plotted values. Future analysis should consider incorporating error bars representing variance across multiple experimental runs to strengthen statistical validity.

\begin{figure}[!htbp]
\centering
\includegraphics[width=0.95\linewidth]{images/2000_nodes_Swapping_Time/(e)_Module_swapped_bar_chart.png}
\caption{(e) Module swapped}
\label{fig:2000_nodes_Swapping_Time_e__Module_swapped_bar_chart}
\end{figure}

This bar chart presents the relationship between swapping time and total module swapped for the 2000 nodes Swapping Time experimental scenario. The x-axis displays swapping time values ranging from 1 to 4, while the y-axis quantifies total module swapped. The single-series visualization facilitates analysis of how the dependent variable responds to changes in the independent parameter setting.

Analysis of the plotted data reveals that total module swapped ranges from 100.00 (at swapping time = 1) to 100.00 (at swapping time = 1), representing a span of 0.00 units. The values remain relatively stable across the parameter range, with minimal net change between initial (100.00) and final (100.00) settings. 

These results indicate that swapping time configuration meaningfully impacts total module swapped in this experimental context. The relatively modest variation suggests that this parameter has limited influence on the measured metric within the tested range. Confidence in these findings is high given the direct correspondence between CSV data and plotted values. Future analysis should consider incorporating error bars representing variance across multiple experimental runs to strengthen statistical validity.


\clearpage

\subsection{2000 nodes Threshold}

\begin{figure}[!htbp]
\centering
\includegraphics[width=0.95\linewidth]{images/2000_nodes_Threshold/(a)_Travel_time_bar_chart.png}
\caption{(a) Travel time}
\label{fig:2000_nodes_Threshold_a__Travel_time_bar_chart}
\end{figure}

This bar chart presents the relationship between threshold and total travel time for the 2000 nodes Threshold experimental scenario. The x-axis displays threshold values ranging from 5 to 20, while the y-axis quantifies total travel time. The single-series visualization facilitates analysis of how the dependent variable responds to changes in the independent parameter setting.

Analysis of the plotted data reveals that total travel time ranges from 13059.88 (at threshold = 5) to 13381.98 (at threshold = 20), representing a span of 322.10 units. The overall trend is increasing, with values rising from 13059.88 at the initial setting to 13381.98 at the final setting. 

These results indicate that threshold configuration meaningfully impacts total travel time in this experimental context. The relatively modest variation suggests that this parameter has limited influence on the measured metric within the tested range. Confidence in these findings is high given the direct correspondence between CSV data and plotted values. Future analysis should consider incorporating error bars representing variance across multiple experimental runs to strengthen statistical validity.

\begin{figure}[!htbp]
\centering
\includegraphics[width=0.95\linewidth]{images/2000_nodes_Threshold/(b)_Energy_consumed_bar_chart.png}
\caption{(b) Energy consumed}
\label{fig:2000_nodes_Threshold_b__Energy_consumed_bar_chart}
\end{figure}

This bar chart presents the relationship between threshold and total energy consumed for the 2000 nodes Threshold experimental scenario. The x-axis displays threshold values ranging from 5 to 20, while the y-axis quantifies total energy consumed. The single-series visualization facilitates analysis of how the dependent variable responds to changes in the independent parameter setting.

Analysis of the plotted data reveals that total energy consumed ranges from 2020.24 (at threshold = 15) to 2057.78 (at threshold = 10), representing a span of 37.54 units. The overall trend is increasing, with values rising from 2027.67 at the initial setting to 2057.45 at the final setting. Notably, the minimum value occurs at an intermediate threshold setting (15), suggesting non-monotonic behavior that warrants further investigation.

These results indicate that threshold configuration meaningfully impacts total energy consumed in this experimental context. The relatively modest variation suggests that this parameter has limited influence on the measured metric within the tested range. Confidence in these findings is high given the direct correspondence between CSV data and plotted values. Future analysis should consider incorporating error bars representing variance across multiple experimental runs to strengthen statistical validity.

\begin{figure}[!htbp]
\centering
\includegraphics[width=0.95\linewidth]{images/2000_nodes_Threshold/(c)_Distance_covered_bar_chart.png}
\caption{(c) Distance covered}
\label{fig:2000_nodes_Threshold_c__Distance_covered_bar_chart}
\end{figure}

This bar chart presents the relationship between threshold and total distance covered for the 2000 nodes Threshold experimental scenario. The x-axis displays threshold values ranging from 5 to 20, while the y-axis quantifies total distance covered. The single-series visualization facilitates analysis of how the dependent variable responds to changes in the independent parameter setting.

Analysis of the plotted data reveals that total distance covered ranges from 8527.92 (at threshold = 5) to 8685.62 (at threshold = 10), representing a span of 157.70 units. The overall trend is increasing, with values rising from 8527.92 at the initial setting to 8652.17 at the final setting. 

These results indicate that threshold configuration meaningfully impacts total distance covered in this experimental context. The relatively modest variation suggests that this parameter has limited influence on the measured metric within the tested range. Confidence in these findings is high given the direct correspondence between CSV data and plotted values. Future analysis should consider incorporating error bars representing variance across multiple experimental runs to strengthen statistical validity.

\begin{figure}[!htbp]
\centering
\includegraphics[width=0.95\linewidth]{images/2000_nodes_Threshold/(d)_Run_time_bar_chart.png}
\caption{(d) Run time}
\label{fig:2000_nodes_Threshold_d__Run_time_bar_chart}
\end{figure}

This bar chart presents the relationship between threshold and total travel time for the 2000 nodes Threshold experimental scenario. The x-axis displays threshold values ranging from 5 to 20, while the y-axis quantifies total travel time. The single-series visualization facilitates analysis of how the dependent variable responds to changes in the independent parameter setting.

Analysis of the plotted data reveals that total travel time ranges from 13059.88 (at threshold = 5) to 13381.98 (at threshold = 20), representing a span of 322.10 units. The overall trend is increasing, with values rising from 13059.88 at the initial setting to 13381.98 at the final setting. 

These results indicate that threshold configuration meaningfully impacts total travel time in this experimental context. The relatively modest variation suggests that this parameter has limited influence on the measured metric within the tested range. Confidence in these findings is high given the direct correspondence between CSV data and plotted values. Future analysis should consider incorporating error bars representing variance across multiple experimental runs to strengthen statistical validity.

\begin{figure}[!htbp]
\centering
\includegraphics[width=0.95\linewidth]{images/2000_nodes_Threshold/(e)_Module_swapped_bar_chart.png}
\caption{(e) Module swapped}
\label{fig:2000_nodes_Threshold_e__Module_swapped_bar_chart}
\end{figure}

This bar chart presents the relationship between threshold and total module swapped for the 2000 nodes Threshold experimental scenario. The x-axis displays threshold values ranging from 5 to 20, while the y-axis quantifies total module swapped. The single-series visualization facilitates analysis of how the dependent variable responds to changes in the independent parameter setting.

Analysis of the plotted data reveals that total module swapped ranges from 97.00 (at threshold = 15) to 100.00 (at threshold = 20), representing a span of 3.00 units. The overall trend is increasing, with values rising from 98.00 at the initial setting to 100.00 at the final setting. Notably, the minimum value occurs at an intermediate threshold setting (15), suggesting non-monotonic behavior that warrants further investigation.

These results indicate that threshold configuration meaningfully impacts total module swapped in this experimental context. The relatively modest variation suggests that this parameter has limited influence on the measured metric within the tested range. Confidence in these findings is high given the direct correspondence between CSV data and plotted values. Future analysis should consider incorporating error bars representing variance across multiple experimental runs to strengthen statistical validity.


\clearpage

\subsection{2000 nodes Traffic}

\begin{figure}[!htbp]
\centering
\includegraphics[width=0.95\linewidth]{images/2000_nodes_Traffic/(a)_Travel_time_bar_chart.png}
\caption{(a) Travel time}
\label{fig:2000_nodes_Traffic_a__Travel_time_bar_chart}
\end{figure}

This bar chart presents the relationship between traffic and total travel time for the 2000 nodes Traffic experimental scenario. The x-axis displays traffic values ranging from High to Low, while the y-axis quantifies total travel time. The single-series visualization facilitates analysis of how the dependent variable responds to changes in the independent parameter setting.

Analysis of the plotted data reveals that total travel time ranges from 11773.91 (at traffic = Low) to 15961.57 (at traffic = High), representing a span of 4187.66 units. The overall trend is decreasing, with values declining from 15961.57 at the initial setting to 11773.91 at the final setting. 

These results indicate that traffic configuration meaningfully impacts total travel time in this experimental context. The substantial variation observed (coefficient of variation exceeding 20\%) suggests that parameter tuning could yield significant performance improvements. Confidence in these findings is high given the direct correspondence between CSV data and plotted values. Future analysis should consider incorporating error bars representing variance across multiple experimental runs to strengthen statistical validity.

\begin{figure}[!htbp]
\centering
\includegraphics[width=0.95\linewidth]{images/2000_nodes_Traffic/(b)_Energy_consumed_bar_chart.png}
\caption{(b) Energy consumed}
\label{fig:2000_nodes_Traffic_b__Energy_consumed_bar_chart}
\end{figure}

This bar chart presents the relationship between traffic and total energy consumed for the 2000 nodes Traffic experimental scenario. The x-axis displays traffic values ranging from High to Low, while the y-axis quantifies total energy consumed. The single-series visualization facilitates analysis of how the dependent variable responds to changes in the independent parameter setting.

Analysis of the plotted data reveals that total energy consumed ranges from 1974.45 (at traffic = Mid) to 2187.71 (at traffic = Low), representing a span of 213.26 units. The overall trend is increasing, with values rising from 1975.64 at the initial setting to 2187.71 at the final setting. Notably, the minimum value occurs at an intermediate traffic setting (Mid), suggesting non-monotonic behavior that warrants further investigation.

These results indicate that traffic configuration meaningfully impacts total energy consumed in this experimental context. The relatively modest variation suggests that this parameter has limited influence on the measured metric within the tested range. Confidence in these findings is high given the direct correspondence between CSV data and plotted values. Future analysis should consider incorporating error bars representing variance across multiple experimental runs to strengthen statistical validity.

\begin{figure}[!htbp]
\centering
\includegraphics[width=0.95\linewidth]{images/2000_nodes_Traffic/(c)_Distance_covered_bar_chart.png}
\caption{(c) Distance covered}
\label{fig:2000_nodes_Traffic_c__Distance_covered_bar_chart}
\end{figure}

This bar chart presents the relationship between traffic and total distance covered for the 2000 nodes Traffic experimental scenario. The x-axis displays traffic values ranging from High to Low, while the y-axis quantifies total distance covered. The single-series visualization facilitates analysis of how the dependent variable responds to changes in the independent parameter setting.

Analysis of the plotted data reveals that total distance covered ranges from 8689.81 (at traffic = High) to 8994.35 (at traffic = Low), representing a span of 304.54 units. The overall trend is increasing, with values rising from 8689.81 at the initial setting to 8994.35 at the final setting. 

These results indicate that traffic configuration meaningfully impacts total distance covered in this experimental context. The relatively modest variation suggests that this parameter has limited influence on the measured metric within the tested range. Confidence in these findings is high given the direct correspondence between CSV data and plotted values. Future analysis should consider incorporating error bars representing variance across multiple experimental runs to strengthen statistical validity.

\begin{figure}[!htbp]
\centering
\includegraphics[width=0.95\linewidth]{images/2000_nodes_Traffic/(d)_Run_time_bar_chart.png}
\caption{(d) Run time}
\label{fig:2000_nodes_Traffic_d__Run_time_bar_chart}
\end{figure}

This bar chart presents the relationship between traffic and total travel time for the 2000 nodes Traffic experimental scenario. The x-axis displays traffic values ranging from High to Low, while the y-axis quantifies total travel time. The single-series visualization facilitates analysis of how the dependent variable responds to changes in the independent parameter setting.

Analysis of the plotted data reveals that total travel time ranges from 11773.91 (at traffic = Low) to 15961.57 (at traffic = High), representing a span of 4187.66 units. The overall trend is decreasing, with values declining from 15961.57 at the initial setting to 11773.91 at the final setting. 

These results indicate that traffic configuration meaningfully impacts total travel time in this experimental context. The substantial variation observed (coefficient of variation exceeding 20\%) suggests that parameter tuning could yield significant performance improvements. Confidence in these findings is high given the direct correspondence between CSV data and plotted values. Future analysis should consider incorporating error bars representing variance across multiple experimental runs to strengthen statistical validity.

\begin{figure}[!htbp]
\centering
\includegraphics[width=0.95\linewidth]{images/2000_nodes_Traffic/(e)_Module_swapped_bar_chart.png}
\caption{(e) Module swapped}
\label{fig:2000_nodes_Traffic_e__Module_swapped_bar_chart}
\end{figure}

This bar chart presents the relationship between traffic and total module swapped for the 2000 nodes Traffic experimental scenario. The x-axis displays traffic values ranging from High to Low, while the y-axis quantifies total module swapped. The single-series visualization facilitates analysis of how the dependent variable responds to changes in the independent parameter setting.

Analysis of the plotted data reveals that total module swapped ranges from 96.00 (at traffic = High) to 107.00 (at traffic = Low), representing a span of 11.00 units. The overall trend is increasing, with values rising from 96.00 at the initial setting to 107.00 at the final setting. 

These results indicate that traffic configuration meaningfully impacts total module swapped in this experimental context. The relatively modest variation suggests that this parameter has limited influence on the measured metric within the tested range. Confidence in these findings is high given the direct correspondence between CSV data and plotted values. Future analysis should consider incorporating error bars representing variance across multiple experimental runs to strengthen statistical validity.


\clearpage

\subsection{500 nodes Module Change}

\begin{figure}[!htbp]
\centering
\includegraphics[width=0.95\linewidth]{images/500_nodes_Module_Change/(a)_Travel_Time_bar_chart.png}
\caption{(a) Travel Time}
\label{fig:500_nodes_Module_Change_a__Travel_Time_bar_chart}
\end{figure}

This bar chart presents the relationship between modules and total travel time for the 500 nodes Module Change experimental scenario. The x-axis displays modules values ranging from 4 to 7, while the y-axis quantifies total travel time. The single-series visualization facilitates analysis of how the dependent variable responds to changes in the independent parameter setting.

Analysis of the plotted data reveals that total travel time ranges from 3287.54 (at modules = 4) to 3287.54 (at modules = 4), representing a span of 0.00 units. The values remain relatively stable across the parameter range, with minimal net change between initial (3287.54) and final (3287.54) settings. 

These results indicate that modules configuration meaningfully impacts total travel time in this experimental context. The relatively modest variation suggests that this parameter has limited influence on the measured metric within the tested range. Confidence in these findings is high given the direct correspondence between CSV data and plotted values. Future analysis should consider incorporating error bars representing variance across multiple experimental runs to strengthen statistical validity.

\begin{figure}[!htbp]
\centering
\includegraphics[width=0.95\linewidth]{images/500_nodes_Module_Change/(b)_Energy_consumption_bar_chart.png}
\caption{(b) Energy consumption}
\label{fig:500_nodes_Module_Change_b__Energy_consumption_bar_chart}
\end{figure}

This bar chart presents the relationship between modules and total energy consumed for the 500 nodes Module Change experimental scenario. The x-axis displays modules values ranging from 4 to 7, while the y-axis quantifies total energy consumed. The single-series visualization facilitates analysis of how the dependent variable responds to changes in the independent parameter setting.

Analysis of the plotted data reveals that total energy consumed ranges from 323.50 (at modules = 4) to 323.50 (at modules = 4), representing a span of 0.00 units. The values remain relatively stable across the parameter range, with minimal net change between initial (323.50) and final (323.50) settings. 

These results indicate that modules configuration meaningfully impacts total energy consumed in this experimental context. The relatively modest variation suggests that this parameter has limited influence on the measured metric within the tested range. Confidence in these findings is high given the direct correspondence between CSV data and plotted values. Future analysis should consider incorporating error bars representing variance across multiple experimental runs to strengthen statistical validity.

\begin{figure}[!htbp]
\centering
\includegraphics[width=0.95\linewidth]{images/500_nodes_Module_Change/(c)_Distance_covered_bar_chart.png}
\caption{(c) Distance covered}
\label{fig:500_nodes_Module_Change_c__Distance_covered_bar_chart}
\end{figure}

This bar chart presents the relationship between modules and total distance covered for the 500 nodes Module Change experimental scenario. The x-axis displays modules values ranging from 4 to 7, while the y-axis quantifies total distance covered. The single-series visualization facilitates analysis of how the dependent variable responds to changes in the independent parameter setting.

Analysis of the plotted data reveals that total distance covered ranges from 2138.91 (at modules = 4) to 2138.91 (at modules = 4), representing a span of 0.00 units. The values remain relatively stable across the parameter range, with minimal net change between initial (2138.91) and final (2138.91) settings. 

These results indicate that modules configuration meaningfully impacts total distance covered in this experimental context. The relatively modest variation suggests that this parameter has limited influence on the measured metric within the tested range. Confidence in these findings is high given the direct correspondence between CSV data and plotted values. Future analysis should consider incorporating error bars representing variance across multiple experimental runs to strengthen statistical validity.

\begin{figure}[!htbp]
\centering
\includegraphics[width=0.95\linewidth]{images/500_nodes_Module_Change/(d)_Runtime_bar_chart.png}
\caption{(d) Runtime}
\label{fig:500_nodes_Module_Change_d__Runtime_bar_chart}
\end{figure}

This bar chart presents the relationship between modules and run time for the 500 nodes Module Change experimental scenario. The x-axis displays modules values ranging from 4 to 7, while the y-axis quantifies run time. The single-series visualization facilitates analysis of how the dependent variable responds to changes in the independent parameter setting.

Analysis of the plotted data reveals that run time ranges from 29.02 (at modules = 5) to 62.16 (at modules = 6), representing a span of 33.14 units. The overall trend is decreasing, with values declining from 58.53 at the initial setting to 36.14 at the final setting. Notably, the minimum value occurs at an intermediate modules setting (5), suggesting non-monotonic behavior that warrants further investigation.

These results indicate that modules configuration meaningfully impacts run time in this experimental context. The substantial variation observed (coefficient of variation exceeding 20\%) suggests that parameter tuning could yield significant performance improvements. Confidence in these findings is high given the direct correspondence between CSV data and plotted values. Future analysis should consider incorporating error bars representing variance across multiple experimental runs to strengthen statistical validity.

\begin{figure}[!htbp]
\centering
\includegraphics[width=0.95\linewidth]{images/500_nodes_Module_Change/(e)_Number_of_module_swapped_bar_chart.png}
\caption{(e) Number of module swapped}
\label{fig:500_nodes_Module_Change_e__Number_of_module_swapped_bar_chart}
\end{figure}

This bar chart presents the relationship between modules and modules for the 500 nodes Module Change experimental scenario. The x-axis displays modules values ranging from 4 to 7, while the y-axis quantifies modules. The single-series visualization facilitates analysis of how the dependent variable responds to changes in the independent parameter setting.

Analysis of the plotted data reveals that modules ranges from 4.00 (at modules = 4) to 7.00 (at modules = 7), representing a span of 3.00 units. The overall trend is increasing, with values rising from 4.00 at the initial setting to 7.00 at the final setting. 

These results indicate that modules configuration meaningfully impacts modules in this experimental context. The substantial variation observed (coefficient of variation exceeding 20\%) suggests that parameter tuning could yield significant performance improvements. Confidence in these findings is high given the direct correspondence between CSV data and plotted values. Future analysis should consider incorporating error bars representing variance across multiple experimental runs to strengthen statistical validity.


\clearpage

\subsection{500 nodes Swapping Time}

\begin{figure}[!htbp]
\centering
\includegraphics[width=0.95\linewidth]{images/500_nodes_Swapping_Time/(a)_Travel_Time_bar_chart.png}
\caption{(a) Travel Time}
\label{fig:500_nodes_Swapping_Time_a__Travel_Time_bar_chart}
\end{figure}

This bar chart presents the relationship between swapping time and total travel time for the 500 nodes Swapping Time experimental scenario. The x-axis displays swapping time values ranging from 1 to 4, while the y-axis quantifies total travel time. The single-series visualization facilitates analysis of how the dependent variable responds to changes in the independent parameter setting.

Analysis of the plotted data reveals that total travel time ranges from 3268.67 (at swapping time = 1) to 3317.54 (at swapping time = 4), representing a span of 48.87 units. The overall trend is increasing, with values rising from 3268.67 at the initial setting to 3317.54 at the final setting. 

These results indicate that swapping time configuration meaningfully impacts total travel time in this experimental context. The relatively modest variation suggests that this parameter has limited influence on the measured metric within the tested range. Confidence in these findings is high given the direct correspondence between CSV data and plotted values. Future analysis should consider incorporating error bars representing variance across multiple experimental runs to strengthen statistical validity.

\begin{figure}[!htbp]
\centering
\includegraphics[width=0.95\linewidth]{images/500_nodes_Swapping_Time/(b)_Energy_consumption_bar_chart.png}
\caption{(b) Energy consumption}
\label{fig:500_nodes_Swapping_Time_b__Energy_consumption_bar_chart}
\end{figure}

This bar chart presents the relationship between swapping time and total energy consumed for the 500 nodes Swapping Time experimental scenario. The x-axis displays swapping time values ranging from 1 to 4, while the y-axis quantifies total energy consumed. The single-series visualization facilitates analysis of how the dependent variable responds to changes in the independent parameter setting.

Analysis of the plotted data reveals that total energy consumed ranges from 323.44 (at swapping time = 1) to 323.50 (at swapping time = 2), representing a span of 0.06 units. The overall trend is increasing, with values rising from 323.44 at the initial setting to 323.50 at the final setting. 

These results indicate that swapping time configuration meaningfully impacts total energy consumed in this experimental context. The relatively modest variation suggests that this parameter has limited influence on the measured metric within the tested range. Confidence in these findings is high given the direct correspondence between CSV data and plotted values. Future analysis should consider incorporating error bars representing variance across multiple experimental runs to strengthen statistical validity.

\begin{figure}[!htbp]
\centering
\includegraphics[width=0.95\linewidth]{images/500_nodes_Swapping_Time/(c)_Distance_covered_bar_chart.png}
\caption{(c) Distance covered}
\label{fig:500_nodes_Swapping_Time_c__Distance_covered_bar_chart}
\end{figure}

This bar chart presents the relationship between swapping time and total distance covered for the 500 nodes Swapping Time experimental scenario. The x-axis displays swapping time values ranging from 1 to 4, while the y-axis quantifies total distance covered. The single-series visualization facilitates analysis of how the dependent variable responds to changes in the independent parameter setting.

Analysis of the plotted data reveals that total distance covered ranges from 2138.07 (at swapping time = 1) to 2138.91 (at swapping time = 2), representing a span of 0.84 units. The overall trend is increasing, with values rising from 2138.07 at the initial setting to 2138.91 at the final setting. 

These results indicate that swapping time configuration meaningfully impacts total distance covered in this experimental context. The relatively modest variation suggests that this parameter has limited influence on the measured metric within the tested range. Confidence in these findings is high given the direct correspondence between CSV data and plotted values. Future analysis should consider incorporating error bars representing variance across multiple experimental runs to strengthen statistical validity.

\begin{figure}[!htbp]
\centering
\includegraphics[width=0.95\linewidth]{images/500_nodes_Swapping_Time/(d)_Runtime_bar_chart.png}
\caption{(d) Runtime}
\label{fig:500_nodes_Swapping_Time_d__Runtime_bar_chart}
\end{figure}

This bar chart presents the relationship between swapping time and run time for the 500 nodes Swapping Time experimental scenario. The x-axis displays swapping time values ranging from 1 to 4, while the y-axis quantifies run time. The single-series visualization facilitates analysis of how the dependent variable responds to changes in the independent parameter setting.

Analysis of the plotted data reveals that run time ranges from 27.73 (at swapping time = 3) to 29.02 (at swapping time = 2), representing a span of 1.28 units. The overall trend is increasing, with values rising from 27.75 at the initial setting to 28.57 at the final setting. Notably, the minimum value occurs at an intermediate swapping time setting (3), suggesting non-monotonic behavior that warrants further investigation.

These results indicate that swapping time configuration meaningfully impacts run time in this experimental context. The relatively modest variation suggests that this parameter has limited influence on the measured metric within the tested range. Confidence in these findings is high given the direct correspondence between CSV data and plotted values. Future analysis should consider incorporating error bars representing variance across multiple experimental runs to strengthen statistical validity.

\begin{figure}[!htbp]
\centering
\includegraphics[width=0.95\linewidth]{images/500_nodes_Swapping_Time/(e)_Number_of_module_swapped_bar_chart.png}
\caption{(e) Number of module swapped}
\label{fig:500_nodes_Swapping_Time_e__Number_of_module_swapped_bar_chart}
\end{figure}

This bar chart presents the relationship between swapping time and total module swapped for the 500 nodes Swapping Time experimental scenario. The x-axis displays swapping time values ranging from 1 to 4, while the y-axis quantifies total module swapped. The single-series visualization facilitates analysis of how the dependent variable responds to changes in the independent parameter setting.

Analysis of the plotted data reveals that total module swapped ranges from 14.00 (at swapping time = 1) to 15.00 (at swapping time = 2), representing a span of 1.00 units. The overall trend is increasing, with values rising from 14.00 at the initial setting to 15.00 at the final setting. 

These results indicate that swapping time configuration meaningfully impacts total module swapped in this experimental context. The relatively modest variation suggests that this parameter has limited influence on the measured metric within the tested range. Confidence in these findings is high given the direct correspondence between CSV data and plotted values. Future analysis should consider incorporating error bars representing variance across multiple experimental runs to strengthen statistical validity.


\clearpage

\subsection{500 nodes Threshold}

\begin{figure}[!htbp]
\centering
\includegraphics[width=0.95\linewidth]{images/500_nodes_Threshold/(a)_Travel_Time_bar_chart.png}
\caption{(a) Travel Time}
\label{fig:500_nodes_Threshold_a__Travel_Time_bar_chart}
\end{figure}

This bar chart presents the relationship between threshold and total travel time for the 500 nodes Threshold experimental scenario. The x-axis displays threshold values ranging from 5 to 20, while the y-axis quantifies total travel time. The single-series visualization facilitates analysis of how the dependent variable responds to changes in the independent parameter setting.

Analysis of the plotted data reveals that total travel time ranges from 3287.54 (at threshold = 5) to 3287.54 (at threshold = 5), representing a span of 0.00 units. The values remain relatively stable across the parameter range, with minimal net change between initial (3287.54) and final (3287.54) settings. 

These results indicate that threshold configuration meaningfully impacts total travel time in this experimental context. The relatively modest variation suggests that this parameter has limited influence on the measured metric within the tested range. Confidence in these findings is high given the direct correspondence between CSV data and plotted values. Future analysis should consider incorporating error bars representing variance across multiple experimental runs to strengthen statistical validity.

\begin{figure}[!htbp]
\centering
\includegraphics[width=0.95\linewidth]{images/500_nodes_Threshold/(b)_Energy_consumption_bar_chart.png}
\caption{(b) Energy consumption}
\label{fig:500_nodes_Threshold_b__Energy_consumption_bar_chart}
\end{figure}

This bar chart presents the relationship between threshold and total energy consumed for the 500 nodes Threshold experimental scenario. The x-axis displays threshold values ranging from 5 to 20, while the y-axis quantifies total energy consumed. The single-series visualization facilitates analysis of how the dependent variable responds to changes in the independent parameter setting.

Analysis of the plotted data reveals that total energy consumed ranges from 323.50 (at threshold = 5) to 323.50 (at threshold = 5), representing a span of 0.00 units. The values remain relatively stable across the parameter range, with minimal net change between initial (323.50) and final (323.50) settings. 

These results indicate that threshold configuration meaningfully impacts total energy consumed in this experimental context. The relatively modest variation suggests that this parameter has limited influence on the measured metric within the tested range. Confidence in these findings is high given the direct correspondence between CSV data and plotted values. Future analysis should consider incorporating error bars representing variance across multiple experimental runs to strengthen statistical validity.

\begin{figure}[!htbp]
\centering
\includegraphics[width=0.95\linewidth]{images/500_nodes_Threshold/(c)_Distance_covered_bar_chart.png}
\caption{(c) Distance covered}
\label{fig:500_nodes_Threshold_c__Distance_covered_bar_chart}
\end{figure}

This bar chart presents the relationship between threshold and total distance covered for the 500 nodes Threshold experimental scenario. The x-axis displays threshold values ranging from 5 to 20, while the y-axis quantifies total distance covered. The single-series visualization facilitates analysis of how the dependent variable responds to changes in the independent parameter setting.

Analysis of the plotted data reveals that total distance covered ranges from 2138.91 (at threshold = 5) to 2138.91 (at threshold = 5), representing a span of 0.00 units. The values remain relatively stable across the parameter range, with minimal net change between initial (2138.91) and final (2138.91) settings. 

These results indicate that threshold configuration meaningfully impacts total distance covered in this experimental context. The relatively modest variation suggests that this parameter has limited influence on the measured metric within the tested range. Confidence in these findings is high given the direct correspondence between CSV data and plotted values. Future analysis should consider incorporating error bars representing variance across multiple experimental runs to strengthen statistical validity.

\begin{figure}[!htbp]
\centering
\includegraphics[width=0.95\linewidth]{images/500_nodes_Threshold/(d)_Runtime_bar_chart.png}
\caption{(d) Runtime}
\label{fig:500_nodes_Threshold_d__Runtime_bar_chart}
\end{figure}

This bar chart presents the relationship between threshold and run time for the 500 nodes Threshold experimental scenario. The x-axis displays threshold values ranging from 5 to 20, while the y-axis quantifies run time. The single-series visualization facilitates analysis of how the dependent variable responds to changes in the independent parameter setting.

Analysis of the plotted data reveals that run time ranges from 29.02 (at threshold = 20) to 66.47 (at threshold = 10), representing a span of 37.45 units. The overall trend is decreasing, with values declining from 58.34 at the initial setting to 29.02 at the final setting. 

These results indicate that threshold configuration meaningfully impacts run time in this experimental context. The substantial variation observed (coefficient of variation exceeding 20\%) suggests that parameter tuning could yield significant performance improvements. Confidence in these findings is high given the direct correspondence between CSV data and plotted values. Future analysis should consider incorporating error bars representing variance across multiple experimental runs to strengthen statistical validity.

\begin{figure}[!htbp]
\centering
\includegraphics[width=0.95\linewidth]{images/500_nodes_Threshold/(e)_Number_of_module_swapped_bar_chart.png}
\caption{(e) Number of module swapped}
\label{fig:500_nodes_Threshold_e__Number_of_module_swapped_bar_chart}
\end{figure}

This bar chart presents the relationship between threshold and total module swapped for the 500 nodes Threshold experimental scenario. The x-axis displays threshold values ranging from 5 to 20, while the y-axis quantifies total module swapped. The single-series visualization facilitates analysis of how the dependent variable responds to changes in the independent parameter setting.

Analysis of the plotted data reveals that total module swapped ranges from 15.00 (at threshold = 5) to 15.00 (at threshold = 5), representing a span of 0.00 units. The values remain relatively stable across the parameter range, with minimal net change between initial (15.00) and final (15.00) settings. 

These results indicate that threshold configuration meaningfully impacts total module swapped in this experimental context. The relatively modest variation suggests that this parameter has limited influence on the measured metric within the tested range. Confidence in these findings is high given the direct correspondence between CSV data and plotted values. Future analysis should consider incorporating error bars representing variance across multiple experimental runs to strengthen statistical validity.


\clearpage

\subsection{500 nodes Traffic}

\begin{figure}[!htbp]
\centering
\includegraphics[width=0.95\linewidth]{images/500_nodes_Traffic/(a)_Travel_Time_bar_chart.png}
\caption{(a) Travel Time}
\label{fig:500_nodes_Traffic_a__Travel_Time_bar_chart}
\end{figure}

This bar chart presents the relationship between traffic and total travel time for the 500 nodes Traffic experimental scenario. The x-axis displays traffic values ranging from High to Low, while the y-axis quantifies total travel time. The single-series visualization facilitates analysis of how the dependent variable responds to changes in the independent parameter setting.

Analysis of the plotted data reveals that total travel time ranges from 2915.88 (at traffic = Low) to 3618.28 (at traffic = High), representing a span of 702.40 units. The overall trend is decreasing, with values declining from 3618.28 at the initial setting to 2915.88 at the final setting. 

These results indicate that traffic configuration meaningfully impacts total travel time in this experimental context. The substantial variation observed (coefficient of variation exceeding 20\%) suggests that parameter tuning could yield significant performance improvements. Confidence in these findings is high given the direct correspondence between CSV data and plotted values. Future analysis should consider incorporating error bars representing variance across multiple experimental runs to strengthen statistical validity.

\begin{figure}[!htbp]
\centering
\includegraphics[width=0.95\linewidth]{images/500_nodes_Traffic/(b)_Energy_consumption_bar_chart.png}
\caption{(b) Energy consumption}
\label{fig:500_nodes_Traffic_b__Energy_consumption_bar_chart}
\end{figure}

This bar chart presents the relationship between traffic and total energy consumed for the 500 nodes Traffic experimental scenario. The x-axis displays traffic values ranging from High to Low, while the y-axis quantifies total energy consumed. The single-series visualization facilitates analysis of how the dependent variable responds to changes in the independent parameter setting.

Analysis of the plotted data reveals that total energy consumed ranges from 287.80 (at traffic = High) to 351.10 (at traffic = Low), representing a span of 63.29 units. The overall trend is increasing, with values rising from 287.80 at the initial setting to 351.10 at the final setting. 

These results indicate that traffic configuration meaningfully impacts total energy consumed in this experimental context. The substantial variation observed (coefficient of variation exceeding 20\%) suggests that parameter tuning could yield significant performance improvements. Confidence in these findings is high given the direct correspondence between CSV data and plotted values. Future analysis should consider incorporating error bars representing variance across multiple experimental runs to strengthen statistical validity.

\begin{figure}[!htbp]
\centering
\includegraphics[width=0.95\linewidth]{images/500_nodes_Traffic/(c)_Distance_covered_bar_chart.png}
\caption{(c) Distance covered}
\label{fig:500_nodes_Traffic_c__Distance_covered_bar_chart}
\end{figure}

This bar chart presents the relationship between traffic and total distance covered for the 500 nodes Traffic experimental scenario. The x-axis displays traffic values ranging from High to Low, while the y-axis quantifies total distance covered. The single-series visualization facilitates analysis of how the dependent variable responds to changes in the independent parameter setting.

Analysis of the plotted data reveals that total distance covered ranges from 1986.78 (at traffic = High) to 2241.45 (at traffic = Low), representing a span of 254.67 units. The overall trend is increasing, with values rising from 1986.78 at the initial setting to 2241.45 at the final setting. 

These results indicate that traffic configuration meaningfully impacts total distance covered in this experimental context. The relatively modest variation suggests that this parameter has limited influence on the measured metric within the tested range. Confidence in these findings is high given the direct correspondence between CSV data and plotted values. Future analysis should consider incorporating error bars representing variance across multiple experimental runs to strengthen statistical validity.

\begin{figure}[!htbp]
\centering
\includegraphics[width=0.95\linewidth]{images/500_nodes_Traffic/(d)_Runtime_bar_chart.png}
\caption{(d) Runtime}
\label{fig:500_nodes_Traffic_d__Runtime_bar_chart}
\end{figure}

This bar chart presents the relationship between traffic and run time for the 500 nodes Traffic experimental scenario. The x-axis displays traffic values ranging from High to Low, while the y-axis quantifies run time. The single-series visualization facilitates analysis of how the dependent variable responds to changes in the independent parameter setting.

Analysis of the plotted data reveals that run time ranges from 37.23 (at traffic = High) to 54.49 (at traffic = Low), representing a span of 17.26 units. The overall trend is increasing, with values rising from 37.23 at the initial setting to 54.49 at the final setting. 

These results indicate that traffic configuration meaningfully impacts run time in this experimental context. The substantial variation observed (coefficient of variation exceeding 20\%) suggests that parameter tuning could yield significant performance improvements. Confidence in these findings is high given the direct correspondence between CSV data and plotted values. Future analysis should consider incorporating error bars representing variance across multiple experimental runs to strengthen statistical validity.

\begin{figure}[!htbp]
\centering
\includegraphics[width=0.95\linewidth]{images/500_nodes_Traffic/(e)_Number_of_module_swapped_bar_chart.png}
\caption{(e) Number of module swapped}
\label{fig:500_nodes_Traffic_e__Number_of_module_swapped_bar_chart}
\end{figure}

This bar chart presents the relationship between traffic and total module swapped for the 500 nodes Traffic experimental scenario. The x-axis displays traffic values ranging from High to Low, while the y-axis quantifies total module swapped. The single-series visualization facilitates analysis of how the dependent variable responds to changes in the independent parameter setting.

Analysis of the plotted data reveals that total module swapped ranges from 12.00 (at traffic = High) to 16.00 (at traffic = Low), representing a span of 4.00 units. The overall trend is increasing, with values rising from 12.00 at the initial setting to 16.00 at the final setting. 

These results indicate that traffic configuration meaningfully impacts total module swapped in this experimental context. The substantial variation observed (coefficient of variation exceeding 20\%) suggests that parameter tuning could yield significant performance improvements. Confidence in these findings is high given the direct correspondence between CSV data and plotted values. Future analysis should consider incorporating error bars representing variance across multiple experimental runs to strengthen statistical validity.


\clearpage

\subsection{MT2TE and Other Program Module Change}

\begin{figure}[!htbp]
\centering
\includegraphics[width=0.95\linewidth]{images/MT2TE_and_Other_Program_Module_Change/(a)_Total_Travel_Time.png}
\caption{(a) Total Travel Time}
\label{fig:MT2TE_and_Other_Program_Module_Change_a__Total_Travel_Time}
\end{figure}

This grouped bar chart presents a comparative analysis of total travel time across multiple algorithms within the MT2TE and Other Program Module Change experimental configuration. The x-axis represents the number of nodes in the network, while the y-axis quantifies the total travel time metric. The legend identifies 4 distinct algorithms: Genetic Algorithm, EVRPBSS, Ant Colony, Clarke and Wright algorithm. This visualization enables direct comparison of algorithmic performance under identical network conditions.

Quantitative analysis reveals significant performance disparities among the evaluated algorithms. Clarke and Wright algorithm demonstrates superior performance with a mean total travel time of 688.78, while Genetic Algorithm exhibits the highest values averaging 1366.67. This represents an improvement of approximately 49.6\% when comparing the best to worst performing algorithms. The relative ranking of algorithms remains largely consistent across different node configurations, suggesting robust performance characteristics.

These findings support the hypothesis that algorithmic choice significantly impacts system performance metrics. Future work should incorporate statistical significance testing and confidence intervals to strengthen these comparative conclusions. Additionally, examining the computational complexity trade-offs between algorithms would provide valuable context for practical deployment decisions.

\begin{figure}[!htbp]
\centering
\includegraphics[width=0.95\linewidth]{images/MT2TE_and_Other_Program_Module_Change/(b)_Energy.png}
\caption{(b) Energy}
\label{fig:MT2TE_and_Other_Program_Module_Change_b__Energy}
\end{figure}

This grouped bar chart presents a comparative analysis of energy across multiple algorithms within the MT2TE and Other Program Module Change experimental configuration. The x-axis represents the number of nodes in the network, while the y-axis quantifies the energy metric. The legend identifies 4 distinct algorithms: Genetic Algorithm, EVRPBSS, Ant Colony, Clarke and Wright algorithm. This visualization enables direct comparison of algorithmic performance under identical network conditions.

Quantitative analysis reveals significant performance disparities among the evaluated algorithms. Clarke and Wright algorithm demonstrates superior performance with a mean energy of 688.78, while Genetic Algorithm exhibits the highest values averaging 1366.67. This represents an improvement of approximately 49.6\% when comparing the best to worst performing algorithms. The relative ranking of algorithms remains largely consistent across different node configurations, suggesting robust performance characteristics.

These findings support the hypothesis that algorithmic choice significantly impacts system performance metrics. Future work should incorporate statistical significance testing and confidence intervals to strengthen these comparative conclusions. Additionally, examining the computational complexity trade-offs between algorithms would provide valuable context for practical deployment decisions.

\begin{figure}[!htbp]
\centering
\includegraphics[width=0.95\linewidth]{images/MT2TE_and_Other_Program_Module_Change/(c)_Distance.png}
\caption{(c) Distance}
\label{fig:MT2TE_and_Other_Program_Module_Change_c__Distance}
\end{figure}

This grouped bar chart presents a comparative analysis of distance across multiple algorithms within the MT2TE and Other Program Module Change experimental configuration. The x-axis represents the number of nodes in the network, while the y-axis quantifies the distance metric. The legend identifies 4 distinct algorithms: Genetic Algorithm, EVRPBSS, Ant Colony, Clarke and Wright algorithm. This visualization enables direct comparison of algorithmic performance under identical network conditions.

Quantitative analysis reveals significant performance disparities among the evaluated algorithms. Clarke and Wright algorithm demonstrates superior performance with a mean distance of 688.78, while Genetic Algorithm exhibits the highest values averaging 1366.67. This represents an improvement of approximately 49.6\% when comparing the best to worst performing algorithms. The relative ranking of algorithms remains largely consistent across different node configurations, suggesting robust performance characteristics.

These findings support the hypothesis that algorithmic choice significantly impacts system performance metrics. Future work should incorporate statistical significance testing and confidence intervals to strengthen these comparative conclusions. Additionally, examining the computational complexity trade-offs between algorithms would provide valuable context for practical deployment decisions.

\begin{figure}[!htbp]
\centering
\includegraphics[width=0.95\linewidth]{images/MT2TE_and_Other_Program_Module_Change/(d)_Module_Swapped.png}
\caption{(d) Module Swapped}
\label{fig:MT2TE_and_Other_Program_Module_Change_d__Module_Swapped}
\end{figure}

This grouped bar chart presents a comparative analysis of module swapped across multiple algorithms within the MT2TE and Other Program Module Change experimental configuration. The x-axis represents the number of nodes in the network, while the y-axis quantifies the module swapped metric. The legend identifies 4 distinct algorithms: Genetic Algorithm, EVRPBSS, Ant Colony, Clarke and Wright algorithm. This visualization enables direct comparison of algorithmic performance under identical network conditions.

Quantitative analysis reveals significant performance disparities among the evaluated algorithms. Clarke and Wright algorithm demonstrates superior performance with a mean module swapped of 688.78, while Genetic Algorithm exhibits the highest values averaging 1366.67. This represents an improvement of approximately 49.6\% when comparing the best to worst performing algorithms. The relative ranking of algorithms remains largely consistent across different node configurations, suggesting robust performance characteristics.

These findings support the hypothesis that algorithmic choice significantly impacts system performance metrics. Future work should incorporate statistical significance testing and confidence intervals to strengthen these comparative conclusions. Additionally, examining the computational complexity trade-offs between algorithms would provide valuable context for practical deployment decisions.

\begin{figure}[!htbp]
\centering
\includegraphics[width=0.95\linewidth]{images/MT2TE_and_Other_Program_Module_Change/(e)_Execution_Time.png}
\caption{(e) Execution Time}
\label{fig:MT2TE_and_Other_Program_Module_Change_e__Execution_Time}
\end{figure}

This grouped bar chart presents a comparative analysis of execution time across multiple algorithms within the MT2TE and Other Program Module Change experimental configuration. The x-axis represents the number of nodes in the network, while the y-axis quantifies the execution time metric. The legend identifies 4 distinct algorithms: Genetic Algorithm, EVRPBSS, Ant Colony, Clarke and Wright algorithm. This visualization enables direct comparison of algorithmic performance under identical network conditions.

Quantitative analysis reveals significant performance disparities among the evaluated algorithms. Clarke and Wright algorithm demonstrates superior performance with a mean execution time of 688.78, while Genetic Algorithm exhibits the highest values averaging 1366.67. This represents an improvement of approximately 49.6\% when comparing the best to worst performing algorithms. The relative ranking of algorithms remains largely consistent across different node configurations, suggesting robust performance characteristics.

These findings support the hypothesis that algorithmic choice significantly impacts system performance metrics. Future work should incorporate statistical significance testing and confidence intervals to strengthen these comparative conclusions. Additionally, examining the computational complexity trade-offs between algorithms would provide valuable context for practical deployment decisions.

\begin{figure}[!htbp]
\centering
\includegraphics[width=0.95\linewidth]{images/MT2TE_and_Other_Program_Module_Change/(f)_Execution_Time.png}
\caption{(f) Execution Time}
\label{fig:MT2TE_and_Other_Program_Module_Change_f__Execution_Time}
\end{figure}

This grouped bar chart presents a comparative analysis of execution time across multiple algorithms within the MT2TE and Other Program Module Change experimental configuration. The x-axis represents the number of nodes in the network, while the y-axis quantifies the execution time metric. The legend identifies 3 distinct algorithms: EVRPBSS, Ant Colony, Clarke and Wright algorithm. This visualization enables direct comparison of algorithmic performance under identical network conditions.

Quantitative analysis reveals significant performance disparities among the evaluated algorithms. Clarke and Wright algorithm demonstrates superior performance with a mean execution time of 688.78, while Ant Colony exhibits the highest values averaging 1020.01. This represents an improvement of approximately 32.5\% when comparing the best to worst performing algorithms. The relative ranking of algorithms remains largely consistent across different node configurations, suggesting robust performance characteristics.

These findings support the hypothesis that algorithmic choice significantly impacts system performance metrics. Future work should incorporate statistical significance testing and confidence intervals to strengthen these comparative conclusions. Additionally, examining the computational complexity trade-offs between algorithms would provide valuable context for practical deployment decisions.


\clearpage

\subsection{MT2TE and Other Program Node Change}

\begin{figure}[!htbp]
\centering
\includegraphics[width=0.95\linewidth]{images/MT2TE_and_Other_Program_Node_Change/(a)_Total_Travel_Time.png}
\caption{(a) Total Travel Time}
\label{fig:MT2TE_and_Other_Program_Node_Change_a__Total_Travel_Time}
\end{figure}

This grouped bar chart presents a comparative analysis of total travel time across multiple algorithms within the MT2TE and Other Program Node Change experimental configuration. The x-axis represents the number of nodes in the network, while the y-axis quantifies the total travel time metric. The legend identifies 4 distinct algorithms: Genetic Algorithm, EVRPBSS, Ant Colony, Clarke and Wright algorithm. This visualization enables direct comparison of algorithmic performance under identical network conditions.

Quantitative analysis reveals significant performance disparities among the evaluated algorithms. Clarke and Wright algorithm demonstrates superior performance with a mean total travel time of 862.43, while Genetic Algorithm exhibits the highest values averaging 2037.20. This represents an improvement of approximately 57.7\% when comparing the best to worst performing algorithms. The relative ranking of algorithms remains largely consistent across different node configurations, suggesting robust performance characteristics.

These findings support the hypothesis that algorithmic choice significantly impacts system performance metrics. Future work should incorporate statistical significance testing and confidence intervals to strengthen these comparative conclusions. Additionally, examining the computational complexity trade-offs between algorithms would provide valuable context for practical deployment decisions.

\begin{figure}[!htbp]
\centering
\includegraphics[width=0.95\linewidth]{images/MT2TE_and_Other_Program_Node_Change/(b)_Energy.png}
\caption{(b) Energy}
\label{fig:MT2TE_and_Other_Program_Node_Change_b__Energy}
\end{figure}

This grouped bar chart presents a comparative analysis of energy across multiple algorithms within the MT2TE and Other Program Node Change experimental configuration. The x-axis represents the number of nodes in the network, while the y-axis quantifies the energy metric. The legend identifies 4 distinct algorithms: Genetic Algorithm, EVRPBSS, Ant Colony, Clarke and Wright algorithm. This visualization enables direct comparison of algorithmic performance under identical network conditions.

Quantitative analysis reveals significant performance disparities among the evaluated algorithms. Clarke and Wright algorithm demonstrates superior performance with a mean energy of 862.43, while Genetic Algorithm exhibits the highest values averaging 2037.20. This represents an improvement of approximately 57.7\% when comparing the best to worst performing algorithms. The relative ranking of algorithms remains largely consistent across different node configurations, suggesting robust performance characteristics.

These findings support the hypothesis that algorithmic choice significantly impacts system performance metrics. Future work should incorporate statistical significance testing and confidence intervals to strengthen these comparative conclusions. Additionally, examining the computational complexity trade-offs between algorithms would provide valuable context for practical deployment decisions.

\begin{figure}[!htbp]
\centering
\includegraphics[width=0.95\linewidth]{images/MT2TE_and_Other_Program_Node_Change/(c)_Distance.png}
\caption{(c) Distance}
\label{fig:MT2TE_and_Other_Program_Node_Change_c__Distance}
\end{figure}

This grouped bar chart presents a comparative analysis of distance across multiple algorithms within the MT2TE and Other Program Node Change experimental configuration. The x-axis represents the number of nodes in the network, while the y-axis quantifies the distance metric. The legend identifies 4 distinct algorithms: Genetic Algorithm, EVRPBSS, Ant Colony, Clarke and Wright algorithm. This visualization enables direct comparison of algorithmic performance under identical network conditions.

Quantitative analysis reveals significant performance disparities among the evaluated algorithms. Clarke and Wright algorithm demonstrates superior performance with a mean distance of 862.43, while Genetic Algorithm exhibits the highest values averaging 2037.20. This represents an improvement of approximately 57.7\% when comparing the best to worst performing algorithms. The relative ranking of algorithms remains largely consistent across different node configurations, suggesting robust performance characteristics.

These findings support the hypothesis that algorithmic choice significantly impacts system performance metrics. Future work should incorporate statistical significance testing and confidence intervals to strengthen these comparative conclusions. Additionally, examining the computational complexity trade-offs between algorithms would provide valuable context for practical deployment decisions.

\begin{figure}[!htbp]
\centering
\includegraphics[width=0.95\linewidth]{images/MT2TE_and_Other_Program_Node_Change/(d)_Execution_Time.png}
\caption{(d) Execution Time}
\label{fig:MT2TE_and_Other_Program_Node_Change_d__Execution_Time}
\end{figure}

This grouped bar chart presents a comparative analysis of execution time across multiple algorithms within the MT2TE and Other Program Node Change experimental configuration. The x-axis represents the number of nodes in the network, while the y-axis quantifies the execution time metric. The legend identifies 4 distinct algorithms: Genetic Algorithm, EVRPBSS, Ant Colony, Clarke and Wright algorithm. This visualization enables direct comparison of algorithmic performance under identical network conditions.

Quantitative analysis reveals significant performance disparities among the evaluated algorithms. Clarke and Wright algorithm demonstrates superior performance with a mean execution time of 862.43, while Genetic Algorithm exhibits the highest values averaging 2037.20. This represents an improvement of approximately 57.7\% when comparing the best to worst performing algorithms. The relative ranking of algorithms remains largely consistent across different node configurations, suggesting robust performance characteristics.

These findings support the hypothesis that algorithmic choice significantly impacts system performance metrics. Future work should incorporate statistical significance testing and confidence intervals to strengthen these comparative conclusions. Additionally, examining the computational complexity trade-offs between algorithms would provide valuable context for practical deployment decisions.

\begin{figure}[!htbp]
\centering
\includegraphics[width=0.95\linewidth]{images/MT2TE_and_Other_Program_Node_Change/(e)_Total_Module_Swapped.png}
\caption{(e) Total Module Swapped}
\label{fig:MT2TE_and_Other_Program_Node_Change_e__Total_Module_Swapped}
\end{figure}

This grouped bar chart presents a comparative analysis of total module swapped across multiple algorithms within the MT2TE and Other Program Node Change experimental configuration. The x-axis represents the number of nodes in the network, while the y-axis quantifies the total module swapped metric. The legend identifies 4 distinct algorithms: Genetic Algorithm, EVRPBSS, Ant Colony, Clarke and Wright algorithm. This visualization enables direct comparison of algorithmic performance under identical network conditions.

Quantitative analysis reveals significant performance disparities among the evaluated algorithms. Clarke and Wright algorithm demonstrates superior performance with a mean total module swapped of 862.43, while Genetic Algorithm exhibits the highest values averaging 2037.20. This represents an improvement of approximately 57.7\% when comparing the best to worst performing algorithms. The relative ranking of algorithms remains largely consistent across different node configurations, suggesting robust performance characteristics.

These findings support the hypothesis that algorithmic choice significantly impacts system performance metrics. Future work should incorporate statistical significance testing and confidence intervals to strengthen these comparative conclusions. Additionally, examining the computational complexity trade-offs between algorithms would provide valuable context for practical deployment decisions.

\begin{figure}[!htbp]
\centering
\includegraphics[width=0.95\linewidth]{images/MT2TE_and_Other_Program_Node_Change/(f)_Execution_Time.png}
\caption{(f) Execution Time}
\label{fig:MT2TE_and_Other_Program_Node_Change_f__Execution_Time}
\end{figure}

This grouped bar chart presents a comparative analysis of execution time across multiple algorithms within the MT2TE and Other Program Node Change experimental configuration. The x-axis represents the number of nodes in the network, while the y-axis quantifies the execution time metric. The legend identifies 4 distinct algorithms: Genetic Algorithm, EVRPBSS, Ant Colony, Clarke and Wright algorithm. This visualization enables direct comparison of algorithmic performance under identical network conditions.

Quantitative analysis reveals significant performance disparities among the evaluated algorithms. Clarke and Wright algorithm demonstrates superior performance with a mean execution time of 862.43, while Genetic Algorithm exhibits the highest values averaging 2037.20. This represents an improvement of approximately 57.7\% when comparing the best to worst performing algorithms. The relative ranking of algorithms remains largely consistent across different node configurations, suggesting robust performance characteristics.

These findings support the hypothesis that algorithmic choice significantly impacts system performance metrics. Future work should incorporate statistical significance testing and confidence intervals to strengthen these comparative conclusions. Additionally, examining the computational complexity trade-offs between algorithms would provide valuable context for practical deployment decisions.


\clearpage

\subsection{MT2TE and Other Program Swap Time Change}

\begin{figure}[!htbp]
\centering
\includegraphics[width=0.95\linewidth]{images/MT2TE_and_Other_Program_Swap_Time_Change/(a)_Total_Travel_Time.png}
\caption{(a) Total Travel Time}
\label{fig:MT2TE_and_Other_Program_Swap_Time_Change_a__Total_Travel_Time}
\end{figure}

This grouped bar chart presents a comparative analysis of total travel time across multiple algorithms within the MT2TE and Other Program Swap Time Change experimental configuration. The x-axis represents the number of nodes in the network, while the y-axis quantifies the total travel time metric. The legend identifies 4 distinct algorithms: Genetic Algorithm, EVRPBSS, Ant Colony, Clarke and Wright algorithm. This visualization enables direct comparison of algorithmic performance under identical network conditions.

Quantitative analysis reveals significant performance disparities among the evaluated algorithms. Clarke and Wright algorithm demonstrates superior performance with a mean total travel time of 697.62, while Genetic Algorithm exhibits the highest values averaging 1438.33. This represents an improvement of approximately 51.5\% when comparing the best to worst performing algorithms. The relative ranking of algorithms remains largely consistent across different node configurations, suggesting robust performance characteristics.

These findings support the hypothesis that algorithmic choice significantly impacts system performance metrics. Future work should incorporate statistical significance testing and confidence intervals to strengthen these comparative conclusions. Additionally, examining the computational complexity trade-offs between algorithms would provide valuable context for practical deployment decisions.

\begin{figure}[!htbp]
\centering
\includegraphics[width=0.95\linewidth]{images/MT2TE_and_Other_Program_Swap_Time_Change/(b)_Energy.png}
\caption{(b) Energy}
\label{fig:MT2TE_and_Other_Program_Swap_Time_Change_b__Energy}
\end{figure}

This grouped bar chart presents a comparative analysis of energy across multiple algorithms within the MT2TE and Other Program Swap Time Change experimental configuration. The x-axis represents the number of nodes in the network, while the y-axis quantifies the energy metric. The legend identifies 4 distinct algorithms: Genetic Algorithm, EVRPBSS, Ant Colony, Clarke and Wright algorithm. This visualization enables direct comparison of algorithmic performance under identical network conditions.

Quantitative analysis reveals significant performance disparities among the evaluated algorithms. Clarke and Wright algorithm demonstrates superior performance with a mean energy of 697.62, while Genetic Algorithm exhibits the highest values averaging 1438.33. This represents an improvement of approximately 51.5\% when comparing the best to worst performing algorithms. The relative ranking of algorithms remains largely consistent across different node configurations, suggesting robust performance characteristics.

These findings support the hypothesis that algorithmic choice significantly impacts system performance metrics. Future work should incorporate statistical significance testing and confidence intervals to strengthen these comparative conclusions. Additionally, examining the computational complexity trade-offs between algorithms would provide valuable context for practical deployment decisions.

\begin{figure}[!htbp]
\centering
\includegraphics[width=0.95\linewidth]{images/MT2TE_and_Other_Program_Swap_Time_Change/(c)_Distance.png}
\caption{(c) Distance}
\label{fig:MT2TE_and_Other_Program_Swap_Time_Change_c__Distance}
\end{figure}

This grouped bar chart presents a comparative analysis of distance across multiple algorithms within the MT2TE and Other Program Swap Time Change experimental configuration. The x-axis represents the number of nodes in the network, while the y-axis quantifies the distance metric. The legend identifies 4 distinct algorithms: Genetic Algorithm, EVRPBSS, Ant Colony, Clarke and Wright algorithm. This visualization enables direct comparison of algorithmic performance under identical network conditions.

Quantitative analysis reveals significant performance disparities among the evaluated algorithms. Clarke and Wright algorithm demonstrates superior performance with a mean distance of 697.62, while Genetic Algorithm exhibits the highest values averaging 1438.33. This represents an improvement of approximately 51.5\% when comparing the best to worst performing algorithms. The relative ranking of algorithms remains largely consistent across different node configurations, suggesting robust performance characteristics.

These findings support the hypothesis that algorithmic choice significantly impacts system performance metrics. Future work should incorporate statistical significance testing and confidence intervals to strengthen these comparative conclusions. Additionally, examining the computational complexity trade-offs between algorithms would provide valuable context for practical deployment decisions.

\begin{figure}[!htbp]
\centering
\includegraphics[width=0.95\linewidth]{images/MT2TE_and_Other_Program_Swap_Time_Change/(d)_Total_Module_Swapped.png}
\caption{(d) Total Module Swapped}
\label{fig:MT2TE_and_Other_Program_Swap_Time_Change_d__Total_Module_Swapped}
\end{figure}

This grouped bar chart presents a comparative analysis of total module swapped across multiple algorithms within the MT2TE and Other Program Swap Time Change experimental configuration. The x-axis represents the number of nodes in the network, while the y-axis quantifies the total module swapped metric. The legend identifies 4 distinct algorithms: Genetic Algorithm, EVRPBSS, Ant Colony, Clarke and Wright algorithm. This visualization enables direct comparison of algorithmic performance under identical network conditions.

Quantitative analysis reveals significant performance disparities among the evaluated algorithms. Clarke and Wright algorithm demonstrates superior performance with a mean total module swapped of 697.62, while Genetic Algorithm exhibits the highest values averaging 1438.33. This represents an improvement of approximately 51.5\% when comparing the best to worst performing algorithms. The relative ranking of algorithms remains largely consistent across different node configurations, suggesting robust performance characteristics.

These findings support the hypothesis that algorithmic choice significantly impacts system performance metrics. Future work should incorporate statistical significance testing and confidence intervals to strengthen these comparative conclusions. Additionally, examining the computational complexity trade-offs between algorithms would provide valuable context for practical deployment decisions.

\begin{figure}[!htbp]
\centering
\includegraphics[width=0.95\linewidth]{images/MT2TE_and_Other_Program_Swap_Time_Change/(e)_Execution_Time.png}
\caption{(e) Execution Time}
\label{fig:MT2TE_and_Other_Program_Swap_Time_Change_e__Execution_Time}
\end{figure}

This grouped bar chart presents a comparative analysis of execution time across multiple algorithms within the MT2TE and Other Program Swap Time Change experimental configuration. The x-axis represents the number of nodes in the network, while the y-axis quantifies the execution time metric. The legend identifies 4 distinct algorithms: Genetic Algorithm, EVRPBSS, Ant Colony, Clarke and Wright algorithm. This visualization enables direct comparison of algorithmic performance under identical network conditions.

Quantitative analysis reveals significant performance disparities among the evaluated algorithms. Clarke and Wright algorithm demonstrates superior performance with a mean execution time of 697.62, while Genetic Algorithm exhibits the highest values averaging 1438.33. This represents an improvement of approximately 51.5\% when comparing the best to worst performing algorithms. The relative ranking of algorithms remains largely consistent across different node configurations, suggesting robust performance characteristics.

These findings support the hypothesis that algorithmic choice significantly impacts system performance metrics. Future work should incorporate statistical significance testing and confidence intervals to strengthen these comparative conclusions. Additionally, examining the computational complexity trade-offs between algorithms would provide valuable context for practical deployment decisions.

\begin{figure}[!htbp]
\centering
\includegraphics[width=0.95\linewidth]{images/MT2TE_and_Other_Program_Swap_Time_Change/(f)_Execution_Time.png}
\caption{(f) Execution Time}
\label{fig:MT2TE_and_Other_Program_Swap_Time_Change_f__Execution_Time}
\end{figure}

This grouped bar chart presents a comparative analysis of execution time across multiple algorithms within the MT2TE and Other Program Swap Time Change experimental configuration. The x-axis represents the number of nodes in the network, while the y-axis quantifies the execution time metric. The legend identifies 3 distinct algorithms: EVRPBSS, Ant Colony, Clarke and Wright algorithm. This visualization enables direct comparison of algorithmic performance under identical network conditions.

Quantitative analysis reveals significant performance disparities among the evaluated algorithms. Clarke and Wright algorithm demonstrates superior performance with a mean execution time of 697.62, while Ant Colony exhibits the highest values averaging 1026.84. This represents an improvement of approximately 32.1\% when comparing the best to worst performing algorithms. The relative ranking of algorithms remains largely consistent across different node configurations, suggesting robust performance characteristics.

These findings support the hypothesis that algorithmic choice significantly impacts system performance metrics. Future work should incorporate statistical significance testing and confidence intervals to strengthen these comparative conclusions. Additionally, examining the computational complexity trade-offs between algorithms would provide valuable context for practical deployment decisions.


\clearpage

\subsection{MT2TE and Other Program Threshold Change}

\begin{figure}[!htbp]
\centering
\includegraphics[width=0.95\linewidth]{images/MT2TE_and_Other_Program_Threshold_Change/(a)_Total_Travel_Time.png}
\caption{(a) Total Travel Time}
\label{fig:MT2TE_and_Other_Program_Threshold_Change_a__Total_Travel_Time}
\end{figure}

This grouped bar chart presents a comparative analysis of total travel time across multiple algorithms within the MT2TE and Other Program Threshold Change experimental configuration. The x-axis represents the number of nodes in the network, while the y-axis quantifies the total travel time metric. The legend identifies 4 distinct algorithms: Genetic Algorithm, EVRPBSS, Ant Colony, Clarke and Wright algorithm. This visualization enables direct comparison of algorithmic performance under identical network conditions.

Quantitative analysis reveals significant performance disparities among the evaluated algorithms. Clarke and Wright algorithm demonstrates superior performance with a mean total travel time of 690.83, while Genetic Algorithm exhibits the highest values averaging 1400.93. This represents an improvement of approximately 50.7\% when comparing the best to worst performing algorithms. The relative ranking of algorithms remains largely consistent across different node configurations, suggesting robust performance characteristics.

These findings support the hypothesis that algorithmic choice significantly impacts system performance metrics. Future work should incorporate statistical significance testing and confidence intervals to strengthen these comparative conclusions. Additionally, examining the computational complexity trade-offs between algorithms would provide valuable context for practical deployment decisions.

\begin{figure}[!htbp]
\centering
\includegraphics[width=0.95\linewidth]{images/MT2TE_and_Other_Program_Threshold_Change/(b)_Energy.png}
\caption{(b) Energy}
\label{fig:MT2TE_and_Other_Program_Threshold_Change_b__Energy}
\end{figure}

This grouped bar chart presents a comparative analysis of energy across multiple algorithms within the MT2TE and Other Program Threshold Change experimental configuration. The x-axis represents the number of nodes in the network, while the y-axis quantifies the energy metric. The legend identifies 4 distinct algorithms: Genetic Algorithm, EVRPBSS, Ant Colony, Clarke and Wright algorithm. This visualization enables direct comparison of algorithmic performance under identical network conditions.

Quantitative analysis reveals significant performance disparities among the evaluated algorithms. Clarke and Wright algorithm demonstrates superior performance with a mean energy of 690.83, while Genetic Algorithm exhibits the highest values averaging 1400.93. This represents an improvement of approximately 50.7\% when comparing the best to worst performing algorithms. The relative ranking of algorithms remains largely consistent across different node configurations, suggesting robust performance characteristics.

These findings support the hypothesis that algorithmic choice significantly impacts system performance metrics. Future work should incorporate statistical significance testing and confidence intervals to strengthen these comparative conclusions. Additionally, examining the computational complexity trade-offs between algorithms would provide valuable context for practical deployment decisions.

\begin{figure}[!htbp]
\centering
\includegraphics[width=0.95\linewidth]{images/MT2TE_and_Other_Program_Threshold_Change/(c)_Distance.png}
\caption{(c) Distance}
\label{fig:MT2TE_and_Other_Program_Threshold_Change_c__Distance}
\end{figure}

This grouped bar chart presents a comparative analysis of distance across multiple algorithms within the MT2TE and Other Program Threshold Change experimental configuration. The x-axis represents the number of nodes in the network, while the y-axis quantifies the distance metric. The legend identifies 4 distinct algorithms: Genetic Algorithm, EVRPBSS, Ant Colony, Clarke and Wright algorithm. This visualization enables direct comparison of algorithmic performance under identical network conditions.

Quantitative analysis reveals significant performance disparities among the evaluated algorithms. Clarke and Wright algorithm demonstrates superior performance with a mean distance of 690.83, while Genetic Algorithm exhibits the highest values averaging 1400.93. This represents an improvement of approximately 50.7\% when comparing the best to worst performing algorithms. The relative ranking of algorithms remains largely consistent across different node configurations, suggesting robust performance characteristics.

These findings support the hypothesis that algorithmic choice significantly impacts system performance metrics. Future work should incorporate statistical significance testing and confidence intervals to strengthen these comparative conclusions. Additionally, examining the computational complexity trade-offs between algorithms would provide valuable context for practical deployment decisions.

\begin{figure}[!htbp]
\centering
\includegraphics[width=0.95\linewidth]{images/MT2TE_and_Other_Program_Threshold_Change/(d)_Module_Swapped.png}
\caption{(d) Module Swapped}
\label{fig:MT2TE_and_Other_Program_Threshold_Change_d__Module_Swapped}
\end{figure}

This grouped bar chart presents a comparative analysis of module swapped across multiple algorithms within the MT2TE and Other Program Threshold Change experimental configuration. The x-axis represents the number of nodes in the network, while the y-axis quantifies the module swapped metric. The legend identifies 4 distinct algorithms: Genetic Algorithm, EVRPBSS, Ant Colony, Clarke and Wright algorithm. This visualization enables direct comparison of algorithmic performance under identical network conditions.

Quantitative analysis reveals significant performance disparities among the evaluated algorithms. Clarke and Wright algorithm demonstrates superior performance with a mean module swapped of 690.83, while Genetic Algorithm exhibits the highest values averaging 1400.93. This represents an improvement of approximately 50.7\% when comparing the best to worst performing algorithms. The relative ranking of algorithms remains largely consistent across different node configurations, suggesting robust performance characteristics.

These findings support the hypothesis that algorithmic choice significantly impacts system performance metrics. Future work should incorporate statistical significance testing and confidence intervals to strengthen these comparative conclusions. Additionally, examining the computational complexity trade-offs between algorithms would provide valuable context for practical deployment decisions.

\begin{figure}[!htbp]
\centering
\includegraphics[width=0.95\linewidth]{images/MT2TE_and_Other_Program_Threshold_Change/(e)_Execution_Time.png}
\caption{(e) Execution Time}
\label{fig:MT2TE_and_Other_Program_Threshold_Change_e__Execution_Time}
\end{figure}

This grouped bar chart presents a comparative analysis of execution time across multiple algorithms within the MT2TE and Other Program Threshold Change experimental configuration. The x-axis represents the number of nodes in the network, while the y-axis quantifies the execution time metric. The legend identifies 4 distinct algorithms: Genetic Algorithm, EVRPBSS, Ant Colony, Clarke and Wright algorithm. This visualization enables direct comparison of algorithmic performance under identical network conditions.

Quantitative analysis reveals significant performance disparities among the evaluated algorithms. Clarke and Wright algorithm demonstrates superior performance with a mean execution time of 690.83, while Genetic Algorithm exhibits the highest values averaging 1400.93. This represents an improvement of approximately 50.7\% when comparing the best to worst performing algorithms. The relative ranking of algorithms remains largely consistent across different node configurations, suggesting robust performance characteristics.

These findings support the hypothesis that algorithmic choice significantly impacts system performance metrics. Future work should incorporate statistical significance testing and confidence intervals to strengthen these comparative conclusions. Additionally, examining the computational complexity trade-offs between algorithms would provide valuable context for practical deployment decisions.

\begin{figure}[!htbp]
\centering
\includegraphics[width=0.95\linewidth]{images/MT2TE_and_Other_Program_Threshold_Change/(f)_Execution_Time.png}
\caption{(f) Execution Time}
\label{fig:MT2TE_and_Other_Program_Threshold_Change_f__Execution_Time}
\end{figure}

This grouped bar chart presents a comparative analysis of execution time across multiple algorithms within the MT2TE and Other Program Threshold Change experimental configuration. The x-axis represents the number of nodes in the network, while the y-axis quantifies the execution time metric. The legend identifies 3 distinct algorithms: EVRPBSS, Ant Colony, Clarke and Wright algorithm. This visualization enables direct comparison of algorithmic performance under identical network conditions.

Quantitative analysis reveals significant performance disparities among the evaluated algorithms. Clarke and Wright algorithm demonstrates superior performance with a mean execution time of 690.83, while Ant Colony exhibits the highest values averaging 1020.01. This represents an improvement of approximately 32.3\% when comparing the best to worst performing algorithms. The relative ranking of algorithms remains largely consistent across different node configurations, suggesting robust performance characteristics.

These findings support the hypothesis that algorithmic choice significantly impacts system performance metrics. Future work should incorporate statistical significance testing and confidence intervals to strengthen these comparative conclusions. Additionally, examining the computational complexity trade-offs between algorithms would provide valuable context for practical deployment decisions.


\clearpage

\subsection{MT2TE and Other Program Traffic}

\begin{figure}[!htbp]
\centering
\includegraphics[width=0.95\linewidth]{images/MT2TE_and_Other_Program_Traffic/(a)_Travel_Time.png}
\caption{(a) Travel Time}
\label{fig:MT2TE_and_Other_Program_Traffic_a__Travel_Time}
\end{figure}

This grouped bar chart presents a comparative analysis of travel time across multiple algorithms within the MT2TE and Other Program Traffic experimental configuration. The x-axis represents the number of nodes in the network, while the y-axis quantifies the travel time metric. The legend identifies 4 distinct algorithms: Genetic Algorithm, EVRPBSS, Ant Colony, Clarke and Wright algorithm. This visualization enables direct comparison of algorithmic performance under identical network conditions.

Quantitative analysis reveals significant performance disparities among the evaluated algorithms. Clarke and Wright algorithm demonstrates superior performance with a mean travel time of 697.95, while Genetic Algorithm exhibits the highest values averaging 1420.64. This represents an improvement of approximately 50.9\% when comparing the best to worst performing algorithms. The relative ranking of algorithms remains largely consistent across different node configurations, suggesting robust performance characteristics.

These findings support the hypothesis that algorithmic choice significantly impacts system performance metrics. Future work should incorporate statistical significance testing and confidence intervals to strengthen these comparative conclusions. Additionally, examining the computational complexity trade-offs between algorithms would provide valuable context for practical deployment decisions.

\begin{figure}[!htbp]
\centering
\includegraphics[width=0.95\linewidth]{images/MT2TE_and_Other_Program_Traffic/(b)_Energy.png}
\caption{(b) Energy}
\label{fig:MT2TE_and_Other_Program_Traffic_b__Energy}
\end{figure}

This grouped bar chart presents a comparative analysis of energy across multiple algorithms within the MT2TE and Other Program Traffic experimental configuration. The x-axis represents the number of nodes in the network, while the y-axis quantifies the energy metric. The legend identifies 4 distinct algorithms: Genetic Algorithm, EVRPBSS, Ant Colony, Clarke and Wright algorithm. This visualization enables direct comparison of algorithmic performance under identical network conditions.

Quantitative analysis reveals significant performance disparities among the evaluated algorithms. Clarke and Wright algorithm demonstrates superior performance with a mean energy of 697.95, while Genetic Algorithm exhibits the highest values averaging 1420.64. This represents an improvement of approximately 50.9\% when comparing the best to worst performing algorithms. The relative ranking of algorithms remains largely consistent across different node configurations, suggesting robust performance characteristics.

These findings support the hypothesis that algorithmic choice significantly impacts system performance metrics. Future work should incorporate statistical significance testing and confidence intervals to strengthen these comparative conclusions. Additionally, examining the computational complexity trade-offs between algorithms would provide valuable context for practical deployment decisions.

\begin{figure}[!htbp]
\centering
\includegraphics[width=0.95\linewidth]{images/MT2TE_and_Other_Program_Traffic/(c)_Distance.png}
\caption{(c) Distance}
\label{fig:MT2TE_and_Other_Program_Traffic_c__Distance}
\end{figure}

This grouped bar chart presents a comparative analysis of distance across multiple algorithms within the MT2TE and Other Program Traffic experimental configuration. The x-axis represents the number of nodes in the network, while the y-axis quantifies the distance metric. The legend identifies 4 distinct algorithms: Genetic Algorithm, EVRPBSS, Ant Colony, Clarke and Wright algorithm. This visualization enables direct comparison of algorithmic performance under identical network conditions.

Quantitative analysis reveals significant performance disparities among the evaluated algorithms. Clarke and Wright algorithm demonstrates superior performance with a mean distance of 697.95, while Genetic Algorithm exhibits the highest values averaging 1420.64. This represents an improvement of approximately 50.9\% when comparing the best to worst performing algorithms. The relative ranking of algorithms remains largely consistent across different node configurations, suggesting robust performance characteristics.

These findings support the hypothesis that algorithmic choice significantly impacts system performance metrics. Future work should incorporate statistical significance testing and confidence intervals to strengthen these comparative conclusions. Additionally, examining the computational complexity trade-offs between algorithms would provide valuable context for practical deployment decisions.

\begin{figure}[!htbp]
\centering
\includegraphics[width=0.95\linewidth]{images/MT2TE_and_Other_Program_Traffic/(d)_Total_Module_Swapped.png}
\caption{(d) Total Module Swapped}
\label{fig:MT2TE_and_Other_Program_Traffic_d__Total_Module_Swapped}
\end{figure}

This grouped bar chart presents a comparative analysis of total module swapped across multiple algorithms within the MT2TE and Other Program Traffic experimental configuration. The x-axis represents the number of nodes in the network, while the y-axis quantifies the total module swapped metric. The legend identifies 4 distinct algorithms: Genetic Algorithm, EVRPBSS, Ant Colony, Clarke and Wright algorithm. This visualization enables direct comparison of algorithmic performance under identical network conditions.

Quantitative analysis reveals significant performance disparities among the evaluated algorithms. Clarke and Wright algorithm demonstrates superior performance with a mean total module swapped of 697.95, while Genetic Algorithm exhibits the highest values averaging 1420.64. This represents an improvement of approximately 50.9\% when comparing the best to worst performing algorithms. The relative ranking of algorithms remains largely consistent across different node configurations, suggesting robust performance characteristics.

These findings support the hypothesis that algorithmic choice significantly impacts system performance metrics. Future work should incorporate statistical significance testing and confidence intervals to strengthen these comparative conclusions. Additionally, examining the computational complexity trade-offs between algorithms would provide valuable context for practical deployment decisions.

\begin{figure}[!htbp]
\centering
\includegraphics[width=0.95\linewidth]{images/MT2TE_and_Other_Program_Traffic/(e)_Execution_Time.png}
\caption{(e) Execution Time}
\label{fig:MT2TE_and_Other_Program_Traffic_e__Execution_Time}
\end{figure}

This grouped bar chart presents a comparative analysis of execution time across multiple algorithms within the MT2TE and Other Program Traffic experimental configuration. The x-axis represents the number of nodes in the network, while the y-axis quantifies the execution time metric. The legend identifies 4 distinct algorithms: Genetic Algorithm, EVRPBSS, Ant Colony, Clarke and Wright algorithm. This visualization enables direct comparison of algorithmic performance under identical network conditions.

Quantitative analysis reveals significant performance disparities among the evaluated algorithms. Clarke and Wright algorithm demonstrates superior performance with a mean execution time of 697.95, while Genetic Algorithm exhibits the highest values averaging 1420.64. This represents an improvement of approximately 50.9\% when comparing the best to worst performing algorithms. The relative ranking of algorithms remains largely consistent across different node configurations, suggesting robust performance characteristics.

These findings support the hypothesis that algorithmic choice significantly impacts system performance metrics. Future work should incorporate statistical significance testing and confidence intervals to strengthen these comparative conclusions. Additionally, examining the computational complexity trade-offs between algorithms would provide valuable context for practical deployment decisions.

\begin{figure}[!htbp]
\centering
\includegraphics[width=0.95\linewidth]{images/MT2TE_and_Other_Program_Traffic/(f)_Execution_Time.png}
\caption{(f) Execution Time}
\label{fig:MT2TE_and_Other_Program_Traffic_f__Execution_Time}
\end{figure}

This grouped bar chart presents a comparative analysis of execution time across multiple algorithms within the MT2TE and Other Program Traffic experimental configuration. The x-axis represents the number of nodes in the network, while the y-axis quantifies the execution time metric. The legend identifies 4 distinct algorithms: Genetic Algorithm, EVRPBSS, Ant Colony, Clarke and Wright algorithm. This visualization enables direct comparison of algorithmic performance under identical network conditions.

Quantitative analysis reveals significant performance disparities among the evaluated algorithms. Clarke and Wright algorithm demonstrates superior performance with a mean execution time of 862.43, while Genetic Algorithm exhibits the highest values averaging 2037.20. This represents an improvement of approximately 57.7\% when comparing the best to worst performing algorithms. The relative ranking of algorithms remains largely consistent across different node configurations, suggesting robust performance characteristics.

These findings support the hypothesis that algorithmic choice significantly impacts system performance metrics. Future work should incorporate statistical significance testing and confidence intervals to strengthen these comparative conclusions. Additionally, examining the computational complexity trade-offs between algorithms would provide valuable context for practical deployment decisions.


\clearpage

\subsection{MT2TE Multi-Line Charts}

\begin{figure}[!htbp]
\centering
\includegraphics[width=0.95\linewidth]{images/MT2TE_Multi-Line_Charts/(a)_Travel_time_bar_chart.png}
\caption{(a) Travel time}
\label{fig:MT2TE_Multi_Line_Charts_a__Travel_time_bar_chart}
\end{figure}

This bar chart presents the relationship between 4 module and 5 module for the MT2TE Multi-Line Charts experimental scenario. The x-axis displays 4 module values ranging from 3287.54 to 14159.32, while the y-axis quantifies 5 module. The single-series visualization facilitates analysis of how the dependent variable responds to changes in the independent parameter setting.

Analysis of the plotted data reveals that 5 module ranges from 3287.54 (at 4 module = 3287.54) to 13381.98 (at 4 module = 14159.32), representing a span of 10094.44 units. The overall trend is increasing, with values rising from 3287.54 at the initial setting to 13381.98 at the final setting. 

These results indicate that 4 module configuration meaningfully impacts 5 module in this experimental context. The substantial variation observed (coefficient of variation exceeding 20\%) suggests that parameter tuning could yield significant performance improvements. Confidence in these findings is high given the direct correspondence between CSV data and plotted values. Future analysis should consider incorporating error bars representing variance across multiple experimental runs to strengthen statistical validity.

\begin{figure}[!htbp]
\centering
\includegraphics[width=0.95\linewidth]{images/MT2TE_Multi-Line_Charts/(b)_Energy_consumed_bar_chart.png}
\caption{(b) Energy consumed}
\label{fig:MT2TE_Multi_Line_Charts_b__Energy_consumed_bar_chart}
\end{figure}

This bar chart presents the relationship between 4 module and 5 module for the MT2TE Multi-Line Charts experimental scenario. The x-axis displays 4 module values ranging from 323.496 to 2154.006, while the y-axis quantifies 5 module. The single-series visualization facilitates analysis of how the dependent variable responds to changes in the independent parameter setting.

Analysis of the plotted data reveals that 5 module ranges from 323.50 (at 4 module = 323.496) to 2057.45 (at 4 module = 2154.006), representing a span of 1733.96 units. The overall trend is increasing, with values rising from 323.50 at the initial setting to 2057.45 at the final setting. 

These results indicate that 4 module configuration meaningfully impacts 5 module in this experimental context. The substantial variation observed (coefficient of variation exceeding 20\%) suggests that parameter tuning could yield significant performance improvements. Confidence in these findings is high given the direct correspondence between CSV data and plotted values. Future analysis should consider incorporating error bars representing variance across multiple experimental runs to strengthen statistical validity.

\begin{figure}[!htbp]
\centering
\includegraphics[width=0.95\linewidth]{images/MT2TE_Multi-Line_Charts/(c)_Distance_covered_bar_chart.png}
\caption{(c) Distance covered}
\label{fig:MT2TE_Multi_Line_Charts_c__Distance_covered_bar_chart}
\end{figure}

This bar chart presents the relationship between 4 module and 5 module for the MT2TE Multi-Line Charts experimental scenario. The x-axis displays 4 module values ranging from 2138.91 to 9234.69, while the y-axis quantifies 5 module. The single-series visualization facilitates analysis of how the dependent variable responds to changes in the independent parameter setting.

Analysis of the plotted data reveals that 5 module ranges from 2138.91 (at 4 module = 2138.91) to 8652.17 (at 4 module = 9234.69), representing a span of 6513.26 units. The overall trend is increasing, with values rising from 2138.91 at the initial setting to 8652.17 at the final setting. 

These results indicate that 4 module configuration meaningfully impacts 5 module in this experimental context. The substantial variation observed (coefficient of variation exceeding 20\%) suggests that parameter tuning could yield significant performance improvements. Confidence in these findings is high given the direct correspondence between CSV data and plotted values. Future analysis should consider incorporating error bars representing variance across multiple experimental runs to strengthen statistical validity.

\begin{figure}[!htbp]
\centering
\includegraphics[width=0.95\linewidth]{images/MT2TE_Multi-Line_Charts/(d)_Run_time_bar_chart.png}
\caption{(d) Run time}
\label{fig:MT2TE_Multi_Line_Charts_d__Run_time_bar_chart}
\end{figure}

This bar chart presents the relationship between 4 module and 5 module for the MT2TE Multi-Line Charts experimental scenario. The x-axis displays 4 module values ranging from 58.526 to 1545.515, while the y-axis quantifies 5 module. The single-series visualization facilitates analysis of how the dependent variable responds to changes in the independent parameter setting.

Analysis of the plotted data reveals that 5 module ranges from 29.02 (at 4 module = 58.526) to 1236.34 (at 4 module = 1545.515), representing a span of 1207.32 units. The overall trend is increasing, with values rising from 29.02 at the initial setting to 1236.34 at the final setting. 

These results indicate that 4 module configuration meaningfully impacts 5 module in this experimental context. The substantial variation observed (coefficient of variation exceeding 20\%) suggests that parameter tuning could yield significant performance improvements. Confidence in these findings is high given the direct correspondence between CSV data and plotted values. Future analysis should consider incorporating error bars representing variance across multiple experimental runs to strengthen statistical validity.

\begin{figure}[!htbp]
\centering
\includegraphics[width=0.95\linewidth]{images/MT2TE_Multi-Line_Charts/(e)_Module_swapped_bar_chart.png}
\caption{(e) Module swapped}
\label{fig:MT2TE_Multi_Line_Charts_e__Module_swapped_bar_chart}
\end{figure}

This bar chart presents the relationship between 4 module and 4 module for the MT2TE Multi-Line Charts experimental scenario. The x-axis displays 4 module values ranging from 15 to 106, while the y-axis quantifies 4 module. The single-series visualization facilitates analysis of how the dependent variable responds to changes in the independent parameter setting.

Analysis of the plotted data reveals that 4 module ranges from 15.00 (at 4 module = 15) to 106.00 (at 4 module = 106), representing a span of 91.00 units. The overall trend is increasing, with values rising from 15.00 at the initial setting to 106.00 at the final setting. 

These results indicate that 4 module configuration meaningfully impacts 4 module in this experimental context. The substantial variation observed (coefficient of variation exceeding 20\%) suggests that parameter tuning could yield significant performance improvements. Confidence in these findings is high given the direct correspondence between CSV data and plotted values. Future analysis should consider incorporating error bars representing variance across multiple experimental runs to strengthen statistical validity.


\clearpage

\subsection{MT2TE Node Change}

\begin{figure}[!htbp]
\centering
\includegraphics[width=0.95\linewidth]{images/MT2TE_Node_Change/(a)_Travel_time_bar_chart.png}
\caption{(a) Travel time}
\label{fig:MT2TE_Node_Change_a__Travel_time_bar_chart}
\end{figure}

This bar chart presents the relationship between number of nodes and total travel time for the MT2TE Node Change experimental scenario. The x-axis displays number of nodes values ranging from 500 to 2000, while the y-axis quantifies total travel time. The single-series visualization facilitates analysis of how the dependent variable responds to changes in the independent parameter setting.

Analysis of the plotted data reveals that total travel time ranges from 3228.14 (at number of nodes = 500) to 13338.35 (at number of nodes = 2000), representing a span of 10110.21 units. The overall trend is increasing, with values rising from 3228.14 at the initial setting to 13338.35 at the final setting. 

These results indicate that number of nodes configuration meaningfully impacts total travel time in this experimental context. The substantial variation observed (coefficient of variation exceeding 20\%) suggests that parameter tuning could yield significant performance improvements. Confidence in these findings is high given the direct correspondence between CSV data and plotted values. Future analysis should consider incorporating error bars representing variance across multiple experimental runs to strengthen statistical validity.

\begin{figure}[!htbp]
\centering
\includegraphics[width=0.95\linewidth]{images/MT2TE_Node_Change/(b)_Energy_consumed_bar_chart.png}
\caption{(b) Energy consumed}
\label{fig:MT2TE_Node_Change_b__Energy_consumed_bar_chart}
\end{figure}

This bar chart presents the relationship between number of nodes and total energy consumed for the MT2TE Node Change experimental scenario. The x-axis displays number of nodes values ranging from 500 to 2000, while the y-axis quantifies total energy consumed. The single-series visualization facilitates analysis of how the dependent variable responds to changes in the independent parameter setting.

Analysis of the plotted data reveals that total energy consumed ranges from 317.62 (at number of nodes = 500) to 2036.78 (at number of nodes = 2000), representing a span of 1719.17 units. The overall trend is increasing, with values rising from 317.62 at the initial setting to 2036.78 at the final setting. 

These results indicate that number of nodes configuration meaningfully impacts total energy consumed in this experimental context. The substantial variation observed (coefficient of variation exceeding 20\%) suggests that parameter tuning could yield significant performance improvements. Confidence in these findings is high given the direct correspondence between CSV data and plotted values. Future analysis should consider incorporating error bars representing variance across multiple experimental runs to strengthen statistical validity.

\begin{figure}[!htbp]
\centering
\includegraphics[width=0.95\linewidth]{images/MT2TE_Node_Change/(c)_Distance_covered_bar_chart.png}
\caption{(c) Distance covered}
\label{fig:MT2TE_Node_Change_c__Distance_covered_bar_chart}
\end{figure}

This bar chart presents the relationship between number of nodes and total distance covered for the MT2TE Node Change experimental scenario. The x-axis displays number of nodes values ranging from 500 to 2000, while the y-axis quantifies total distance covered. The single-series visualization facilitates analysis of how the dependent variable responds to changes in the independent parameter setting.

Analysis of the plotted data reveals that total distance covered ranges from 2105.32 (at number of nodes = 500) to 8672.23 (at number of nodes = 2000), representing a span of 6566.90 units. The overall trend is increasing, with values rising from 2105.32 at the initial setting to 8672.23 at the final setting. 

These results indicate that number of nodes configuration meaningfully impacts total distance covered in this experimental context. The substantial variation observed (coefficient of variation exceeding 20\%) suggests that parameter tuning could yield significant performance improvements. Confidence in these findings is high given the direct correspondence between CSV data and plotted values. Future analysis should consider incorporating error bars representing variance across multiple experimental runs to strengthen statistical validity.

\begin{figure}[!htbp]
\centering
\includegraphics[width=0.95\linewidth]{images/MT2TE_Node_Change/(d)_Run_time_bar_chart.png}
\caption{(d) Run time}
\label{fig:MT2TE_Node_Change_d__Run_time_bar_chart}
\end{figure}

This bar chart presents the relationship between number of nodes and run time for the MT2TE Node Change experimental scenario. The x-axis displays number of nodes values ranging from 500 to 2000, while the y-axis quantifies run time. The single-series visualization facilitates analysis of how the dependent variable responds to changes in the independent parameter setting.

Analysis of the plotted data reveals that run time ranges from 42.87 (at number of nodes = 500) to 1257.88 (at number of nodes = 2000), representing a span of 1215.01 units. The overall trend is increasing, with values rising from 42.87 at the initial setting to 1257.88 at the final setting. 

These results indicate that number of nodes configuration meaningfully impacts run time in this experimental context. The substantial variation observed (coefficient of variation exceeding 20\%) suggests that parameter tuning could yield significant performance improvements. Confidence in these findings is high given the direct correspondence between CSV data and plotted values. Future analysis should consider incorporating error bars representing variance across multiple experimental runs to strengthen statistical validity.

\begin{figure}[!htbp]
\centering
\includegraphics[width=0.95\linewidth]{images/MT2TE_Node_Change/(e)_Module_swapped_bar_chart.png}
\caption{(e) Module swapped}
\label{fig:MT2TE_Node_Change_e__Module_swapped_bar_chart}
\end{figure}

This bar chart presents the relationship between number of nodes and total module swapped for the MT2TE Node Change experimental scenario. The x-axis displays number of nodes values ranging from 500 to 2000, while the y-axis quantifies total module swapped. The single-series visualization facilitates analysis of how the dependent variable responds to changes in the independent parameter setting.

Analysis of the plotted data reveals that total module swapped ranges from 14.50 (at number of nodes = 500) to 98.75 (at number of nodes = 2000), representing a span of 84.25 units. The overall trend is increasing, with values rising from 14.50 at the initial setting to 98.75 at the final setting. 

These results indicate that number of nodes configuration meaningfully impacts total module swapped in this experimental context. The substantial variation observed (coefficient of variation exceeding 20\%) suggests that parameter tuning could yield significant performance improvements. Confidence in these findings is high given the direct correspondence between CSV data and plotted values. Future analysis should consider incorporating error bars representing variance across multiple experimental runs to strengthen statistical validity.


\clearpage

\subsection{Optimum and MT2TE Module Change}

\begin{figure}[!htbp]
\centering
\includegraphics[width=0.95\linewidth]{images/Optimum_and_MT2TE_Module_Change/(a)_Travel_Time.png}
\caption{(a) Travel Time}
\label{fig:Optimum_and_MT2TE_Module_Change_a__Travel_Time}
\end{figure}

This bar chart presents the relationship between module and optimum for the Optimum and MT2TE Module Change experimental scenario. The x-axis displays module values ranging from 3 to 6, while the y-axis quantifies optimum. The single-series visualization facilitates analysis of how the dependent variable responds to changes in the independent parameter setting.

Analysis of the plotted data reveals that optimum ranges from 111.20 (at module = 3) to 111.20 (at module = 3), representing a span of 0.00 units. The values remain relatively stable across the parameter range, with minimal net change between initial (111.20) and final (111.20) settings. 

These results indicate that module configuration meaningfully impacts optimum in this experimental context. The relatively modest variation suggests that this parameter has limited influence on the measured metric within the tested range. Confidence in these findings is high given the direct correspondence between CSV data and plotted values. Future analysis should consider incorporating error bars representing variance across multiple experimental runs to strengthen statistical validity.

\begin{figure}[!htbp]
\centering
\includegraphics[width=0.95\linewidth]{images/Optimum_and_MT2TE_Module_Change/(b)_Energy.png}
\caption{(b) Energy}
\label{fig:Optimum_and_MT2TE_Module_Change_b__Energy}
\end{figure}

This bar chart presents the relationship between module and optimum for the Optimum and MT2TE Module Change experimental scenario. The x-axis displays module values ranging from 3 to 6, while the y-axis quantifies optimum. The single-series visualization facilitates analysis of how the dependent variable responds to changes in the independent parameter setting.

Analysis of the plotted data reveals that optimum ranges from 9.01 (at module = 3) to 9.01 (at module = 3), representing a span of 0.00 units. The values remain relatively stable across the parameter range, with minimal net change between initial (9.01) and final (9.01) settings. 

These results indicate that module configuration meaningfully impacts optimum in this experimental context. The relatively modest variation suggests that this parameter has limited influence on the measured metric within the tested range. Confidence in these findings is high given the direct correspondence between CSV data and plotted values. Future analysis should consider incorporating error bars representing variance across multiple experimental runs to strengthen statistical validity.

\begin{figure}[!htbp]
\centering
\includegraphics[width=0.95\linewidth]{images/Optimum_and_MT2TE_Module_Change/(c)_Distance.png}
\caption{(c) Distance}
\label{fig:Optimum_and_MT2TE_Module_Change_c__Distance}
\end{figure}

This bar chart presents the relationship between module and optimum for the Optimum and MT2TE Module Change experimental scenario. The x-axis displays module values ranging from 3 to 6, while the y-axis quantifies optimum. The single-series visualization facilitates analysis of how the dependent variable responds to changes in the independent parameter setting.

Analysis of the plotted data reveals that optimum ranges from 74.38 (at module = 3) to 74.38 (at module = 3), representing a span of 0.00 units. The values remain relatively stable across the parameter range, with minimal net change between initial (74.38) and final (74.38) settings. 

These results indicate that module configuration meaningfully impacts optimum in this experimental context. The relatively modest variation suggests that this parameter has limited influence on the measured metric within the tested range. Confidence in these findings is high given the direct correspondence between CSV data and plotted values. Future analysis should consider incorporating error bars representing variance across multiple experimental runs to strengthen statistical validity.


\clearpage

\subsection{Optimum and MT2TE Node Change}

\begin{figure}[!htbp]
\centering
\includegraphics[width=0.95\linewidth]{images/Optimum_and_MT2TE_Node_Change/(a)_Total_Travel_Time_Optimum_and_MT2TE_Node_Change.png}
\caption{(a) Total Travel Time Optimum and MT2TE Node Change}
\label{fig:Optimum_and_MT2TE_Node_Change_a__Total_Travel_Time_Optimum_and_MT2TE_Node_Change}
\end{figure}

This bar chart presents the relationship between node and optimum for the Optimum and MT2TE Node Change experimental scenario. The x-axis displays node values ranging from 12 to 24, while the y-axis quantifies optimum. The single-series visualization facilitates analysis of how the dependent variable responds to changes in the independent parameter setting.

Analysis of the plotted data reveals that optimum ranges from 106.61 (at node = 12) to 153.26 (at node = 24), representing a span of 46.65 units. The overall trend is increasing, with values rising from 106.61 at the initial setting to 153.26 at the final setting. 

These results indicate that node configuration meaningfully impacts optimum in this experimental context. The substantial variation observed (coefficient of variation exceeding 20\%) suggests that parameter tuning could yield significant performance improvements. Confidence in these findings is high given the direct correspondence between CSV data and plotted values. Future analysis should consider incorporating error bars representing variance across multiple experimental runs to strengthen statistical validity.

\begin{figure}[!htbp]
\centering
\includegraphics[width=0.95\linewidth]{images/Optimum_and_MT2TE_Node_Change/(b)_Total_Energy_Consumed_Optimum_and_MT2TE_Node_Change.png}
\caption{(b) Total Energy Consumed Optimum and MT2TE Node Change}
\label{fig:Optimum_and_MT2TE_Node_Change_b__Total_Energy_Consumed_Optimum_and_MT2TE_Node_Change}
\end{figure}

This bar chart presents the relationship between node and optimum for the Optimum and MT2TE Node Change experimental scenario. The x-axis displays node values ranging from 12 to 24, while the y-axis quantifies optimum. The single-series visualization facilitates analysis of how the dependent variable responds to changes in the independent parameter setting.

Analysis of the plotted data reveals that optimum ranges from 8.72 (at node = 12) to 11.84 (at node = 24), representing a span of 3.12 units. The overall trend is increasing, with values rising from 8.72 at the initial setting to 11.84 at the final setting. 

These results indicate that node configuration meaningfully impacts optimum in this experimental context. The substantial variation observed (coefficient of variation exceeding 20\%) suggests that parameter tuning could yield significant performance improvements. Confidence in these findings is high given the direct correspondence between CSV data and plotted values. Future analysis should consider incorporating error bars representing variance across multiple experimental runs to strengthen statistical validity.

\begin{figure}[!htbp]
\centering
\includegraphics[width=0.95\linewidth]{images/Optimum_and_MT2TE_Node_Change/(c)_Total_Distance_Optimum_and_MT2TE_Node_Change.png}
\caption{(c) Total Distance Optimum and MT2TE Node Change}
\label{fig:Optimum_and_MT2TE_Node_Change_c__Total_Distance_Optimum_and_MT2TE_Node_Change}
\end{figure}

This bar chart presents the relationship between node and optimum for the Optimum and MT2TE Node Change experimental scenario. The x-axis displays node values ranging from 12 to 24, while the y-axis quantifies optimum. The single-series visualization facilitates analysis of how the dependent variable responds to changes in the independent parameter setting.

Analysis of the plotted data reveals that optimum ranges from 65.57 (at node = 12) to 98.75 (at node = 24), representing a span of 33.18 units. The overall trend is increasing, with values rising from 65.57 at the initial setting to 98.75 at the final setting. 

These results indicate that node configuration meaningfully impacts optimum in this experimental context. The substantial variation observed (coefficient of variation exceeding 20\%) suggests that parameter tuning could yield significant performance improvements. Confidence in these findings is high given the direct correspondence between CSV data and plotted values. Future analysis should consider incorporating error bars representing variance across multiple experimental runs to strengthen statistical validity.

\begin{figure}[!htbp]
\centering
\includegraphics[width=0.95\linewidth]{images/Optimum_and_MT2TE_Node_Change/(d)_Execution_Time_Optimum_and_MT2TE_Node_Change.png}
\caption{(d) Execution Time Optimum and MT2TE Node Change}
\label{fig:Optimum_and_MT2TE_Node_Change_d__Execution_Time_Optimum_and_MT2TE_Node_Change}
\end{figure}

This bar chart presents the relationship between node and optimum for the Optimum and MT2TE Node Change experimental scenario. The x-axis displays node values ranging from 12 to 28, while the y-axis quantifies optimum. The single-series visualization facilitates analysis of how the dependent variable responds to changes in the independent parameter setting.

Analysis of the plotted data reveals that optimum ranges from 12.00 (at node = 12) to 298800.00 (at node = 28), representing a span of 298788.00 units. The overall trend is increasing, with values rising from 12.00 at the initial setting to 298800.00 at the final setting. 

These results indicate that node configuration meaningfully impacts optimum in this experimental context. The substantial variation observed (coefficient of variation exceeding 20\%) suggests that parameter tuning could yield significant performance improvements. Confidence in these findings is high given the direct correspondence between CSV data and plotted values. Future analysis should consider incorporating error bars representing variance across multiple experimental runs to strengthen statistical validity.


\clearpage

\subsection{Optimum and MT2TE Swap Time Change}

\begin{figure}[!htbp]
\centering
\includegraphics[width=0.95\linewidth]{images/Optimum_and_MT2TE_Swap_Time_Change/(a)_Travel_Time.png}
\caption{(a) Travel Time}
\label{fig:Optimum_and_MT2TE_Swap_Time_Change_a__Travel_Time}
\end{figure}

This bar chart presents the relationship between swaptime and optimum for the Optimum and MT2TE Swap Time Change experimental scenario. The x-axis displays swaptime values ranging from 1 to 4, while the y-axis quantifies optimum. The single-series visualization facilitates analysis of how the dependent variable responds to changes in the independent parameter setting.

Analysis of the plotted data reveals that optimum ranges from 111.20 (at swaptime = 1) to 111.20 (at swaptime = 2), representing a span of 0.00 units. The overall trend is increasing, with values rising from 111.20 at the initial setting to 111.20 at the final setting. 

These results indicate that swaptime configuration meaningfully impacts optimum in this experimental context. The relatively modest variation suggests that this parameter has limited influence on the measured metric within the tested range. Confidence in these findings is high given the direct correspondence between CSV data and plotted values. Future analysis should consider incorporating error bars representing variance across multiple experimental runs to strengthen statistical validity.

\begin{figure}[!htbp]
\centering
\includegraphics[width=0.95\linewidth]{images/Optimum_and_MT2TE_Swap_Time_Change/(b)_Energy.png}
\caption{(b) Energy}
\label{fig:Optimum_and_MT2TE_Swap_Time_Change_b__Energy}
\end{figure}

This bar chart presents the relationship between swaptime and optimum for the Optimum and MT2TE Swap Time Change experimental scenario. The x-axis displays swaptime values ranging from 1 to 4, while the y-axis quantifies optimum. The single-series visualization facilitates analysis of how the dependent variable responds to changes in the independent parameter setting.

Analysis of the plotted data reveals that optimum ranges from 9.01 (at swaptime = 1) to 9.01 (at swaptime = 1), representing a span of 0.00 units. The values remain relatively stable across the parameter range, with minimal net change between initial (9.01) and final (9.01) settings. 

These results indicate that swaptime configuration meaningfully impacts optimum in this experimental context. The relatively modest variation suggests that this parameter has limited influence on the measured metric within the tested range. Confidence in these findings is high given the direct correspondence between CSV data and plotted values. Future analysis should consider incorporating error bars representing variance across multiple experimental runs to strengthen statistical validity.

\begin{figure}[!htbp]
\centering
\includegraphics[width=0.95\linewidth]{images/Optimum_and_MT2TE_Swap_Time_Change/(c)_Distance.png}
\caption{(c) Distance}
\label{fig:Optimum_and_MT2TE_Swap_Time_Change_c__Distance}
\end{figure}

This bar chart presents the relationship between swaptime and optimum for the Optimum and MT2TE Swap Time Change experimental scenario. The x-axis displays swaptime values ranging from 1 to 4, while the y-axis quantifies optimum. The single-series visualization facilitates analysis of how the dependent variable responds to changes in the independent parameter setting.

Analysis of the plotted data reveals that optimum ranges from 74.38 (at swaptime = 1) to 74.38 (at swaptime = 1), representing a span of 0.00 units. The values remain relatively stable across the parameter range, with minimal net change between initial (74.38) and final (74.38) settings. 

These results indicate that swaptime configuration meaningfully impacts optimum in this experimental context. The relatively modest variation suggests that this parameter has limited influence on the measured metric within the tested range. Confidence in these findings is high given the direct correspondence between CSV data and plotted values. Future analysis should consider incorporating error bars representing variance across multiple experimental runs to strengthen statistical validity.


\clearpage

\subsection{Optimum and MT2TE Threshold Change}

\begin{figure}[!htbp]
\centering
\includegraphics[width=0.95\linewidth]{images/Optimum_and_MT2TE_Threshold_Change/(a)_Travel_Time.png}
\caption{(a) Travel Time}
\label{fig:Optimum_and_MT2TE_Threshold_Change_a__Travel_Time}
\end{figure}

This bar chart presents the relationship between threshold and optimum for the Optimum and MT2TE Threshold Change experimental scenario. The x-axis displays threshold values ranging from 5 to 20, while the y-axis quantifies optimum. The single-series visualization facilitates analysis of how the dependent variable responds to changes in the independent parameter setting.

Analysis of the plotted data reveals that optimum ranges from 111.20 (at threshold = 5) to 111.20 (at threshold = 5), representing a span of 0.00 units. The values remain relatively stable across the parameter range, with minimal net change between initial (111.20) and final (111.20) settings. 

These results indicate that threshold configuration meaningfully impacts optimum in this experimental context. The relatively modest variation suggests that this parameter has limited influence on the measured metric within the tested range. Confidence in these findings is high given the direct correspondence between CSV data and plotted values. Future analysis should consider incorporating error bars representing variance across multiple experimental runs to strengthen statistical validity.

\begin{figure}[!htbp]
\centering
\includegraphics[width=0.95\linewidth]{images/Optimum_and_MT2TE_Threshold_Change/(b)_Energy.png}
\caption{(b) Energy}
\label{fig:Optimum_and_MT2TE_Threshold_Change_b__Energy}
\end{figure}

This bar chart presents the relationship between threshold and optimum for the Optimum and MT2TE Threshold Change experimental scenario. The x-axis displays threshold values ranging from 5 to 20, while the y-axis quantifies optimum. The single-series visualization facilitates analysis of how the dependent variable responds to changes in the independent parameter setting.

Analysis of the plotted data reveals that optimum ranges from 9.01 (at threshold = 5) to 9.01 (at threshold = 5), representing a span of 0.00 units. The values remain relatively stable across the parameter range, with minimal net change between initial (9.01) and final (9.01) settings. 

These results indicate that threshold configuration meaningfully impacts optimum in this experimental context. The relatively modest variation suggests that this parameter has limited influence on the measured metric within the tested range. Confidence in these findings is high given the direct correspondence between CSV data and plotted values. Future analysis should consider incorporating error bars representing variance across multiple experimental runs to strengthen statistical validity.

\begin{figure}[!htbp]
\centering
\includegraphics[width=0.95\linewidth]{images/Optimum_and_MT2TE_Threshold_Change/(c)_Distance.png}
\caption{(c) Distance}
\label{fig:Optimum_and_MT2TE_Threshold_Change_c__Distance}
\end{figure}

This bar chart presents the relationship between threshold and optimum for the Optimum and MT2TE Threshold Change experimental scenario. The x-axis displays threshold values ranging from 5 to 20, while the y-axis quantifies optimum. The single-series visualization facilitates analysis of how the dependent variable responds to changes in the independent parameter setting.

Analysis of the plotted data reveals that optimum ranges from 74.38 (at threshold = 5) to 74.38 (at threshold = 5), representing a span of 0.00 units. The values remain relatively stable across the parameter range, with minimal net change between initial (74.38) and final (74.38) settings. 

These results indicate that threshold configuration meaningfully impacts optimum in this experimental context. The relatively modest variation suggests that this parameter has limited influence on the measured metric within the tested range. Confidence in these findings is high given the direct correspondence between CSV data and plotted values. Future analysis should consider incorporating error bars representing variance across multiple experimental runs to strengthen statistical validity.


\clearpage

\subsection{Optimum and MT2TE Traffic Change}

\begin{figure}[!htbp]
\centering
\includegraphics[width=0.95\linewidth]{images/Optimum_and_MT2TE_Traffic_Change/(a)_Travel_Time.png}
\caption{(a) Travel Time}
\label{fig:Optimum_and_MT2TE_Traffic_Change_a__Travel_Time}
\end{figure}

This bar chart presents the relationship between trafficlevel and optimum for the Optimum and MT2TE Traffic Change experimental scenario. The x-axis displays trafficlevel values ranging from Low to High, while the y-axis quantifies optimum. The single-series visualization facilitates analysis of how the dependent variable responds to changes in the independent parameter setting.

Analysis of the plotted data reveals that optimum ranges from 94.72 (at trafficlevel = Low) to 133.80 (at trafficlevel = High), representing a span of 39.08 units. The overall trend is increasing, with values rising from 94.72 at the initial setting to 133.80 at the final setting. 

These results indicate that trafficlevel configuration meaningfully impacts optimum in this experimental context. The substantial variation observed (coefficient of variation exceeding 20\%) suggests that parameter tuning could yield significant performance improvements. Confidence in these findings is high given the direct correspondence between CSV data and plotted values. Future analysis should consider incorporating error bars representing variance across multiple experimental runs to strengthen statistical validity.

\begin{figure}[!htbp]
\centering
\includegraphics[width=0.95\linewidth]{images/Optimum_and_MT2TE_Traffic_Change/(b)_Energy.png}
\caption{(b) Energy}
\label{fig:Optimum_and_MT2TE_Traffic_Change_b__Energy}
\end{figure}

This bar chart presents the relationship between trafficlevel and optimum for the Optimum and MT2TE Traffic Change experimental scenario. The x-axis displays trafficlevel values ranging from Low to High, while the y-axis quantifies optimum. The single-series visualization facilitates analysis of how the dependent variable responds to changes in the independent parameter setting.

Analysis of the plotted data reveals that optimum ranges from 7.68 (at trafficlevel = Low) to 11.09 (at trafficlevel = High), representing a span of 3.41 units. The overall trend is increasing, with values rising from 7.68 at the initial setting to 11.09 at the final setting. 

These results indicate that trafficlevel configuration meaningfully impacts optimum in this experimental context. The substantial variation observed (coefficient of variation exceeding 20\%) suggests that parameter tuning could yield significant performance improvements. Confidence in these findings is high given the direct correspondence between CSV data and plotted values. Future analysis should consider incorporating error bars representing variance across multiple experimental runs to strengthen statistical validity.

\begin{figure}[!htbp]
\centering
\includegraphics[width=0.95\linewidth]{images/Optimum_and_MT2TE_Traffic_Change/(c)_Distance.png}
\caption{(c) Distance}
\label{fig:Optimum_and_MT2TE_Traffic_Change_c__Distance}
\end{figure}

This bar chart presents the relationship between traffic and total distance covered for the Optimum and MT2TE Traffic Change experimental scenario. The x-axis displays traffic values ranging from High to Low, while the y-axis quantifies total distance covered. The single-series visualization facilitates analysis of how the dependent variable responds to changes in the independent parameter setting.

Analysis of the plotted data reveals that total distance covered ranges from 8689.81 (at traffic = High) to 8994.35 (at traffic = Low), representing a span of 304.54 units. The overall trend is increasing, with values rising from 8689.81 at the initial setting to 8994.35 at the final setting. 

These results indicate that traffic configuration meaningfully impacts total distance covered in this experimental context. The relatively modest variation suggests that this parameter has limited influence on the measured metric within the tested range. Confidence in these findings is high given the direct correspondence between CSV data and plotted values. Future analysis should consider incorporating error bars representing variance across multiple experimental runs to strengthen statistical validity.


\clearpage

\subsection{Real Time Sumo Module Change}

\begin{figure}[!htbp]
\centering
\includegraphics[width=0.95\linewidth]{images/Real_Time_Sumo_Module_Change/(a)_Travel_Time.png}
\caption{(a) Travel Time}
\label{fig:Real_Time_Sumo_Module_Change_a__Travel_Time}
\end{figure}

This bar chart presents the relationship between modules and total travel time for the Real Time Sumo Module Change experimental scenario. The x-axis displays modules values ranging from 3 to 6, while the y-axis quantifies total travel time. The single-series visualization facilitates analysis of how the dependent variable responds to changes in the independent parameter setting.

Analysis of the plotted data reveals that total travel time ranges from 87.96 (at modules = 3) to 90.06 (at modules = 4), representing a span of 2.10 units. The overall trend is increasing, with values rising from 87.96 at the initial setting to 88.79 at the final setting. 

These results indicate that modules configuration meaningfully impacts total travel time in this experimental context. The relatively modest variation suggests that this parameter has limited influence on the measured metric within the tested range. Confidence in these findings is high given the direct correspondence between CSV data and plotted values. Future analysis should consider incorporating error bars representing variance across multiple experimental runs to strengthen statistical validity.

\begin{figure}[!htbp]
\centering
\includegraphics[width=0.95\linewidth]{images/Real_Time_Sumo_Module_Change/(b)_energy.png}
\caption{(b) energy}
\label{fig:Real_Time_Sumo_Module_Change_b__energy}
\end{figure}

This bar chart presents the relationship between modules and total energy consumed for the Real Time Sumo Module Change experimental scenario. The x-axis displays modules values ranging from 3 to 6, while the y-axis quantifies total energy consumed. The single-series visualization facilitates analysis of how the dependent variable responds to changes in the independent parameter setting.

Analysis of the plotted data reveals that total energy consumed ranges from 12.54 (at modules = 3) to 13.11 (at modules = 4), representing a span of 0.58 units. The overall trend is increasing, with values rising from 12.54 at the initial setting to 12.84 at the final setting. 

These results indicate that modules configuration meaningfully impacts total energy consumed in this experimental context. The relatively modest variation suggests that this parameter has limited influence on the measured metric within the tested range. Confidence in these findings is high given the direct correspondence between CSV data and plotted values. Future analysis should consider incorporating error bars representing variance across multiple experimental runs to strengthen statistical validity.

\begin{figure}[!htbp]
\centering
\includegraphics[width=0.95\linewidth]{images/Real_Time_Sumo_Module_Change/(c)_Distance.png}
\caption{(c) Distance}
\label{fig:Real_Time_Sumo_Module_Change_c__Distance}
\end{figure}

This bar chart presents the relationship between modules and total distance covered for the Real Time Sumo Module Change experimental scenario. The x-axis displays modules values ranging from 3 to 6, while the y-axis quantifies total distance covered. The single-series visualization facilitates analysis of how the dependent variable responds to changes in the independent parameter setting.

Analysis of the plotted data reveals that total distance covered ranges from 60.75 (at modules = 3) to 62.76 (at modules = 4), representing a span of 2.01 units. The overall trend is increasing, with values rising from 60.75 at the initial setting to 62.00 at the final setting. 

These results indicate that modules configuration meaningfully impacts total distance covered in this experimental context. The relatively modest variation suggests that this parameter has limited influence on the measured metric within the tested range. Confidence in these findings is high given the direct correspondence between CSV data and plotted values. Future analysis should consider incorporating error bars representing variance across multiple experimental runs to strengthen statistical validity.

\begin{figure}[!htbp]
\centering
\includegraphics[width=0.95\linewidth]{images/Real_Time_Sumo_Module_Change/(d)_Run_time.png}
\caption{(d) Run time}
\label{fig:Real_Time_Sumo_Module_Change_d__Run_time}
\end{figure}

This bar chart presents the relationship between modules and total travel time for the Real Time Sumo Module Change experimental scenario. The x-axis displays modules values ranging from 3 to 6, while the y-axis quantifies total travel time. The single-series visualization facilitates analysis of how the dependent variable responds to changes in the independent parameter setting.

Analysis of the plotted data reveals that total travel time ranges from 87.96 (at modules = 3) to 90.06 (at modules = 4), representing a span of 2.10 units. The overall trend is increasing, with values rising from 87.96 at the initial setting to 88.79 at the final setting. 

These results indicate that modules configuration meaningfully impacts total travel time in this experimental context. The relatively modest variation suggests that this parameter has limited influence on the measured metric within the tested range. Confidence in these findings is high given the direct correspondence between CSV data and plotted values. Future analysis should consider incorporating error bars representing variance across multiple experimental runs to strengthen statistical validity.


\clearpage

\subsection{Real Time Sumo Swap Time}

\begin{figure}[!htbp]
\centering
\includegraphics[width=0.95\linewidth]{images/Real_Time_Sumo_Swap_Time/(a)_Travel_Time.png}
\caption{(a) Travel Time}
\label{fig:Real_Time_Sumo_Swap_Time_a__Travel_Time}
\end{figure}

This bar chart presents the relationship between swap time (min) and total travel time for the Real Time Sumo Swap Time experimental scenario. The x-axis displays swap time (min) values ranging from 1 to 4, while the y-axis quantifies total travel time. The single-series visualization facilitates analysis of how the dependent variable responds to changes in the independent parameter setting.

Analysis of the plotted data reveals that total travel time ranges from 89.22 (at swap time (min) = 2) to 92.84 (at swap time (min) = 4), representing a span of 3.62 units. The overall trend is increasing, with values rising from 89.94 at the initial setting to 92.84 at the final setting. Notably, the minimum value occurs at an intermediate swap time (min) setting (2), suggesting non-monotonic behavior that warrants further investigation.

These results indicate that swap time (min) configuration meaningfully impacts total travel time in this experimental context. The relatively modest variation suggests that this parameter has limited influence on the measured metric within the tested range. Confidence in these findings is high given the direct correspondence between CSV data and plotted values. Future analysis should consider incorporating error bars representing variance across multiple experimental runs to strengthen statistical validity.

\begin{figure}[!htbp]
\centering
\includegraphics[width=0.95\linewidth]{images/Real_Time_Sumo_Swap_Time/(b)_energy.png}
\caption{(b) energy}
\label{fig:Real_Time_Sumo_Swap_Time_b__energy}
\end{figure}

This bar chart presents the relationship between swap time (min) and total energy consumed for the Real Time Sumo Swap Time experimental scenario. The x-axis displays swap time (min) values ranging from 1 to 4, while the y-axis quantifies total energy consumed. The single-series visualization facilitates analysis of how the dependent variable responds to changes in the independent parameter setting.

Analysis of the plotted data reveals that total energy consumed ranges from 12.57 (at swap time (min) = 3) to 13.18 (at swap time (min) = 4), representing a span of 0.60 units. The overall trend is increasing, with values rising from 12.92 at the initial setting to 13.18 at the final setting. Notably, the minimum value occurs at an intermediate swap time (min) setting (3), suggesting non-monotonic behavior that warrants further investigation.

These results indicate that swap time (min) configuration meaningfully impacts total energy consumed in this experimental context. The relatively modest variation suggests that this parameter has limited influence on the measured metric within the tested range. Confidence in these findings is high given the direct correspondence between CSV data and plotted values. Future analysis should consider incorporating error bars representing variance across multiple experimental runs to strengthen statistical validity.

\begin{figure}[!htbp]
\centering
\includegraphics[width=0.95\linewidth]{images/Real_Time_Sumo_Swap_Time/(c)_Distance.png}
\caption{(c) Distance}
\label{fig:Real_Time_Sumo_Swap_Time_c__Distance}
\end{figure}

This bar chart presents the relationship between swap time (min) and total distance covered for the Real Time Sumo Swap Time experimental scenario. The x-axis displays swap time (min) values ranging from 1 to 4, while the y-axis quantifies total distance covered. The single-series visualization facilitates analysis of how the dependent variable responds to changes in the independent parameter setting.

Analysis of the plotted data reveals that total distance covered ranges from 61.23 (at swap time (min) = 3) to 64.06 (at swap time (min) = 4), representing a span of 2.83 units. The overall trend is increasing, with values rising from 62.42 at the initial setting to 64.06 at the final setting. Notably, the minimum value occurs at an intermediate swap time (min) setting (3), suggesting non-monotonic behavior that warrants further investigation.

These results indicate that swap time (min) configuration meaningfully impacts total distance covered in this experimental context. The relatively modest variation suggests that this parameter has limited influence on the measured metric within the tested range. Confidence in these findings is high given the direct correspondence between CSV data and plotted values. Future analysis should consider incorporating error bars representing variance across multiple experimental runs to strengthen statistical validity.

\begin{figure}[!htbp]
\centering
\includegraphics[width=0.95\linewidth]{images/Real_Time_Sumo_Swap_Time/(d)_Runtime.png}
\caption{(d) Runtime}
\label{fig:Real_Time_Sumo_Swap_Time_d__Runtime}
\end{figure}

This bar chart presents the relationship between swap time (min) and run time for the Real Time Sumo Swap Time experimental scenario. The x-axis displays swap time (min) values ranging from 1 to 4, while the y-axis quantifies run time. The single-series visualization facilitates analysis of how the dependent variable responds to changes in the independent parameter setting.

Analysis of the plotted data reveals that run time ranges from 540.79 (at swap time (min) = 1) to 563.35 (at swap time (min) = 4), representing a span of 22.55 units. The overall trend is increasing, with values rising from 540.79 at the initial setting to 563.35 at the final setting. 

These results indicate that swap time (min) configuration meaningfully impacts run time in this experimental context. The relatively modest variation suggests that this parameter has limited influence on the measured metric within the tested range. Confidence in these findings is high given the direct correspondence between CSV data and plotted values. Future analysis should consider incorporating error bars representing variance across multiple experimental runs to strengthen statistical validity.


\clearpage

\subsection{Real Time Sumo Threshold}

\begin{figure}[!htbp]
\centering
\includegraphics[width=0.95\linewidth]{images/Real_Time_Sumo_Threshold/(a)_Travel_Time.png}
\caption{(a) Travel Time}
\label{fig:Real_Time_Sumo_Threshold_a__Travel_Time}
\end{figure}

This bar chart presents the relationship between threshold and total travel time for the Real Time Sumo Threshold experimental scenario. The x-axis displays threshold values ranging from 5 to 20, while the y-axis quantifies total travel time. The single-series visualization facilitates analysis of how the dependent variable responds to changes in the independent parameter setting.

Analysis of the plotted data reveals that total travel time ranges from 86.63 (at threshold = 5) to 92.25 (at threshold = 15), representing a span of 5.62 units. The overall trend is increasing, with values rising from 86.63 at the initial setting to 89.22 at the final setting. 

These results indicate that threshold configuration meaningfully impacts total travel time in this experimental context. The relatively modest variation suggests that this parameter has limited influence on the measured metric within the tested range. Confidence in these findings is high given the direct correspondence between CSV data and plotted values. Future analysis should consider incorporating error bars representing variance across multiple experimental runs to strengthen statistical validity.

\begin{figure}[!htbp]
\centering
\includegraphics[width=0.95\linewidth]{images/Real_Time_Sumo_Threshold/(b)_Energy.png}
\caption{(b) Energy}
\label{fig:Real_Time_Sumo_Threshold_b__Energy}
\end{figure}

This bar chart presents the relationship between threshold and total energy consumed for the Real Time Sumo Threshold experimental scenario. The x-axis displays threshold values ranging from 5 to 20, while the y-axis quantifies total energy consumed. The single-series visualization facilitates analysis of how the dependent variable responds to changes in the independent parameter setting.

Analysis of the plotted data reveals that total energy consumed ranges from 12.59 (at threshold = 5) to 13.17 (at threshold = 15), representing a span of 0.58 units. The overall trend is increasing, with values rising from 12.59 at the initial setting to 12.93 at the final setting. 

These results indicate that threshold configuration meaningfully impacts total energy consumed in this experimental context. The relatively modest variation suggests that this parameter has limited influence on the measured metric within the tested range. Confidence in these findings is high given the direct correspondence between CSV data and plotted values. Future analysis should consider incorporating error bars representing variance across multiple experimental runs to strengthen statistical validity.

\begin{figure}[!htbp]
\centering
\includegraphics[width=0.95\linewidth]{images/Real_Time_Sumo_Threshold/(c)_Distance.png}
\caption{(c) Distance}
\label{fig:Real_Time_Sumo_Threshold_c__Distance}
\end{figure}

This bar chart presents the relationship between threshold and total distance covered for the Real Time Sumo Threshold experimental scenario. The x-axis displays threshold values ranging from 5 to 20, while the y-axis quantifies total distance covered. The single-series visualization facilitates analysis of how the dependent variable responds to changes in the independent parameter setting.

Analysis of the plotted data reveals that total distance covered ranges from 60.06 (at threshold = 5) to 63.76 (at threshold = 15), representing a span of 3.70 units. The overall trend is increasing, with values rising from 60.06 at the initial setting to 61.70 at the final setting. 

These results indicate that threshold configuration meaningfully impacts total distance covered in this experimental context. The relatively modest variation suggests that this parameter has limited influence on the measured metric within the tested range. Confidence in these findings is high given the direct correspondence between CSV data and plotted values. Future analysis should consider incorporating error bars representing variance across multiple experimental runs to strengthen statistical validity.

\begin{figure}[!htbp]
\centering
\includegraphics[width=0.95\linewidth]{images/Real_Time_Sumo_Threshold/(d)_runtime.png}
\caption{(d) runtime}
\label{fig:Real_Time_Sumo_Threshold_d__runtime}
\end{figure}

This bar chart presents the relationship between threshold and run time for the Real Time Sumo Threshold experimental scenario. The x-axis displays threshold values ranging from 5 to 20, while the y-axis quantifies run time. The single-series visualization facilitates analysis of how the dependent variable responds to changes in the independent parameter setting.

Analysis of the plotted data reveals that run time ranges from 546.38 (at threshold = 10) to 568.65 (at threshold = 15), representing a span of 22.27 units. The overall trend is increasing, with values rising from 547.79 at the initial setting to 553.99 at the final setting. Notably, the minimum value occurs at an intermediate threshold setting (10), suggesting non-monotonic behavior that warrants further investigation.

These results indicate that threshold configuration meaningfully impacts run time in this experimental context. The relatively modest variation suggests that this parameter has limited influence on the measured metric within the tested range. Confidence in these findings is high given the direct correspondence between CSV data and plotted values. Future analysis should consider incorporating error bars representing variance across multiple experimental runs to strengthen statistical validity.


\clearpage

\subsection{Sumo Static Module Change}

\begin{figure}[!htbp]
\centering
\includegraphics[width=0.95\linewidth]{images/Sumo_Static_Module_Change/(a)_Travel_Time.png}
\caption{(a) Travel Time}
\label{fig:Sumo_Static_Module_Change_a__Travel_Time}
\end{figure}

This bar chart presents the relationship between modules and total travel time for the Sumo Static Module Change experimental scenario. The x-axis displays modules values ranging from 3 to 6, while the y-axis quantifies total travel time. The single-series visualization facilitates analysis of how the dependent variable responds to changes in the independent parameter setting.

Analysis of the plotted data reveals that total travel time ranges from 92.73 (at modules = 5) to 108.71 (at modules = 6), representing a span of 15.98 units. The overall trend is increasing, with values rising from 94.47 at the initial setting to 108.71 at the final setting. Notably, the minimum value occurs at an intermediate modules setting (5), suggesting non-monotonic behavior that warrants further investigation.

These results indicate that modules configuration meaningfully impacts total travel time in this experimental context. The relatively modest variation suggests that this parameter has limited influence on the measured metric within the tested range. Confidence in these findings is high given the direct correspondence between CSV data and plotted values. Future analysis should consider incorporating error bars representing variance across multiple experimental runs to strengthen statistical validity.

\begin{figure}[!htbp]
\centering
\includegraphics[width=0.95\linewidth]{images/Sumo_Static_Module_Change/(b)_Energy.png}
\caption{(b) Energy}
\label{fig:Sumo_Static_Module_Change_b__Energy}
\end{figure}

This bar chart presents the relationship between modules and total energy consumed for the Sumo Static Module Change experimental scenario. The x-axis displays modules values ranging from 3 to 6, while the y-axis quantifies total energy consumed. The single-series visualization facilitates analysis of how the dependent variable responds to changes in the independent parameter setting.

Analysis of the plotted data reveals that total energy consumed ranges from 9.05 (at modules = 5) to 11.61 (at modules = 6), representing a span of 2.55 units. The overall trend is increasing, with values rising from 9.29 at the initial setting to 11.61 at the final setting. Notably, the minimum value occurs at an intermediate modules setting (5), suggesting non-monotonic behavior that warrants further investigation.

These results indicate that modules configuration meaningfully impacts total energy consumed in this experimental context. The substantial variation observed (coefficient of variation exceeding 20\%) suggests that parameter tuning could yield significant performance improvements. Confidence in these findings is high given the direct correspondence between CSV data and plotted values. Future analysis should consider incorporating error bars representing variance across multiple experimental runs to strengthen statistical validity.

\begin{figure}[!htbp]
\centering
\includegraphics[width=0.95\linewidth]{images/Sumo_Static_Module_Change/(c)_Distance.png}
\caption{(c) Distance}
\label{fig:Sumo_Static_Module_Change_c__Distance}
\end{figure}

This bar chart presents the relationship between modules and total distance covered for the Sumo Static Module Change experimental scenario. The x-axis displays modules values ranging from 3 to 6, while the y-axis quantifies total distance covered. The single-series visualization facilitates analysis of how the dependent variable responds to changes in the independent parameter setting.

Analysis of the plotted data reveals that total distance covered ranges from 57.52 (at modules = 5) to 69.86 (at modules = 6), representing a span of 12.34 units. The overall trend is increasing, with values rising from 59.51 at the initial setting to 69.86 at the final setting. Notably, the minimum value occurs at an intermediate modules setting (5), suggesting non-monotonic behavior that warrants further investigation.

These results indicate that modules configuration meaningfully impacts total distance covered in this experimental context. The relatively modest variation suggests that this parameter has limited influence on the measured metric within the tested range. Confidence in these findings is high given the direct correspondence between CSV data and plotted values. Future analysis should consider incorporating error bars representing variance across multiple experimental runs to strengthen statistical validity.

\begin{figure}[!htbp]
\centering
\includegraphics[width=0.95\linewidth]{images/Sumo_Static_Module_Change/(d)_Module_Swapped.png}
\caption{(d) Module Swapped}
\label{fig:Sumo_Static_Module_Change_d__Module_Swapped}
\end{figure}

This bar chart presents the relationship between modules and modules for the Sumo Static Module Change experimental scenario. The x-axis displays modules values ranging from 3 to 6, while the y-axis quantifies modules. The single-series visualization facilitates analysis of how the dependent variable responds to changes in the independent parameter setting.

Analysis of the plotted data reveals that modules ranges from 3.00 (at modules = 3) to 6.00 (at modules = 6), representing a span of 3.00 units. The overall trend is increasing, with values rising from 3.00 at the initial setting to 6.00 at the final setting. 

These results indicate that modules configuration meaningfully impacts modules in this experimental context. The substantial variation observed (coefficient of variation exceeding 20\%) suggests that parameter tuning could yield significant performance improvements. Confidence in these findings is high given the direct correspondence between CSV data and plotted values. Future analysis should consider incorporating error bars representing variance across multiple experimental runs to strengthen statistical validity.

\begin{figure}[!htbp]
\centering
\includegraphics[width=0.95\linewidth]{images/Sumo_Static_Module_Change/(e)_Run_Time.png}
\caption{(e) Run Time}
\label{fig:Sumo_Static_Module_Change_e__Run_Time}
\end{figure}

This bar chart presents the relationship between modules and total travel time for the Sumo Static Module Change experimental scenario. The x-axis displays modules values ranging from 3 to 6, while the y-axis quantifies total travel time. The single-series visualization facilitates analysis of how the dependent variable responds to changes in the independent parameter setting.

Analysis of the plotted data reveals that total travel time ranges from 92.73 (at modules = 5) to 108.71 (at modules = 6), representing a span of 15.98 units. The overall trend is increasing, with values rising from 94.47 at the initial setting to 108.71 at the final setting. Notably, the minimum value occurs at an intermediate modules setting (5), suggesting non-monotonic behavior that warrants further investigation.

These results indicate that modules configuration meaningfully impacts total travel time in this experimental context. The relatively modest variation suggests that this parameter has limited influence on the measured metric within the tested range. Confidence in these findings is high given the direct correspondence between CSV data and plotted values. Future analysis should consider incorporating error bars representing variance across multiple experimental runs to strengthen statistical validity.


\clearpage

\subsection{Sumo Static Swap Time}

\begin{figure}[!htbp]
\centering
\includegraphics[width=0.95\linewidth]{images/Sumo_Static_Swap_Time/(a)_Travel_Time.png}
\caption{(a) Travel Time}
\label{fig:Sumo_Static_Swap_Time_a__Travel_Time}
\end{figure}

This bar chart presents the relationship between swap time (min) and total travel time for the Sumo Static Swap Time experimental scenario. The x-axis displays swap time (min) values ranging from 1 to 4, while the y-axis quantifies total travel time. The single-series visualization facilitates analysis of how the dependent variable responds to changes in the independent parameter setting.

Analysis of the plotted data reveals that total travel time ranges from 85.81 (at swap time (min) = 4) to 100.69 (at swap time (min) = 3), representing a span of 14.88 units. The overall trend is decreasing, with values declining from 100.19 at the initial setting to 85.81 at the final setting. 

These results indicate that swap time (min) configuration meaningfully impacts total travel time in this experimental context. The relatively modest variation suggests that this parameter has limited influence on the measured metric within the tested range. Confidence in these findings is high given the direct correspondence between CSV data and plotted values. Future analysis should consider incorporating error bars representing variance across multiple experimental runs to strengthen statistical validity.

\begin{figure}[!htbp]
\centering
\includegraphics[width=0.95\linewidth]{images/Sumo_Static_Swap_Time/(b)_Energy.png}
\caption{(b) Energy}
\label{fig:Sumo_Static_Swap_Time_b__Energy}
\end{figure}

This bar chart presents the relationship between swap time (min) and total energy consumed for the Sumo Static Swap Time experimental scenario. The x-axis displays swap time (min) values ranging from 1 to 4, while the y-axis quantifies total energy consumed. The single-series visualization facilitates analysis of how the dependent variable responds to changes in the independent parameter setting.

Analysis of the plotted data reveals that total energy consumed ranges from 9.05 (at swap time (min) = 2) to 11.51 (at swap time (min) = 3), representing a span of 2.46 units. The overall trend is decreasing, with values declining from 11.00 at the initial setting to 9.75 at the final setting. Notably, the minimum value occurs at an intermediate swap time (min) setting (2), suggesting non-monotonic behavior that warrants further investigation.

These results indicate that swap time (min) configuration meaningfully impacts total energy consumed in this experimental context. The substantial variation observed (coefficient of variation exceeding 20\%) suggests that parameter tuning could yield significant performance improvements. Confidence in these findings is high given the direct correspondence between CSV data and plotted values. Future analysis should consider incorporating error bars representing variance across multiple experimental runs to strengthen statistical validity.

\begin{figure}[!htbp]
\centering
\includegraphics[width=0.95\linewidth]{images/Sumo_Static_Swap_Time/(c)_Distance.png}
\caption{(c) Distance}
\label{fig:Sumo_Static_Swap_Time_c__Distance}
\end{figure}

This bar chart presents the relationship between swap time (min) and total distance covered for the Sumo Static Swap Time experimental scenario. The x-axis displays swap time (min) values ranging from 1 to 4, while the y-axis quantifies total distance covered. The single-series visualization facilitates analysis of how the dependent variable responds to changes in the independent parameter setting.

Analysis of the plotted data reveals that total distance covered ranges from 57.11 (at swap time (min) = 4) to 66.91 (at swap time (min) = 3), representing a span of 9.80 units. The overall trend is decreasing, with values declining from 64.69 at the initial setting to 57.11 at the final setting. 

These results indicate that swap time (min) configuration meaningfully impacts total distance covered in this experimental context. The relatively modest variation suggests that this parameter has limited influence on the measured metric within the tested range. Confidence in these findings is high given the direct correspondence between CSV data and plotted values. Future analysis should consider incorporating error bars representing variance across multiple experimental runs to strengthen statistical validity.

\begin{figure}[!htbp]
\centering
\includegraphics[width=0.95\linewidth]{images/Sumo_Static_Swap_Time/(d)_Module_Swapped.png}
\caption{(d) Module Swapped}
\label{fig:Sumo_Static_Swap_Time_d__Module_Swapped}
\end{figure}

This bar chart presents the relationship between swap time (min) and total module swapped for the Sumo Static Swap Time experimental scenario. The x-axis displays swap time (min) values ranging from 1 to 4, while the y-axis quantifies total module swapped. The single-series visualization facilitates analysis of how the dependent variable responds to changes in the independent parameter setting.

Analysis of the plotted data reveals that total module swapped ranges from 0.00 (at swap time (min) = 1) to 0.00 (at swap time (min) = 1), representing a span of 0.00 units. The values remain relatively stable across the parameter range, with minimal net change between initial (0.00) and final (0.00) settings. 

These results indicate that swap time (min) configuration meaningfully impacts total module swapped in this experimental context. The relatively modest variation suggests that this parameter has limited influence on the measured metric within the tested range. Confidence in these findings is high given the direct correspondence between CSV data and plotted values. Future analysis should consider incorporating error bars representing variance across multiple experimental runs to strengthen statistical validity.

\begin{figure}[!htbp]
\centering
\includegraphics[width=0.95\linewidth]{images/Sumo_Static_Swap_Time/(e)_Run_Time.png}
\caption{(e) Run Time}
\label{fig:Sumo_Static_Swap_Time_e__Run_Time}
\end{figure}

This bar chart presents the relationship between swap time (min) and total travel time for the Sumo Static Swap Time experimental scenario. The x-axis displays swap time (min) values ranging from 1 to 4, while the y-axis quantifies total travel time. The single-series visualization facilitates analysis of how the dependent variable responds to changes in the independent parameter setting.

Analysis of the plotted data reveals that total travel time ranges from 85.81 (at swap time (min) = 4) to 100.69 (at swap time (min) = 3), representing a span of 14.88 units. The overall trend is decreasing, with values declining from 100.19 at the initial setting to 85.81 at the final setting. 

These results indicate that swap time (min) configuration meaningfully impacts total travel time in this experimental context. The relatively modest variation suggests that this parameter has limited influence on the measured metric within the tested range. Confidence in these findings is high given the direct correspondence between CSV data and plotted values. Future analysis should consider incorporating error bars representing variance across multiple experimental runs to strengthen statistical validity.


\clearpage

\subsection{Sumo Static Threshold}

\begin{figure}[!htbp]
\centering
\includegraphics[width=0.95\linewidth]{images/Sumo_Static_Threshold/(a)_Travel_Time.png}
\caption{(a) Travel Time}
\label{fig:Sumo_Static_Threshold_a__Travel_Time}
\end{figure}

This bar chart presents the relationship between threshold and total travel time for the Sumo Static Threshold experimental scenario. The x-axis displays threshold values ranging from 5 to 20, while the y-axis quantifies total travel time. The single-series visualization facilitates analysis of how the dependent variable responds to changes in the independent parameter setting.

Analysis of the plotted data reveals that total travel time ranges from 86.54 (at threshold = 10) to 106.44 (at threshold = 5), representing a span of 19.90 units. The overall trend is decreasing, with values declining from 106.44 at the initial setting to 92.73 at the final setting. Notably, the minimum value occurs at an intermediate threshold setting (10), suggesting non-monotonic behavior that warrants further investigation.

These results indicate that threshold configuration meaningfully impacts total travel time in this experimental context. The substantial variation observed (coefficient of variation exceeding 20\%) suggests that parameter tuning could yield significant performance improvements. Confidence in these findings is high given the direct correspondence between CSV data and plotted values. Future analysis should consider incorporating error bars representing variance across multiple experimental runs to strengthen statistical validity.

\begin{figure}[!htbp]
\centering
\includegraphics[width=0.95\linewidth]{images/Sumo_Static_Threshold/(b)_Energy.png}
\caption{(b) Energy}
\label{fig:Sumo_Static_Threshold_b__Energy}
\end{figure}

This bar chart presents the relationship between threshold and total energy consumed for the Sumo Static Threshold experimental scenario. The x-axis displays threshold values ranging from 5 to 20, while the y-axis quantifies total energy consumed. The single-series visualization facilitates analysis of how the dependent variable responds to changes in the independent parameter setting.

Analysis of the plotted data reveals that total energy consumed ranges from 8.80 (at threshold = 10) to 11.34 (at threshold = 5), representing a span of 2.54 units. The overall trend is decreasing, with values declining from 11.34 at the initial setting to 9.05 at the final setting. Notably, the minimum value occurs at an intermediate threshold setting (10), suggesting non-monotonic behavior that warrants further investigation.

These results indicate that threshold configuration meaningfully impacts total energy consumed in this experimental context. The substantial variation observed (coefficient of variation exceeding 20\%) suggests that parameter tuning could yield significant performance improvements. Confidence in these findings is high given the direct correspondence between CSV data and plotted values. Future analysis should consider incorporating error bars representing variance across multiple experimental runs to strengthen statistical validity.

\begin{figure}[!htbp]
\centering
\includegraphics[width=0.95\linewidth]{images/Sumo_Static_Threshold/(c)_Distance.png}
\caption{(c) Distance}
\label{fig:Sumo_Static_Threshold_c__Distance}
\end{figure}

This bar chart presents the relationship between threshold and total distance covered for the Sumo Static Threshold experimental scenario. The x-axis displays threshold values ranging from 5 to 20, while the y-axis quantifies total distance covered. The single-series visualization facilitates analysis of how the dependent variable responds to changes in the independent parameter setting.

Analysis of the plotted data reveals that total distance covered ranges from 55.14 (at threshold = 10) to 69.18 (at threshold = 5), representing a span of 14.04 units. The overall trend is decreasing, with values declining from 69.18 at the initial setting to 57.52 at the final setting. Notably, the minimum value occurs at an intermediate threshold setting (10), suggesting non-monotonic behavior that warrants further investigation.

These results indicate that threshold configuration meaningfully impacts total distance covered in this experimental context. The substantial variation observed (coefficient of variation exceeding 20\%) suggests that parameter tuning could yield significant performance improvements. Confidence in these findings is high given the direct correspondence between CSV data and plotted values. Future analysis should consider incorporating error bars representing variance across multiple experimental runs to strengthen statistical validity.

\begin{figure}[!htbp]
\centering
\includegraphics[width=0.95\linewidth]{images/Sumo_Static_Threshold/(d)_Module_Swapped.png}
\caption{(d) Module Swapped}
\label{fig:Sumo_Static_Threshold_d__Module_Swapped}
\end{figure}

This bar chart presents the relationship between threshold and total module swapped for the Sumo Static Threshold experimental scenario. The x-axis displays threshold values ranging from 5 to 20, while the y-axis quantifies total module swapped. The single-series visualization facilitates analysis of how the dependent variable responds to changes in the independent parameter setting.

Analysis of the plotted data reveals that total module swapped ranges from 0.00 (at threshold = 5) to 0.00 (at threshold = 5), representing a span of 0.00 units. The values remain relatively stable across the parameter range, with minimal net change between initial (0.00) and final (0.00) settings. 

These results indicate that threshold configuration meaningfully impacts total module swapped in this experimental context. The relatively modest variation suggests that this parameter has limited influence on the measured metric within the tested range. Confidence in these findings is high given the direct correspondence between CSV data and plotted values. Future analysis should consider incorporating error bars representing variance across multiple experimental runs to strengthen statistical validity.

\begin{figure}[!htbp]
\centering
\includegraphics[width=0.95\linewidth]{images/Sumo_Static_Threshold/(e)_Run_Time.png}
\caption{(e) Run Time}
\label{fig:Sumo_Static_Threshold_e__Run_Time}
\end{figure}

This bar chart presents the relationship between threshold and total travel time for the Sumo Static Threshold experimental scenario. The x-axis displays threshold values ranging from 5 to 20, while the y-axis quantifies total travel time. The single-series visualization facilitates analysis of how the dependent variable responds to changes in the independent parameter setting.

Analysis of the plotted data reveals that total travel time ranges from 86.54 (at threshold = 10) to 106.44 (at threshold = 5), representing a span of 19.90 units. The overall trend is decreasing, with values declining from 106.44 at the initial setting to 92.73 at the final setting. Notably, the minimum value occurs at an intermediate threshold setting (10), suggesting non-monotonic behavior that warrants further investigation.

These results indicate that threshold configuration meaningfully impacts total travel time in this experimental context. The substantial variation observed (coefficient of variation exceeding 20\%) suggests that parameter tuning could yield significant performance improvements. Confidence in these findings is high given the direct correspondence between CSV data and plotted values. Future analysis should consider incorporating error bars representing variance across multiple experimental runs to strengthen statistical validity.


\clearpage

% Requires: \usepackage{graphicx}

\subsection{1000 nodes Module Change}

\begin{figure}[!htbp]
\centering
\includegraphics[width=0.95\linewidth]{images/1000_nodes_Module_Change/(a)_Travel_Time_bar_chart.png}
\caption{(a) Travel Time}
\label{fig:1000_nodes_Module_Change_a__Travel_Time_bar_chart}
\end{figure}

This bar chart presents the relationship between modules and total travel time for the 1000 nodes Module Change experimental scenario. The x-axis displays modules values ranging from 4 to 7, while the y-axis quantifies total travel time. The single-series visualization facilitates analysis of how the dependent variable responds to changes in the independent parameter setting.

Analysis of the plotted data reveals that total travel time ranges from 6779.83 (at modules = 4) to 6808.90 (at modules = 5), representing a span of 29.07 units. The overall trend is increasing, with values rising from 6779.83 at the initial setting to 6808.90 at the final setting. 

These results indicate that modules configuration meaningfully impacts total travel time in this experimental context. The relatively modest variation suggests that this parameter has limited influence on the measured metric within the tested range. Confidence in these findings is high given the direct correspondence between CSV data and plotted values. Future analysis should consider incorporating error bars representing variance across multiple experimental runs to strengthen statistical validity.

\begin{figure}[!htbp]
\centering
\includegraphics[width=0.95\linewidth]{images/1000_nodes_Module_Change/(b)_Energy_consumption_bar_chart.png}
\caption{(b) Energy consumption}
\label{fig:1000_nodes_Module_Change_b__Energy_consumption_bar_chart}
\end{figure}

This bar chart presents the relationship between modules and total energy consumed for the 1000 nodes Module Change experimental scenario. The x-axis displays modules values ranging from 4 to 7, while the y-axis quantifies total energy consumed. The single-series visualization facilitates analysis of how the dependent variable responds to changes in the independent parameter setting.

Analysis of the plotted data reveals that total energy consumed ranges from 793.93 (at modules = 4) to 795.16 (at modules = 5), representing a span of 1.22 units. The overall trend is increasing, with values rising from 793.93 at the initial setting to 795.16 at the final setting. 

These results indicate that modules configuration meaningfully impacts total energy consumed in this experimental context. The relatively modest variation suggests that this parameter has limited influence on the measured metric within the tested range. Confidence in these findings is high given the direct correspondence between CSV data and plotted values. Future analysis should consider incorporating error bars representing variance across multiple experimental runs to strengthen statistical validity.

\begin{figure}[!htbp]
\centering
\includegraphics[width=0.95\linewidth]{images/1000_nodes_Module_Change/(c)_Distance_covered_bar_chart.png}
\caption{(c) Distance covered}
\label{fig:1000_nodes_Module_Change_c__Distance_covered_bar_chart}
\end{figure}

This bar chart presents the relationship between modules and total distance covered for the 1000 nodes Module Change experimental scenario. The x-axis displays modules values ranging from 4 to 7, while the y-axis quantifies total distance covered. The single-series visualization facilitates analysis of how the dependent variable responds to changes in the independent parameter setting.

Analysis of the plotted data reveals that total distance covered ranges from 4422.50 (at modules = 4) to 4437.52 (at modules = 5), representing a span of 15.02 units. The overall trend is increasing, with values rising from 4422.50 at the initial setting to 4437.52 at the final setting. 

These results indicate that modules configuration meaningfully impacts total distance covered in this experimental context. The relatively modest variation suggests that this parameter has limited influence on the measured metric within the tested range. Confidence in these findings is high given the direct correspondence between CSV data and plotted values. Future analysis should consider incorporating error bars representing variance across multiple experimental runs to strengthen statistical validity.

\begin{figure}[!htbp]
\centering
\includegraphics[width=0.95\linewidth]{images/1000_nodes_Module_Change/(d)_Runtime_bar_chart.png}
\caption{(d) Runtime}
\label{fig:1000_nodes_Module_Change_d__Runtime_bar_chart}
\end{figure}

This bar chart presents the relationship between modules and run time for the 1000 nodes Module Change experimental scenario. The x-axis displays modules values ranging from 4 to 7, while the y-axis quantifies run time. The single-series visualization facilitates analysis of how the dependent variable responds to changes in the independent parameter setting.

Analysis of the plotted data reveals that run time ranges from 187.29 (at modules = 4) to 396.10 (at modules = 5), representing a span of 208.81 units. The overall trend is increasing, with values rising from 187.29 at the initial setting to 226.36 at the final setting. 

These results indicate that modules configuration meaningfully impacts run time in this experimental context. The substantial variation observed (coefficient of variation exceeding 20\%) suggests that parameter tuning could yield significant performance improvements. Confidence in these findings is high given the direct correspondence between CSV data and plotted values. Future analysis should consider incorporating error bars representing variance across multiple experimental runs to strengthen statistical validity.

\begin{figure}[!htbp]
\centering
\includegraphics[width=0.95\linewidth]{images/1000_nodes_Module_Change/(e)_Number_of_module_swapped_bar_chart.png}
\caption{(e) Number of module swapped}
\label{fig:1000_nodes_Module_Change_e__Number_of_module_swapped_bar_chart}
\end{figure}

This bar chart presents the relationship between modules and modules for the 1000 nodes Module Change experimental scenario. The x-axis displays modules values ranging from 4 to 7, while the y-axis quantifies modules. The single-series visualization facilitates analysis of how the dependent variable responds to changes in the independent parameter setting.

Analysis of the plotted data reveals that modules ranges from 4.00 (at modules = 4) to 7.00 (at modules = 7), representing a span of 3.00 units. The overall trend is increasing, with values rising from 4.00 at the initial setting to 7.00 at the final setting. 

These results indicate that modules configuration meaningfully impacts modules in this experimental context. The substantial variation observed (coefficient of variation exceeding 20\%) suggests that parameter tuning could yield significant performance improvements. Confidence in these findings is high given the direct correspondence between CSV data and plotted values. Future analysis should consider incorporating error bars representing variance across multiple experimental runs to strengthen statistical validity.


\clearpage

\subsection{1000 nodes Swapping Time}

\begin{figure}[!htbp]
\centering
\includegraphics[width=0.95\linewidth]{images/1000_nodes_Swapping_Time/(a)_Travel_Time_bar_chart.png}
\caption{(a) Travel Time}
\label{fig:1000_nodes_Swapping_Time_a__Travel_Time_bar_chart}
\end{figure}

This bar chart presents the relationship between swapping time and total travel time for the 1000 nodes Swapping Time experimental scenario. The x-axis displays swapping time values ranging from 1 to 4, while the y-axis quantifies total travel time. The single-series visualization facilitates analysis of how the dependent variable responds to changes in the independent parameter setting.

Analysis of the plotted data reveals that total travel time ranges from 6771.90 (at swapping time = 1) to 6882.78 (at swapping time = 4), representing a span of 110.88 units. The overall trend is increasing, with values rising from 6771.90 at the initial setting to 6882.78 at the final setting. 

These results indicate that swapping time configuration meaningfully impacts total travel time in this experimental context. The relatively modest variation suggests that this parameter has limited influence on the measured metric within the tested range. Confidence in these findings is high given the direct correspondence between CSV data and plotted values. Future analysis should consider incorporating error bars representing variance across multiple experimental runs to strengthen statistical validity.

\begin{figure}[!htbp]
\centering
\includegraphics[width=0.95\linewidth]{images/1000_nodes_Swapping_Time/(b)_Energy_consumption_bar_chart.png}
\caption{(b) Energy consumption}
\label{fig:1000_nodes_Swapping_Time_b__Energy_consumption_bar_chart}
\end{figure}

This bar chart presents the relationship between swapping time and total energy consumed for the 1000 nodes Swapping Time experimental scenario. The x-axis displays swapping time values ranging from 1 to 4, while the y-axis quantifies total energy consumed. The single-series visualization facilitates analysis of how the dependent variable responds to changes in the independent parameter setting.

Analysis of the plotted data reveals that total energy consumed ranges from 794.98 (at swapping time = 3) to 795.16 (at swapping time = 1), representing a span of 0.18 units. The overall trend is decreasing, with values declining from 795.16 at the initial setting to 794.98 at the final setting. Notably, the minimum value occurs at an intermediate swapping time setting (3), suggesting non-monotonic behavior that warrants further investigation.

These results indicate that swapping time configuration meaningfully impacts total energy consumed in this experimental context. The relatively modest variation suggests that this parameter has limited influence on the measured metric within the tested range. Confidence in these findings is high given the direct correspondence between CSV data and plotted values. Future analysis should consider incorporating error bars representing variance across multiple experimental runs to strengthen statistical validity.

\begin{figure}[!htbp]
\centering
\includegraphics[width=0.95\linewidth]{images/1000_nodes_Swapping_Time/(c)_Distance_covered_bar_chart.png}
\caption{(c) Distance covered}
\label{fig:1000_nodes_Swapping_Time_c__Distance_covered_bar_chart}
\end{figure}

This bar chart presents the relationship between swapping time and total distance covered for the 1000 nodes Swapping Time experimental scenario. The x-axis displays swapping time values ranging from 1 to 4, while the y-axis quantifies total distance covered. The single-series visualization facilitates analysis of how the dependent variable responds to changes in the independent parameter setting.

Analysis of the plotted data reveals that total distance covered ranges from 4436.55 (at swapping time = 3) to 4437.52 (at swapping time = 1), representing a span of 0.97 units. The overall trend is decreasing, with values declining from 4437.52 at the initial setting to 4436.55 at the final setting. Notably, the minimum value occurs at an intermediate swapping time setting (3), suggesting non-monotonic behavior that warrants further investigation.

These results indicate that swapping time configuration meaningfully impacts total distance covered in this experimental context. The relatively modest variation suggests that this parameter has limited influence on the measured metric within the tested range. Confidence in these findings is high given the direct correspondence between CSV data and plotted values. Future analysis should consider incorporating error bars representing variance across multiple experimental runs to strengthen statistical validity.

\begin{figure}[!htbp]
\centering
\includegraphics[width=0.95\linewidth]{images/1000_nodes_Swapping_Time/(d)_Runtime_bar_chart.png}
\caption{(d) Runtime}
\label{fig:1000_nodes_Swapping_Time_d__Runtime_bar_chart}
\end{figure}

This bar chart presents the relationship between swapping time and run time for the 1000 nodes Swapping Time experimental scenario. The x-axis displays swapping time values ranging from 1 to 4, while the y-axis quantifies run time. The single-series visualization facilitates analysis of how the dependent variable responds to changes in the independent parameter setting.

Analysis of the plotted data reveals that run time ranges from 190.96 (at swapping time = 3) to 399.38 (at swapping time = 4), representing a span of 208.42 units. The overall trend is increasing, with values rising from 363.28 at the initial setting to 399.38 at the final setting. Notably, the minimum value occurs at an intermediate swapping time setting (3), suggesting non-monotonic behavior that warrants further investigation.

These results indicate that swapping time configuration meaningfully impacts run time in this experimental context. The substantial variation observed (coefficient of variation exceeding 20\%) suggests that parameter tuning could yield significant performance improvements. Confidence in these findings is high given the direct correspondence between CSV data and plotted values. Future analysis should consider incorporating error bars representing variance across multiple experimental runs to strengthen statistical validity.

\begin{figure}[!htbp]
\centering
\includegraphics[width=0.95\linewidth]{images/1000_nodes_Swapping_Time/(e)_Number_of_module_swapped_bar_chart.png}
\caption{(e) Number of module swapped}
\label{fig:1000_nodes_Swapping_Time_e__Number_of_module_swapped_bar_chart}
\end{figure}

This bar chart presents the relationship between swapping time and total module swapped for the 1000 nodes Swapping Time experimental scenario. The x-axis displays swapping time values ranging from 1 to 4, while the y-axis quantifies total module swapped. The single-series visualization facilitates analysis of how the dependent variable responds to changes in the independent parameter setting.

Analysis of the plotted data reveals that total module swapped ranges from 37.00 (at swapping time = 1) to 37.00 (at swapping time = 1), representing a span of 0.00 units. The values remain relatively stable across the parameter range, with minimal net change between initial (37.00) and final (37.00) settings. 

These results indicate that swapping time configuration meaningfully impacts total module swapped in this experimental context. The relatively modest variation suggests that this parameter has limited influence on the measured metric within the tested range. Confidence in these findings is high given the direct correspondence between CSV data and plotted values. Future analysis should consider incorporating error bars representing variance across multiple experimental runs to strengthen statistical validity.


\clearpage

\subsection{1000 nodes Threshold}

\begin{figure}[!htbp]
\centering
\includegraphics[width=0.95\linewidth]{images/1000_nodes_Threshold/(a)_Travel_Time_bar_chart.png}
\caption{(a) Travel Time}
\label{fig:1000_nodes_Threshold_a__Travel_Time_bar_chart}
\end{figure}

This bar chart presents the relationship between threshold and total travel time for the 1000 nodes Threshold experimental scenario. The x-axis displays threshold values ranging from 5 to 20, while the y-axis quantifies total travel time. The single-series visualization facilitates analysis of how the dependent variable responds to changes in the independent parameter setting.

Analysis of the plotted data reveals that total travel time ranges from 6808.90 (at threshold = 5) to 6808.90 (at threshold = 5), representing a span of 0.00 units. The values remain relatively stable across the parameter range, with minimal net change between initial (6808.90) and final (6808.90) settings. 

These results indicate that threshold configuration meaningfully impacts total travel time in this experimental context. The relatively modest variation suggests that this parameter has limited influence on the measured metric within the tested range. Confidence in these findings is high given the direct correspondence between CSV data and plotted values. Future analysis should consider incorporating error bars representing variance across multiple experimental runs to strengthen statistical validity.

\begin{figure}[!htbp]
\centering
\includegraphics[width=0.95\linewidth]{images/1000_nodes_Threshold/(b)_Energy_consumption_bar_chart.png}
\caption{(b) Energy consumption}
\label{fig:1000_nodes_Threshold_b__Energy_consumption_bar_chart}
\end{figure}

This bar chart presents the relationship between threshold and total energy consumed for the 1000 nodes Threshold experimental scenario. The x-axis displays threshold values ranging from 5 to 20, while the y-axis quantifies total energy consumed. The single-series visualization facilitates analysis of how the dependent variable responds to changes in the independent parameter setting.

Analysis of the plotted data reveals that total energy consumed ranges from 795.16 (at threshold = 5) to 795.16 (at threshold = 5), representing a span of 0.00 units. The values remain relatively stable across the parameter range, with minimal net change between initial (795.16) and final (795.16) settings. 

These results indicate that threshold configuration meaningfully impacts total energy consumed in this experimental context. The relatively modest variation suggests that this parameter has limited influence on the measured metric within the tested range. Confidence in these findings is high given the direct correspondence between CSV data and plotted values. Future analysis should consider incorporating error bars representing variance across multiple experimental runs to strengthen statistical validity.

\begin{figure}[!htbp]
\centering
\includegraphics[width=0.95\linewidth]{images/1000_nodes_Threshold/(c)_Distance_covered_bar_chart.png}
\caption{(c) Distance covered}
\label{fig:1000_nodes_Threshold_c__Distance_covered_bar_chart}
\end{figure}

This bar chart presents the relationship between threshold and total distance covered for the 1000 nodes Threshold experimental scenario. The x-axis displays threshold values ranging from 5 to 20, while the y-axis quantifies total distance covered. The single-series visualization facilitates analysis of how the dependent variable responds to changes in the independent parameter setting.

Analysis of the plotted data reveals that total distance covered ranges from 4437.52 (at threshold = 5) to 4437.52 (at threshold = 5), representing a span of 0.00 units. The values remain relatively stable across the parameter range, with minimal net change between initial (4437.52) and final (4437.52) settings. 

These results indicate that threshold configuration meaningfully impacts total distance covered in this experimental context. The relatively modest variation suggests that this parameter has limited influence on the measured metric within the tested range. Confidence in these findings is high given the direct correspondence between CSV data and plotted values. Future analysis should consider incorporating error bars representing variance across multiple experimental runs to strengthen statistical validity.

\begin{figure}[!htbp]
\centering
\includegraphics[width=0.95\linewidth]{images/1000_nodes_Threshold/(d)_Runtime_bar_chart.png}
\caption{(d) Runtime}
\label{fig:1000_nodes_Threshold_d__Runtime_bar_chart}
\end{figure}

This bar chart presents the relationship between threshold and run time for the 1000 nodes Threshold experimental scenario. The x-axis displays threshold values ranging from 5 to 20, while the y-axis quantifies run time. The single-series visualization facilitates analysis of how the dependent variable responds to changes in the independent parameter setting.

Analysis of the plotted data reveals that run time ranges from 225.32 (at threshold = 15) to 396.10 (at threshold = 20), representing a span of 170.78 units. The overall trend is increasing, with values rising from 287.89 at the initial setting to 396.10 at the final setting. Notably, the minimum value occurs at an intermediate threshold setting (15), suggesting non-monotonic behavior that warrants further investigation.

These results indicate that threshold configuration meaningfully impacts run time in this experimental context. The substantial variation observed (coefficient of variation exceeding 20\%) suggests that parameter tuning could yield significant performance improvements. Confidence in these findings is high given the direct correspondence between CSV data and plotted values. Future analysis should consider incorporating error bars representing variance across multiple experimental runs to strengthen statistical validity.

\begin{figure}[!htbp]
\centering
\includegraphics[width=0.95\linewidth]{images/1000_nodes_Threshold/(e)_Number_of_module_swapped_bar_chart.png}
\caption{(e) Number of module swapped}
\label{fig:1000_nodes_Threshold_e__Number_of_module_swapped_bar_chart}
\end{figure}

This bar chart presents the relationship between threshold and total module swapped for the 1000 nodes Threshold experimental scenario. The x-axis displays threshold values ranging from 5 to 20, while the y-axis quantifies total module swapped. The single-series visualization facilitates analysis of how the dependent variable responds to changes in the independent parameter setting.

Analysis of the plotted data reveals that total module swapped ranges from 37.00 (at threshold = 5) to 37.00 (at threshold = 5), representing a span of 0.00 units. The values remain relatively stable across the parameter range, with minimal net change between initial (37.00) and final (37.00) settings. 

These results indicate that threshold configuration meaningfully impacts total module swapped in this experimental context. The relatively modest variation suggests that this parameter has limited influence on the measured metric within the tested range. Confidence in these findings is high given the direct correspondence between CSV data and plotted values. Future analysis should consider incorporating error bars representing variance across multiple experimental runs to strengthen statistical validity.


\clearpage

\subsection{1000 nodes Traffic}

\begin{figure}[!htbp]
\centering
\includegraphics[width=0.95\linewidth]{images/1000_nodes_Traffic/(a)_Travel_Time_bar_chart.png}
\caption{(a) Travel Time}
\label{fig:1000_nodes_Traffic_a__Travel_Time_bar_chart}
\end{figure}

This bar chart presents the relationship between traffic and total travel time for the 1000 nodes Traffic experimental scenario. The x-axis displays traffic values ranging from High to Low, while the y-axis quantifies total travel time. The single-series visualization facilitates analysis of how the dependent variable responds to changes in the independent parameter setting.

Analysis of the plotted data reveals that total travel time ranges from 6269.66 (at traffic = Low) to 8004.34 (at traffic = High), representing a span of 1734.68 units. The overall trend is decreasing, with values declining from 8004.34 at the initial setting to 6269.66 at the final setting. 

These results indicate that traffic configuration meaningfully impacts total travel time in this experimental context. The substantial variation observed (coefficient of variation exceeding 20\%) suggests that parameter tuning could yield significant performance improvements. Confidence in these findings is high given the direct correspondence between CSV data and plotted values. Future analysis should consider incorporating error bars representing variance across multiple experimental runs to strengthen statistical validity.

\begin{figure}[!htbp]
\centering
\includegraphics[width=0.95\linewidth]{images/1000_nodes_Traffic/(b)_Energy_consumption_bar_chart.png}
\caption{(b) Energy consumption}
\label{fig:1000_nodes_Traffic_b__Energy_consumption_bar_chart}
\end{figure}

This bar chart presents the relationship between traffic and total energy consumed for the 1000 nodes Traffic experimental scenario. The x-axis displays traffic values ranging from High to Low, while the y-axis quantifies total energy consumed. The single-series visualization facilitates analysis of how the dependent variable responds to changes in the independent parameter setting.

Analysis of the plotted data reveals that total energy consumed ranges from 757.04 (at traffic = High) to 883.10 (at traffic = Low), representing a span of 126.06 units. The overall trend is increasing, with values rising from 757.04 at the initial setting to 883.10 at the final setting. 

These results indicate that traffic configuration meaningfully impacts total energy consumed in this experimental context. The relatively modest variation suggests that this parameter has limited influence on the measured metric within the tested range. Confidence in these findings is high given the direct correspondence between CSV data and plotted values. Future analysis should consider incorporating error bars representing variance across multiple experimental runs to strengthen statistical validity.

\begin{figure}[!htbp]
\centering
\includegraphics[width=0.95\linewidth]{images/1000_nodes_Traffic/(c)_Distance_covered_bar_chart.png}
\caption{(c) Distance covered}
\label{fig:1000_nodes_Traffic_c__Distance_covered_bar_chart}
\end{figure}

This bar chart presents the relationship between traffic and total distance covered for the 1000 nodes Traffic experimental scenario. The x-axis displays traffic values ranging from High to Low, while the y-axis quantifies total distance covered. The single-series visualization facilitates analysis of how the dependent variable responds to changes in the independent parameter setting.

Analysis of the plotted data reveals that total distance covered ranges from 4374.17 (at traffic = High) to 4806.46 (at traffic = Low), representing a span of 432.29 units. The overall trend is increasing, with values rising from 4374.17 at the initial setting to 4806.46 at the final setting. 

These results indicate that traffic configuration meaningfully impacts total distance covered in this experimental context. The relatively modest variation suggests that this parameter has limited influence on the measured metric within the tested range. Confidence in these findings is high given the direct correspondence between CSV data and plotted values. Future analysis should consider incorporating error bars representing variance across multiple experimental runs to strengthen statistical validity.

\begin{figure}[!htbp]
\centering
\includegraphics[width=0.95\linewidth]{images/1000_nodes_Traffic/(d)_Runtime_bar_chart.png}
\caption{(d) Runtime}
\label{fig:1000_nodes_Traffic_d__Runtime_bar_chart}
\end{figure}

This bar chart presents the relationship between traffic and run time for the 1000 nodes Traffic experimental scenario. The x-axis displays traffic values ranging from High to Low, while the y-axis quantifies run time. The single-series visualization facilitates analysis of how the dependent variable responds to changes in the independent parameter setting.

Analysis of the plotted data reveals that run time ranges from 368.03 (at traffic = High) to 407.25 (at traffic = Mid), representing a span of 39.22 units. The overall trend is increasing, with values rising from 368.03 at the initial setting to 382.14 at the final setting. 

These results indicate that traffic configuration meaningfully impacts run time in this experimental context. The relatively modest variation suggests that this parameter has limited influence on the measured metric within the tested range. Confidence in these findings is high given the direct correspondence between CSV data and plotted values. Future analysis should consider incorporating error bars representing variance across multiple experimental runs to strengthen statistical validity.

\begin{figure}[!htbp]
\centering
\includegraphics[width=0.95\linewidth]{images/1000_nodes_Traffic/(e)_Number_of_module_swapped_bar_chart.png}
\caption{(e) Number of module swapped}
\label{fig:1000_nodes_Traffic_e__Number_of_module_swapped_bar_chart}
\end{figure}

This bar chart presents the relationship between traffic and total module swapped for the 1000 nodes Traffic experimental scenario. The x-axis displays traffic values ranging from High to Low, while the y-axis quantifies total module swapped. The single-series visualization facilitates analysis of how the dependent variable responds to changes in the independent parameter setting.

Analysis of the plotted data reveals that total module swapped ranges from 35.00 (at traffic = High) to 41.00 (at traffic = Low), representing a span of 6.00 units. The overall trend is increasing, with values rising from 35.00 at the initial setting to 41.00 at the final setting. 

These results indicate that traffic configuration meaningfully impacts total module swapped in this experimental context. The relatively modest variation suggests that this parameter has limited influence on the measured metric within the tested range. Confidence in these findings is high given the direct correspondence between CSV data and plotted values. Future analysis should consider incorporating error bars representing variance across multiple experimental runs to strengthen statistical validity.


\clearpage

\subsection{1500 nodes Module Change}

\begin{figure}[!htbp]
\centering
\includegraphics[width=0.95\linewidth]{images/1500_nodes_Module_Change/(a)_Travel_time_bar_chart.png}
\caption{(a) Travel time}
\label{fig:1500_nodes_Module_Change_a__Travel_time_bar_chart}
\end{figure}

This bar chart presents the relationship between modules and total travel time for the 1500 nodes Module Change experimental scenario. The x-axis displays modules values ranging from 4 to 7, while the y-axis quantifies total travel time. The single-series visualization facilitates analysis of how the dependent variable responds to changes in the independent parameter setting.

Analysis of the plotted data reveals that total travel time ranges from 10133.53 (at modules = 5) to 10275.22 (at modules = 4), representing a span of 141.69 units. The overall trend is decreasing, with values declining from 10275.22 at the initial setting to 10148.53 at the final setting. Notably, the minimum value occurs at an intermediate modules setting (5), suggesting non-monotonic behavior that warrants further investigation.

These results indicate that modules configuration meaningfully impacts total travel time in this experimental context. The relatively modest variation suggests that this parameter has limited influence on the measured metric within the tested range. Confidence in these findings is high given the direct correspondence between CSV data and plotted values. Future analysis should consider incorporating error bars representing variance across multiple experimental runs to strengthen statistical validity.

\begin{figure}[!htbp]
\centering
\includegraphics[width=0.95\linewidth]{images/1500_nodes_Module_Change/(b)_Energy_consumed_bar_chart.png}
\caption{(b) Energy consumed}
\label{fig:1500_nodes_Module_Change_b__Energy_consumed_bar_chart}
\end{figure}

This bar chart presents the relationship between modules and total energy consumed for the 1500 nodes Module Change experimental scenario. The x-axis displays modules values ranging from 4 to 7, while the y-axis quantifies total energy consumed. The single-series visualization facilitates analysis of how the dependent variable responds to changes in the independent parameter setting.

Analysis of the plotted data reveals that total energy consumed ranges from 1346.58 (at modules = 5) to 1351.12 (at modules = 4), representing a span of 4.55 units. The overall trend is decreasing, with values declining from 1351.12 at the initial setting to 1347.93 at the final setting. Notably, the minimum value occurs at an intermediate modules setting (5), suggesting non-monotonic behavior that warrants further investigation.

These results indicate that modules configuration meaningfully impacts total energy consumed in this experimental context. The relatively modest variation suggests that this parameter has limited influence on the measured metric within the tested range. Confidence in these findings is high given the direct correspondence between CSV data and plotted values. Future analysis should consider incorporating error bars representing variance across multiple experimental runs to strengthen statistical validity.

\begin{figure}[!htbp]
\centering
\includegraphics[width=0.95\linewidth]{images/1500_nodes_Module_Change/(c)_Distance_covered_bar_chart.png}
\caption{(c) Distance covered}
\label{fig:1500_nodes_Module_Change_c__Distance_covered_bar_chart}
\end{figure}

This bar chart presents the relationship between modules and total distance covered for the 1500 nodes Module Change experimental scenario. The x-axis displays modules values ranging from 4 to 7, while the y-axis quantifies total distance covered. The single-series visualization facilitates analysis of how the dependent variable responds to changes in the independent parameter setting.

Analysis of the plotted data reveals that total distance covered ranges from 6533.82 (at modules = 5) to 6628.04 (at modules = 4), representing a span of 94.22 units. The overall trend is decreasing, with values declining from 6628.04 at the initial setting to 6545.22 at the final setting. Notably, the minimum value occurs at an intermediate modules setting (5), suggesting non-monotonic behavior that warrants further investigation.

These results indicate that modules configuration meaningfully impacts total distance covered in this experimental context. The relatively modest variation suggests that this parameter has limited influence on the measured metric within the tested range. Confidence in these findings is high given the direct correspondence between CSV data and plotted values. Future analysis should consider incorporating error bars representing variance across multiple experimental runs to strengthen statistical validity.

\begin{figure}[!htbp]
\centering
\includegraphics[width=0.95\linewidth]{images/1500_nodes_Module_Change/(d)_Run_time_bar_chart.png}
\caption{(d) Run time}
\label{fig:1500_nodes_Module_Change_d__Run_time_bar_chart}
\end{figure}

This bar chart presents the relationship between modules and total travel time for the 1500 nodes Module Change experimental scenario. The x-axis displays modules values ranging from 4 to 7, while the y-axis quantifies total travel time. The single-series visualization facilitates analysis of how the dependent variable responds to changes in the independent parameter setting.

Analysis of the plotted data reveals that total travel time ranges from 10133.53 (at modules = 5) to 10275.22 (at modules = 4), representing a span of 141.69 units. The overall trend is decreasing, with values declining from 10275.22 at the initial setting to 10148.53 at the final setting. Notably, the minimum value occurs at an intermediate modules setting (5), suggesting non-monotonic behavior that warrants further investigation.

These results indicate that modules configuration meaningfully impacts total travel time in this experimental context. The relatively modest variation suggests that this parameter has limited influence on the measured metric within the tested range. Confidence in these findings is high given the direct correspondence between CSV data and plotted values. Future analysis should consider incorporating error bars representing variance across multiple experimental runs to strengthen statistical validity.

\begin{figure}[!htbp]
\centering
\includegraphics[width=0.95\linewidth]{images/1500_nodes_Module_Change/(e)_Module_swapped_bar_chart.png}
\caption{(e) Module swapped}
\label{fig:1500_nodes_Module_Change_e__Module_swapped_bar_chart}
\end{figure}

This bar chart presents the relationship between modules and modules for the 1500 nodes Module Change experimental scenario. The x-axis displays modules values ranging from 4 to 7, while the y-axis quantifies modules. The single-series visualization facilitates analysis of how the dependent variable responds to changes in the independent parameter setting.

Analysis of the plotted data reveals that modules ranges from 4.00 (at modules = 4) to 7.00 (at modules = 7), representing a span of 3.00 units. The overall trend is increasing, with values rising from 4.00 at the initial setting to 7.00 at the final setting. 

These results indicate that modules configuration meaningfully impacts modules in this experimental context. The substantial variation observed (coefficient of variation exceeding 20\%) suggests that parameter tuning could yield significant performance improvements. Confidence in these findings is high given the direct correspondence between CSV data and plotted values. Future analysis should consider incorporating error bars representing variance across multiple experimental runs to strengthen statistical validity.


\clearpage

\subsection{1500 nodes Swapping Time}

\begin{figure}[!htbp]
\centering
\includegraphics[width=0.95\linewidth]{images/1500_nodes_Swapping_Time/(a)_Travel_time_bar_chart.png}
\caption{(a) Travel time}
\label{fig:1500_nodes_Swapping_Time_a__Travel_time_bar_chart}
\end{figure}

This bar chart presents the relationship between swapping time and total travel time for the 1500 nodes Swapping Time experimental scenario. The x-axis displays swapping time values ranging from 1 to 4, while the y-axis quantifies total travel time. The single-series visualization facilitates analysis of how the dependent variable responds to changes in the independent parameter setting.

Analysis of the plotted data reveals that total travel time ranges from 10089.02 (at swapping time = 1) to 10262.19 (at swapping time = 4), representing a span of 173.17 units. The overall trend is increasing, with values rising from 10089.02 at the initial setting to 10262.19 at the final setting. 

These results indicate that swapping time configuration meaningfully impacts total travel time in this experimental context. The relatively modest variation suggests that this parameter has limited influence on the measured metric within the tested range. Confidence in these findings is high given the direct correspondence between CSV data and plotted values. Future analysis should consider incorporating error bars representing variance across multiple experimental runs to strengthen statistical validity.

\begin{figure}[!htbp]
\centering
\includegraphics[width=0.95\linewidth]{images/1500_nodes_Swapping_Time/(b)_Energy_consumed_bar_chart.png}
\caption{(b) Energy consumed}
\label{fig:1500_nodes_Swapping_Time_b__Energy_consumed_bar_chart}
\end{figure}

This bar chart presents the relationship between swapping time and total energy consumed for the 1500 nodes Swapping Time experimental scenario. The x-axis displays swapping time values ranging from 1 to 4, while the y-axis quantifies total energy consumed. The single-series visualization facilitates analysis of how the dependent variable responds to changes in the independent parameter setting.

Analysis of the plotted data reveals that total energy consumed ranges from 1346.58 (at swapping time = 2) to 1349.76 (at swapping time = 1), representing a span of 3.18 units. The overall trend is decreasing, with values declining from 1349.76 at the initial setting to 1346.59 at the final setting. Notably, the minimum value occurs at an intermediate swapping time setting (2), suggesting non-monotonic behavior that warrants further investigation.

These results indicate that swapping time configuration meaningfully impacts total energy consumed in this experimental context. The relatively modest variation suggests that this parameter has limited influence on the measured metric within the tested range. Confidence in these findings is high given the direct correspondence between CSV data and plotted values. Future analysis should consider incorporating error bars representing variance across multiple experimental runs to strengthen statistical validity.

\begin{figure}[!htbp]
\centering
\includegraphics[width=0.95\linewidth]{images/1500_nodes_Swapping_Time/(c)_Distance_covered_bar_chart.png}
\caption{(c) Distance covered}
\label{fig:1500_nodes_Swapping_Time_c__Distance_covered_bar_chart}
\end{figure}

This bar chart presents the relationship between swapping time and total distance covered for the 1500 nodes Swapping Time experimental scenario. The x-axis displays swapping time values ranging from 1 to 4, while the y-axis quantifies total distance covered. The single-series visualization facilitates analysis of how the dependent variable responds to changes in the independent parameter setting.

Analysis of the plotted data reveals that total distance covered ranges from 6533.82 (at swapping time = 2) to 6552.92 (at swapping time = 1), representing a span of 19.10 units. The overall trend is decreasing, with values declining from 6552.92 at the initial setting to 6534.02 at the final setting. Notably, the minimum value occurs at an intermediate swapping time setting (2), suggesting non-monotonic behavior that warrants further investigation.

These results indicate that swapping time configuration meaningfully impacts total distance covered in this experimental context. The relatively modest variation suggests that this parameter has limited influence on the measured metric within the tested range. Confidence in these findings is high given the direct correspondence between CSV data and plotted values. Future analysis should consider incorporating error bars representing variance across multiple experimental runs to strengthen statistical validity.

\begin{figure}[!htbp]
\centering
\includegraphics[width=0.95\linewidth]{images/1500_nodes_Swapping_Time/(d)_Run_time_bar_chart.png}
\caption{(d) Run time}
\label{fig:1500_nodes_Swapping_Time_d__Run_time_bar_chart}
\end{figure}

This bar chart presents the relationship between swapping time and total travel time for the 1500 nodes Swapping Time experimental scenario. The x-axis displays swapping time values ranging from 1 to 4, while the y-axis quantifies total travel time. The single-series visualization facilitates analysis of how the dependent variable responds to changes in the independent parameter setting.

Analysis of the plotted data reveals that total travel time ranges from 10089.02 (at swapping time = 1) to 10262.19 (at swapping time = 4), representing a span of 173.17 units. The overall trend is increasing, with values rising from 10089.02 at the initial setting to 10262.19 at the final setting. 

These results indicate that swapping time configuration meaningfully impacts total travel time in this experimental context. The relatively modest variation suggests that this parameter has limited influence on the measured metric within the tested range. Confidence in these findings is high given the direct correspondence between CSV data and plotted values. Future analysis should consider incorporating error bars representing variance across multiple experimental runs to strengthen statistical validity.

\begin{figure}[!htbp]
\centering
\includegraphics[width=0.95\linewidth]{images/1500_nodes_Swapping_Time/(e)_Module_swapped_bar_chart.png}
\caption{(e) Module swapped}
\label{fig:1500_nodes_Swapping_Time_e__Module_swapped_bar_chart}
\end{figure}

This bar chart presents the relationship between swapping time and total module swapped for the 1500 nodes Swapping Time experimental scenario. The x-axis displays swapping time values ranging from 1 to 4, while the y-axis quantifies total module swapped. The single-series visualization facilitates analysis of how the dependent variable responds to changes in the independent parameter setting.

Analysis of the plotted data reveals that total module swapped ranges from 64.00 (at swapping time = 1) to 64.00 (at swapping time = 1), representing a span of 0.00 units. The values remain relatively stable across the parameter range, with minimal net change between initial (64.00) and final (64.00) settings. 

These results indicate that swapping time configuration meaningfully impacts total module swapped in this experimental context. The relatively modest variation suggests that this parameter has limited influence on the measured metric within the tested range. Confidence in these findings is high given the direct correspondence between CSV data and plotted values. Future analysis should consider incorporating error bars representing variance across multiple experimental runs to strengthen statistical validity.


\clearpage

\subsection{1500 nodes Threshold}

\begin{figure}[!htbp]
\centering
\includegraphics[width=0.95\linewidth]{images/1500_nodes_Threshold/(a)_Travel_time_bar_chart.png}
\caption{(a) Travel time}
\label{fig:1500_nodes_Threshold_a__Travel_time_bar_chart}
\end{figure}

This bar chart presents the relationship between threshold and total travel time for the 1500 nodes Threshold experimental scenario. The x-axis displays threshold values ranging from 5 to 20, while the y-axis quantifies total travel time. The single-series visualization facilitates analysis of how the dependent variable responds to changes in the independent parameter setting.

Analysis of the plotted data reveals that total travel time ranges from 10133.53 (at threshold = 10) to 10148.53 (at threshold = 5), representing a span of 15.00 units. The overall trend is decreasing, with values declining from 10148.53 at the initial setting to 10133.53 at the final setting. Notably, the minimum value occurs at an intermediate threshold setting (10), suggesting non-monotonic behavior that warrants further investigation.

These results indicate that threshold configuration meaningfully impacts total travel time in this experimental context. The relatively modest variation suggests that this parameter has limited influence on the measured metric within the tested range. Confidence in these findings is high given the direct correspondence between CSV data and plotted values. Future analysis should consider incorporating error bars representing variance across multiple experimental runs to strengthen statistical validity.

\begin{figure}[!htbp]
\centering
\includegraphics[width=0.95\linewidth]{images/1500_nodes_Threshold/(b)_Energy_consumed_bar_chart.png}
\caption{(b) Energy consumed}
\label{fig:1500_nodes_Threshold_b__Energy_consumed_bar_chart}
\end{figure}

This bar chart presents the relationship between threshold and total energy consumed for the 1500 nodes Threshold experimental scenario. The x-axis displays threshold values ranging from 5 to 20, while the y-axis quantifies total energy consumed. The single-series visualization facilitates analysis of how the dependent variable responds to changes in the independent parameter setting.

Analysis of the plotted data reveals that total energy consumed ranges from 1346.58 (at threshold = 10) to 1347.93 (at threshold = 5), representing a span of 1.36 units. The overall trend is decreasing, with values declining from 1347.93 at the initial setting to 1346.58 at the final setting. Notably, the minimum value occurs at an intermediate threshold setting (10), suggesting non-monotonic behavior that warrants further investigation.

These results indicate that threshold configuration meaningfully impacts total energy consumed in this experimental context. The relatively modest variation suggests that this parameter has limited influence on the measured metric within the tested range. Confidence in these findings is high given the direct correspondence between CSV data and plotted values. Future analysis should consider incorporating error bars representing variance across multiple experimental runs to strengthen statistical validity.

\begin{figure}[!htbp]
\centering
\includegraphics[width=0.95\linewidth]{images/1500_nodes_Threshold/(c)_Distance_covered_bar_chart.png}
\caption{(c) Distance covered}
\label{fig:1500_nodes_Threshold_c__Distance_covered_bar_chart}
\end{figure}

This bar chart presents the relationship between threshold and total distance covered for the 1500 nodes Threshold experimental scenario. The x-axis displays threshold values ranging from 5 to 20, while the y-axis quantifies total distance covered. The single-series visualization facilitates analysis of how the dependent variable responds to changes in the independent parameter setting.

Analysis of the plotted data reveals that total distance covered ranges from 6533.82 (at threshold = 10) to 6545.22 (at threshold = 5), representing a span of 11.40 units. The overall trend is decreasing, with values declining from 6545.22 at the initial setting to 6533.82 at the final setting. Notably, the minimum value occurs at an intermediate threshold setting (10), suggesting non-monotonic behavior that warrants further investigation.

These results indicate that threshold configuration meaningfully impacts total distance covered in this experimental context. The relatively modest variation suggests that this parameter has limited influence on the measured metric within the tested range. Confidence in these findings is high given the direct correspondence between CSV data and plotted values. Future analysis should consider incorporating error bars representing variance across multiple experimental runs to strengthen statistical validity.

\begin{figure}[!htbp]
\centering
\includegraphics[width=0.95\linewidth]{images/1500_nodes_Threshold/(d)_Run_time_bar_chart.png}
\caption{(d) Run time}
\label{fig:1500_nodes_Threshold_d__Run_time_bar_chart}
\end{figure}

This bar chart presents the relationship between threshold and total travel time for the 1500 nodes Threshold experimental scenario. The x-axis displays threshold values ranging from 5 to 20, while the y-axis quantifies total travel time. The single-series visualization facilitates analysis of how the dependent variable responds to changes in the independent parameter setting.

Analysis of the plotted data reveals that total travel time ranges from 10133.53 (at threshold = 10) to 10148.53 (at threshold = 5), representing a span of 15.00 units. The overall trend is decreasing, with values declining from 10148.53 at the initial setting to 10133.53 at the final setting. Notably, the minimum value occurs at an intermediate threshold setting (10), suggesting non-monotonic behavior that warrants further investigation.

These results indicate that threshold configuration meaningfully impacts total travel time in this experimental context. The relatively modest variation suggests that this parameter has limited influence on the measured metric within the tested range. Confidence in these findings is high given the direct correspondence between CSV data and plotted values. Future analysis should consider incorporating error bars representing variance across multiple experimental runs to strengthen statistical validity.

\begin{figure}[!htbp]
\centering
\includegraphics[width=0.95\linewidth]{images/1500_nodes_Threshold/(e)_Module_swapped_bar_chart.png}
\caption{(e) Module swapped}
\label{fig:1500_nodes_Threshold_e__Module_swapped_bar_chart}
\end{figure}

This bar chart presents the relationship between threshold and total module swapped for the 1500 nodes Threshold experimental scenario. The x-axis displays threshold values ranging from 5 to 20, while the y-axis quantifies total module swapped. The single-series visualization facilitates analysis of how the dependent variable responds to changes in the independent parameter setting.

Analysis of the plotted data reveals that total module swapped ranges from 63.00 (at threshold = 5) to 64.00 (at threshold = 10), representing a span of 1.00 units. The overall trend is increasing, with values rising from 63.00 at the initial setting to 64.00 at the final setting. 

These results indicate that threshold configuration meaningfully impacts total module swapped in this experimental context. The relatively modest variation suggests that this parameter has limited influence on the measured metric within the tested range. Confidence in these findings is high given the direct correspondence between CSV data and plotted values. Future analysis should consider incorporating error bars representing variance across multiple experimental runs to strengthen statistical validity.


\clearpage

\subsection{1500 nodes Traffic}

\begin{figure}[!htbp]
\centering
\includegraphics[width=0.95\linewidth]{images/1500_nodes_Traffic/(a)_Travel_time_bar_chart.png}
\caption{(a) Travel time}
\label{fig:1500_nodes_Traffic_a__Travel_time_bar_chart}
\end{figure}

This bar chart presents the relationship between traffic and total travel time for the 1500 nodes Traffic experimental scenario. The x-axis displays traffic values ranging from High to Low, while the y-axis quantifies total travel time. The single-series visualization facilitates analysis of how the dependent variable responds to changes in the independent parameter setting.

Analysis of the plotted data reveals that total travel time ranges from 8682.70 (at traffic = Low) to 11868.31 (at traffic = High), representing a span of 3185.61 units. The overall trend is decreasing, with values declining from 11868.31 at the initial setting to 8682.70 at the final setting. 

These results indicate that traffic configuration meaningfully impacts total travel time in this experimental context. The substantial variation observed (coefficient of variation exceeding 20\%) suggests that parameter tuning could yield significant performance improvements. Confidence in these findings is high given the direct correspondence between CSV data and plotted values. Future analysis should consider incorporating error bars representing variance across multiple experimental runs to strengthen statistical validity.

\begin{figure}[!htbp]
\centering
\includegraphics[width=0.95\linewidth]{images/1500_nodes_Traffic/(b)_Energy_consumed_bar_chart.png}
\caption{(b) Energy consumed}
\label{fig:1500_nodes_Traffic_b__Energy_consumed_bar_chart}
\end{figure}

This bar chart presents the relationship between traffic and total energy consumed for the 1500 nodes Traffic experimental scenario. The x-axis displays traffic values ranging from High to Low, while the y-axis quantifies total energy consumed. The single-series visualization facilitates analysis of how the dependent variable responds to changes in the independent parameter setting.

Analysis of the plotted data reveals that total energy consumed ranges from 1303.59 (at traffic = High) to 1443.98 (at traffic = Low), representing a span of 140.40 units. The overall trend is increasing, with values rising from 1303.59 at the initial setting to 1443.98 at the final setting. 

These results indicate that traffic configuration meaningfully impacts total energy consumed in this experimental context. The relatively modest variation suggests that this parameter has limited influence on the measured metric within the tested range. Confidence in these findings is high given the direct correspondence between CSV data and plotted values. Future analysis should consider incorporating error bars representing variance across multiple experimental runs to strengthen statistical validity.

\begin{figure}[!htbp]
\centering
\includegraphics[width=0.95\linewidth]{images/1500_nodes_Traffic/(c)_Distance_covered_bar_chart.png}
\caption{(c) Distance covered}
\label{fig:1500_nodes_Traffic_c__Distance_covered_bar_chart}
\end{figure}

This bar chart presents the relationship between traffic and total distance covered for the 1500 nodes Traffic experimental scenario. The x-axis displays traffic values ranging from High to Low, while the y-axis quantifies total distance covered. The single-series visualization facilitates analysis of how the dependent variable responds to changes in the independent parameter setting.

Analysis of the plotted data reveals that total distance covered ranges from 6463.22 (at traffic = Mid) to 6661.07 (at traffic = Low), representing a span of 197.85 units. The overall trend is increasing, with values rising from 6493.82 at the initial setting to 6661.07 at the final setting. Notably, the minimum value occurs at an intermediate traffic setting (Mid), suggesting non-monotonic behavior that warrants further investigation.

These results indicate that traffic configuration meaningfully impacts total distance covered in this experimental context. The relatively modest variation suggests that this parameter has limited influence on the measured metric within the tested range. Confidence in these findings is high given the direct correspondence between CSV data and plotted values. Future analysis should consider incorporating error bars representing variance across multiple experimental runs to strengthen statistical validity.

\begin{figure}[!htbp]
\centering
\includegraphics[width=0.95\linewidth]{images/1500_nodes_Traffic/(d)_Run_time_bar_chart.png}
\caption{(d) Run time}
\label{fig:1500_nodes_Traffic_d__Run_time_bar_chart}
\end{figure}

This bar chart presents the relationship between traffic and total travel time for the 1500 nodes Traffic experimental scenario. The x-axis displays traffic values ranging from High to Low, while the y-axis quantifies total travel time. The single-series visualization facilitates analysis of how the dependent variable responds to changes in the independent parameter setting.

Analysis of the plotted data reveals that total travel time ranges from 8682.70 (at traffic = Low) to 11868.31 (at traffic = High), representing a span of 3185.61 units. The overall trend is decreasing, with values declining from 11868.31 at the initial setting to 8682.70 at the final setting. 

These results indicate that traffic configuration meaningfully impacts total travel time in this experimental context. The substantial variation observed (coefficient of variation exceeding 20\%) suggests that parameter tuning could yield significant performance improvements. Confidence in these findings is high given the direct correspondence between CSV data and plotted values. Future analysis should consider incorporating error bars representing variance across multiple experimental runs to strengthen statistical validity.

\begin{figure}[!htbp]
\centering
\includegraphics[width=0.95\linewidth]{images/1500_nodes_Traffic/(e)_Module_swapped_bar_chart.png}
\caption{(e) Module swapped}
\label{fig:1500_nodes_Traffic_e__Module_swapped_bar_chart}
\end{figure}

This bar chart presents the relationship between traffic and total module swapped for the 1500 nodes Traffic experimental scenario. The x-axis displays traffic values ranging from High to Low, while the y-axis quantifies total module swapped. The single-series visualization facilitates analysis of how the dependent variable responds to changes in the independent parameter setting.

Analysis of the plotted data reveals that total module swapped ranges from 63.00 (at traffic = High) to 69.00 (at traffic = Low), representing a span of 6.00 units. The overall trend is increasing, with values rising from 63.00 at the initial setting to 69.00 at the final setting. 

These results indicate that traffic configuration meaningfully impacts total module swapped in this experimental context. The relatively modest variation suggests that this parameter has limited influence on the measured metric within the tested range. Confidence in these findings is high given the direct correspondence between CSV data and plotted values. Future analysis should consider incorporating error bars representing variance across multiple experimental runs to strengthen statistical validity.


\clearpage

\subsection{2000 nodes Module Change}

\begin{figure}[!htbp]
\centering
\includegraphics[width=0.95\linewidth]{images/2000_nodes_Module_Change/(a)_Travel_time_bar_chart.png}
\caption{(a) Travel time}
\label{fig:2000_nodes_Module_Change_a__Travel_time_bar_chart}
\end{figure}

This bar chart presents the relationship between modules and total travel time for the 2000 nodes Module Change experimental scenario. The x-axis displays modules values ranging from 4 to 7, while the y-axis quantifies total travel time. The single-series visualization facilitates analysis of how the dependent variable responds to changes in the independent parameter setting.

Analysis of the plotted data reveals that total travel time ranges from 13169.12 (at modules = 7) to 14159.32 (at modules = 4), representing a span of 990.20 units. The overall trend is decreasing, with values declining from 14159.32 at the initial setting to 13169.12 at the final setting. 

These results indicate that modules configuration meaningfully impacts total travel time in this experimental context. The relatively modest variation suggests that this parameter has limited influence on the measured metric within the tested range. Confidence in these findings is high given the direct correspondence between CSV data and plotted values. Future analysis should consider incorporating error bars representing variance across multiple experimental runs to strengthen statistical validity.

\begin{figure}[!htbp]
\centering
\includegraphics[width=0.95\linewidth]{images/2000_nodes_Module_Change/(b)_Energy_consumed_bar_chart.png}
\caption{(b) Energy consumed}
\label{fig:2000_nodes_Module_Change_b__Energy_consumed_bar_chart}
\end{figure}

This bar chart presents the relationship between modules and total energy consumed for the 2000 nodes Module Change experimental scenario. The x-axis displays modules values ranging from 4 to 7, while the y-axis quantifies total energy consumed. The single-series visualization facilitates analysis of how the dependent variable responds to changes in the independent parameter setting.

Analysis of the plotted data reveals that total energy consumed ranges from 2030.72 (at modules = 7) to 2154.01 (at modules = 4), representing a span of 123.28 units. The overall trend is decreasing, with values declining from 2154.01 at the initial setting to 2030.72 at the final setting. 

These results indicate that modules configuration meaningfully impacts total energy consumed in this experimental context. The relatively modest variation suggests that this parameter has limited influence on the measured metric within the tested range. Confidence in these findings is high given the direct correspondence between CSV data and plotted values. Future analysis should consider incorporating error bars representing variance across multiple experimental runs to strengthen statistical validity.

\begin{figure}[!htbp]
\centering
\includegraphics[width=0.95\linewidth]{images/2000_nodes_Module_Change/(c)_Distance_covered_bar_chart.png}
\caption{(c) Distance covered}
\label{fig:2000_nodes_Module_Change_c__Distance_covered_bar_chart}
\end{figure}

This bar chart presents the relationship between modules and total distance covered for the 2000 nodes Module Change experimental scenario. The x-axis displays modules values ranging from 4 to 7, while the y-axis quantifies total distance covered. The single-series visualization facilitates analysis of how the dependent variable responds to changes in the independent parameter setting.

Analysis of the plotted data reveals that total distance covered ranges from 8588.97 (at modules = 7) to 9234.69 (at modules = 4), representing a span of 645.72 units. The overall trend is decreasing, with values declining from 9234.69 at the initial setting to 8588.97 at the final setting. 

These results indicate that modules configuration meaningfully impacts total distance covered in this experimental context. The relatively modest variation suggests that this parameter has limited influence on the measured metric within the tested range. Confidence in these findings is high given the direct correspondence between CSV data and plotted values. Future analysis should consider incorporating error bars representing variance across multiple experimental runs to strengthen statistical validity.

\begin{figure}[!htbp]
\centering
\includegraphics[width=0.95\linewidth]{images/2000_nodes_Module_Change/(d)_Run_time_bar_chart.png}
\caption{(d) Run time}
\label{fig:2000_nodes_Module_Change_d__Run_time_bar_chart}
\end{figure}

This bar chart presents the relationship between modules and total travel time for the 2000 nodes Module Change experimental scenario. The x-axis displays modules values ranging from 4 to 7, while the y-axis quantifies total travel time. The single-series visualization facilitates analysis of how the dependent variable responds to changes in the independent parameter setting.

Analysis of the plotted data reveals that total travel time ranges from 13169.12 (at modules = 7) to 14159.32 (at modules = 4), representing a span of 990.20 units. The overall trend is decreasing, with values declining from 14159.32 at the initial setting to 13169.12 at the final setting. 

These results indicate that modules configuration meaningfully impacts total travel time in this experimental context. The relatively modest variation suggests that this parameter has limited influence on the measured metric within the tested range. Confidence in these findings is high given the direct correspondence between CSV data and plotted values. Future analysis should consider incorporating error bars representing variance across multiple experimental runs to strengthen statistical validity.

\begin{figure}[!htbp]
\centering
\includegraphics[width=0.95\linewidth]{images/2000_nodes_Module_Change/(e)_Module_swapped_bar_chart.png}
\caption{(e) Module swapped}
\label{fig:2000_nodes_Module_Change_e__Module_swapped_bar_chart}
\end{figure}

This bar chart presents the relationship between modules and modules for the 2000 nodes Module Change experimental scenario. The x-axis displays modules values ranging from 4 to 7, while the y-axis quantifies modules. The single-series visualization facilitates analysis of how the dependent variable responds to changes in the independent parameter setting.

Analysis of the plotted data reveals that modules ranges from 4.00 (at modules = 4) to 7.00 (at modules = 7), representing a span of 3.00 units. The overall trend is increasing, with values rising from 4.00 at the initial setting to 7.00 at the final setting. 

These results indicate that modules configuration meaningfully impacts modules in this experimental context. The substantial variation observed (coefficient of variation exceeding 20\%) suggests that parameter tuning could yield significant performance improvements. Confidence in these findings is high given the direct correspondence between CSV data and plotted values. Future analysis should consider incorporating error bars representing variance across multiple experimental runs to strengthen statistical validity.


\clearpage

\subsection{2000 nodes Swapping Time}

\begin{figure}[!htbp]
\centering
\includegraphics[width=0.95\linewidth]{images/2000_nodes_Swapping_Time/(a)_Travel_time_bar_chart.png}
\caption{(a) Travel time}
\label{fig:2000_nodes_Swapping_Time_a__Travel_time_bar_chart}
\end{figure}

This bar chart presents the relationship between swapping time and total travel time for the 2000 nodes Swapping Time experimental scenario. The x-axis displays swapping time values ranging from 1 to 4, while the y-axis quantifies total travel time. The single-series visualization facilitates analysis of how the dependent variable responds to changes in the independent parameter setting.

Analysis of the plotted data reveals that total travel time ranges from 13181.25 (at swapping time = 1) to 13540.28 (at swapping time = 4), representing a span of 359.03 units. The overall trend is increasing, with values rising from 13181.25 at the initial setting to 13540.28 at the final setting. 

These results indicate that swapping time configuration meaningfully impacts total travel time in this experimental context. The relatively modest variation suggests that this parameter has limited influence on the measured metric within the tested range. Confidence in these findings is high given the direct correspondence between CSV data and plotted values. Future analysis should consider incorporating error bars representing variance across multiple experimental runs to strengthen statistical validity.

\begin{figure}[!htbp]
\centering
\includegraphics[width=0.95\linewidth]{images/2000_nodes_Swapping_Time/(b)_Energy_consumed_bar_chart.png}
\caption{(b) Energy consumed}
\label{fig:2000_nodes_Swapping_Time_b__Energy_consumed_bar_chart}
\end{figure}

This bar chart presents the relationship between swapping time and total energy consumed for the 2000 nodes Swapping Time experimental scenario. The x-axis displays swapping time values ranging from 1 to 4, while the y-axis quantifies total energy consumed. The single-series visualization facilitates analysis of how the dependent variable responds to changes in the independent parameter setting.

Analysis of the plotted data reveals that total energy consumed ranges from 2057.20 (at swapping time = 1) to 2062.48 (at swapping time = 4), representing a span of 5.28 units. The overall trend is increasing, with values rising from 2057.20 at the initial setting to 2062.48 at the final setting. 

These results indicate that swapping time configuration meaningfully impacts total energy consumed in this experimental context. The relatively modest variation suggests that this parameter has limited influence on the measured metric within the tested range. Confidence in these findings is high given the direct correspondence between CSV data and plotted values. Future analysis should consider incorporating error bars representing variance across multiple experimental runs to strengthen statistical validity.

\begin{figure}[!htbp]
\centering
\includegraphics[width=0.95\linewidth]{images/2000_nodes_Swapping_Time/(c)_Distance_covered_bar_chart.png}
\caption{(c) Distance covered}
\label{fig:2000_nodes_Swapping_Time_c__Distance_covered_bar_chart}
\end{figure}

This bar chart presents the relationship between swapping time and total distance covered for the 2000 nodes Swapping Time experimental scenario. The x-axis displays swapping time values ranging from 1 to 4, while the y-axis quantifies total distance covered. The single-series visualization facilitates analysis of how the dependent variable responds to changes in the independent parameter setting.

Analysis of the plotted data reveals that total distance covered ranges from 8650.95 (at swapping time = 1) to 8701.15 (at swapping time = 4), representing a span of 50.20 units. The overall trend is increasing, with values rising from 8650.95 at the initial setting to 8701.15 at the final setting. 

These results indicate that swapping time configuration meaningfully impacts total distance covered in this experimental context. The relatively modest variation suggests that this parameter has limited influence on the measured metric within the tested range. Confidence in these findings is high given the direct correspondence between CSV data and plotted values. Future analysis should consider incorporating error bars representing variance across multiple experimental runs to strengthen statistical validity.

\begin{figure}[!htbp]
\centering
\includegraphics[width=0.95\linewidth]{images/2000_nodes_Swapping_Time/(d)_Run_time_bar_chart.png}
\caption{(d) Run time}
\label{fig:2000_nodes_Swapping_Time_d__Run_time_bar_chart}
\end{figure}

This bar chart presents the relationship between swapping time and total travel time for the 2000 nodes Swapping Time experimental scenario. The x-axis displays swapping time values ranging from 1 to 4, while the y-axis quantifies total travel time. The single-series visualization facilitates analysis of how the dependent variable responds to changes in the independent parameter setting.

Analysis of the plotted data reveals that total travel time ranges from 13181.25 (at swapping time = 1) to 13540.28 (at swapping time = 4), representing a span of 359.03 units. The overall trend is increasing, with values rising from 13181.25 at the initial setting to 13540.28 at the final setting. 

These results indicate that swapping time configuration meaningfully impacts total travel time in this experimental context. The relatively modest variation suggests that this parameter has limited influence on the measured metric within the tested range. Confidence in these findings is high given the direct correspondence between CSV data and plotted values. Future analysis should consider incorporating error bars representing variance across multiple experimental runs to strengthen statistical validity.

\begin{figure}[!htbp]
\centering
\includegraphics[width=0.95\linewidth]{images/2000_nodes_Swapping_Time/(e)_Module_swapped_bar_chart.png}
\caption{(e) Module swapped}
\label{fig:2000_nodes_Swapping_Time_e__Module_swapped_bar_chart}
\end{figure}

This bar chart presents the relationship between swapping time and total module swapped for the 2000 nodes Swapping Time experimental scenario. The x-axis displays swapping time values ranging from 1 to 4, while the y-axis quantifies total module swapped. The single-series visualization facilitates analysis of how the dependent variable responds to changes in the independent parameter setting.

Analysis of the plotted data reveals that total module swapped ranges from 100.00 (at swapping time = 1) to 100.00 (at swapping time = 1), representing a span of 0.00 units. The values remain relatively stable across the parameter range, with minimal net change between initial (100.00) and final (100.00) settings. 

These results indicate that swapping time configuration meaningfully impacts total module swapped in this experimental context. The relatively modest variation suggests that this parameter has limited influence on the measured metric within the tested range. Confidence in these findings is high given the direct correspondence between CSV data and plotted values. Future analysis should consider incorporating error bars representing variance across multiple experimental runs to strengthen statistical validity.


\clearpage

\subsection{2000 nodes Threshold}

\begin{figure}[!htbp]
\centering
\includegraphics[width=0.95\linewidth]{images/2000_nodes_Threshold/(a)_Travel_time_bar_chart.png}
\caption{(a) Travel time}
\label{fig:2000_nodes_Threshold_a__Travel_time_bar_chart}
\end{figure}

This bar chart presents the relationship between threshold and total travel time for the 2000 nodes Threshold experimental scenario. The x-axis displays threshold values ranging from 5 to 20, while the y-axis quantifies total travel time. The single-series visualization facilitates analysis of how the dependent variable responds to changes in the independent parameter setting.

Analysis of the plotted data reveals that total travel time ranges from 13059.88 (at threshold = 5) to 13381.98 (at threshold = 20), representing a span of 322.10 units. The overall trend is increasing, with values rising from 13059.88 at the initial setting to 13381.98 at the final setting. 

These results indicate that threshold configuration meaningfully impacts total travel time in this experimental context. The relatively modest variation suggests that this parameter has limited influence on the measured metric within the tested range. Confidence in these findings is high given the direct correspondence between CSV data and plotted values. Future analysis should consider incorporating error bars representing variance across multiple experimental runs to strengthen statistical validity.

\begin{figure}[!htbp]
\centering
\includegraphics[width=0.95\linewidth]{images/2000_nodes_Threshold/(b)_Energy_consumed_bar_chart.png}
\caption{(b) Energy consumed}
\label{fig:2000_nodes_Threshold_b__Energy_consumed_bar_chart}
\end{figure}

This bar chart presents the relationship between threshold and total energy consumed for the 2000 nodes Threshold experimental scenario. The x-axis displays threshold values ranging from 5 to 20, while the y-axis quantifies total energy consumed. The single-series visualization facilitates analysis of how the dependent variable responds to changes in the independent parameter setting.

Analysis of the plotted data reveals that total energy consumed ranges from 2020.24 (at threshold = 15) to 2057.78 (at threshold = 10), representing a span of 37.54 units. The overall trend is increasing, with values rising from 2027.67 at the initial setting to 2057.45 at the final setting. Notably, the minimum value occurs at an intermediate threshold setting (15), suggesting non-monotonic behavior that warrants further investigation.

These results indicate that threshold configuration meaningfully impacts total energy consumed in this experimental context. The relatively modest variation suggests that this parameter has limited influence on the measured metric within the tested range. Confidence in these findings is high given the direct correspondence between CSV data and plotted values. Future analysis should consider incorporating error bars representing variance across multiple experimental runs to strengthen statistical validity.

\begin{figure}[!htbp]
\centering
\includegraphics[width=0.95\linewidth]{images/2000_nodes_Threshold/(c)_Distance_covered_bar_chart.png}
\caption{(c) Distance covered}
\label{fig:2000_nodes_Threshold_c__Distance_covered_bar_chart}
\end{figure}

This bar chart presents the relationship between threshold and total distance covered for the 2000 nodes Threshold experimental scenario. The x-axis displays threshold values ranging from 5 to 20, while the y-axis quantifies total distance covered. The single-series visualization facilitates analysis of how the dependent variable responds to changes in the independent parameter setting.

Analysis of the plotted data reveals that total distance covered ranges from 8527.92 (at threshold = 5) to 8685.62 (at threshold = 10), representing a span of 157.70 units. The overall trend is increasing, with values rising from 8527.92 at the initial setting to 8652.17 at the final setting. 

These results indicate that threshold configuration meaningfully impacts total distance covered in this experimental context. The relatively modest variation suggests that this parameter has limited influence on the measured metric within the tested range. Confidence in these findings is high given the direct correspondence between CSV data and plotted values. Future analysis should consider incorporating error bars representing variance across multiple experimental runs to strengthen statistical validity.

\begin{figure}[!htbp]
\centering
\includegraphics[width=0.95\linewidth]{images/2000_nodes_Threshold/(d)_Run_time_bar_chart.png}
\caption{(d) Run time}
\label{fig:2000_nodes_Threshold_d__Run_time_bar_chart}
\end{figure}

This bar chart presents the relationship between threshold and total travel time for the 2000 nodes Threshold experimental scenario. The x-axis displays threshold values ranging from 5 to 20, while the y-axis quantifies total travel time. The single-series visualization facilitates analysis of how the dependent variable responds to changes in the independent parameter setting.

Analysis of the plotted data reveals that total travel time ranges from 13059.88 (at threshold = 5) to 13381.98 (at threshold = 20), representing a span of 322.10 units. The overall trend is increasing, with values rising from 13059.88 at the initial setting to 13381.98 at the final setting. 

These results indicate that threshold configuration meaningfully impacts total travel time in this experimental context. The relatively modest variation suggests that this parameter has limited influence on the measured metric within the tested range. Confidence in these findings is high given the direct correspondence between CSV data and plotted values. Future analysis should consider incorporating error bars representing variance across multiple experimental runs to strengthen statistical validity.

\begin{figure}[!htbp]
\centering
\includegraphics[width=0.95\linewidth]{images/2000_nodes_Threshold/(e)_Module_swapped_bar_chart.png}
\caption{(e) Module swapped}
\label{fig:2000_nodes_Threshold_e__Module_swapped_bar_chart}
\end{figure}

This bar chart presents the relationship between threshold and total module swapped for the 2000 nodes Threshold experimental scenario. The x-axis displays threshold values ranging from 5 to 20, while the y-axis quantifies total module swapped. The single-series visualization facilitates analysis of how the dependent variable responds to changes in the independent parameter setting.

Analysis of the plotted data reveals that total module swapped ranges from 97.00 (at threshold = 15) to 100.00 (at threshold = 20), representing a span of 3.00 units. The overall trend is increasing, with values rising from 98.00 at the initial setting to 100.00 at the final setting. Notably, the minimum value occurs at an intermediate threshold setting (15), suggesting non-monotonic behavior that warrants further investigation.

These results indicate that threshold configuration meaningfully impacts total module swapped in this experimental context. The relatively modest variation suggests that this parameter has limited influence on the measured metric within the tested range. Confidence in these findings is high given the direct correspondence between CSV data and plotted values. Future analysis should consider incorporating error bars representing variance across multiple experimental runs to strengthen statistical validity.


\clearpage

\subsection{2000 nodes Traffic}

\begin{figure}[!htbp]
\centering
\includegraphics[width=0.95\linewidth]{images/2000_nodes_Traffic/(a)_Travel_time_bar_chart.png}
\caption{(a) Travel time}
\label{fig:2000_nodes_Traffic_a__Travel_time_bar_chart}
\end{figure}

This bar chart presents the relationship between traffic and total travel time for the 2000 nodes Traffic experimental scenario. The x-axis displays traffic values ranging from High to Low, while the y-axis quantifies total travel time. The single-series visualization facilitates analysis of how the dependent variable responds to changes in the independent parameter setting.

Analysis of the plotted data reveals that total travel time ranges from 11773.91 (at traffic = Low) to 15961.57 (at traffic = High), representing a span of 4187.66 units. The overall trend is decreasing, with values declining from 15961.57 at the initial setting to 11773.91 at the final setting. 

These results indicate that traffic configuration meaningfully impacts total travel time in this experimental context. The substantial variation observed (coefficient of variation exceeding 20\%) suggests that parameter tuning could yield significant performance improvements. Confidence in these findings is high given the direct correspondence between CSV data and plotted values. Future analysis should consider incorporating error bars representing variance across multiple experimental runs to strengthen statistical validity.

\begin{figure}[!htbp]
\centering
\includegraphics[width=0.95\linewidth]{images/2000_nodes_Traffic/(b)_Energy_consumed_bar_chart.png}
\caption{(b) Energy consumed}
\label{fig:2000_nodes_Traffic_b__Energy_consumed_bar_chart}
\end{figure}

This bar chart presents the relationship between traffic and total energy consumed for the 2000 nodes Traffic experimental scenario. The x-axis displays traffic values ranging from High to Low, while the y-axis quantifies total energy consumed. The single-series visualization facilitates analysis of how the dependent variable responds to changes in the independent parameter setting.

Analysis of the plotted data reveals that total energy consumed ranges from 1974.45 (at traffic = Mid) to 2187.71 (at traffic = Low), representing a span of 213.26 units. The overall trend is increasing, with values rising from 1975.64 at the initial setting to 2187.71 at the final setting. Notably, the minimum value occurs at an intermediate traffic setting (Mid), suggesting non-monotonic behavior that warrants further investigation.

These results indicate that traffic configuration meaningfully impacts total energy consumed in this experimental context. The relatively modest variation suggests that this parameter has limited influence on the measured metric within the tested range. Confidence in these findings is high given the direct correspondence between CSV data and plotted values. Future analysis should consider incorporating error bars representing variance across multiple experimental runs to strengthen statistical validity.

\begin{figure}[!htbp]
\centering
\includegraphics[width=0.95\linewidth]{images/2000_nodes_Traffic/(c)_Distance_covered_bar_chart.png}
\caption{(c) Distance covered}
\label{fig:2000_nodes_Traffic_c__Distance_covered_bar_chart}
\end{figure}

This bar chart presents the relationship between traffic and total distance covered for the 2000 nodes Traffic experimental scenario. The x-axis displays traffic values ranging from High to Low, while the y-axis quantifies total distance covered. The single-series visualization facilitates analysis of how the dependent variable responds to changes in the independent parameter setting.

Analysis of the plotted data reveals that total distance covered ranges from 8689.81 (at traffic = High) to 8994.35 (at traffic = Low), representing a span of 304.54 units. The overall trend is increasing, with values rising from 8689.81 at the initial setting to 8994.35 at the final setting. 

These results indicate that traffic configuration meaningfully impacts total distance covered in this experimental context. The relatively modest variation suggests that this parameter has limited influence on the measured metric within the tested range. Confidence in these findings is high given the direct correspondence between CSV data and plotted values. Future analysis should consider incorporating error bars representing variance across multiple experimental runs to strengthen statistical validity.

\begin{figure}[!htbp]
\centering
\includegraphics[width=0.95\linewidth]{images/2000_nodes_Traffic/(d)_Run_time_bar_chart.png}
\caption{(d) Run time}
\label{fig:2000_nodes_Traffic_d__Run_time_bar_chart}
\end{figure}

This bar chart presents the relationship between traffic and total travel time for the 2000 nodes Traffic experimental scenario. The x-axis displays traffic values ranging from High to Low, while the y-axis quantifies total travel time. The single-series visualization facilitates analysis of how the dependent variable responds to changes in the independent parameter setting.

Analysis of the plotted data reveals that total travel time ranges from 11773.91 (at traffic = Low) to 15961.57 (at traffic = High), representing a span of 4187.66 units. The overall trend is decreasing, with values declining from 15961.57 at the initial setting to 11773.91 at the final setting. 

These results indicate that traffic configuration meaningfully impacts total travel time in this experimental context. The substantial variation observed (coefficient of variation exceeding 20\%) suggests that parameter tuning could yield significant performance improvements. Confidence in these findings is high given the direct correspondence between CSV data and plotted values. Future analysis should consider incorporating error bars representing variance across multiple experimental runs to strengthen statistical validity.

\begin{figure}[!htbp]
\centering
\includegraphics[width=0.95\linewidth]{images/2000_nodes_Traffic/(e)_Module_swapped_bar_chart.png}
\caption{(e) Module swapped}
\label{fig:2000_nodes_Traffic_e__Module_swapped_bar_chart}
\end{figure}

This bar chart presents the relationship between traffic and total module swapped for the 2000 nodes Traffic experimental scenario. The x-axis displays traffic values ranging from High to Low, while the y-axis quantifies total module swapped. The single-series visualization facilitates analysis of how the dependent variable responds to changes in the independent parameter setting.

Analysis of the plotted data reveals that total module swapped ranges from 96.00 (at traffic = High) to 107.00 (at traffic = Low), representing a span of 11.00 units. The overall trend is increasing, with values rising from 96.00 at the initial setting to 107.00 at the final setting. 

These results indicate that traffic configuration meaningfully impacts total module swapped in this experimental context. The relatively modest variation suggests that this parameter has limited influence on the measured metric within the tested range. Confidence in these findings is high given the direct correspondence between CSV data and plotted values. Future analysis should consider incorporating error bars representing variance across multiple experimental runs to strengthen statistical validity.


\clearpage

\subsection{500 nodes Module Change}

\begin{figure}[!htbp]
\centering
\includegraphics[width=0.95\linewidth]{images/500_nodes_Module_Change/(a)_Travel_Time_bar_chart.png}
\caption{(a) Travel Time}
\label{fig:500_nodes_Module_Change_a__Travel_Time_bar_chart}
\end{figure}

This bar chart presents the relationship between modules and total travel time for the 500 nodes Module Change experimental scenario. The x-axis displays modules values ranging from 4 to 7, while the y-axis quantifies total travel time. The single-series visualization facilitates analysis of how the dependent variable responds to changes in the independent parameter setting.

Analysis of the plotted data reveals that total travel time ranges from 3287.54 (at modules = 4) to 3287.54 (at modules = 4), representing a span of 0.00 units. The values remain relatively stable across the parameter range, with minimal net change between initial (3287.54) and final (3287.54) settings. 

These results indicate that modules configuration meaningfully impacts total travel time in this experimental context. The relatively modest variation suggests that this parameter has limited influence on the measured metric within the tested range. Confidence in these findings is high given the direct correspondence between CSV data and plotted values. Future analysis should consider incorporating error bars representing variance across multiple experimental runs to strengthen statistical validity.

\begin{figure}[!htbp]
\centering
\includegraphics[width=0.95\linewidth]{images/500_nodes_Module_Change/(b)_Energy_consumption_bar_chart.png}
\caption{(b) Energy consumption}
\label{fig:500_nodes_Module_Change_b__Energy_consumption_bar_chart}
\end{figure}

This bar chart presents the relationship between modules and total energy consumed for the 500 nodes Module Change experimental scenario. The x-axis displays modules values ranging from 4 to 7, while the y-axis quantifies total energy consumed. The single-series visualization facilitates analysis of how the dependent variable responds to changes in the independent parameter setting.

Analysis of the plotted data reveals that total energy consumed ranges from 323.50 (at modules = 4) to 323.50 (at modules = 4), representing a span of 0.00 units. The values remain relatively stable across the parameter range, with minimal net change between initial (323.50) and final (323.50) settings. 

These results indicate that modules configuration meaningfully impacts total energy consumed in this experimental context. The relatively modest variation suggests that this parameter has limited influence on the measured metric within the tested range. Confidence in these findings is high given the direct correspondence between CSV data and plotted values. Future analysis should consider incorporating error bars representing variance across multiple experimental runs to strengthen statistical validity.

\begin{figure}[!htbp]
\centering
\includegraphics[width=0.95\linewidth]{images/500_nodes_Module_Change/(c)_Distance_covered_bar_chart.png}
\caption{(c) Distance covered}
\label{fig:500_nodes_Module_Change_c__Distance_covered_bar_chart}
\end{figure}

This bar chart presents the relationship between modules and total distance covered for the 500 nodes Module Change experimental scenario. The x-axis displays modules values ranging from 4 to 7, while the y-axis quantifies total distance covered. The single-series visualization facilitates analysis of how the dependent variable responds to changes in the independent parameter setting.

Analysis of the plotted data reveals that total distance covered ranges from 2138.91 (at modules = 4) to 2138.91 (at modules = 4), representing a span of 0.00 units. The values remain relatively stable across the parameter range, with minimal net change between initial (2138.91) and final (2138.91) settings. 

These results indicate that modules configuration meaningfully impacts total distance covered in this experimental context. The relatively modest variation suggests that this parameter has limited influence on the measured metric within the tested range. Confidence in these findings is high given the direct correspondence between CSV data and plotted values. Future analysis should consider incorporating error bars representing variance across multiple experimental runs to strengthen statistical validity.

\begin{figure}[!htbp]
\centering
\includegraphics[width=0.95\linewidth]{images/500_nodes_Module_Change/(d)_Runtime_bar_chart.png}
\caption{(d) Runtime}
\label{fig:500_nodes_Module_Change_d__Runtime_bar_chart}
\end{figure}

This bar chart presents the relationship between modules and run time for the 500 nodes Module Change experimental scenario. The x-axis displays modules values ranging from 4 to 7, while the y-axis quantifies run time. The single-series visualization facilitates analysis of how the dependent variable responds to changes in the independent parameter setting.

Analysis of the plotted data reveals that run time ranges from 29.02 (at modules = 5) to 62.16 (at modules = 6), representing a span of 33.14 units. The overall trend is decreasing, with values declining from 58.53 at the initial setting to 36.14 at the final setting. Notably, the minimum value occurs at an intermediate modules setting (5), suggesting non-monotonic behavior that warrants further investigation.

These results indicate that modules configuration meaningfully impacts run time in this experimental context. The substantial variation observed (coefficient of variation exceeding 20\%) suggests that parameter tuning could yield significant performance improvements. Confidence in these findings is high given the direct correspondence between CSV data and plotted values. Future analysis should consider incorporating error bars representing variance across multiple experimental runs to strengthen statistical validity.

\begin{figure}[!htbp]
\centering
\includegraphics[width=0.95\linewidth]{images/500_nodes_Module_Change/(e)_Number_of_module_swapped_bar_chart.png}
\caption{(e) Number of module swapped}
\label{fig:500_nodes_Module_Change_e__Number_of_module_swapped_bar_chart}
\end{figure}

This bar chart presents the relationship between modules and modules for the 500 nodes Module Change experimental scenario. The x-axis displays modules values ranging from 4 to 7, while the y-axis quantifies modules. The single-series visualization facilitates analysis of how the dependent variable responds to changes in the independent parameter setting.

Analysis of the plotted data reveals that modules ranges from 4.00 (at modules = 4) to 7.00 (at modules = 7), representing a span of 3.00 units. The overall trend is increasing, with values rising from 4.00 at the initial setting to 7.00 at the final setting. 

These results indicate that modules configuration meaningfully impacts modules in this experimental context. The substantial variation observed (coefficient of variation exceeding 20\%) suggests that parameter tuning could yield significant performance improvements. Confidence in these findings is high given the direct correspondence between CSV data and plotted values. Future analysis should consider incorporating error bars representing variance across multiple experimental runs to strengthen statistical validity.


\clearpage

\subsection{500 nodes Swapping Time}

\begin{figure}[!htbp]
\centering
\includegraphics[width=0.95\linewidth]{images/500_nodes_Swapping_Time/(a)_Travel_Time_bar_chart.png}
\caption{(a) Travel Time}
\label{fig:500_nodes_Swapping_Time_a__Travel_Time_bar_chart}
\end{figure}

This bar chart presents the relationship between swapping time and total travel time for the 500 nodes Swapping Time experimental scenario. The x-axis displays swapping time values ranging from 1 to 4, while the y-axis quantifies total travel time. The single-series visualization facilitates analysis of how the dependent variable responds to changes in the independent parameter setting.

Analysis of the plotted data reveals that total travel time ranges from 3268.67 (at swapping time = 1) to 3317.54 (at swapping time = 4), representing a span of 48.87 units. The overall trend is increasing, with values rising from 3268.67 at the initial setting to 3317.54 at the final setting. 

These results indicate that swapping time configuration meaningfully impacts total travel time in this experimental context. The relatively modest variation suggests that this parameter has limited influence on the measured metric within the tested range. Confidence in these findings is high given the direct correspondence between CSV data and plotted values. Future analysis should consider incorporating error bars representing variance across multiple experimental runs to strengthen statistical validity.

\begin{figure}[!htbp]
\centering
\includegraphics[width=0.95\linewidth]{images/500_nodes_Swapping_Time/(b)_Energy_consumption_bar_chart.png}
\caption{(b) Energy consumption}
\label{fig:500_nodes_Swapping_Time_b__Energy_consumption_bar_chart}
\end{figure}

This bar chart presents the relationship between swapping time and total energy consumed for the 500 nodes Swapping Time experimental scenario. The x-axis displays swapping time values ranging from 1 to 4, while the y-axis quantifies total energy consumed. The single-series visualization facilitates analysis of how the dependent variable responds to changes in the independent parameter setting.

Analysis of the plotted data reveals that total energy consumed ranges from 323.44 (at swapping time = 1) to 323.50 (at swapping time = 2), representing a span of 0.06 units. The overall trend is increasing, with values rising from 323.44 at the initial setting to 323.50 at the final setting. 

These results indicate that swapping time configuration meaningfully impacts total energy consumed in this experimental context. The relatively modest variation suggests that this parameter has limited influence on the measured metric within the tested range. Confidence in these findings is high given the direct correspondence between CSV data and plotted values. Future analysis should consider incorporating error bars representing variance across multiple experimental runs to strengthen statistical validity.

\begin{figure}[!htbp]
\centering
\includegraphics[width=0.95\linewidth]{images/500_nodes_Swapping_Time/(c)_Distance_covered_bar_chart.png}
\caption{(c) Distance covered}
\label{fig:500_nodes_Swapping_Time_c__Distance_covered_bar_chart}
\end{figure}

This bar chart presents the relationship between swapping time and total distance covered for the 500 nodes Swapping Time experimental scenario. The x-axis displays swapping time values ranging from 1 to 4, while the y-axis quantifies total distance covered. The single-series visualization facilitates analysis of how the dependent variable responds to changes in the independent parameter setting.

Analysis of the plotted data reveals that total distance covered ranges from 2138.07 (at swapping time = 1) to 2138.91 (at swapping time = 2), representing a span of 0.84 units. The overall trend is increasing, with values rising from 2138.07 at the initial setting to 2138.91 at the final setting. 

These results indicate that swapping time configuration meaningfully impacts total distance covered in this experimental context. The relatively modest variation suggests that this parameter has limited influence on the measured metric within the tested range. Confidence in these findings is high given the direct correspondence between CSV data and plotted values. Future analysis should consider incorporating error bars representing variance across multiple experimental runs to strengthen statistical validity.

\begin{figure}[!htbp]
\centering
\includegraphics[width=0.95\linewidth]{images/500_nodes_Swapping_Time/(d)_Runtime_bar_chart.png}
\caption{(d) Runtime}
\label{fig:500_nodes_Swapping_Time_d__Runtime_bar_chart}
\end{figure}

This bar chart presents the relationship between swapping time and run time for the 500 nodes Swapping Time experimental scenario. The x-axis displays swapping time values ranging from 1 to 4, while the y-axis quantifies run time. The single-series visualization facilitates analysis of how the dependent variable responds to changes in the independent parameter setting.

Analysis of the plotted data reveals that run time ranges from 27.73 (at swapping time = 3) to 29.02 (at swapping time = 2), representing a span of 1.28 units. The overall trend is increasing, with values rising from 27.75 at the initial setting to 28.57 at the final setting. Notably, the minimum value occurs at an intermediate swapping time setting (3), suggesting non-monotonic behavior that warrants further investigation.

These results indicate that swapping time configuration meaningfully impacts run time in this experimental context. The relatively modest variation suggests that this parameter has limited influence on the measured metric within the tested range. Confidence in these findings is high given the direct correspondence between CSV data and plotted values. Future analysis should consider incorporating error bars representing variance across multiple experimental runs to strengthen statistical validity.

\begin{figure}[!htbp]
\centering
\includegraphics[width=0.95\linewidth]{images/500_nodes_Swapping_Time/(e)_Number_of_module_swapped_bar_chart.png}
\caption{(e) Number of module swapped}
\label{fig:500_nodes_Swapping_Time_e__Number_of_module_swapped_bar_chart}
\end{figure}

This bar chart presents the relationship between swapping time and total module swapped for the 500 nodes Swapping Time experimental scenario. The x-axis displays swapping time values ranging from 1 to 4, while the y-axis quantifies total module swapped. The single-series visualization facilitates analysis of how the dependent variable responds to changes in the independent parameter setting.

Analysis of the plotted data reveals that total module swapped ranges from 14.00 (at swapping time = 1) to 15.00 (at swapping time = 2), representing a span of 1.00 units. The overall trend is increasing, with values rising from 14.00 at the initial setting to 15.00 at the final setting. 

These results indicate that swapping time configuration meaningfully impacts total module swapped in this experimental context. The relatively modest variation suggests that this parameter has limited influence on the measured metric within the tested range. Confidence in these findings is high given the direct correspondence between CSV data and plotted values. Future analysis should consider incorporating error bars representing variance across multiple experimental runs to strengthen statistical validity.


\clearpage

\subsection{500 nodes Threshold}

\begin{figure}[!htbp]
\centering
\includegraphics[width=0.95\linewidth]{images/500_nodes_Threshold/(a)_Travel_Time_bar_chart.png}
\caption{(a) Travel Time}
\label{fig:500_nodes_Threshold_a__Travel_Time_bar_chart}
\end{figure}

This bar chart presents the relationship between threshold and total travel time for the 500 nodes Threshold experimental scenario. The x-axis displays threshold values ranging from 5 to 20, while the y-axis quantifies total travel time. The single-series visualization facilitates analysis of how the dependent variable responds to changes in the independent parameter setting.

Analysis of the plotted data reveals that total travel time ranges from 3287.54 (at threshold = 5) to 3287.54 (at threshold = 5), representing a span of 0.00 units. The values remain relatively stable across the parameter range, with minimal net change between initial (3287.54) and final (3287.54) settings. 

These results indicate that threshold configuration meaningfully impacts total travel time in this experimental context. The relatively modest variation suggests that this parameter has limited influence on the measured metric within the tested range. Confidence in these findings is high given the direct correspondence between CSV data and plotted values. Future analysis should consider incorporating error bars representing variance across multiple experimental runs to strengthen statistical validity.

\begin{figure}[!htbp]
\centering
\includegraphics[width=0.95\linewidth]{images/500_nodes_Threshold/(b)_Energy_consumption_bar_chart.png}
\caption{(b) Energy consumption}
\label{fig:500_nodes_Threshold_b__Energy_consumption_bar_chart}
\end{figure}

This bar chart presents the relationship between threshold and total energy consumed for the 500 nodes Threshold experimental scenario. The x-axis displays threshold values ranging from 5 to 20, while the y-axis quantifies total energy consumed. The single-series visualization facilitates analysis of how the dependent variable responds to changes in the independent parameter setting.

Analysis of the plotted data reveals that total energy consumed ranges from 323.50 (at threshold = 5) to 323.50 (at threshold = 5), representing a span of 0.00 units. The values remain relatively stable across the parameter range, with minimal net change between initial (323.50) and final (323.50) settings. 

These results indicate that threshold configuration meaningfully impacts total energy consumed in this experimental context. The relatively modest variation suggests that this parameter has limited influence on the measured metric within the tested range. Confidence in these findings is high given the direct correspondence between CSV data and plotted values. Future analysis should consider incorporating error bars representing variance across multiple experimental runs to strengthen statistical validity.

\begin{figure}[!htbp]
\centering
\includegraphics[width=0.95\linewidth]{images/500_nodes_Threshold/(c)_Distance_covered_bar_chart.png}
\caption{(c) Distance covered}
\label{fig:500_nodes_Threshold_c__Distance_covered_bar_chart}
\end{figure}

This bar chart presents the relationship between threshold and total distance covered for the 500 nodes Threshold experimental scenario. The x-axis displays threshold values ranging from 5 to 20, while the y-axis quantifies total distance covered. The single-series visualization facilitates analysis of how the dependent variable responds to changes in the independent parameter setting.

Analysis of the plotted data reveals that total distance covered ranges from 2138.91 (at threshold = 5) to 2138.91 (at threshold = 5), representing a span of 0.00 units. The values remain relatively stable across the parameter range, with minimal net change between initial (2138.91) and final (2138.91) settings. 

These results indicate that threshold configuration meaningfully impacts total distance covered in this experimental context. The relatively modest variation suggests that this parameter has limited influence on the measured metric within the tested range. Confidence in these findings is high given the direct correspondence between CSV data and plotted values. Future analysis should consider incorporating error bars representing variance across multiple experimental runs to strengthen statistical validity.

\begin{figure}[!htbp]
\centering
\includegraphics[width=0.95\linewidth]{images/500_nodes_Threshold/(d)_Runtime_bar_chart.png}
\caption{(d) Runtime}
\label{fig:500_nodes_Threshold_d__Runtime_bar_chart}
\end{figure}

This bar chart presents the relationship between threshold and run time for the 500 nodes Threshold experimental scenario. The x-axis displays threshold values ranging from 5 to 20, while the y-axis quantifies run time. The single-series visualization facilitates analysis of how the dependent variable responds to changes in the independent parameter setting.

Analysis of the plotted data reveals that run time ranges from 29.02 (at threshold = 20) to 66.47 (at threshold = 10), representing a span of 37.45 units. The overall trend is decreasing, with values declining from 58.34 at the initial setting to 29.02 at the final setting. 

These results indicate that threshold configuration meaningfully impacts run time in this experimental context. The substantial variation observed (coefficient of variation exceeding 20\%) suggests that parameter tuning could yield significant performance improvements. Confidence in these findings is high given the direct correspondence between CSV data and plotted values. Future analysis should consider incorporating error bars representing variance across multiple experimental runs to strengthen statistical validity.

\begin{figure}[!htbp]
\centering
\includegraphics[width=0.95\linewidth]{images/500_nodes_Threshold/(e)_Number_of_module_swapped_bar_chart.png}
\caption{(e) Number of module swapped}
\label{fig:500_nodes_Threshold_e__Number_of_module_swapped_bar_chart}
\end{figure}

This bar chart presents the relationship between threshold and total module swapped for the 500 nodes Threshold experimental scenario. The x-axis displays threshold values ranging from 5 to 20, while the y-axis quantifies total module swapped. The single-series visualization facilitates analysis of how the dependent variable responds to changes in the independent parameter setting.

Analysis of the plotted data reveals that total module swapped ranges from 15.00 (at threshold = 5) to 15.00 (at threshold = 5), representing a span of 0.00 units. The values remain relatively stable across the parameter range, with minimal net change between initial (15.00) and final (15.00) settings. 

These results indicate that threshold configuration meaningfully impacts total module swapped in this experimental context. The relatively modest variation suggests that this parameter has limited influence on the measured metric within the tested range. Confidence in these findings is high given the direct correspondence between CSV data and plotted values. Future analysis should consider incorporating error bars representing variance across multiple experimental runs to strengthen statistical validity.


\clearpage

\subsection{500 nodes Traffic}

\begin{figure}[!htbp]
\centering
\includegraphics[width=0.95\linewidth]{images/500_nodes_Traffic/(a)_Travel_Time_bar_chart.png}
\caption{(a) Travel Time}
\label{fig:500_nodes_Traffic_a__Travel_Time_bar_chart}
\end{figure}

This bar chart presents the relationship between traffic and total travel time for the 500 nodes Traffic experimental scenario. The x-axis displays traffic values ranging from High to Low, while the y-axis quantifies total travel time. The single-series visualization facilitates analysis of how the dependent variable responds to changes in the independent parameter setting.

Analysis of the plotted data reveals that total travel time ranges from 2915.88 (at traffic = Low) to 3618.28 (at traffic = High), representing a span of 702.40 units. The overall trend is decreasing, with values declining from 3618.28 at the initial setting to 2915.88 at the final setting. 

These results indicate that traffic configuration meaningfully impacts total travel time in this experimental context. The substantial variation observed (coefficient of variation exceeding 20\%) suggests that parameter tuning could yield significant performance improvements. Confidence in these findings is high given the direct correspondence between CSV data and plotted values. Future analysis should consider incorporating error bars representing variance across multiple experimental runs to strengthen statistical validity.

\begin{figure}[!htbp]
\centering
\includegraphics[width=0.95\linewidth]{images/500_nodes_Traffic/(b)_Energy_consumption_bar_chart.png}
\caption{(b) Energy consumption}
\label{fig:500_nodes_Traffic_b__Energy_consumption_bar_chart}
\end{figure}

This bar chart presents the relationship between traffic and total energy consumed for the 500 nodes Traffic experimental scenario. The x-axis displays traffic values ranging from High to Low, while the y-axis quantifies total energy consumed. The single-series visualization facilitates analysis of how the dependent variable responds to changes in the independent parameter setting.

Analysis of the plotted data reveals that total energy consumed ranges from 287.80 (at traffic = High) to 351.10 (at traffic = Low), representing a span of 63.29 units. The overall trend is increasing, with values rising from 287.80 at the initial setting to 351.10 at the final setting. 

These results indicate that traffic configuration meaningfully impacts total energy consumed in this experimental context. The substantial variation observed (coefficient of variation exceeding 20\%) suggests that parameter tuning could yield significant performance improvements. Confidence in these findings is high given the direct correspondence between CSV data and plotted values. Future analysis should consider incorporating error bars representing variance across multiple experimental runs to strengthen statistical validity.

\begin{figure}[!htbp]
\centering
\includegraphics[width=0.95\linewidth]{images/500_nodes_Traffic/(c)_Distance_covered_bar_chart.png}
\caption{(c) Distance covered}
\label{fig:500_nodes_Traffic_c__Distance_covered_bar_chart}
\end{figure}

This bar chart presents the relationship between traffic and total distance covered for the 500 nodes Traffic experimental scenario. The x-axis displays traffic values ranging from High to Low, while the y-axis quantifies total distance covered. The single-series visualization facilitates analysis of how the dependent variable responds to changes in the independent parameter setting.

Analysis of the plotted data reveals that total distance covered ranges from 1986.78 (at traffic = High) to 2241.45 (at traffic = Low), representing a span of 254.67 units. The overall trend is increasing, with values rising from 1986.78 at the initial setting to 2241.45 at the final setting. 

These results indicate that traffic configuration meaningfully impacts total distance covered in this experimental context. The relatively modest variation suggests that this parameter has limited influence on the measured metric within the tested range. Confidence in these findings is high given the direct correspondence between CSV data and plotted values. Future analysis should consider incorporating error bars representing variance across multiple experimental runs to strengthen statistical validity.

\begin{figure}[!htbp]
\centering
\includegraphics[width=0.95\linewidth]{images/500_nodes_Traffic/(d)_Runtime_bar_chart.png}
\caption{(d) Runtime}
\label{fig:500_nodes_Traffic_d__Runtime_bar_chart}
\end{figure}

This bar chart presents the relationship between traffic and run time for the 500 nodes Traffic experimental scenario. The x-axis displays traffic values ranging from High to Low, while the y-axis quantifies run time. The single-series visualization facilitates analysis of how the dependent variable responds to changes in the independent parameter setting.

Analysis of the plotted data reveals that run time ranges from 37.23 (at traffic = High) to 54.49 (at traffic = Low), representing a span of 17.26 units. The overall trend is increasing, with values rising from 37.23 at the initial setting to 54.49 at the final setting. 

These results indicate that traffic configuration meaningfully impacts run time in this experimental context. The substantial variation observed (coefficient of variation exceeding 20\%) suggests that parameter tuning could yield significant performance improvements. Confidence in these findings is high given the direct correspondence between CSV data and plotted values. Future analysis should consider incorporating error bars representing variance across multiple experimental runs to strengthen statistical validity.

\begin{figure}[!htbp]
\centering
\includegraphics[width=0.95\linewidth]{images/500_nodes_Traffic/(e)_Number_of_module_swapped_bar_chart.png}
\caption{(e) Number of module swapped}
\label{fig:500_nodes_Traffic_e__Number_of_module_swapped_bar_chart}
\end{figure}

This bar chart presents the relationship between traffic and total module swapped for the 500 nodes Traffic experimental scenario. The x-axis displays traffic values ranging from High to Low, while the y-axis quantifies total module swapped. The single-series visualization facilitates analysis of how the dependent variable responds to changes in the independent parameter setting.

Analysis of the plotted data reveals that total module swapped ranges from 12.00 (at traffic = High) to 16.00 (at traffic = Low), representing a span of 4.00 units. The overall trend is increasing, with values rising from 12.00 at the initial setting to 16.00 at the final setting. 

These results indicate that traffic configuration meaningfully impacts total module swapped in this experimental context. The substantial variation observed (coefficient of variation exceeding 20\%) suggests that parameter tuning could yield significant performance improvements. Confidence in these findings is high given the direct correspondence between CSV data and plotted values. Future analysis should consider incorporating error bars representing variance across multiple experimental runs to strengthen statistical validity.


\clearpage

\subsection{MT2TE and Other Program Module Change}

\begin{figure}[!htbp]
\centering
\includegraphics[width=0.95\linewidth]{images/MT2TE_and_Other_Program_Module_Change/(a)_Total_Travel_Time.png}
\caption{(a) Total Travel Time}
\label{fig:MT2TE_and_Other_Program_Module_Change_a__Total_Travel_Time}
\end{figure}

This grouped bar chart presents a comparative analysis of total travel time across multiple algorithms within the MT2TE and Other Program Module Change experimental configuration. The x-axis represents the number of nodes in the network, while the y-axis quantifies the total travel time metric. The legend identifies 4 distinct algorithms: Genetic Algorithm, EVRPBSS, Ant Colony, Clarke and Wright algorithm. This visualization enables direct comparison of algorithmic performance under identical network conditions.

Quantitative analysis reveals significant performance disparities among the evaluated algorithms. Clarke and Wright algorithm demonstrates superior performance with a mean total travel time of 688.78, while Genetic Algorithm exhibits the highest values averaging 1366.67. This represents an improvement of approximately 49.6\% when comparing the best to worst performing algorithms. The relative ranking of algorithms remains largely consistent across different node configurations, suggesting robust performance characteristics.

These findings support the hypothesis that algorithmic choice significantly impacts system performance metrics. Future work should incorporate statistical significance testing and confidence intervals to strengthen these comparative conclusions. Additionally, examining the computational complexity trade-offs between algorithms would provide valuable context for practical deployment decisions.

\begin{figure}[!htbp]
\centering
\includegraphics[width=0.95\linewidth]{images/MT2TE_and_Other_Program_Module_Change/(b)_Energy.png}
\caption{(b) Energy}
\label{fig:MT2TE_and_Other_Program_Module_Change_b__Energy}
\end{figure}

This grouped bar chart presents a comparative analysis of energy across multiple algorithms within the MT2TE and Other Program Module Change experimental configuration. The x-axis represents the number of nodes in the network, while the y-axis quantifies the energy metric. The legend identifies 4 distinct algorithms: Genetic Algorithm, EVRPBSS, Ant Colony, Clarke and Wright algorithm. This visualization enables direct comparison of algorithmic performance under identical network conditions.

Quantitative analysis reveals significant performance disparities among the evaluated algorithms. Clarke and Wright algorithm demonstrates superior performance with a mean energy of 688.78, while Genetic Algorithm exhibits the highest values averaging 1366.67. This represents an improvement of approximately 49.6\% when comparing the best to worst performing algorithms. The relative ranking of algorithms remains largely consistent across different node configurations, suggesting robust performance characteristics.

These findings support the hypothesis that algorithmic choice significantly impacts system performance metrics. Future work should incorporate statistical significance testing and confidence intervals to strengthen these comparative conclusions. Additionally, examining the computational complexity trade-offs between algorithms would provide valuable context for practical deployment decisions.

\begin{figure}[!htbp]
\centering
\includegraphics[width=0.95\linewidth]{images/MT2TE_and_Other_Program_Module_Change/(c)_Distance.png}
\caption{(c) Distance}
\label{fig:MT2TE_and_Other_Program_Module_Change_c__Distance}
\end{figure}

This grouped bar chart presents a comparative analysis of distance across multiple algorithms within the MT2TE and Other Program Module Change experimental configuration. The x-axis represents the number of nodes in the network, while the y-axis quantifies the distance metric. The legend identifies 4 distinct algorithms: Genetic Algorithm, EVRPBSS, Ant Colony, Clarke and Wright algorithm. This visualization enables direct comparison of algorithmic performance under identical network conditions.

Quantitative analysis reveals significant performance disparities among the evaluated algorithms. Clarke and Wright algorithm demonstrates superior performance with a mean distance of 688.78, while Genetic Algorithm exhibits the highest values averaging 1366.67. This represents an improvement of approximately 49.6\% when comparing the best to worst performing algorithms. The relative ranking of algorithms remains largely consistent across different node configurations, suggesting robust performance characteristics.

These findings support the hypothesis that algorithmic choice significantly impacts system performance metrics. Future work should incorporate statistical significance testing and confidence intervals to strengthen these comparative conclusions. Additionally, examining the computational complexity trade-offs between algorithms would provide valuable context for practical deployment decisions.

\begin{figure}[!htbp]
\centering
\includegraphics[width=0.95\linewidth]{images/MT2TE_and_Other_Program_Module_Change/(d)_Module_Swapped.png}
\caption{(d) Module Swapped}
\label{fig:MT2TE_and_Other_Program_Module_Change_d__Module_Swapped}
\end{figure}

This grouped bar chart presents a comparative analysis of module swapped across multiple algorithms within the MT2TE and Other Program Module Change experimental configuration. The x-axis represents the number of nodes in the network, while the y-axis quantifies the module swapped metric. The legend identifies 4 distinct algorithms: Genetic Algorithm, EVRPBSS, Ant Colony, Clarke and Wright algorithm. This visualization enables direct comparison of algorithmic performance under identical network conditions.

Quantitative analysis reveals significant performance disparities among the evaluated algorithms. Clarke and Wright algorithm demonstrates superior performance with a mean module swapped of 688.78, while Genetic Algorithm exhibits the highest values averaging 1366.67. This represents an improvement of approximately 49.6\% when comparing the best to worst performing algorithms. The relative ranking of algorithms remains largely consistent across different node configurations, suggesting robust performance characteristics.

These findings support the hypothesis that algorithmic choice significantly impacts system performance metrics. Future work should incorporate statistical significance testing and confidence intervals to strengthen these comparative conclusions. Additionally, examining the computational complexity trade-offs between algorithms would provide valuable context for practical deployment decisions.

\begin{figure}[!htbp]
\centering
\includegraphics[width=0.95\linewidth]{images/MT2TE_and_Other_Program_Module_Change/(e)_Execution_Time.png}
\caption{(e) Execution Time}
\label{fig:MT2TE_and_Other_Program_Module_Change_e__Execution_Time}
\end{figure}

This grouped bar chart presents a comparative analysis of execution time across multiple algorithms within the MT2TE and Other Program Module Change experimental configuration. The x-axis represents the number of nodes in the network, while the y-axis quantifies the execution time metric. The legend identifies 4 distinct algorithms: Genetic Algorithm, EVRPBSS, Ant Colony, Clarke and Wright algorithm. This visualization enables direct comparison of algorithmic performance under identical network conditions.

Quantitative analysis reveals significant performance disparities among the evaluated algorithms. Clarke and Wright algorithm demonstrates superior performance with a mean execution time of 688.78, while Genetic Algorithm exhibits the highest values averaging 1366.67. This represents an improvement of approximately 49.6\% when comparing the best to worst performing algorithms. The relative ranking of algorithms remains largely consistent across different node configurations, suggesting robust performance characteristics.

These findings support the hypothesis that algorithmic choice significantly impacts system performance metrics. Future work should incorporate statistical significance testing and confidence intervals to strengthen these comparative conclusions. Additionally, examining the computational complexity trade-offs between algorithms would provide valuable context for practical deployment decisions.

\begin{figure}[!htbp]
\centering
\includegraphics[width=0.95\linewidth]{images/MT2TE_and_Other_Program_Module_Change/(f)_Execution_Time.png}
\caption{(f) Execution Time}
\label{fig:MT2TE_and_Other_Program_Module_Change_f__Execution_Time}
\end{figure}

This grouped bar chart presents a comparative analysis of execution time across multiple algorithms within the MT2TE and Other Program Module Change experimental configuration. The x-axis represents the number of nodes in the network, while the y-axis quantifies the execution time metric. The legend identifies 3 distinct algorithms: EVRPBSS, Ant Colony, Clarke and Wright algorithm. This visualization enables direct comparison of algorithmic performance under identical network conditions.

Quantitative analysis reveals significant performance disparities among the evaluated algorithms. Clarke and Wright algorithm demonstrates superior performance with a mean execution time of 688.78, while Ant Colony exhibits the highest values averaging 1020.01. This represents an improvement of approximately 32.5\% when comparing the best to worst performing algorithms. The relative ranking of algorithms remains largely consistent across different node configurations, suggesting robust performance characteristics.

These findings support the hypothesis that algorithmic choice significantly impacts system performance metrics. Future work should incorporate statistical significance testing and confidence intervals to strengthen these comparative conclusions. Additionally, examining the computational complexity trade-offs between algorithms would provide valuable context for practical deployment decisions.


\clearpage

\subsection{MT2TE and Other Program Node Change}

\begin{figure}[!htbp]
\centering
\includegraphics[width=0.95\linewidth]{images/MT2TE_and_Other_Program_Node_Change/(a)_Total_Travel_Time.png}
\caption{(a) Total Travel Time}
\label{fig:MT2TE_and_Other_Program_Node_Change_a__Total_Travel_Time}
\end{figure}

This grouped bar chart presents a comparative analysis of total travel time across multiple algorithms within the MT2TE and Other Program Node Change experimental configuration. The x-axis represents the number of nodes in the network, while the y-axis quantifies the total travel time metric. The legend identifies 4 distinct algorithms: Genetic Algorithm, EVRPBSS, Ant Colony, Clarke and Wright algorithm. This visualization enables direct comparison of algorithmic performance under identical network conditions.

Quantitative analysis reveals significant performance disparities among the evaluated algorithms. Clarke and Wright algorithm demonstrates superior performance with a mean total travel time of 862.43, while Genetic Algorithm exhibits the highest values averaging 2037.20. This represents an improvement of approximately 57.7\% when comparing the best to worst performing algorithms. The relative ranking of algorithms remains largely consistent across different node configurations, suggesting robust performance characteristics.

These findings support the hypothesis that algorithmic choice significantly impacts system performance metrics. Future work should incorporate statistical significance testing and confidence intervals to strengthen these comparative conclusions. Additionally, examining the computational complexity trade-offs between algorithms would provide valuable context for practical deployment decisions.

\begin{figure}[!htbp]
\centering
\includegraphics[width=0.95\linewidth]{images/MT2TE_and_Other_Program_Node_Change/(b)_Energy.png}
\caption{(b) Energy}
\label{fig:MT2TE_and_Other_Program_Node_Change_b__Energy}
\end{figure}

This grouped bar chart presents a comparative analysis of energy across multiple algorithms within the MT2TE and Other Program Node Change experimental configuration. The x-axis represents the number of nodes in the network, while the y-axis quantifies the energy metric. The legend identifies 4 distinct algorithms: Genetic Algorithm, EVRPBSS, Ant Colony, Clarke and Wright algorithm. This visualization enables direct comparison of algorithmic performance under identical network conditions.

Quantitative analysis reveals significant performance disparities among the evaluated algorithms. Clarke and Wright algorithm demonstrates superior performance with a mean energy of 862.43, while Genetic Algorithm exhibits the highest values averaging 2037.20. This represents an improvement of approximately 57.7\% when comparing the best to worst performing algorithms. The relative ranking of algorithms remains largely consistent across different node configurations, suggesting robust performance characteristics.

These findings support the hypothesis that algorithmic choice significantly impacts system performance metrics. Future work should incorporate statistical significance testing and confidence intervals to strengthen these comparative conclusions. Additionally, examining the computational complexity trade-offs between algorithms would provide valuable context for practical deployment decisions.

\begin{figure}[!htbp]
\centering
\includegraphics[width=0.95\linewidth]{images/MT2TE_and_Other_Program_Node_Change/(c)_Distance.png}
\caption{(c) Distance}
\label{fig:MT2TE_and_Other_Program_Node_Change_c__Distance}
\end{figure}

This grouped bar chart presents a comparative analysis of distance across multiple algorithms within the MT2TE and Other Program Node Change experimental configuration. The x-axis represents the number of nodes in the network, while the y-axis quantifies the distance metric. The legend identifies 4 distinct algorithms: Genetic Algorithm, EVRPBSS, Ant Colony, Clarke and Wright algorithm. This visualization enables direct comparison of algorithmic performance under identical network conditions.

Quantitative analysis reveals significant performance disparities among the evaluated algorithms. Clarke and Wright algorithm demonstrates superior performance with a mean distance of 862.43, while Genetic Algorithm exhibits the highest values averaging 2037.20. This represents an improvement of approximately 57.7\% when comparing the best to worst performing algorithms. The relative ranking of algorithms remains largely consistent across different node configurations, suggesting robust performance characteristics.

These findings support the hypothesis that algorithmic choice significantly impacts system performance metrics. Future work should incorporate statistical significance testing and confidence intervals to strengthen these comparative conclusions. Additionally, examining the computational complexity trade-offs between algorithms would provide valuable context for practical deployment decisions.

\begin{figure}[!htbp]
\centering
\includegraphics[width=0.95\linewidth]{images/MT2TE_and_Other_Program_Node_Change/(d)_Execution_Time.png}
\caption{(d) Execution Time}
\label{fig:MT2TE_and_Other_Program_Node_Change_d__Execution_Time}
\end{figure}

This grouped bar chart presents a comparative analysis of execution time across multiple algorithms within the MT2TE and Other Program Node Change experimental configuration. The x-axis represents the number of nodes in the network, while the y-axis quantifies the execution time metric. The legend identifies 4 distinct algorithms: Genetic Algorithm, EVRPBSS, Ant Colony, Clarke and Wright algorithm. This visualization enables direct comparison of algorithmic performance under identical network conditions.

Quantitative analysis reveals significant performance disparities among the evaluated algorithms. Clarke and Wright algorithm demonstrates superior performance with a mean execution time of 862.43, while Genetic Algorithm exhibits the highest values averaging 2037.20. This represents an improvement of approximately 57.7\% when comparing the best to worst performing algorithms. The relative ranking of algorithms remains largely consistent across different node configurations, suggesting robust performance characteristics.

These findings support the hypothesis that algorithmic choice significantly impacts system performance metrics. Future work should incorporate statistical significance testing and confidence intervals to strengthen these comparative conclusions. Additionally, examining the computational complexity trade-offs between algorithms would provide valuable context for practical deployment decisions.

\begin{figure}[!htbp]
\centering
\includegraphics[width=0.95\linewidth]{images/MT2TE_and_Other_Program_Node_Change/(e)_Total_Module_Swapped.png}
\caption{(e) Total Module Swapped}
\label{fig:MT2TE_and_Other_Program_Node_Change_e__Total_Module_Swapped}
\end{figure}

This grouped bar chart presents a comparative analysis of total module swapped across multiple algorithms within the MT2TE and Other Program Node Change experimental configuration. The x-axis represents the number of nodes in the network, while the y-axis quantifies the total module swapped metric. The legend identifies 4 distinct algorithms: Genetic Algorithm, EVRPBSS, Ant Colony, Clarke and Wright algorithm. This visualization enables direct comparison of algorithmic performance under identical network conditions.

Quantitative analysis reveals significant performance disparities among the evaluated algorithms. Clarke and Wright algorithm demonstrates superior performance with a mean total module swapped of 862.43, while Genetic Algorithm exhibits the highest values averaging 2037.20. This represents an improvement of approximately 57.7\% when comparing the best to worst performing algorithms. The relative ranking of algorithms remains largely consistent across different node configurations, suggesting robust performance characteristics.

These findings support the hypothesis that algorithmic choice significantly impacts system performance metrics. Future work should incorporate statistical significance testing and confidence intervals to strengthen these comparative conclusions. Additionally, examining the computational complexity trade-offs between algorithms would provide valuable context for practical deployment decisions.

\begin{figure}[!htbp]
\centering
\includegraphics[width=0.95\linewidth]{images/MT2TE_and_Other_Program_Node_Change/(f)_Execution_Time.png}
\caption{(f) Execution Time}
\label{fig:MT2TE_and_Other_Program_Node_Change_f__Execution_Time}
\end{figure}

This grouped bar chart presents a comparative analysis of execution time across multiple algorithms within the MT2TE and Other Program Node Change experimental configuration. The x-axis represents the number of nodes in the network, while the y-axis quantifies the execution time metric. The legend identifies 4 distinct algorithms: Genetic Algorithm, EVRPBSS, Ant Colony, Clarke and Wright algorithm. This visualization enables direct comparison of algorithmic performance under identical network conditions.

Quantitative analysis reveals significant performance disparities among the evaluated algorithms. Clarke and Wright algorithm demonstrates superior performance with a mean execution time of 862.43, while Genetic Algorithm exhibits the highest values averaging 2037.20. This represents an improvement of approximately 57.7\% when comparing the best to worst performing algorithms. The relative ranking of algorithms remains largely consistent across different node configurations, suggesting robust performance characteristics.

These findings support the hypothesis that algorithmic choice significantly impacts system performance metrics. Future work should incorporate statistical significance testing and confidence intervals to strengthen these comparative conclusions. Additionally, examining the computational complexity trade-offs between algorithms would provide valuable context for practical deployment decisions.


\clearpage

\subsection{MT2TE and Other Program Swap Time Change}

\begin{figure}[!htbp]
\centering
\includegraphics[width=0.95\linewidth]{images/MT2TE_and_Other_Program_Swap_Time_Change/(a)_Total_Travel_Time.png}
\caption{(a) Total Travel Time}
\label{fig:MT2TE_and_Other_Program_Swap_Time_Change_a__Total_Travel_Time}
\end{figure}

This grouped bar chart presents a comparative analysis of total travel time across multiple algorithms within the MT2TE and Other Program Swap Time Change experimental configuration. The x-axis represents the number of nodes in the network, while the y-axis quantifies the total travel time metric. The legend identifies 4 distinct algorithms: Genetic Algorithm, EVRPBSS, Ant Colony, Clarke and Wright algorithm. This visualization enables direct comparison of algorithmic performance under identical network conditions.

Quantitative analysis reveals significant performance disparities among the evaluated algorithms. Clarke and Wright algorithm demonstrates superior performance with a mean total travel time of 697.62, while Genetic Algorithm exhibits the highest values averaging 1438.33. This represents an improvement of approximately 51.5\% when comparing the best to worst performing algorithms. The relative ranking of algorithms remains largely consistent across different node configurations, suggesting robust performance characteristics.

These findings support the hypothesis that algorithmic choice significantly impacts system performance metrics. Future work should incorporate statistical significance testing and confidence intervals to strengthen these comparative conclusions. Additionally, examining the computational complexity trade-offs between algorithms would provide valuable context for practical deployment decisions.

\begin{figure}[!htbp]
\centering
\includegraphics[width=0.95\linewidth]{images/MT2TE_and_Other_Program_Swap_Time_Change/(b)_Energy.png}
\caption{(b) Energy}
\label{fig:MT2TE_and_Other_Program_Swap_Time_Change_b__Energy}
\end{figure}

This grouped bar chart presents a comparative analysis of energy across multiple algorithms within the MT2TE and Other Program Swap Time Change experimental configuration. The x-axis represents the number of nodes in the network, while the y-axis quantifies the energy metric. The legend identifies 4 distinct algorithms: Genetic Algorithm, EVRPBSS, Ant Colony, Clarke and Wright algorithm. This visualization enables direct comparison of algorithmic performance under identical network conditions.

Quantitative analysis reveals significant performance disparities among the evaluated algorithms. Clarke and Wright algorithm demonstrates superior performance with a mean energy of 697.62, while Genetic Algorithm exhibits the highest values averaging 1438.33. This represents an improvement of approximately 51.5\% when comparing the best to worst performing algorithms. The relative ranking of algorithms remains largely consistent across different node configurations, suggesting robust performance characteristics.

These findings support the hypothesis that algorithmic choice significantly impacts system performance metrics. Future work should incorporate statistical significance testing and confidence intervals to strengthen these comparative conclusions. Additionally, examining the computational complexity trade-offs between algorithms would provide valuable context for practical deployment decisions.

\begin{figure}[!htbp]
\centering
\includegraphics[width=0.95\linewidth]{images/MT2TE_and_Other_Program_Swap_Time_Change/(c)_Distance.png}
\caption{(c) Distance}
\label{fig:MT2TE_and_Other_Program_Swap_Time_Change_c__Distance}
\end{figure}

This grouped bar chart presents a comparative analysis of distance across multiple algorithms within the MT2TE and Other Program Swap Time Change experimental configuration. The x-axis represents the number of nodes in the network, while the y-axis quantifies the distance metric. The legend identifies 4 distinct algorithms: Genetic Algorithm, EVRPBSS, Ant Colony, Clarke and Wright algorithm. This visualization enables direct comparison of algorithmic performance under identical network conditions.

Quantitative analysis reveals significant performance disparities among the evaluated algorithms. Clarke and Wright algorithm demonstrates superior performance with a mean distance of 697.62, while Genetic Algorithm exhibits the highest values averaging 1438.33. This represents an improvement of approximately 51.5\% when comparing the best to worst performing algorithms. The relative ranking of algorithms remains largely consistent across different node configurations, suggesting robust performance characteristics.

These findings support the hypothesis that algorithmic choice significantly impacts system performance metrics. Future work should incorporate statistical significance testing and confidence intervals to strengthen these comparative conclusions. Additionally, examining the computational complexity trade-offs between algorithms would provide valuable context for practical deployment decisions.

\begin{figure}[!htbp]
\centering
\includegraphics[width=0.95\linewidth]{images/MT2TE_and_Other_Program_Swap_Time_Change/(d)_Total_Module_Swapped.png}
\caption{(d) Total Module Swapped}
\label{fig:MT2TE_and_Other_Program_Swap_Time_Change_d__Total_Module_Swapped}
\end{figure}

This grouped bar chart presents a comparative analysis of total module swapped across multiple algorithms within the MT2TE and Other Program Swap Time Change experimental configuration. The x-axis represents the number of nodes in the network, while the y-axis quantifies the total module swapped metric. The legend identifies 4 distinct algorithms: Genetic Algorithm, EVRPBSS, Ant Colony, Clarke and Wright algorithm. This visualization enables direct comparison of algorithmic performance under identical network conditions.

Quantitative analysis reveals significant performance disparities among the evaluated algorithms. Clarke and Wright algorithm demonstrates superior performance with a mean total module swapped of 697.62, while Genetic Algorithm exhibits the highest values averaging 1438.33. This represents an improvement of approximately 51.5\% when comparing the best to worst performing algorithms. The relative ranking of algorithms remains largely consistent across different node configurations, suggesting robust performance characteristics.

These findings support the hypothesis that algorithmic choice significantly impacts system performance metrics. Future work should incorporate statistical significance testing and confidence intervals to strengthen these comparative conclusions. Additionally, examining the computational complexity trade-offs between algorithms would provide valuable context for practical deployment decisions.

\begin{figure}[!htbp]
\centering
\includegraphics[width=0.95\linewidth]{images/MT2TE_and_Other_Program_Swap_Time_Change/(e)_Execution_Time.png}
\caption{(e) Execution Time}
\label{fig:MT2TE_and_Other_Program_Swap_Time_Change_e__Execution_Time}
\end{figure}

This grouped bar chart presents a comparative analysis of execution time across multiple algorithms within the MT2TE and Other Program Swap Time Change experimental configuration. The x-axis represents the number of nodes in the network, while the y-axis quantifies the execution time metric. The legend identifies 4 distinct algorithms: Genetic Algorithm, EVRPBSS, Ant Colony, Clarke and Wright algorithm. This visualization enables direct comparison of algorithmic performance under identical network conditions.

Quantitative analysis reveals significant performance disparities among the evaluated algorithms. Clarke and Wright algorithm demonstrates superior performance with a mean execution time of 697.62, while Genetic Algorithm exhibits the highest values averaging 1438.33. This represents an improvement of approximately 51.5\% when comparing the best to worst performing algorithms. The relative ranking of algorithms remains largely consistent across different node configurations, suggesting robust performance characteristics.

These findings support the hypothesis that algorithmic choice significantly impacts system performance metrics. Future work should incorporate statistical significance testing and confidence intervals to strengthen these comparative conclusions. Additionally, examining the computational complexity trade-offs between algorithms would provide valuable context for practical deployment decisions.

\begin{figure}[!htbp]
\centering
\includegraphics[width=0.95\linewidth]{images/MT2TE_and_Other_Program_Swap_Time_Change/(f)_Execution_Time.png}
\caption{(f) Execution Time}
\label{fig:MT2TE_and_Other_Program_Swap_Time_Change_f__Execution_Time}
\end{figure}

This grouped bar chart presents a comparative analysis of execution time across multiple algorithms within the MT2TE and Other Program Swap Time Change experimental configuration. The x-axis represents the number of nodes in the network, while the y-axis quantifies the execution time metric. The legend identifies 3 distinct algorithms: EVRPBSS, Ant Colony, Clarke and Wright algorithm. This visualization enables direct comparison of algorithmic performance under identical network conditions.

Quantitative analysis reveals significant performance disparities among the evaluated algorithms. Clarke and Wright algorithm demonstrates superior performance with a mean execution time of 697.62, while Ant Colony exhibits the highest values averaging 1026.84. This represents an improvement of approximately 32.1\% when comparing the best to worst performing algorithms. The relative ranking of algorithms remains largely consistent across different node configurations, suggesting robust performance characteristics.

These findings support the hypothesis that algorithmic choice significantly impacts system performance metrics. Future work should incorporate statistical significance testing and confidence intervals to strengthen these comparative conclusions. Additionally, examining the computational complexity trade-offs between algorithms would provide valuable context for practical deployment decisions.


\clearpage

\subsection{MT2TE and Other Program Threshold Change}

\begin{figure}[!htbp]
\centering
\includegraphics[width=0.95\linewidth]{images/MT2TE_and_Other_Program_Threshold_Change/(a)_Total_Travel_Time.png}
\caption{(a) Total Travel Time}
\label{fig:MT2TE_and_Other_Program_Threshold_Change_a__Total_Travel_Time}
\end{figure}

This grouped bar chart presents a comparative analysis of total travel time across multiple algorithms within the MT2TE and Other Program Threshold Change experimental configuration. The x-axis represents the number of nodes in the network, while the y-axis quantifies the total travel time metric. The legend identifies 4 distinct algorithms: Genetic Algorithm, EVRPBSS, Ant Colony, Clarke and Wright algorithm. This visualization enables direct comparison of algorithmic performance under identical network conditions.

Quantitative analysis reveals significant performance disparities among the evaluated algorithms. Clarke and Wright algorithm demonstrates superior performance with a mean total travel time of 690.83, while Genetic Algorithm exhibits the highest values averaging 1400.93. This represents an improvement of approximately 50.7\% when comparing the best to worst performing algorithms. The relative ranking of algorithms remains largely consistent across different node configurations, suggesting robust performance characteristics.

These findings support the hypothesis that algorithmic choice significantly impacts system performance metrics. Future work should incorporate statistical significance testing and confidence intervals to strengthen these comparative conclusions. Additionally, examining the computational complexity trade-offs between algorithms would provide valuable context for practical deployment decisions.

\begin{figure}[!htbp]
\centering
\includegraphics[width=0.95\linewidth]{images/MT2TE_and_Other_Program_Threshold_Change/(b)_Energy.png}
\caption{(b) Energy}
\label{fig:MT2TE_and_Other_Program_Threshold_Change_b__Energy}
\end{figure}

This grouped bar chart presents a comparative analysis of energy across multiple algorithms within the MT2TE and Other Program Threshold Change experimental configuration. The x-axis represents the number of nodes in the network, while the y-axis quantifies the energy metric. The legend identifies 4 distinct algorithms: Genetic Algorithm, EVRPBSS, Ant Colony, Clarke and Wright algorithm. This visualization enables direct comparison of algorithmic performance under identical network conditions.

Quantitative analysis reveals significant performance disparities among the evaluated algorithms. Clarke and Wright algorithm demonstrates superior performance with a mean energy of 690.83, while Genetic Algorithm exhibits the highest values averaging 1400.93. This represents an improvement of approximately 50.7\% when comparing the best to worst performing algorithms. The relative ranking of algorithms remains largely consistent across different node configurations, suggesting robust performance characteristics.

These findings support the hypothesis that algorithmic choice significantly impacts system performance metrics. Future work should incorporate statistical significance testing and confidence intervals to strengthen these comparative conclusions. Additionally, examining the computational complexity trade-offs between algorithms would provide valuable context for practical deployment decisions.

\begin{figure}[!htbp]
\centering
\includegraphics[width=0.95\linewidth]{images/MT2TE_and_Other_Program_Threshold_Change/(c)_Distance.png}
\caption{(c) Distance}
\label{fig:MT2TE_and_Other_Program_Threshold_Change_c__Distance}
\end{figure}

This grouped bar chart presents a comparative analysis of distance across multiple algorithms within the MT2TE and Other Program Threshold Change experimental configuration. The x-axis represents the number of nodes in the network, while the y-axis quantifies the distance metric. The legend identifies 4 distinct algorithms: Genetic Algorithm, EVRPBSS, Ant Colony, Clarke and Wright algorithm. This visualization enables direct comparison of algorithmic performance under identical network conditions.

Quantitative analysis reveals significant performance disparities among the evaluated algorithms. Clarke and Wright algorithm demonstrates superior performance with a mean distance of 690.83, while Genetic Algorithm exhibits the highest values averaging 1400.93. This represents an improvement of approximately 50.7\% when comparing the best to worst performing algorithms. The relative ranking of algorithms remains largely consistent across different node configurations, suggesting robust performance characteristics.

These findings support the hypothesis that algorithmic choice significantly impacts system performance metrics. Future work should incorporate statistical significance testing and confidence intervals to strengthen these comparative conclusions. Additionally, examining the computational complexity trade-offs between algorithms would provide valuable context for practical deployment decisions.

\begin{figure}[!htbp]
\centering
\includegraphics[width=0.95\linewidth]{images/MT2TE_and_Other_Program_Threshold_Change/(d)_Module_Swapped.png}
\caption{(d) Module Swapped}
\label{fig:MT2TE_and_Other_Program_Threshold_Change_d__Module_Swapped}
\end{figure}

This grouped bar chart presents a comparative analysis of module swapped across multiple algorithms within the MT2TE and Other Program Threshold Change experimental configuration. The x-axis represents the number of nodes in the network, while the y-axis quantifies the module swapped metric. The legend identifies 4 distinct algorithms: Genetic Algorithm, EVRPBSS, Ant Colony, Clarke and Wright algorithm. This visualization enables direct comparison of algorithmic performance under identical network conditions.

Quantitative analysis reveals significant performance disparities among the evaluated algorithms. Clarke and Wright algorithm demonstrates superior performance with a mean module swapped of 690.83, while Genetic Algorithm exhibits the highest values averaging 1400.93. This represents an improvement of approximately 50.7\% when comparing the best to worst performing algorithms. The relative ranking of algorithms remains largely consistent across different node configurations, suggesting robust performance characteristics.

These findings support the hypothesis that algorithmic choice significantly impacts system performance metrics. Future work should incorporate statistical significance testing and confidence intervals to strengthen these comparative conclusions. Additionally, examining the computational complexity trade-offs between algorithms would provide valuable context for practical deployment decisions.

\begin{figure}[!htbp]
\centering
\includegraphics[width=0.95\linewidth]{images/MT2TE_and_Other_Program_Threshold_Change/(e)_Execution_Time.png}
\caption{(e) Execution Time}
\label{fig:MT2TE_and_Other_Program_Threshold_Change_e__Execution_Time}
\end{figure}

This grouped bar chart presents a comparative analysis of execution time across multiple algorithms within the MT2TE and Other Program Threshold Change experimental configuration. The x-axis represents the number of nodes in the network, while the y-axis quantifies the execution time metric. The legend identifies 4 distinct algorithms: Genetic Algorithm, EVRPBSS, Ant Colony, Clarke and Wright algorithm. This visualization enables direct comparison of algorithmic performance under identical network conditions.

Quantitative analysis reveals significant performance disparities among the evaluated algorithms. Clarke and Wright algorithm demonstrates superior performance with a mean execution time of 690.83, while Genetic Algorithm exhibits the highest values averaging 1400.93. This represents an improvement of approximately 50.7\% when comparing the best to worst performing algorithms. The relative ranking of algorithms remains largely consistent across different node configurations, suggesting robust performance characteristics.

These findings support the hypothesis that algorithmic choice significantly impacts system performance metrics. Future work should incorporate statistical significance testing and confidence intervals to strengthen these comparative conclusions. Additionally, examining the computational complexity trade-offs between algorithms would provide valuable context for practical deployment decisions.

\begin{figure}[!htbp]
\centering
\includegraphics[width=0.95\linewidth]{images/MT2TE_and_Other_Program_Threshold_Change/(f)_Execution_Time.png}
\caption{(f) Execution Time}
\label{fig:MT2TE_and_Other_Program_Threshold_Change_f__Execution_Time}
\end{figure}

This grouped bar chart presents a comparative analysis of execution time across multiple algorithms within the MT2TE and Other Program Threshold Change experimental configuration. The x-axis represents the number of nodes in the network, while the y-axis quantifies the execution time metric. The legend identifies 3 distinct algorithms: EVRPBSS, Ant Colony, Clarke and Wright algorithm. This visualization enables direct comparison of algorithmic performance under identical network conditions.

Quantitative analysis reveals significant performance disparities among the evaluated algorithms. Clarke and Wright algorithm demonstrates superior performance with a mean execution time of 690.83, while Ant Colony exhibits the highest values averaging 1020.01. This represents an improvement of approximately 32.3\% when comparing the best to worst performing algorithms. The relative ranking of algorithms remains largely consistent across different node configurations, suggesting robust performance characteristics.

These findings support the hypothesis that algorithmic choice significantly impacts system performance metrics. Future work should incorporate statistical significance testing and confidence intervals to strengthen these comparative conclusions. Additionally, examining the computational complexity trade-offs between algorithms would provide valuable context for practical deployment decisions.


\clearpage

\subsection{MT2TE and Other Program Traffic}

\begin{figure}[!htbp]
\centering
\includegraphics[width=0.95\linewidth]{images/MT2TE_and_Other_Program_Traffic/(a)_Travel_Time.png}
\caption{(a) Travel Time}
\label{fig:MT2TE_and_Other_Program_Traffic_a__Travel_Time}
\end{figure}

This grouped bar chart presents a comparative analysis of travel time across multiple algorithms within the MT2TE and Other Program Traffic experimental configuration. The x-axis represents the number of nodes in the network, while the y-axis quantifies the travel time metric. The legend identifies 4 distinct algorithms: Genetic Algorithm, EVRPBSS, Ant Colony, Clarke and Wright algorithm. This visualization enables direct comparison of algorithmic performance under identical network conditions.

Quantitative analysis reveals significant performance disparities among the evaluated algorithms. Clarke and Wright algorithm demonstrates superior performance with a mean travel time of 697.95, while Genetic Algorithm exhibits the highest values averaging 1420.64. This represents an improvement of approximately 50.9\% when comparing the best to worst performing algorithms. The relative ranking of algorithms remains largely consistent across different node configurations, suggesting robust performance characteristics.

These findings support the hypothesis that algorithmic choice significantly impacts system performance metrics. Future work should incorporate statistical significance testing and confidence intervals to strengthen these comparative conclusions. Additionally, examining the computational complexity trade-offs between algorithms would provide valuable context for practical deployment decisions.

\begin{figure}[!htbp]
\centering
\includegraphics[width=0.95\linewidth]{images/MT2TE_and_Other_Program_Traffic/(b)_Energy.png}
\caption{(b) Energy}
\label{fig:MT2TE_and_Other_Program_Traffic_b__Energy}
\end{figure}

This grouped bar chart presents a comparative analysis of energy across multiple algorithms within the MT2TE and Other Program Traffic experimental configuration. The x-axis represents the number of nodes in the network, while the y-axis quantifies the energy metric. The legend identifies 4 distinct algorithms: Genetic Algorithm, EVRPBSS, Ant Colony, Clarke and Wright algorithm. This visualization enables direct comparison of algorithmic performance under identical network conditions.

Quantitative analysis reveals significant performance disparities among the evaluated algorithms. Clarke and Wright algorithm demonstrates superior performance with a mean energy of 697.95, while Genetic Algorithm exhibits the highest values averaging 1420.64. This represents an improvement of approximately 50.9\% when comparing the best to worst performing algorithms. The relative ranking of algorithms remains largely consistent across different node configurations, suggesting robust performance characteristics.

These findings support the hypothesis that algorithmic choice significantly impacts system performance metrics. Future work should incorporate statistical significance testing and confidence intervals to strengthen these comparative conclusions. Additionally, examining the computational complexity trade-offs between algorithms would provide valuable context for practical deployment decisions.

\begin{figure}[!htbp]
\centering
\includegraphics[width=0.95\linewidth]{images/MT2TE_and_Other_Program_Traffic/(c)_Distance.png}
\caption{(c) Distance}
\label{fig:MT2TE_and_Other_Program_Traffic_c__Distance}
\end{figure}

This grouped bar chart presents a comparative analysis of distance across multiple algorithms within the MT2TE and Other Program Traffic experimental configuration. The x-axis represents the number of nodes in the network, while the y-axis quantifies the distance metric. The legend identifies 4 distinct algorithms: Genetic Algorithm, EVRPBSS, Ant Colony, Clarke and Wright algorithm. This visualization enables direct comparison of algorithmic performance under identical network conditions.

Quantitative analysis reveals significant performance disparities among the evaluated algorithms. Clarke and Wright algorithm demonstrates superior performance with a mean distance of 697.95, while Genetic Algorithm exhibits the highest values averaging 1420.64. This represents an improvement of approximately 50.9\% when comparing the best to worst performing algorithms. The relative ranking of algorithms remains largely consistent across different node configurations, suggesting robust performance characteristics.

These findings support the hypothesis that algorithmic choice significantly impacts system performance metrics. Future work should incorporate statistical significance testing and confidence intervals to strengthen these comparative conclusions. Additionally, examining the computational complexity trade-offs between algorithms would provide valuable context for practical deployment decisions.

\begin{figure}[!htbp]
\centering
\includegraphics[width=0.95\linewidth]{images/MT2TE_and_Other_Program_Traffic/(d)_Total_Module_Swapped.png}
\caption{(d) Total Module Swapped}
\label{fig:MT2TE_and_Other_Program_Traffic_d__Total_Module_Swapped}
\end{figure}

This grouped bar chart presents a comparative analysis of total module swapped across multiple algorithms within the MT2TE and Other Program Traffic experimental configuration. The x-axis represents the number of nodes in the network, while the y-axis quantifies the total module swapped metric. The legend identifies 4 distinct algorithms: Genetic Algorithm, EVRPBSS, Ant Colony, Clarke and Wright algorithm. This visualization enables direct comparison of algorithmic performance under identical network conditions.

Quantitative analysis reveals significant performance disparities among the evaluated algorithms. Clarke and Wright algorithm demonstrates superior performance with a mean total module swapped of 697.95, while Genetic Algorithm exhibits the highest values averaging 1420.64. This represents an improvement of approximately 50.9\% when comparing the best to worst performing algorithms. The relative ranking of algorithms remains largely consistent across different node configurations, suggesting robust performance characteristics.

These findings support the hypothesis that algorithmic choice significantly impacts system performance metrics. Future work should incorporate statistical significance testing and confidence intervals to strengthen these comparative conclusions. Additionally, examining the computational complexity trade-offs between algorithms would provide valuable context for practical deployment decisions.

\begin{figure}[!htbp]
\centering
\includegraphics[width=0.95\linewidth]{images/MT2TE_and_Other_Program_Traffic/(e)_Execution_Time.png}
\caption{(e) Execution Time}
\label{fig:MT2TE_and_Other_Program_Traffic_e__Execution_Time}
\end{figure}

This grouped bar chart presents a comparative analysis of execution time across multiple algorithms within the MT2TE and Other Program Traffic experimental configuration. The x-axis represents the number of nodes in the network, while the y-axis quantifies the execution time metric. The legend identifies 4 distinct algorithms: Genetic Algorithm, EVRPBSS, Ant Colony, Clarke and Wright algorithm. This visualization enables direct comparison of algorithmic performance under identical network conditions.

Quantitative analysis reveals significant performance disparities among the evaluated algorithms. Clarke and Wright algorithm demonstrates superior performance with a mean execution time of 697.95, while Genetic Algorithm exhibits the highest values averaging 1420.64. This represents an improvement of approximately 50.9\% when comparing the best to worst performing algorithms. The relative ranking of algorithms remains largely consistent across different node configurations, suggesting robust performance characteristics.

These findings support the hypothesis that algorithmic choice significantly impacts system performance metrics. Future work should incorporate statistical significance testing and confidence intervals to strengthen these comparative conclusions. Additionally, examining the computational complexity trade-offs between algorithms would provide valuable context for practical deployment decisions.

\begin{figure}[!htbp]
\centering
\includegraphics[width=0.95\linewidth]{images/MT2TE_and_Other_Program_Traffic/(f)_Execution_Time.png}
\caption{(f) Execution Time}
\label{fig:MT2TE_and_Other_Program_Traffic_f__Execution_Time}
\end{figure}

This grouped bar chart presents a comparative analysis of execution time across multiple algorithms within the MT2TE and Other Program Traffic experimental configuration. The x-axis represents the number of nodes in the network, while the y-axis quantifies the execution time metric. The legend identifies 4 distinct algorithms: Genetic Algorithm, EVRPBSS, Ant Colony, Clarke and Wright algorithm. This visualization enables direct comparison of algorithmic performance under identical network conditions.

Quantitative analysis reveals significant performance disparities among the evaluated algorithms. Clarke and Wright algorithm demonstrates superior performance with a mean execution time of 862.43, while Genetic Algorithm exhibits the highest values averaging 2037.20. This represents an improvement of approximately 57.7\% when comparing the best to worst performing algorithms. The relative ranking of algorithms remains largely consistent across different node configurations, suggesting robust performance characteristics.

These findings support the hypothesis that algorithmic choice significantly impacts system performance metrics. Future work should incorporate statistical significance testing and confidence intervals to strengthen these comparative conclusions. Additionally, examining the computational complexity trade-offs between algorithms would provide valuable context for practical deployment decisions.


\clearpage

\subsection{MT2TE Multi-Line Charts}

\begin{figure}[!htbp]
\centering
\includegraphics[width=0.95\linewidth]{images/MT2TE_Multi-Line_Charts/(a)_Travel_time_bar_chart.png}
\caption{(a) Travel time}
\label{fig:MT2TE_Multi_Line_Charts_a__Travel_time_bar_chart}
\end{figure}

This bar chart presents the relationship between 4 module and 5 module for the MT2TE Multi-Line Charts experimental scenario. The x-axis displays 4 module values ranging from 3287.54 to 14159.32, while the y-axis quantifies 5 module. The single-series visualization facilitates analysis of how the dependent variable responds to changes in the independent parameter setting.

Analysis of the plotted data reveals that 5 module ranges from 3287.54 (at 4 module = 3287.54) to 13381.98 (at 4 module = 14159.32), representing a span of 10094.44 units. The overall trend is increasing, with values rising from 3287.54 at the initial setting to 13381.98 at the final setting. 

These results indicate that 4 module configuration meaningfully impacts 5 module in this experimental context. The substantial variation observed (coefficient of variation exceeding 20\%) suggests that parameter tuning could yield significant performance improvements. Confidence in these findings is high given the direct correspondence between CSV data and plotted values. Future analysis should consider incorporating error bars representing variance across multiple experimental runs to strengthen statistical validity.

\begin{figure}[!htbp]
\centering
\includegraphics[width=0.95\linewidth]{images/MT2TE_Multi-Line_Charts/(b)_Energy_consumed_bar_chart.png}
\caption{(b) Energy consumed}
\label{fig:MT2TE_Multi_Line_Charts_b__Energy_consumed_bar_chart}
\end{figure}

This bar chart presents the relationship between 4 module and 5 module for the MT2TE Multi-Line Charts experimental scenario. The x-axis displays 4 module values ranging from 323.496 to 2154.006, while the y-axis quantifies 5 module. The single-series visualization facilitates analysis of how the dependent variable responds to changes in the independent parameter setting.

Analysis of the plotted data reveals that 5 module ranges from 323.50 (at 4 module = 323.496) to 2057.45 (at 4 module = 2154.006), representing a span of 1733.96 units. The overall trend is increasing, with values rising from 323.50 at the initial setting to 2057.45 at the final setting. 

These results indicate that 4 module configuration meaningfully impacts 5 module in this experimental context. The substantial variation observed (coefficient of variation exceeding 20\%) suggests that parameter tuning could yield significant performance improvements. Confidence in these findings is high given the direct correspondence between CSV data and plotted values. Future analysis should consider incorporating error bars representing variance across multiple experimental runs to strengthen statistical validity.

\begin{figure}[!htbp]
\centering
\includegraphics[width=0.95\linewidth]{images/MT2TE_Multi-Line_Charts/(c)_Distance_covered_bar_chart.png}
\caption{(c) Distance covered}
\label{fig:MT2TE_Multi_Line_Charts_c__Distance_covered_bar_chart}
\end{figure}

This bar chart presents the relationship between 4 module and 5 module for the MT2TE Multi-Line Charts experimental scenario. The x-axis displays 4 module values ranging from 2138.91 to 9234.69, while the y-axis quantifies 5 module. The single-series visualization facilitates analysis of how the dependent variable responds to changes in the independent parameter setting.

Analysis of the plotted data reveals that 5 module ranges from 2138.91 (at 4 module = 2138.91) to 8652.17 (at 4 module = 9234.69), representing a span of 6513.26 units. The overall trend is increasing, with values rising from 2138.91 at the initial setting to 8652.17 at the final setting. 

These results indicate that 4 module configuration meaningfully impacts 5 module in this experimental context. The substantial variation observed (coefficient of variation exceeding 20\%) suggests that parameter tuning could yield significant performance improvements. Confidence in these findings is high given the direct correspondence between CSV data and plotted values. Future analysis should consider incorporating error bars representing variance across multiple experimental runs to strengthen statistical validity.

\begin{figure}[!htbp]
\centering
\includegraphics[width=0.95\linewidth]{images/MT2TE_Multi-Line_Charts/(d)_Run_time_bar_chart.png}
\caption{(d) Run time}
\label{fig:MT2TE_Multi_Line_Charts_d__Run_time_bar_chart}
\end{figure}

This bar chart presents the relationship between 4 module and 5 module for the MT2TE Multi-Line Charts experimental scenario. The x-axis displays 4 module values ranging from 58.526 to 1545.515, while the y-axis quantifies 5 module. The single-series visualization facilitates analysis of how the dependent variable responds to changes in the independent parameter setting.

Analysis of the plotted data reveals that 5 module ranges from 29.02 (at 4 module = 58.526) to 1236.34 (at 4 module = 1545.515), representing a span of 1207.32 units. The overall trend is increasing, with values rising from 29.02 at the initial setting to 1236.34 at the final setting. 

These results indicate that 4 module configuration meaningfully impacts 5 module in this experimental context. The substantial variation observed (coefficient of variation exceeding 20\%) suggests that parameter tuning could yield significant performance improvements. Confidence in these findings is high given the direct correspondence between CSV data and plotted values. Future analysis should consider incorporating error bars representing variance across multiple experimental runs to strengthen statistical validity.

\begin{figure}[!htbp]
\centering
\includegraphics[width=0.95\linewidth]{images/MT2TE_Multi-Line_Charts/(e)_Module_swapped_bar_chart.png}
\caption{(e) Module swapped}
\label{fig:MT2TE_Multi_Line_Charts_e__Module_swapped_bar_chart}
\end{figure}

This bar chart presents the relationship between 4 module and 4 module for the MT2TE Multi-Line Charts experimental scenario. The x-axis displays 4 module values ranging from 15 to 106, while the y-axis quantifies 4 module. The single-series visualization facilitates analysis of how the dependent variable responds to changes in the independent parameter setting.

Analysis of the plotted data reveals that 4 module ranges from 15.00 (at 4 module = 15) to 106.00 (at 4 module = 106), representing a span of 91.00 units. The overall trend is increasing, with values rising from 15.00 at the initial setting to 106.00 at the final setting. 

These results indicate that 4 module configuration meaningfully impacts 4 module in this experimental context. The substantial variation observed (coefficient of variation exceeding 20\%) suggests that parameter tuning could yield significant performance improvements. Confidence in these findings is high given the direct correspondence between CSV data and plotted values. Future analysis should consider incorporating error bars representing variance across multiple experimental runs to strengthen statistical validity.


\clearpage

\subsection{MT2TE Node Change}

\begin{figure}[!htbp]
\centering
\includegraphics[width=0.95\linewidth]{images/MT2TE_Node_Change/(a)_Travel_time_bar_chart.png}
\caption{(a) Travel time}
\label{fig:MT2TE_Node_Change_a__Travel_time_bar_chart}
\end{figure}

This bar chart presents the relationship between number of nodes and total travel time for the MT2TE Node Change experimental scenario. The x-axis displays number of nodes values ranging from 500 to 2000, while the y-axis quantifies total travel time. The single-series visualization facilitates analysis of how the dependent variable responds to changes in the independent parameter setting.

Analysis of the plotted data reveals that total travel time ranges from 3228.14 (at number of nodes = 500) to 13338.35 (at number of nodes = 2000), representing a span of 10110.21 units. The overall trend is increasing, with values rising from 3228.14 at the initial setting to 13338.35 at the final setting. 

These results indicate that number of nodes configuration meaningfully impacts total travel time in this experimental context. The substantial variation observed (coefficient of variation exceeding 20\%) suggests that parameter tuning could yield significant performance improvements. Confidence in these findings is high given the direct correspondence between CSV data and plotted values. Future analysis should consider incorporating error bars representing variance across multiple experimental runs to strengthen statistical validity.

\begin{figure}[!htbp]
\centering
\includegraphics[width=0.95\linewidth]{images/MT2TE_Node_Change/(b)_Energy_consumed_bar_chart.png}
\caption{(b) Energy consumed}
\label{fig:MT2TE_Node_Change_b__Energy_consumed_bar_chart}
\end{figure}

This bar chart presents the relationship between number of nodes and total energy consumed for the MT2TE Node Change experimental scenario. The x-axis displays number of nodes values ranging from 500 to 2000, while the y-axis quantifies total energy consumed. The single-series visualization facilitates analysis of how the dependent variable responds to changes in the independent parameter setting.

Analysis of the plotted data reveals that total energy consumed ranges from 317.62 (at number of nodes = 500) to 2036.78 (at number of nodes = 2000), representing a span of 1719.17 units. The overall trend is increasing, with values rising from 317.62 at the initial setting to 2036.78 at the final setting. 

These results indicate that number of nodes configuration meaningfully impacts total energy consumed in this experimental context. The substantial variation observed (coefficient of variation exceeding 20\%) suggests that parameter tuning could yield significant performance improvements. Confidence in these findings is high given the direct correspondence between CSV data and plotted values. Future analysis should consider incorporating error bars representing variance across multiple experimental runs to strengthen statistical validity.

\begin{figure}[!htbp]
\centering
\includegraphics[width=0.95\linewidth]{images/MT2TE_Node_Change/(c)_Distance_covered_bar_chart.png}
\caption{(c) Distance covered}
\label{fig:MT2TE_Node_Change_c__Distance_covered_bar_chart}
\end{figure}

This bar chart presents the relationship between number of nodes and total distance covered for the MT2TE Node Change experimental scenario. The x-axis displays number of nodes values ranging from 500 to 2000, while the y-axis quantifies total distance covered. The single-series visualization facilitates analysis of how the dependent variable responds to changes in the independent parameter setting.

Analysis of the plotted data reveals that total distance covered ranges from 2105.32 (at number of nodes = 500) to 8672.23 (at number of nodes = 2000), representing a span of 6566.90 units. The overall trend is increasing, with values rising from 2105.32 at the initial setting to 8672.23 at the final setting. 

These results indicate that number of nodes configuration meaningfully impacts total distance covered in this experimental context. The substantial variation observed (coefficient of variation exceeding 20\%) suggests that parameter tuning could yield significant performance improvements. Confidence in these findings is high given the direct correspondence between CSV data and plotted values. Future analysis should consider incorporating error bars representing variance across multiple experimental runs to strengthen statistical validity.

\begin{figure}[!htbp]
\centering
\includegraphics[width=0.95\linewidth]{images/MT2TE_Node_Change/(d)_Run_time_bar_chart.png}
\caption{(d) Run time}
\label{fig:MT2TE_Node_Change_d__Run_time_bar_chart}
\end{figure}

This bar chart presents the relationship between number of nodes and run time for the MT2TE Node Change experimental scenario. The x-axis displays number of nodes values ranging from 500 to 2000, while the y-axis quantifies run time. The single-series visualization facilitates analysis of how the dependent variable responds to changes in the independent parameter setting.

Analysis of the plotted data reveals that run time ranges from 42.87 (at number of nodes = 500) to 1257.88 (at number of nodes = 2000), representing a span of 1215.01 units. The overall trend is increasing, with values rising from 42.87 at the initial setting to 1257.88 at the final setting. 

These results indicate that number of nodes configuration meaningfully impacts run time in this experimental context. The substantial variation observed (coefficient of variation exceeding 20\%) suggests that parameter tuning could yield significant performance improvements. Confidence in these findings is high given the direct correspondence between CSV data and plotted values. Future analysis should consider incorporating error bars representing variance across multiple experimental runs to strengthen statistical validity.

\begin{figure}[!htbp]
\centering
\includegraphics[width=0.95\linewidth]{images/MT2TE_Node_Change/(e)_Module_swapped_bar_chart.png}
\caption{(e) Module swapped}
\label{fig:MT2TE_Node_Change_e__Module_swapped_bar_chart}
\end{figure}

This bar chart presents the relationship between number of nodes and total module swapped for the MT2TE Node Change experimental scenario. The x-axis displays number of nodes values ranging from 500 to 2000, while the y-axis quantifies total module swapped. The single-series visualization facilitates analysis of how the dependent variable responds to changes in the independent parameter setting.

Analysis of the plotted data reveals that total module swapped ranges from 14.50 (at number of nodes = 500) to 98.75 (at number of nodes = 2000), representing a span of 84.25 units. The overall trend is increasing, with values rising from 14.50 at the initial setting to 98.75 at the final setting. 

These results indicate that number of nodes configuration meaningfully impacts total module swapped in this experimental context. The substantial variation observed (coefficient of variation exceeding 20\%) suggests that parameter tuning could yield significant performance improvements. Confidence in these findings is high given the direct correspondence between CSV data and plotted values. Future analysis should consider incorporating error bars representing variance across multiple experimental runs to strengthen statistical validity.


\clearpage

\subsection{Optimum and MT2TE Module Change}

\begin{figure}[!htbp]
\centering
\includegraphics[width=0.95\linewidth]{images/Optimum_and_MT2TE_Module_Change/(a)_Travel_Time.png}
\caption{(a) Travel Time}
\label{fig:Optimum_and_MT2TE_Module_Change_a__Travel_Time}
\end{figure}

This bar chart presents the relationship between module and optimum for the Optimum and MT2TE Module Change experimental scenario. The x-axis displays module values ranging from 3 to 6, while the y-axis quantifies optimum. The single-series visualization facilitates analysis of how the dependent variable responds to changes in the independent parameter setting.

Analysis of the plotted data reveals that optimum ranges from 111.20 (at module = 3) to 111.20 (at module = 3), representing a span of 0.00 units. The values remain relatively stable across the parameter range, with minimal net change between initial (111.20) and final (111.20) settings. 

These results indicate that module configuration meaningfully impacts optimum in this experimental context. The relatively modest variation suggests that this parameter has limited influence on the measured metric within the tested range. Confidence in these findings is high given the direct correspondence between CSV data and plotted values. Future analysis should consider incorporating error bars representing variance across multiple experimental runs to strengthen statistical validity.

\begin{figure}[!htbp]
\centering
\includegraphics[width=0.95\linewidth]{images/Optimum_and_MT2TE_Module_Change/(b)_Energy.png}
\caption{(b) Energy}
\label{fig:Optimum_and_MT2TE_Module_Change_b__Energy}
\end{figure}

This bar chart presents the relationship between module and optimum for the Optimum and MT2TE Module Change experimental scenario. The x-axis displays module values ranging from 3 to 6, while the y-axis quantifies optimum. The single-series visualization facilitates analysis of how the dependent variable responds to changes in the independent parameter setting.

Analysis of the plotted data reveals that optimum ranges from 9.01 (at module = 3) to 9.01 (at module = 3), representing a span of 0.00 units. The values remain relatively stable across the parameter range, with minimal net change between initial (9.01) and final (9.01) settings. 

These results indicate that module configuration meaningfully impacts optimum in this experimental context. The relatively modest variation suggests that this parameter has limited influence on the measured metric within the tested range. Confidence in these findings is high given the direct correspondence between CSV data and plotted values. Future analysis should consider incorporating error bars representing variance across multiple experimental runs to strengthen statistical validity.

\begin{figure}[!htbp]
\centering
\includegraphics[width=0.95\linewidth]{images/Optimum_and_MT2TE_Module_Change/(c)_Distance.png}
\caption{(c) Distance}
\label{fig:Optimum_and_MT2TE_Module_Change_c__Distance}
\end{figure}

This bar chart presents the relationship between module and optimum for the Optimum and MT2TE Module Change experimental scenario. The x-axis displays module values ranging from 3 to 6, while the y-axis quantifies optimum. The single-series visualization facilitates analysis of how the dependent variable responds to changes in the independent parameter setting.

Analysis of the plotted data reveals that optimum ranges from 74.38 (at module = 3) to 74.38 (at module = 3), representing a span of 0.00 units. The values remain relatively stable across the parameter range, with minimal net change between initial (74.38) and final (74.38) settings. 

These results indicate that module configuration meaningfully impacts optimum in this experimental context. The relatively modest variation suggests that this parameter has limited influence on the measured metric within the tested range. Confidence in these findings is high given the direct correspondence between CSV data and plotted values. Future analysis should consider incorporating error bars representing variance across multiple experimental runs to strengthen statistical validity.


\clearpage

\subsection{Optimum and MT2TE Node Change}

\begin{figure}[!htbp]
\centering
\includegraphics[width=0.95\linewidth]{images/Optimum_and_MT2TE_Node_Change/(a)_Total_Travel_Time_Optimum_and_MT2TE_Node_Change.png}
\caption{(a) Total Travel Time Optimum and MT2TE Node Change}
\label{fig:Optimum_and_MT2TE_Node_Change_a__Total_Travel_Time_Optimum_and_MT2TE_Node_Change}
\end{figure}

This bar chart presents the relationship between node and optimum for the Optimum and MT2TE Node Change experimental scenario. The x-axis displays node values ranging from 12 to 24, while the y-axis quantifies optimum. The single-series visualization facilitates analysis of how the dependent variable responds to changes in the independent parameter setting.

Analysis of the plotted data reveals that optimum ranges from 106.61 (at node = 12) to 153.26 (at node = 24), representing a span of 46.65 units. The overall trend is increasing, with values rising from 106.61 at the initial setting to 153.26 at the final setting. 

These results indicate that node configuration meaningfully impacts optimum in this experimental context. The substantial variation observed (coefficient of variation exceeding 20\%) suggests that parameter tuning could yield significant performance improvements. Confidence in these findings is high given the direct correspondence between CSV data and plotted values. Future analysis should consider incorporating error bars representing variance across multiple experimental runs to strengthen statistical validity.

\begin{figure}[!htbp]
\centering
\includegraphics[width=0.95\linewidth]{images/Optimum_and_MT2TE_Node_Change/(b)_Total_Energy_Consumed_Optimum_and_MT2TE_Node_Change.png}
\caption{(b) Total Energy Consumed Optimum and MT2TE Node Change}
\label{fig:Optimum_and_MT2TE_Node_Change_b__Total_Energy_Consumed_Optimum_and_MT2TE_Node_Change}
\end{figure}

This bar chart presents the relationship between node and optimum for the Optimum and MT2TE Node Change experimental scenario. The x-axis displays node values ranging from 12 to 24, while the y-axis quantifies optimum. The single-series visualization facilitates analysis of how the dependent variable responds to changes in the independent parameter setting.

Analysis of the plotted data reveals that optimum ranges from 8.72 (at node = 12) to 11.84 (at node = 24), representing a span of 3.12 units. The overall trend is increasing, with values rising from 8.72 at the initial setting to 11.84 at the final setting. 

These results indicate that node configuration meaningfully impacts optimum in this experimental context. The substantial variation observed (coefficient of variation exceeding 20\%) suggests that parameter tuning could yield significant performance improvements. Confidence in these findings is high given the direct correspondence between CSV data and plotted values. Future analysis should consider incorporating error bars representing variance across multiple experimental runs to strengthen statistical validity.

\begin{figure}[!htbp]
\centering
\includegraphics[width=0.95\linewidth]{images/Optimum_and_MT2TE_Node_Change/(c)_Total_Distance_Optimum_and_MT2TE_Node_Change.png}
\caption{(c) Total Distance Optimum and MT2TE Node Change}
\label{fig:Optimum_and_MT2TE_Node_Change_c__Total_Distance_Optimum_and_MT2TE_Node_Change}
\end{figure}

This bar chart presents the relationship between node and optimum for the Optimum and MT2TE Node Change experimental scenario. The x-axis displays node values ranging from 12 to 24, while the y-axis quantifies optimum. The single-series visualization facilitates analysis of how the dependent variable responds to changes in the independent parameter setting.

Analysis of the plotted data reveals that optimum ranges from 65.57 (at node = 12) to 98.75 (at node = 24), representing a span of 33.18 units. The overall trend is increasing, with values rising from 65.57 at the initial setting to 98.75 at the final setting. 

These results indicate that node configuration meaningfully impacts optimum in this experimental context. The substantial variation observed (coefficient of variation exceeding 20\%) suggests that parameter tuning could yield significant performance improvements. Confidence in these findings is high given the direct correspondence between CSV data and plotted values. Future analysis should consider incorporating error bars representing variance across multiple experimental runs to strengthen statistical validity.

\begin{figure}[!htbp]
\centering
\includegraphics[width=0.95\linewidth]{images/Optimum_and_MT2TE_Node_Change/(d)_Execution_Time_Optimum_and_MT2TE_Node_Change.png}
\caption{(d) Execution Time Optimum and MT2TE Node Change}
\label{fig:Optimum_and_MT2TE_Node_Change_d__Execution_Time_Optimum_and_MT2TE_Node_Change}
\end{figure}

This bar chart presents the relationship between node and optimum for the Optimum and MT2TE Node Change experimental scenario. The x-axis displays node values ranging from 12 to 28, while the y-axis quantifies optimum. The single-series visualization facilitates analysis of how the dependent variable responds to changes in the independent parameter setting.

Analysis of the plotted data reveals that optimum ranges from 12.00 (at node = 12) to 298800.00 (at node = 28), representing a span of 298788.00 units. The overall trend is increasing, with values rising from 12.00 at the initial setting to 298800.00 at the final setting. 

These results indicate that node configuration meaningfully impacts optimum in this experimental context. The substantial variation observed (coefficient of variation exceeding 20\%) suggests that parameter tuning could yield significant performance improvements. Confidence in these findings is high given the direct correspondence between CSV data and plotted values. Future analysis should consider incorporating error bars representing variance across multiple experimental runs to strengthen statistical validity.


\clearpage

\subsection{Optimum and MT2TE Swap Time Change}

\begin{figure}[!htbp]
\centering
\includegraphics[width=0.95\linewidth]{images/Optimum_and_MT2TE_Swap_Time_Change/(a)_Travel_Time.png}
\caption{(a) Travel Time}
\label{fig:Optimum_and_MT2TE_Swap_Time_Change_a__Travel_Time}
\end{figure}

This bar chart presents the relationship between swaptime and optimum for the Optimum and MT2TE Swap Time Change experimental scenario. The x-axis displays swaptime values ranging from 1 to 4, while the y-axis quantifies optimum. The single-series visualization facilitates analysis of how the dependent variable responds to changes in the independent parameter setting.

Analysis of the plotted data reveals that optimum ranges from 111.20 (at swaptime = 1) to 111.20 (at swaptime = 2), representing a span of 0.00 units. The overall trend is increasing, with values rising from 111.20 at the initial setting to 111.20 at the final setting. 

These results indicate that swaptime configuration meaningfully impacts optimum in this experimental context. The relatively modest variation suggests that this parameter has limited influence on the measured metric within the tested range. Confidence in these findings is high given the direct correspondence between CSV data and plotted values. Future analysis should consider incorporating error bars representing variance across multiple experimental runs to strengthen statistical validity.

\begin{figure}[!htbp]
\centering
\includegraphics[width=0.95\linewidth]{images/Optimum_and_MT2TE_Swap_Time_Change/(b)_Energy.png}
\caption{(b) Energy}
\label{fig:Optimum_and_MT2TE_Swap_Time_Change_b__Energy}
\end{figure}

This bar chart presents the relationship between swaptime and optimum for the Optimum and MT2TE Swap Time Change experimental scenario. The x-axis displays swaptime values ranging from 1 to 4, while the y-axis quantifies optimum. The single-series visualization facilitates analysis of how the dependent variable responds to changes in the independent parameter setting.

Analysis of the plotted data reveals that optimum ranges from 9.01 (at swaptime = 1) to 9.01 (at swaptime = 1), representing a span of 0.00 units. The values remain relatively stable across the parameter range, with minimal net change between initial (9.01) and final (9.01) settings. 

These results indicate that swaptime configuration meaningfully impacts optimum in this experimental context. The relatively modest variation suggests that this parameter has limited influence on the measured metric within the tested range. Confidence in these findings is high given the direct correspondence between CSV data and plotted values. Future analysis should consider incorporating error bars representing variance across multiple experimental runs to strengthen statistical validity.

\begin{figure}[!htbp]
\centering
\includegraphics[width=0.95\linewidth]{images/Optimum_and_MT2TE_Swap_Time_Change/(c)_Distance.png}
\caption{(c) Distance}
\label{fig:Optimum_and_MT2TE_Swap_Time_Change_c__Distance}
\end{figure}

This bar chart presents the relationship between swaptime and optimum for the Optimum and MT2TE Swap Time Change experimental scenario. The x-axis displays swaptime values ranging from 1 to 4, while the y-axis quantifies optimum. The single-series visualization facilitates analysis of how the dependent variable responds to changes in the independent parameter setting.

Analysis of the plotted data reveals that optimum ranges from 74.38 (at swaptime = 1) to 74.38 (at swaptime = 1), representing a span of 0.00 units. The values remain relatively stable across the parameter range, with minimal net change between initial (74.38) and final (74.38) settings. 

These results indicate that swaptime configuration meaningfully impacts optimum in this experimental context. The relatively modest variation suggests that this parameter has limited influence on the measured metric within the tested range. Confidence in these findings is high given the direct correspondence between CSV data and plotted values. Future analysis should consider incorporating error bars representing variance across multiple experimental runs to strengthen statistical validity.


\clearpage

\subsection{Optimum and MT2TE Threshold Change}

\begin{figure}[!htbp]
\centering
\includegraphics[width=0.95\linewidth]{images/Optimum_and_MT2TE_Threshold_Change/(a)_Travel_Time.png}
\caption{(a) Travel Time}
\label{fig:Optimum_and_MT2TE_Threshold_Change_a__Travel_Time}
\end{figure}

This bar chart presents the relationship between threshold and optimum for the Optimum and MT2TE Threshold Change experimental scenario. The x-axis displays threshold values ranging from 5 to 20, while the y-axis quantifies optimum. The single-series visualization facilitates analysis of how the dependent variable responds to changes in the independent parameter setting.

Analysis of the plotted data reveals that optimum ranges from 111.20 (at threshold = 5) to 111.20 (at threshold = 5), representing a span of 0.00 units. The values remain relatively stable across the parameter range, with minimal net change between initial (111.20) and final (111.20) settings. 

These results indicate that threshold configuration meaningfully impacts optimum in this experimental context. The relatively modest variation suggests that this parameter has limited influence on the measured metric within the tested range. Confidence in these findings is high given the direct correspondence between CSV data and plotted values. Future analysis should consider incorporating error bars representing variance across multiple experimental runs to strengthen statistical validity.

\begin{figure}[!htbp]
\centering
\includegraphics[width=0.95\linewidth]{images/Optimum_and_MT2TE_Threshold_Change/(b)_Energy.png}
\caption{(b) Energy}
\label{fig:Optimum_and_MT2TE_Threshold_Change_b__Energy}
\end{figure}

This bar chart presents the relationship between threshold and optimum for the Optimum and MT2TE Threshold Change experimental scenario. The x-axis displays threshold values ranging from 5 to 20, while the y-axis quantifies optimum. The single-series visualization facilitates analysis of how the dependent variable responds to changes in the independent parameter setting.

Analysis of the plotted data reveals that optimum ranges from 9.01 (at threshold = 5) to 9.01 (at threshold = 5), representing a span of 0.00 units. The values remain relatively stable across the parameter range, with minimal net change between initial (9.01) and final (9.01) settings. 

These results indicate that threshold configuration meaningfully impacts optimum in this experimental context. The relatively modest variation suggests that this parameter has limited influence on the measured metric within the tested range. Confidence in these findings is high given the direct correspondence between CSV data and plotted values. Future analysis should consider incorporating error bars representing variance across multiple experimental runs to strengthen statistical validity.

\begin{figure}[!htbp]
\centering
\includegraphics[width=0.95\linewidth]{images/Optimum_and_MT2TE_Threshold_Change/(c)_Distance.png}
\caption{(c) Distance}
\label{fig:Optimum_and_MT2TE_Threshold_Change_c__Distance}
\end{figure}

This bar chart presents the relationship between threshold and optimum for the Optimum and MT2TE Threshold Change experimental scenario. The x-axis displays threshold values ranging from 5 to 20, while the y-axis quantifies optimum. The single-series visualization facilitates analysis of how the dependent variable responds to changes in the independent parameter setting.

Analysis of the plotted data reveals that optimum ranges from 74.38 (at threshold = 5) to 74.38 (at threshold = 5), representing a span of 0.00 units. The values remain relatively stable across the parameter range, with minimal net change between initial (74.38) and final (74.38) settings. 

These results indicate that threshold configuration meaningfully impacts optimum in this experimental context. The relatively modest variation suggests that this parameter has limited influence on the measured metric within the tested range. Confidence in these findings is high given the direct correspondence between CSV data and plotted values. Future analysis should consider incorporating error bars representing variance across multiple experimental runs to strengthen statistical validity.


\clearpage

\subsection{Optimum and MT2TE Traffic Change}

\begin{figure}[!htbp]
\centering
\includegraphics[width=0.95\linewidth]{images/Optimum_and_MT2TE_Traffic_Change/(a)_Travel_Time.png}
\caption{(a) Travel Time}
\label{fig:Optimum_and_MT2TE_Traffic_Change_a__Travel_Time}
\end{figure}

This bar chart presents the relationship between trafficlevel and optimum for the Optimum and MT2TE Traffic Change experimental scenario. The x-axis displays trafficlevel values ranging from Low to High, while the y-axis quantifies optimum. The single-series visualization facilitates analysis of how the dependent variable responds to changes in the independent parameter setting.

Analysis of the plotted data reveals that optimum ranges from 94.72 (at trafficlevel = Low) to 133.80 (at trafficlevel = High), representing a span of 39.08 units. The overall trend is increasing, with values rising from 94.72 at the initial setting to 133.80 at the final setting. 

These results indicate that trafficlevel configuration meaningfully impacts optimum in this experimental context. The substantial variation observed (coefficient of variation exceeding 20\%) suggests that parameter tuning could yield significant performance improvements. Confidence in these findings is high given the direct correspondence between CSV data and plotted values. Future analysis should consider incorporating error bars representing variance across multiple experimental runs to strengthen statistical validity.

\begin{figure}[!htbp]
\centering
\includegraphics[width=0.95\linewidth]{images/Optimum_and_MT2TE_Traffic_Change/(b)_Energy.png}
\caption{(b) Energy}
\label{fig:Optimum_and_MT2TE_Traffic_Change_b__Energy}
\end{figure}

This bar chart presents the relationship between trafficlevel and optimum for the Optimum and MT2TE Traffic Change experimental scenario. The x-axis displays trafficlevel values ranging from Low to High, while the y-axis quantifies optimum. The single-series visualization facilitates analysis of how the dependent variable responds to changes in the independent parameter setting.

Analysis of the plotted data reveals that optimum ranges from 7.68 (at trafficlevel = Low) to 11.09 (at trafficlevel = High), representing a span of 3.41 units. The overall trend is increasing, with values rising from 7.68 at the initial setting to 11.09 at the final setting. 

These results indicate that trafficlevel configuration meaningfully impacts optimum in this experimental context. The substantial variation observed (coefficient of variation exceeding 20\%) suggests that parameter tuning could yield significant performance improvements. Confidence in these findings is high given the direct correspondence between CSV data and plotted values. Future analysis should consider incorporating error bars representing variance across multiple experimental runs to strengthen statistical validity.

\begin{figure}[!htbp]
\centering
\includegraphics[width=0.95\linewidth]{images/Optimum_and_MT2TE_Traffic_Change/(c)_Distance.png}
\caption{(c) Distance}
\label{fig:Optimum_and_MT2TE_Traffic_Change_c__Distance}
\end{figure}

This bar chart presents the relationship between traffic and total distance covered for the Optimum and MT2TE Traffic Change experimental scenario. The x-axis displays traffic values ranging from High to Low, while the y-axis quantifies total distance covered. The single-series visualization facilitates analysis of how the dependent variable responds to changes in the independent parameter setting.

Analysis of the plotted data reveals that total distance covered ranges from 8689.81 (at traffic = High) to 8994.35 (at traffic = Low), representing a span of 304.54 units. The overall trend is increasing, with values rising from 8689.81 at the initial setting to 8994.35 at the final setting. 

These results indicate that traffic configuration meaningfully impacts total distance covered in this experimental context. The relatively modest variation suggests that this parameter has limited influence on the measured metric within the tested range. Confidence in these findings is high given the direct correspondence between CSV data and plotted values. Future analysis should consider incorporating error bars representing variance across multiple experimental runs to strengthen statistical validity.


\clearpage

\subsection{Real Time Sumo Module Change}

\begin{figure}[!htbp]
\centering
\includegraphics[width=0.95\linewidth]{images/Real_Time_Sumo_Module_Change/(a)_Travel_Time.png}
\caption{(a) Travel Time}
\label{fig:Real_Time_Sumo_Module_Change_a__Travel_Time}
\end{figure}

This bar chart presents the relationship between modules and total travel time for the Real Time Sumo Module Change experimental scenario. The x-axis displays modules values ranging from 3 to 6, while the y-axis quantifies total travel time. The single-series visualization facilitates analysis of how the dependent variable responds to changes in the independent parameter setting.

Analysis of the plotted data reveals that total travel time ranges from 87.96 (at modules = 3) to 90.06 (at modules = 4), representing a span of 2.10 units. The overall trend is increasing, with values rising from 87.96 at the initial setting to 88.79 at the final setting. 

These results indicate that modules configuration meaningfully impacts total travel time in this experimental context. The relatively modest variation suggests that this parameter has limited influence on the measured metric within the tested range. Confidence in these findings is high given the direct correspondence between CSV data and plotted values. Future analysis should consider incorporating error bars representing variance across multiple experimental runs to strengthen statistical validity.

\begin{figure}[!htbp]
\centering
\includegraphics[width=0.95\linewidth]{images/Real_Time_Sumo_Module_Change/(b)_energy.png}
\caption{(b) energy}
\label{fig:Real_Time_Sumo_Module_Change_b__energy}
\end{figure}

This bar chart presents the relationship between modules and total energy consumed for the Real Time Sumo Module Change experimental scenario. The x-axis displays modules values ranging from 3 to 6, while the y-axis quantifies total energy consumed. The single-series visualization facilitates analysis of how the dependent variable responds to changes in the independent parameter setting.

Analysis of the plotted data reveals that total energy consumed ranges from 12.54 (at modules = 3) to 13.11 (at modules = 4), representing a span of 0.58 units. The overall trend is increasing, with values rising from 12.54 at the initial setting to 12.84 at the final setting. 

These results indicate that modules configuration meaningfully impacts total energy consumed in this experimental context. The relatively modest variation suggests that this parameter has limited influence on the measured metric within the tested range. Confidence in these findings is high given the direct correspondence between CSV data and plotted values. Future analysis should consider incorporating error bars representing variance across multiple experimental runs to strengthen statistical validity.

\begin{figure}[!htbp]
\centering
\includegraphics[width=0.95\linewidth]{images/Real_Time_Sumo_Module_Change/(c)_Distance.png}
\caption{(c) Distance}
\label{fig:Real_Time_Sumo_Module_Change_c__Distance}
\end{figure}

This bar chart presents the relationship between modules and total distance covered for the Real Time Sumo Module Change experimental scenario. The x-axis displays modules values ranging from 3 to 6, while the y-axis quantifies total distance covered. The single-series visualization facilitates analysis of how the dependent variable responds to changes in the independent parameter setting.

Analysis of the plotted data reveals that total distance covered ranges from 60.75 (at modules = 3) to 62.76 (at modules = 4), representing a span of 2.01 units. The overall trend is increasing, with values rising from 60.75 at the initial setting to 62.00 at the final setting. 

These results indicate that modules configuration meaningfully impacts total distance covered in this experimental context. The relatively modest variation suggests that this parameter has limited influence on the measured metric within the tested range. Confidence in these findings is high given the direct correspondence between CSV data and plotted values. Future analysis should consider incorporating error bars representing variance across multiple experimental runs to strengthen statistical validity.

\begin{figure}[!htbp]
\centering
\includegraphics[width=0.95\linewidth]{images/Real_Time_Sumo_Module_Change/(d)_Run_time.png}
\caption{(d) Run time}
\label{fig:Real_Time_Sumo_Module_Change_d__Run_time}
\end{figure}

This bar chart presents the relationship between modules and total travel time for the Real Time Sumo Module Change experimental scenario. The x-axis displays modules values ranging from 3 to 6, while the y-axis quantifies total travel time. The single-series visualization facilitates analysis of how the dependent variable responds to changes in the independent parameter setting.

Analysis of the plotted data reveals that total travel time ranges from 87.96 (at modules = 3) to 90.06 (at modules = 4), representing a span of 2.10 units. The overall trend is increasing, with values rising from 87.96 at the initial setting to 88.79 at the final setting. 

These results indicate that modules configuration meaningfully impacts total travel time in this experimental context. The relatively modest variation suggests that this parameter has limited influence on the measured metric within the tested range. Confidence in these findings is high given the direct correspondence between CSV data and plotted values. Future analysis should consider incorporating error bars representing variance across multiple experimental runs to strengthen statistical validity.


\clearpage

\subsection{Real Time Sumo Swap Time}

\begin{figure}[!htbp]
\centering
\includegraphics[width=0.95\linewidth]{images/Real_Time_Sumo_Swap_Time/(a)_Travel_Time.png}
\caption{(a) Travel Time}
\label{fig:Real_Time_Sumo_Swap_Time_a__Travel_Time}
\end{figure}

This bar chart presents the relationship between swap time (min) and total travel time for the Real Time Sumo Swap Time experimental scenario. The x-axis displays swap time (min) values ranging from 1 to 4, while the y-axis quantifies total travel time. The single-series visualization facilitates analysis of how the dependent variable responds to changes in the independent parameter setting.

Analysis of the plotted data reveals that total travel time ranges from 89.22 (at swap time (min) = 2) to 92.84 (at swap time (min) = 4), representing a span of 3.62 units. The overall trend is increasing, with values rising from 89.94 at the initial setting to 92.84 at the final setting. Notably, the minimum value occurs at an intermediate swap time (min) setting (2), suggesting non-monotonic behavior that warrants further investigation.

These results indicate that swap time (min) configuration meaningfully impacts total travel time in this experimental context. The relatively modest variation suggests that this parameter has limited influence on the measured metric within the tested range. Confidence in these findings is high given the direct correspondence between CSV data and plotted values. Future analysis should consider incorporating error bars representing variance across multiple experimental runs to strengthen statistical validity.

\begin{figure}[!htbp]
\centering
\includegraphics[width=0.95\linewidth]{images/Real_Time_Sumo_Swap_Time/(b)_energy.png}
\caption{(b) energy}
\label{fig:Real_Time_Sumo_Swap_Time_b__energy}
\end{figure}

This bar chart presents the relationship between swap time (min) and total energy consumed for the Real Time Sumo Swap Time experimental scenario. The x-axis displays swap time (min) values ranging from 1 to 4, while the y-axis quantifies total energy consumed. The single-series visualization facilitates analysis of how the dependent variable responds to changes in the independent parameter setting.

Analysis of the plotted data reveals that total energy consumed ranges from 12.57 (at swap time (min) = 3) to 13.18 (at swap time (min) = 4), representing a span of 0.60 units. The overall trend is increasing, with values rising from 12.92 at the initial setting to 13.18 at the final setting. Notably, the minimum value occurs at an intermediate swap time (min) setting (3), suggesting non-monotonic behavior that warrants further investigation.

These results indicate that swap time (min) configuration meaningfully impacts total energy consumed in this experimental context. The relatively modest variation suggests that this parameter has limited influence on the measured metric within the tested range. Confidence in these findings is high given the direct correspondence between CSV data and plotted values. Future analysis should consider incorporating error bars representing variance across multiple experimental runs to strengthen statistical validity.

\begin{figure}[!htbp]
\centering
\includegraphics[width=0.95\linewidth]{images/Real_Time_Sumo_Swap_Time/(c)_Distance.png}
\caption{(c) Distance}
\label{fig:Real_Time_Sumo_Swap_Time_c__Distance}
\end{figure}

This bar chart presents the relationship between swap time (min) and total distance covered for the Real Time Sumo Swap Time experimental scenario. The x-axis displays swap time (min) values ranging from 1 to 4, while the y-axis quantifies total distance covered. The single-series visualization facilitates analysis of how the dependent variable responds to changes in the independent parameter setting.

Analysis of the plotted data reveals that total distance covered ranges from 61.23 (at swap time (min) = 3) to 64.06 (at swap time (min) = 4), representing a span of 2.83 units. The overall trend is increasing, with values rising from 62.42 at the initial setting to 64.06 at the final setting. Notably, the minimum value occurs at an intermediate swap time (min) setting (3), suggesting non-monotonic behavior that warrants further investigation.

These results indicate that swap time (min) configuration meaningfully impacts total distance covered in this experimental context. The relatively modest variation suggests that this parameter has limited influence on the measured metric within the tested range. Confidence in these findings is high given the direct correspondence between CSV data and plotted values. Future analysis should consider incorporating error bars representing variance across multiple experimental runs to strengthen statistical validity.

\begin{figure}[!htbp]
\centering
\includegraphics[width=0.95\linewidth]{images/Real_Time_Sumo_Swap_Time/(d)_Runtime.png}
\caption{(d) Runtime}
\label{fig:Real_Time_Sumo_Swap_Time_d__Runtime}
\end{figure}

This bar chart presents the relationship between swap time (min) and run time for the Real Time Sumo Swap Time experimental scenario. The x-axis displays swap time (min) values ranging from 1 to 4, while the y-axis quantifies run time. The single-series visualization facilitates analysis of how the dependent variable responds to changes in the independent parameter setting.

Analysis of the plotted data reveals that run time ranges from 540.79 (at swap time (min) = 1) to 563.35 (at swap time (min) = 4), representing a span of 22.55 units. The overall trend is increasing, with values rising from 540.79 at the initial setting to 563.35 at the final setting. 

These results indicate that swap time (min) configuration meaningfully impacts run time in this experimental context. The relatively modest variation suggests that this parameter has limited influence on the measured metric within the tested range. Confidence in these findings is high given the direct correspondence between CSV data and plotted values. Future analysis should consider incorporating error bars representing variance across multiple experimental runs to strengthen statistical validity.


\clearpage

\subsection{Real Time Sumo Threshold}

\begin{figure}[!htbp]
\centering
\includegraphics[width=0.95\linewidth]{images/Real_Time_Sumo_Threshold/(a)_Travel_Time.png}
\caption{(a) Travel Time}
\label{fig:Real_Time_Sumo_Threshold_a__Travel_Time}
\end{figure}

This bar chart presents the relationship between threshold and total travel time for the Real Time Sumo Threshold experimental scenario. The x-axis displays threshold values ranging from 5 to 20, while the y-axis quantifies total travel time. The single-series visualization facilitates analysis of how the dependent variable responds to changes in the independent parameter setting.

Analysis of the plotted data reveals that total travel time ranges from 86.63 (at threshold = 5) to 92.25 (at threshold = 15), representing a span of 5.62 units. The overall trend is increasing, with values rising from 86.63 at the initial setting to 89.22 at the final setting. 

These results indicate that threshold configuration meaningfully impacts total travel time in this experimental context. The relatively modest variation suggests that this parameter has limited influence on the measured metric within the tested range. Confidence in these findings is high given the direct correspondence between CSV data and plotted values. Future analysis should consider incorporating error bars representing variance across multiple experimental runs to strengthen statistical validity.

\begin{figure}[!htbp]
\centering
\includegraphics[width=0.95\linewidth]{images/Real_Time_Sumo_Threshold/(b)_Energy.png}
\caption{(b) Energy}
\label{fig:Real_Time_Sumo_Threshold_b__Energy}
\end{figure}

This bar chart presents the relationship between threshold and total energy consumed for the Real Time Sumo Threshold experimental scenario. The x-axis displays threshold values ranging from 5 to 20, while the y-axis quantifies total energy consumed. The single-series visualization facilitates analysis of how the dependent variable responds to changes in the independent parameter setting.

Analysis of the plotted data reveals that total energy consumed ranges from 12.59 (at threshold = 5) to 13.17 (at threshold = 15), representing a span of 0.58 units. The overall trend is increasing, with values rising from 12.59 at the initial setting to 12.93 at the final setting. 

These results indicate that threshold configuration meaningfully impacts total energy consumed in this experimental context. The relatively modest variation suggests that this parameter has limited influence on the measured metric within the tested range. Confidence in these findings is high given the direct correspondence between CSV data and plotted values. Future analysis should consider incorporating error bars representing variance across multiple experimental runs to strengthen statistical validity.

\begin{figure}[!htbp]
\centering
\includegraphics[width=0.95\linewidth]{images/Real_Time_Sumo_Threshold/(c)_Distance.png}
\caption{(c) Distance}
\label{fig:Real_Time_Sumo_Threshold_c__Distance}
\end{figure}

This bar chart presents the relationship between threshold and total distance covered for the Real Time Sumo Threshold experimental scenario. The x-axis displays threshold values ranging from 5 to 20, while the y-axis quantifies total distance covered. The single-series visualization facilitates analysis of how the dependent variable responds to changes in the independent parameter setting.

Analysis of the plotted data reveals that total distance covered ranges from 60.06 (at threshold = 5) to 63.76 (at threshold = 15), representing a span of 3.70 units. The overall trend is increasing, with values rising from 60.06 at the initial setting to 61.70 at the final setting. 

These results indicate that threshold configuration meaningfully impacts total distance covered in this experimental context. The relatively modest variation suggests that this parameter has limited influence on the measured metric within the tested range. Confidence in these findings is high given the direct correspondence between CSV data and plotted values. Future analysis should consider incorporating error bars representing variance across multiple experimental runs to strengthen statistical validity.

\begin{figure}[!htbp]
\centering
\includegraphics[width=0.95\linewidth]{images/Real_Time_Sumo_Threshold/(d)_runtime.png}
\caption{(d) runtime}
\label{fig:Real_Time_Sumo_Threshold_d__runtime}
\end{figure}

This bar chart presents the relationship between threshold and run time for the Real Time Sumo Threshold experimental scenario. The x-axis displays threshold values ranging from 5 to 20, while the y-axis quantifies run time. The single-series visualization facilitates analysis of how the dependent variable responds to changes in the independent parameter setting.

Analysis of the plotted data reveals that run time ranges from 546.38 (at threshold = 10) to 568.65 (at threshold = 15), representing a span of 22.27 units. The overall trend is increasing, with values rising from 547.79 at the initial setting to 553.99 at the final setting. Notably, the minimum value occurs at an intermediate threshold setting (10), suggesting non-monotonic behavior that warrants further investigation.

These results indicate that threshold configuration meaningfully impacts run time in this experimental context. The relatively modest variation suggests that this parameter has limited influence on the measured metric within the tested range. Confidence in these findings is high given the direct correspondence between CSV data and plotted values. Future analysis should consider incorporating error bars representing variance across multiple experimental runs to strengthen statistical validity.


\clearpage

\subsection{Sumo Static Module Change}

\begin{figure}[!htbp]
\centering
\includegraphics[width=0.95\linewidth]{images/Sumo_Static_Module_Change/(a)_Travel_Time.png}
\caption{(a) Travel Time}
\label{fig:Sumo_Static_Module_Change_a__Travel_Time}
\end{figure}

This bar chart presents the relationship between modules and total travel time for the Sumo Static Module Change experimental scenario. The x-axis displays modules values ranging from 3 to 6, while the y-axis quantifies total travel time. The single-series visualization facilitates analysis of how the dependent variable responds to changes in the independent parameter setting.

Analysis of the plotted data reveals that total travel time ranges from 92.73 (at modules = 5) to 108.71 (at modules = 6), representing a span of 15.98 units. The overall trend is increasing, with values rising from 94.47 at the initial setting to 108.71 at the final setting. Notably, the minimum value occurs at an intermediate modules setting (5), suggesting non-monotonic behavior that warrants further investigation.

These results indicate that modules configuration meaningfully impacts total travel time in this experimental context. The relatively modest variation suggests that this parameter has limited influence on the measured metric within the tested range. Confidence in these findings is high given the direct correspondence between CSV data and plotted values. Future analysis should consider incorporating error bars representing variance across multiple experimental runs to strengthen statistical validity.

\begin{figure}[!htbp]
\centering
\includegraphics[width=0.95\linewidth]{images/Sumo_Static_Module_Change/(b)_Energy.png}
\caption{(b) Energy}
\label{fig:Sumo_Static_Module_Change_b__Energy}
\end{figure}

This bar chart presents the relationship between modules and total energy consumed for the Sumo Static Module Change experimental scenario. The x-axis displays modules values ranging from 3 to 6, while the y-axis quantifies total energy consumed. The single-series visualization facilitates analysis of how the dependent variable responds to changes in the independent parameter setting.

Analysis of the plotted data reveals that total energy consumed ranges from 9.05 (at modules = 5) to 11.61 (at modules = 6), representing a span of 2.55 units. The overall trend is increasing, with values rising from 9.29 at the initial setting to 11.61 at the final setting. Notably, the minimum value occurs at an intermediate modules setting (5), suggesting non-monotonic behavior that warrants further investigation.

These results indicate that modules configuration meaningfully impacts total energy consumed in this experimental context. The substantial variation observed (coefficient of variation exceeding 20\%) suggests that parameter tuning could yield significant performance improvements. Confidence in these findings is high given the direct correspondence between CSV data and plotted values. Future analysis should consider incorporating error bars representing variance across multiple experimental runs to strengthen statistical validity.

\begin{figure}[!htbp]
\centering
\includegraphics[width=0.95\linewidth]{images/Sumo_Static_Module_Change/(c)_Distance.png}
\caption{(c) Distance}
\label{fig:Sumo_Static_Module_Change_c__Distance}
\end{figure}

This bar chart presents the relationship between modules and total distance covered for the Sumo Static Module Change experimental scenario. The x-axis displays modules values ranging from 3 to 6, while the y-axis quantifies total distance covered. The single-series visualization facilitates analysis of how the dependent variable responds to changes in the independent parameter setting.

Analysis of the plotted data reveals that total distance covered ranges from 57.52 (at modules = 5) to 69.86 (at modules = 6), representing a span of 12.34 units. The overall trend is increasing, with values rising from 59.51 at the initial setting to 69.86 at the final setting. Notably, the minimum value occurs at an intermediate modules setting (5), suggesting non-monotonic behavior that warrants further investigation.

These results indicate that modules configuration meaningfully impacts total distance covered in this experimental context. The relatively modest variation suggests that this parameter has limited influence on the measured metric within the tested range. Confidence in these findings is high given the direct correspondence between CSV data and plotted values. Future analysis should consider incorporating error bars representing variance across multiple experimental runs to strengthen statistical validity.

\begin{figure}[!htbp]
\centering
\includegraphics[width=0.95\linewidth]{images/Sumo_Static_Module_Change/(d)_Module_Swapped.png}
\caption{(d) Module Swapped}
\label{fig:Sumo_Static_Module_Change_d__Module_Swapped}
\end{figure}

This bar chart presents the relationship between modules and modules for the Sumo Static Module Change experimental scenario. The x-axis displays modules values ranging from 3 to 6, while the y-axis quantifies modules. The single-series visualization facilitates analysis of how the dependent variable responds to changes in the independent parameter setting.

Analysis of the plotted data reveals that modules ranges from 3.00 (at modules = 3) to 6.00 (at modules = 6), representing a span of 3.00 units. The overall trend is increasing, with values rising from 3.00 at the initial setting to 6.00 at the final setting. 

These results indicate that modules configuration meaningfully impacts modules in this experimental context. The substantial variation observed (coefficient of variation exceeding 20\%) suggests that parameter tuning could yield significant performance improvements. Confidence in these findings is high given the direct correspondence between CSV data and plotted values. Future analysis should consider incorporating error bars representing variance across multiple experimental runs to strengthen statistical validity.

\begin{figure}[!htbp]
\centering
\includegraphics[width=0.95\linewidth]{images/Sumo_Static_Module_Change/(e)_Run_Time.png}
\caption{(e) Run Time}
\label{fig:Sumo_Static_Module_Change_e__Run_Time}
\end{figure}

This bar chart presents the relationship between modules and total travel time for the Sumo Static Module Change experimental scenario. The x-axis displays modules values ranging from 3 to 6, while the y-axis quantifies total travel time. The single-series visualization facilitates analysis of how the dependent variable responds to changes in the independent parameter setting.

Analysis of the plotted data reveals that total travel time ranges from 92.73 (at modules = 5) to 108.71 (at modules = 6), representing a span of 15.98 units. The overall trend is increasing, with values rising from 94.47 at the initial setting to 108.71 at the final setting. Notably, the minimum value occurs at an intermediate modules setting (5), suggesting non-monotonic behavior that warrants further investigation.

These results indicate that modules configuration meaningfully impacts total travel time in this experimental context. The relatively modest variation suggests that this parameter has limited influence on the measured metric within the tested range. Confidence in these findings is high given the direct correspondence between CSV data and plotted values. Future analysis should consider incorporating error bars representing variance across multiple experimental runs to strengthen statistical validity.


\clearpage

\subsection{Sumo Static Swap Time}

\begin{figure}[!htbp]
\centering
\includegraphics[width=0.95\linewidth]{images/Sumo_Static_Swap_Time/(a)_Travel_Time.png}
\caption{(a) Travel Time}
\label{fig:Sumo_Static_Swap_Time_a__Travel_Time}
\end{figure}

This bar chart presents the relationship between swap time (min) and total travel time for the Sumo Static Swap Time experimental scenario. The x-axis displays swap time (min) values ranging from 1 to 4, while the y-axis quantifies total travel time. The single-series visualization facilitates analysis of how the dependent variable responds to changes in the independent parameter setting.

Analysis of the plotted data reveals that total travel time ranges from 85.81 (at swap time (min) = 4) to 100.69 (at swap time (min) = 3), representing a span of 14.88 units. The overall trend is decreasing, with values declining from 100.19 at the initial setting to 85.81 at the final setting. 

These results indicate that swap time (min) configuration meaningfully impacts total travel time in this experimental context. The relatively modest variation suggests that this parameter has limited influence on the measured metric within the tested range. Confidence in these findings is high given the direct correspondence between CSV data and plotted values. Future analysis should consider incorporating error bars representing variance across multiple experimental runs to strengthen statistical validity.

\begin{figure}[!htbp]
\centering
\includegraphics[width=0.95\linewidth]{images/Sumo_Static_Swap_Time/(b)_Energy.png}
\caption{(b) Energy}
\label{fig:Sumo_Static_Swap_Time_b__Energy}
\end{figure}

This bar chart presents the relationship between swap time (min) and total energy consumed for the Sumo Static Swap Time experimental scenario. The x-axis displays swap time (min) values ranging from 1 to 4, while the y-axis quantifies total energy consumed. The single-series visualization facilitates analysis of how the dependent variable responds to changes in the independent parameter setting.

Analysis of the plotted data reveals that total energy consumed ranges from 9.05 (at swap time (min) = 2) to 11.51 (at swap time (min) = 3), representing a span of 2.46 units. The overall trend is decreasing, with values declining from 11.00 at the initial setting to 9.75 at the final setting. Notably, the minimum value occurs at an intermediate swap time (min) setting (2), suggesting non-monotonic behavior that warrants further investigation.

These results indicate that swap time (min) configuration meaningfully impacts total energy consumed in this experimental context. The substantial variation observed (coefficient of variation exceeding 20\%) suggests that parameter tuning could yield significant performance improvements. Confidence in these findings is high given the direct correspondence between CSV data and plotted values. Future analysis should consider incorporating error bars representing variance across multiple experimental runs to strengthen statistical validity.

\begin{figure}[!htbp]
\centering
\includegraphics[width=0.95\linewidth]{images/Sumo_Static_Swap_Time/(c)_Distance.png}
\caption{(c) Distance}
\label{fig:Sumo_Static_Swap_Time_c__Distance}
\end{figure}

This bar chart presents the relationship between swap time (min) and total distance covered for the Sumo Static Swap Time experimental scenario. The x-axis displays swap time (min) values ranging from 1 to 4, while the y-axis quantifies total distance covered. The single-series visualization facilitates analysis of how the dependent variable responds to changes in the independent parameter setting.

Analysis of the plotted data reveals that total distance covered ranges from 57.11 (at swap time (min) = 4) to 66.91 (at swap time (min) = 3), representing a span of 9.80 units. The overall trend is decreasing, with values declining from 64.69 at the initial setting to 57.11 at the final setting. 

These results indicate that swap time (min) configuration meaningfully impacts total distance covered in this experimental context. The relatively modest variation suggests that this parameter has limited influence on the measured metric within the tested range. Confidence in these findings is high given the direct correspondence between CSV data and plotted values. Future analysis should consider incorporating error bars representing variance across multiple experimental runs to strengthen statistical validity.

\begin{figure}[!htbp]
\centering
\includegraphics[width=0.95\linewidth]{images/Sumo_Static_Swap_Time/(d)_Module_Swapped.png}
\caption{(d) Module Swapped}
\label{fig:Sumo_Static_Swap_Time_d__Module_Swapped}
\end{figure}

This bar chart presents the relationship between swap time (min) and total module swapped for the Sumo Static Swap Time experimental scenario. The x-axis displays swap time (min) values ranging from 1 to 4, while the y-axis quantifies total module swapped. The single-series visualization facilitates analysis of how the dependent variable responds to changes in the independent parameter setting.

Analysis of the plotted data reveals that total module swapped ranges from 0.00 (at swap time (min) = 1) to 0.00 (at swap time (min) = 1), representing a span of 0.00 units. The values remain relatively stable across the parameter range, with minimal net change between initial (0.00) and final (0.00) settings. 

These results indicate that swap time (min) configuration meaningfully impacts total module swapped in this experimental context. The relatively modest variation suggests that this parameter has limited influence on the measured metric within the tested range. Confidence in these findings is high given the direct correspondence between CSV data and plotted values. Future analysis should consider incorporating error bars representing variance across multiple experimental runs to strengthen statistical validity.

\begin{figure}[!htbp]
\centering
\includegraphics[width=0.95\linewidth]{images/Sumo_Static_Swap_Time/(e)_Run_Time.png}
\caption{(e) Run Time}
\label{fig:Sumo_Static_Swap_Time_e__Run_Time}
\end{figure}

This bar chart presents the relationship between swap time (min) and total travel time for the Sumo Static Swap Time experimental scenario. The x-axis displays swap time (min) values ranging from 1 to 4, while the y-axis quantifies total travel time. The single-series visualization facilitates analysis of how the dependent variable responds to changes in the independent parameter setting.

Analysis of the plotted data reveals that total travel time ranges from 85.81 (at swap time (min) = 4) to 100.69 (at swap time (min) = 3), representing a span of 14.88 units. The overall trend is decreasing, with values declining from 100.19 at the initial setting to 85.81 at the final setting. 

These results indicate that swap time (min) configuration meaningfully impacts total travel time in this experimental context. The relatively modest variation suggests that this parameter has limited influence on the measured metric within the tested range. Confidence in these findings is high given the direct correspondence between CSV data and plotted values. Future analysis should consider incorporating error bars representing variance across multiple experimental runs to strengthen statistical validity.


\clearpage

\subsection{Sumo Static Threshold}

\begin{figure}[!htbp]
\centering
\includegraphics[width=0.95\linewidth]{images/Sumo_Static_Threshold/(a)_Travel_Time.png}
\caption{(a) Travel Time}
\label{fig:Sumo_Static_Threshold_a__Travel_Time}
\end{figure}

This bar chart presents the relationship between threshold and total travel time for the Sumo Static Threshold experimental scenario. The x-axis displays threshold values ranging from 5 to 20, while the y-axis quantifies total travel time. The single-series visualization facilitates analysis of how the dependent variable responds to changes in the independent parameter setting.

Analysis of the plotted data reveals that total travel time ranges from 86.54 (at threshold = 10) to 106.44 (at threshold = 5), representing a span of 19.90 units. The overall trend is decreasing, with values declining from 106.44 at the initial setting to 92.73 at the final setting. Notably, the minimum value occurs at an intermediate threshold setting (10), suggesting non-monotonic behavior that warrants further investigation.

These results indicate that threshold configuration meaningfully impacts total travel time in this experimental context. The substantial variation observed (coefficient of variation exceeding 20\%) suggests that parameter tuning could yield significant performance improvements. Confidence in these findings is high given the direct correspondence between CSV data and plotted values. Future analysis should consider incorporating error bars representing variance across multiple experimental runs to strengthen statistical validity.

\begin{figure}[!htbp]
\centering
\includegraphics[width=0.95\linewidth]{images/Sumo_Static_Threshold/(b)_Energy.png}
\caption{(b) Energy}
\label{fig:Sumo_Static_Threshold_b__Energy}
\end{figure}

This bar chart presents the relationship between threshold and total energy consumed for the Sumo Static Threshold experimental scenario. The x-axis displays threshold values ranging from 5 to 20, while the y-axis quantifies total energy consumed. The single-series visualization facilitates analysis of how the dependent variable responds to changes in the independent parameter setting.

Analysis of the plotted data reveals that total energy consumed ranges from 8.80 (at threshold = 10) to 11.34 (at threshold = 5), representing a span of 2.54 units. The overall trend is decreasing, with values declining from 11.34 at the initial setting to 9.05 at the final setting. Notably, the minimum value occurs at an intermediate threshold setting (10), suggesting non-monotonic behavior that warrants further investigation.

These results indicate that threshold configuration meaningfully impacts total energy consumed in this experimental context. The substantial variation observed (coefficient of variation exceeding 20\%) suggests that parameter tuning could yield significant performance improvements. Confidence in these findings is high given the direct correspondence between CSV data and plotted values. Future analysis should consider incorporating error bars representing variance across multiple experimental runs to strengthen statistical validity.

\begin{figure}[!htbp]
\centering
\includegraphics[width=0.95\linewidth]{images/Sumo_Static_Threshold/(c)_Distance.png}
\caption{(c) Distance}
\label{fig:Sumo_Static_Threshold_c__Distance}
\end{figure}

This bar chart presents the relationship between threshold and total distance covered for the Sumo Static Threshold experimental scenario. The x-axis displays threshold values ranging from 5 to 20, while the y-axis quantifies total distance covered. The single-series visualization facilitates analysis of how the dependent variable responds to changes in the independent parameter setting.

Analysis of the plotted data reveals that total distance covered ranges from 55.14 (at threshold = 10) to 69.18 (at threshold = 5), representing a span of 14.04 units. The overall trend is decreasing, with values declining from 69.18 at the initial setting to 57.52 at the final setting. Notably, the minimum value occurs at an intermediate threshold setting (10), suggesting non-monotonic behavior that warrants further investigation.

These results indicate that threshold configuration meaningfully impacts total distance covered in this experimental context. The substantial variation observed (coefficient of variation exceeding 20\%) suggests that parameter tuning could yield significant performance improvements. Confidence in these findings is high given the direct correspondence between CSV data and plotted values. Future analysis should consider incorporating error bars representing variance across multiple experimental runs to strengthen statistical validity.

\begin{figure}[!htbp]
\centering
\includegraphics[width=0.95\linewidth]{images/Sumo_Static_Threshold/(d)_Module_Swapped.png}
\caption{(d) Module Swapped}
\label{fig:Sumo_Static_Threshold_d__Module_Swapped}
\end{figure}

This bar chart presents the relationship between threshold and total module swapped for the Sumo Static Threshold experimental scenario. The x-axis displays threshold values ranging from 5 to 20, while the y-axis quantifies total module swapped. The single-series visualization facilitates analysis of how the dependent variable responds to changes in the independent parameter setting.

Analysis of the plotted data reveals that total module swapped ranges from 0.00 (at threshold = 5) to 0.00 (at threshold = 5), representing a span of 0.00 units. The values remain relatively stable across the parameter range, with minimal net change between initial (0.00) and final (0.00) settings. 

These results indicate that threshold configuration meaningfully impacts total module swapped in this experimental context. The relatively modest variation suggests that this parameter has limited influence on the measured metric within the tested range. Confidence in these findings is high given the direct correspondence between CSV data and plotted values. Future analysis should consider incorporating error bars representing variance across multiple experimental runs to strengthen statistical validity.

\begin{figure}[!htbp]
\centering
\includegraphics[width=0.95\linewidth]{images/Sumo_Static_Threshold/(e)_Run_Time.png}
\caption{(e) Run Time}
\label{fig:Sumo_Static_Threshold_e__Run_Time}
\end{figure}

This bar chart presents the relationship between threshold and total travel time for the Sumo Static Threshold experimental scenario. The x-axis displays threshold values ranging from 5 to 20, while the y-axis quantifies total travel time. The single-series visualization facilitates analysis of how the dependent variable responds to changes in the independent parameter setting.

Analysis of the plotted data reveals that total travel time ranges from 86.54 (at threshold = 10) to 106.44 (at threshold = 5), representing a span of 19.90 units. The overall trend is decreasing, with values declining from 106.44 at the initial setting to 92.73 at the final setting. Notably, the minimum value occurs at an intermediate threshold setting (10), suggesting non-monotonic behavior that warrants further investigation.

These results indicate that threshold configuration meaningfully impacts total travel time in this experimental context. The substantial variation observed (coefficient of variation exceeding 20\%) suggests that parameter tuning could yield significant performance improvements. Confidence in these findings is high given the direct correspondence between CSV data and plotted values. Future analysis should consider incorporating error bars representing variance across multiple experimental runs to strengthen statistical validity.


\clearpage
